\documentclass[8pt,aspectratio=169]{beamer}
\usetheme{Madrid}
\usepackage[utf8]{inputenc}
\usepackage{graphicx}
\usepackage{booktabs}
\usepackage{hyperref}
\usepackage{amsmath}

\newcommand{\bottomnote}[1]{\vfill\par\noindent\footnotesize\textit{#1}}

\title{L46: Swiss FINMA and EU MiCA}
\subtitle{Module G: Regulation \& Future}
\author{Blockchain \& Cryptocurrency Course}
\date{December 2025}

\begin{document}

\begin{frame}
\titlepage
\end{frame}

\begin{frame}{Learning Objectives}
\begin{itemize}
\item Understand Swiss FINMA's principles-based approach to crypto regulation
\item Analyze token classification frameworks (FINMA vs MiCA)
\item Evaluate EU MiCA as comprehensive crypto-asset regulation
\item Compare stablecoin requirements between frameworks
\item Assess CASP licensing and compliance requirements
\item Understand real-world implementation challenges (2024-2025)
\end{itemize}
\end{frame}

\begin{frame}{Why Focus on Switzerland and EU?}
\begin{columns}[T]
\begin{column}{0.48\textwidth}
\textbf{Switzerland}
\begin{itemize}
\item Crypto Valley (Zug): 1,000+ blockchain companies
\item Clear legal framework since 2019
\item Home to Ethereum, Cardano, Tezos foundations
\item Principles-based, flexible approach
\end{itemize}
\end{column}
\begin{column}{0.48\textwidth}
\textbf{EU MiCA}
\begin{itemize}
\item First comprehensive crypto framework
\item 27 member states (450M people)
\item Template for other jurisdictions
\item Rules-based, detailed requirements
\end{itemize}
\end{column}
\end{columns}
\vspace{0.3cm}
\textbf{Key Contrast}: Principles-based (Switzerland) vs rules-based (EU)
\end{frame}

\begin{frame}[t]{Token Classification Comparison}
\begin{center}
\includegraphics[width=0.65\textwidth]{charts/01_token_classification/chart.pdf}
\end{center}
\bottomnote{Economic function determines regulatory treatment, not technology}
\end{frame}

\begin{frame}{FINMA Token Classification}
\begin{itemize}
\item \textbf{Three Token Categories} (not mutually exclusive):
\begin{enumerate}
\item \textbf{Payment Tokens}: Means of payment, transfer of value
\begin{itemize}
\item Examples: Bitcoin, Litecoin
\item Regulation: AML only (not securities law)
\end{itemize}
\item \textbf{Utility Tokens}: Access to application or service
\begin{itemize}
\item Examples: Filecoin, Golem
\item Regulation: Minimal (if truly utility)
\end{itemize}
\item \textbf{Asset Tokens}: Represent assets, claims, or equity
\begin{itemize}
\item Examples: Tokenized securities, equity tokens
\item Regulation: Securities law (prospectus, licensing)
\end{itemize}
\end{enumerate}
\item \textbf{Hybrid Tokens}: Can have multiple characteristics
\end{itemize}
\end{frame}

\begin{frame}{DLT Act (2021): Swiss Blockchain Legislation}
\begin{itemize}
\item \textbf{Purpose}: Adapt Swiss law to blockchain technology
\item \textbf{Key Innovations}:
\begin{enumerate}
\item \textbf{DLT Securities}: Legal recognition of tokenized securities
\begin{itemize}
\item Rights register maintained on blockchain
\item Same legal status as traditional securities
\end{itemize}
\item \textbf{DLT Trading Facilities}: New license category
\begin{itemize}
\item Lower barriers than traditional exchanges
\item Capital: CHF 500,000 minimum
\end{itemize}
\item \textbf{Crypto Asset Segregation in Bankruptcy}
\begin{itemize}
\item Customer assets segregated from estate
\end{itemize}
\end{enumerate}
\item \textbf{Impact}: Legal certainty for tokenization, institutional custody
\end{itemize}
\end{frame}

\begin{frame}[t]{Swiss Crypto Valley Growth}
\begin{center}
\includegraphics[width=0.55\textwidth]{charts/04_crypto_valley/chart.pdf}
\end{center}
\bottomnote{Regulatory clarity attracted global blockchain companies to Switzerland}
\end{frame}

\begin{frame}{MiCA: Overview}
\begin{itemize}
\item \textbf{MiCA (Markets in Crypto-Assets Regulation)}: EU-wide framework
\item \textbf{Timeline}:
\begin{itemize}
\item Proposed: September 2020
\item Approved: April 2023
\item Full implementation: December 30, 2024
\end{itemize}
\item \textbf{Objectives}:
\begin{enumerate}
\item Legal certainty for crypto assets
\item Consumer and investor protection
\item Financial stability safeguards
\item Support innovation and competition
\end{enumerate}
\item \textbf{Scope}: Crypto-assets, issuers, CASPs
\item \textbf{Exclusions}: NFTs (unless fractionalized), CBDCs
\end{itemize}
\end{frame}

\begin{frame}{MiCA: Crypto-Asset Definitions}
\begin{enumerate}
\item \textbf{E-Money Tokens (EMTs)}:
\begin{itemize}
\item Stable value referencing single fiat currency
\item Examples: USDC, USDT
\item Regulation: Strictest (banking-like)
\end{itemize}
\item \textbf{Asset-Referenced Tokens (ARTs)}:
\begin{itemize}
\item Stable value referencing basket/commodities
\item Regulation: Capital, reserve, governance
\end{itemize}
\item \textbf{Other Crypto-Assets}:
\begin{itemize}
\item All other tokens (BTC, ETH, utility)
\item Regulation: Disclosure (white paper)
\end{itemize}
\end{enumerate}
\end{frame}

\begin{frame}[t]{MiCA CASP Capital Requirements}
\begin{center}
\includegraphics[width=0.55\textwidth]{charts/02_casp_requirements/chart.pdf}
\end{center}
\bottomnote{Single authorization valid across all 27 EU member states (passporting)}
\end{frame}

\begin{frame}{MiCA: CASP Services}
\begin{itemize}
\item \textbf{CASP Services} (require authorization):
\begin{enumerate}
\item Custody and administration of crypto-assets
\item Operation of trading platform
\item Exchange (crypto-fiat, crypto-crypto)
\item Execution of orders on behalf of clients
\item Placing of crypto-assets
\item Providing advice on crypto-assets
\item Portfolio management
\item Transfer services
\end{enumerate}
\item \textbf{Passporting}: Single authorization valid EU-wide
\item \textbf{Grandfathering}: 18-month transition for existing providers
\end{itemize}
\end{frame}

\begin{frame}{MiCA: E-Money Token Requirements}
\begin{itemize}
\item \textbf{Issuer Authorization}:
\begin{itemize}
\item Must be credit institution or e-money institution
\item CASP license NOT sufficient
\end{itemize}
\item \textbf{Reserve Requirements}:
\begin{itemize}
\item 1:1 backing in high-quality liquid assets
\item Segregated from issuer's assets
\item Daily reconciliation
\end{itemize}
\item \textbf{Redemption Rights}:
\begin{itemize}
\item Redeem at par value at any time
\item No fees for redemption
\end{itemize}
\item \textbf{Significant EMTs}: Enhanced EBA supervision
\item \textbf{Prohibition}: Interest payments on stablecoins
\end{itemize}
\end{frame}

\begin{frame}[t]{Stablecoin MiCA Compliance (2024)}
\begin{center}
\includegraphics[width=0.55\textwidth]{charts/03_stablecoin_compliance/chart.pdf}
\end{center}
\bottomnote{USDT delisted from EU exchanges; USDC positioned as MiCA-compliant}
\end{frame}

\begin{frame}{Stablecoin Compliance Reality}
\begin{itemize}
\item \textbf{Circle (USDC)}:
\begin{itemize}
\item Obtained e-money license in France (July 2024)
\item First global stablecoin issuer MiCA-compliant
\item USDC and EURC fully authorized
\end{itemize}
\item \textbf{Tether (USDT)}:
\begin{itemize}
\item Did NOT obtain EU e-money license
\item Delisted from EU exchanges (December 2024)
\item Coinbase Europe, Crypto.com removed USDT
\end{itemize}
\item \textbf{Market Impact}:
\begin{itemize}
\item USDC gaining EU market share
\item Euro stablecoins emerging (EURC, EURS)
\item Compliance as competitive advantage
\end{itemize}
\end{itemize}
\end{frame}

\begin{frame}{MiCA: Market Abuse Provisions}
\begin{itemize}
\item \textbf{Prohibited Conduct}:
\begin{enumerate}
\item \textbf{Insider Dealing}:
\begin{itemize}
\item Trading on material non-public information
\item Same rules as traditional securities
\end{itemize}
\item \textbf{Market Manipulation}:
\begin{itemize}
\item Wash trading, spoofing, layering
\item Pump-and-dump schemes
\item Spreading false information
\end{itemize}
\item \textbf{Unlawful Disclosure}
\end{enumerate}
\item \textbf{Enforcement}: National authorities, criminal penalties
\item \textbf{Surveillance}: CASPs must monitor and report
\end{itemize}
\end{frame}

\begin{frame}[t]{Framework Comparison}
\begin{center}
\includegraphics[width=0.55\textwidth]{charts/05_framework_comparison/chart.pdf}
\end{center}
\bottomnote{Both frameworks provide regulatory clarity; MiCA offers broader market access}
\end{frame}

\begin{frame}{Swiss vs EU: Key Differences}
\begin{table}
\centering
\small
\begin{tabular}{lll}
\toprule
\textbf{Aspect} & \textbf{Switzerland} & \textbf{EU MiCA} \\
\midrule
Approach & Principles-based & Rules-based \\
Token Classification & Flexible (3 types) & Rigid (3 types) \\
Stablecoin Rules & Case-by-case & Banking-like \\
Market Access & Swiss only & 27 EU countries \\
DeFi Treatment & Unclear & Excluded (for now) \\
Compliance Cost & Moderate & Higher \\
\bottomrule
\end{tabular}
\end{table}
\vspace{0.3cm}
\textbf{Strategy}: Swiss firms seeking EU presence for MiCA passport
\end{frame}

\begin{frame}{CASP Authorization Progress (2024-2025)}
\begin{itemize}
\item \textbf{Major Exchanges Seeking Authorization}:
\begin{itemize}
\item Binance: Applied in multiple EU jurisdictions
\item Kraken: Pursuing licenses in Germany, Ireland
\item Coinbase: Leveraging Irish e-money license
\item Bitstamp: Already licensed in Luxembourg
\end{itemize}
\item \textbf{Bottlenecks}:
\begin{itemize}
\item National regulators overwhelmed with applications
\item Varying interpretation across member states
\item Technical standards still being finalized
\end{itemize}
\item \textbf{Preferred Jurisdictions}: Ireland, France, Germany
\end{itemize}
\end{frame}

\begin{frame}{MiCA 2.0: What's Next}
\begin{itemize}
\item \textbf{European Commission Review} (expected 2025):
\begin{itemize}
\item DeFi regulation (currently excluded)
\item NFT treatment (fractionalization rules)
\item Staking and lending services
\end{itemize}
\item \textbf{Transfer of Funds Regulation (TFR)}:
\begin{itemize}
\item Travel Rule for crypto (sender/receiver info)
\item Self-hosted wallet verification above thresholds
\end{itemize}
\item \textbf{Technical Standards} (ESMA, EBA):
\begin{itemize}
\item Custody specifications
\item Sustainability disclosures (PoW energy)
\end{itemize}
\item \textbf{Global Influence}: MiCA as template for UK, Japan, Singapore
\end{itemize}
\end{frame}

\begin{frame}{Summary}
\textbf{Key Takeaways:}
\begin{itemize}
\item \textbf{Swiss FINMA}: Principles-based, flexible, DLT Act innovation
\item \textbf{Token classification}: Payment, Utility, Asset (function-based)
\item \textbf{EU MiCA}: Comprehensive rules-based framework, 27 countries
\item \textbf{MiCA categories}: EMTs, ARTs, Other crypto-assets
\item \textbf{Stablecoin regulation}: Banking-like requirements, reserve backing
\item \textbf{December 2024}: MiCA fully implemented, USDT delisted
\item \textbf{USDC vs USDT}: Compliance as competitive advantage
\item \textbf{Crypto Valley}: 1,100+ companies, regulatory clarity advantage
\item \textbf{Future}: MiCA 2.0 (DeFi, NFTs), Travel Rule enforcement
\end{itemize}
\end{frame}

\begin{frame}{Questions for Reflection}
\begin{enumerate}
\item What are the trade-offs between principles-based vs rules-based regulation?
\item Why did Circle succeed in obtaining MiCA compliance while Tether did not?
\item How does the EU passport benefit crypto firms compared to Swiss-only?
\item Should DeFi protocols be included in MiCA 2.0?
\item What makes Crypto Valley attractive despite lacking EU market access?
\end{enumerate}
\end{frame}

\end{document}
