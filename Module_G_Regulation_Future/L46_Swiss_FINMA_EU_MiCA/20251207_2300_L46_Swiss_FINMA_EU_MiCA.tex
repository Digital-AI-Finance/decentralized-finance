\documentclass[8pt,aspectratio=169]{beamer}
\usetheme{Madrid}
\usepackage[utf8]{inputenc}
\usepackage{graphicx}
\usepackage{booktabs}
\usepackage{hyperref}
\usepackage{amsmath}

\title{L46: Swiss FINMA and EU MiCA}
\subtitle{Module G: Regulation \& Future}
\author{Blockchain \& Cryptocurrency Course}
\date{December 2025}

\begin{document}

\begin{frame}
\titlepage
\end{frame}

\begin{frame}{Why Focus on Switzerland and EU?}
\begin{itemize}
\item \textbf{Switzerland}: Global leader in crypto regulation clarity
\begin{itemize}
\item Crypto Valley (Zug): 1,000+ blockchain companies
\item Clear legal framework since 2019
\item Home to Ethereum Foundation, Cardano Foundation, Tezos
\end{itemize}
\item \textbf{EU MiCA}: First comprehensive crypto regulatory framework
\begin{itemize}
\item Applies to 27 member states (450M people)
\item Template for other jurisdictions globally
\item Effective 2024-2025
\end{itemize}
\vspace{0.3cm}
\item \textbf{Relevance}: Both frameworks offer regulatory certainty for builders and investors
\item \textbf{Contrast}: Principles-based (Switzerland) vs rules-based (EU)
\end{itemize}
\end{frame}

\section{Switzerland: FINMA Framework}

\begin{frame}{FINMA: Swiss Financial Regulator}
\begin{itemize}
\item \textbf{FINMA (Swiss Financial Market Supervisory Authority)}: Independent regulator
\item \textbf{Mandate}: Protect creditors, investors, and insured parties; ensure financial market functionality
\item \textbf{Crypto Regulation Milestones}:
\begin{itemize}
\item 2017: First ICO guidelines
\item 2019: Comprehensive token classification guidance
\item 2021: DLT Act (blockchain-specific legislation)
\item 2023: Updated guidance on DeFi and stablecoins
\end{itemize}
\item \textbf{Approach}: Technology-neutral, principles-based
\item \textbf{Philosophy}: ``Same business, same risk, same rules''
\end{itemize}
\end{frame}

\begin{frame}{FINMA Token Classification}
\begin{itemize}
\item \textbf{Three Token Categories} (not mutually exclusive):
\begin{enumerate}
\item \textbf{Payment Tokens}: Means of payment, transfer of value
\begin{itemize}
\item Examples: Bitcoin, Litecoin
\item Regulation: AML (not securities law)
\end{itemize}
\item \textbf{Utility Tokens}: Access to application or service
\begin{itemize}
\item Examples: Filecoin (storage), Golem (compute)
\item Regulation: No specific financial regulation (if truly utility)
\end{itemize}
\item \textbf{Asset Tokens}: Represent assets, claims, or equity
\begin{itemize}
\item Examples: Tokenized securities, equity tokens
\item Regulation: Securities law (prospectus, licensing)
\end{itemize}
\end{enumerate}
\item \textbf{Hybrid Tokens}: Can have multiple characteristics (e.g., utility + payment)
\end{itemize}
\end{frame}

\begin{frame}{Token Classification: Decision Tree}
\begin{enumerate}
\item \textbf{Does token represent an asset or economic right?}
\begin{itemize}
\item Yes $\rightarrow$ \textbf{Asset Token} (securities regulation)
\item No $\rightarrow$ Continue to question 2
\end{itemize}
\item \textbf{Is token primarily used for payments or value transfer?}
\begin{itemize}
\item Yes $\rightarrow$ \textbf{Payment Token} (AML only)
\item No $\rightarrow$ Continue to question 3
\end{itemize}
\item \textbf{Does token grant access to application or service?}
\begin{itemize}
\item Yes, and platform exists $\rightarrow$ \textbf{Utility Token} (minimal regulation)
\item Yes, but platform doesn't exist $\rightarrow$ May be asset token (investment contract)
\end{itemize}
\end{enumerate}
\vspace{0.3cm}
\textbf{Key Insight}: Economic function determines regulatory treatment, not technology
\end{frame}

\begin{frame}{FINMA: ICO Guidelines}
\begin{itemize}
\item \textbf{Key Requirements for Token Issuances}:
\begin{enumerate}
\item \textbf{AML Compliance}:
\begin{itemize}
\item KYC for token purchasers (if payment tokens)
\item Anti-money laundering checks
\item Affiliation with Swiss financial intermediary
\end{itemize}
\item \textbf{Securities Law} (if asset tokens):
\begin{itemize}
\item Prospectus requirement (unless exemption)
\item Authorization as securities house (if professional issuer)
\end{itemize}
\item \textbf{Banking Law}:
\begin{itemize}
\item Public deposit-taking requires banking license
\item Exception: Pre-sale if tokens delivered within 6 months
\end{itemize}
\end{enumerate}
\item \textbf{Transparency}: FINMA publishes guidance, no-action letters
\item \textbf{Flexibility}: Case-by-case assessment, no rigid rules
\end{itemize}
\end{frame}

\begin{frame}{DLT Act (2021): Blockchain-Specific Legislation}
\begin{itemize}
\item \textbf{Purpose}: Adapt Swiss law to blockchain technology
\item \textbf{Key Innovations}:
\begin{enumerate}
\item \textbf{DLT Securities}: Legal recognition of tokenized securities
\begin{itemize}
\item Rights register maintained on blockchain (not central registry)
\item Same legal status as traditional securities
\end{itemize}
\item \textbf{DLT Trading Facilities}: New license category
\begin{itemize}
\item For platforms trading tokenized securities
\item Lower barriers than traditional exchanges
\item Prudential requirements, custody standards
\end{itemize}
\item \textbf{Segregation of Crypto Assets in Bankruptcy}:
\begin{itemize}
\item Customer crypto assets segregated from bankruptcy estate
\item Protection against insolvency (unlike general creditor claim)
\end{itemize}
\end{enumerate}
\item \textbf{Impact}: Legal certainty for tokenization, institutional custody
\end{itemize}
\end{frame}

\begin{frame}{DLT Trading Facility License}
\begin{columns}[T]
\begin{column}{0.48\textwidth}
\textbf{Requirements}
\begin{itemize}
\item Minimum capital: CHF 500,000
\item Fit-and-proper management
\item Adequate organizational structure
\item Risk management and internal controls
\item Custody arrangements (segregation)
\item AML/CFT compliance
\end{itemize}
\vspace{0.3cm}
\textbf{Scope}
\begin{itemize}
\item Trading of DLT securities
\item Central custody (optional)
\item Order matching
\item Settlement on DLT
\end{itemize}
\end{column}
\begin{column}{0.48\textwidth}
\textbf{Advantages}
\begin{itemize}
\item Lighter than traditional exchange license
\item Faster authorization process
\item Innovation-friendly framework
\item Legal certainty for participants
\end{itemize}
\vspace{0.3cm}
\textbf{Examples}
\begin{itemize}
\item SDX (SIX Digital Exchange): First FINMA-licensed DLT trading facility
\item Tokenized bonds, funds, securities
\item Regulated secondary market
\end{itemize}
\end{column}
\end{columns}
\end{frame}

\begin{frame}{FINMA: Stablecoin Regulation}
\begin{itemize}
\item \textbf{Classification}: Depends on structure
\begin{itemize}
\item \textbf{Fiat-backed (e.g., USDC)}: Payment token, may trigger e-money rules
\item \textbf{Asset-backed}: Asset token (securities regulation)
\item \textbf{Algorithmic}: Likely asset token (if investment scheme)
\end{itemize}
\item \textbf{Banking Law Trigger}:
\begin{itemize}
\item If accepts public deposits $\rightarrow$ Banking license required
\item Exception: Fully reserved, no maturity transformation, no credit risk
\end{itemize}
\item \textbf{Payment System Regulation}:
\begin{itemize}
\item Large-scale payment tokens may be systemically important
\item FINMA oversight of payment infrastructure
\end{itemize}
\item \textbf{Case Study}: Libra/Diem project (Facebook)
\begin{itemize}
\item Chose Switzerland for regulatory clarity
\item FINMA assessed as payment system (not bank)
\item Project ultimately abandoned for regulatory reasons globally
\end{itemize}
\end{itemize}
\end{frame}

\begin{frame}{Swiss Crypto Ecosystem}
\begin{itemize}
\item \textbf{Crypto Valley (Zug)}: Global blockchain hub since 2013
\begin{itemize}
\item 1,000+ blockchain companies
\item Ethereum Foundation, Cardano, Polkadot, Tezos foundations
\item Major exchanges: Bitcoin Suisse, Crypto Finance
\end{itemize}
\item \textbf{Supportive Factors}:
\begin{itemize}
\item Regulatory clarity and consistency
\item Political stability, rule of law
\item Banking infrastructure (crypto-friendly banks)
\item Tax competitiveness (low corporate tax in Zug)
\item Skilled workforce and universities
\end{itemize}
\item \textbf{Institutional Adoption}:
\begin{itemize}
\item SIX (Swiss stock exchange) operates SDX (digital asset exchange)
\item Swiss banks offer crypto custody (e.g., Julius Baer, Maerki Baumann)
\item Pension funds permitted to invest in crypto (small allocations)
\end{itemize}
\end{itemize}
\end{frame}

\section{EU: Markets in Crypto-Assets (MiCA)}

\begin{frame}{MiCA: Overview}
\begin{itemize}
\item \textbf{MiCA (Markets in Crypto-Assets Regulation)}: EU-wide framework
\item \textbf{Legislative Process}:
\begin{itemize}
\item Proposed: September 2020
\item Approved: April 2023 (EU Parliament and Council)
\item Entry into force: Phased 2024-2025
\end{itemize}
\item \textbf{Objectives}:
\begin{enumerate}
\item Legal certainty for crypto assets not covered by existing financial regulation
\item Support innovation and fair competition
\item Consumer and investor protection
\item Financial stability safeguards
\end{enumerate}
\item \textbf{Scope}: Crypto-assets, issuers, CASPs (Crypto-Asset Service Providers)
\item \textbf{Exclusions}: NFTs (unless fractionalized), CBDCs, utility tokens for specific services
\end{itemize}
\end{frame}

\begin{frame}{MiCA: Crypto-Asset Definitions}
\begin{enumerate}
\item \textbf{E-Money Tokens (EMTs)}:
\begin{itemize}
\item Maintain stable value by referencing single fiat currency
\item Examples: USDC, USDT (if compliant)
\item Regulation: Strictest (banking-like requirements)
\end{itemize}
\item \textbf{Asset-Referenced Tokens (ARTs)}:
\begin{itemize}
\item Maintain stable value by referencing basket of assets, commodities, or multiple currencies
\item Examples: Hypothetical basket-backed stablecoins
\item Regulation: Strict (capital, reserve, governance)
\end{itemize}
\item \textbf{Other Crypto-Assets}:
\begin{itemize}
\item All other tokens not EMTs or ARTs
\item Examples: Bitcoin, Ethereum, utility tokens
\item Regulation: Disclosure requirements (white paper)
\end{itemize}
\end{enumerate}
\end{frame}

\begin{frame}{MiCA: E-Money Token (EMT) Requirements}
\begin{itemize}
\item \textbf{Issuer Authorization}:
\begin{itemize}
\item Must be authorized credit institution or e-money institution
\item CASP license NOT sufficient for issuance
\end{itemize}
\item \textbf{Reserve Requirements}:
\begin{itemize}
\item 1:1 backing in high-quality liquid assets
\item Segregated from issuer's own assets
\item Custody with authorized entities
\item Daily reconciliation
\end{itemize}
\item \textbf{Redemption Rights}:
\begin{itemize}
\item Holders can redeem at par value at any time
\item No fees for redemption
\item Redemption in referenced fiat currency
\end{itemize}
\item \textbf{Supervision}: National competent authority (NCA) of home member state
\item \textbf{Impact}: Tether (USDT) may struggle to comply; Circle (USDC) better positioned
\end{itemize}
\end{frame}

\begin{frame}{MiCA: Significant E-Money Tokens}
\begin{itemize}
\item \textbf{Significance Criteria} (any one triggers):
\begin{itemize}
\item Customer base >10 million
\item Reserve assets value >5 billion EUR
\item Daily transactions >2.5 million, value >500 million EUR
\item Interconnectedness with financial system
\end{itemize}
\item \textbf{Enhanced Requirements}:
\begin{itemize}
\item EBA (European Banking Authority) supervision
\item Higher capital requirements
\item Liquidity management plan
\item Recovery and redemption plan
\item Stress testing
\item Interoperability requirements
\end{itemize}
\item \textbf{Limits on Use}:
\begin{itemize}
\item Restrictions on use as means of exchange in single member state (prevent EUR substitution)
\end{itemize}
\end{itemize}
\end{frame}

\begin{frame}{MiCA: Asset-Referenced Token (ART) Requirements}
\begin{itemize}
\item \textbf{Issuer Authorization}:
\begin{itemize}
\item Dedicated authorization (not credit institution)
\item Own funds requirement: 350,000 EUR or 2\% of reserve assets
\item Fit-and-proper management
\end{itemize}
\item \textbf{Reserve Assets}:
\begin{itemize}
\item Composition matching referenced assets
\item Custody with authorized entities
\item Segregation from issuer's assets
\item Investment policy (limits on asset types)
\end{itemize}
\item \textbf{Governance}:
\begin{itemize}
\item White paper approval by competent authority
\item Disclosure of reserve composition (quarterly reports)
\item Audit of reserves (annual)
\end{itemize}
\item \textbf{Prohibition}: Stablecoins offering interest payments (prevent bank-like deposit taking)
\end{itemize}
\end{frame}

\begin{frame}{MiCA: Crypto-Asset Service Providers (CASPs)}
\begin{itemize}
\item \textbf{CASP Services} (require authorization):
\begin{enumerate}
\item Custody and administration of crypto-assets
\item Operation of trading platform
\item Exchange of crypto-assets for fiat or other crypto
\item Execution of orders on behalf of clients
\item Placing of crypto-assets
\item Reception and transmission of orders
\item Providing advice on crypto-assets
\item Portfolio management
\item Transfer services
\end{enumerate}
\item \textbf{Authorization}: Single authorization valid across all EU member states (passporting)
\item \textbf{Grandfathering}: Existing providers have 18 months to comply
\end{itemize}
\end{frame}

\begin{frame}{MiCA: CASP Requirements}
\begin{columns}[T]
\begin{column}{0.48\textwidth}
\textbf{Organizational}
\begin{itemize}
\item Minimum capital: 50,000-150,000 EUR (service-dependent)
\item Fit-and-proper management
\item Conflict of interest policies
\item Complaint handling procedures
\item Outsourcing governance
\end{itemize}
\vspace{0.3cm}
\textbf{Operational}
\begin{itemize}
\item Custody requirements (segregation, cold storage)
\item Cybersecurity measures
\item Business continuity plans
\item AML/CFT compliance
\end{itemize}
\end{column}
\begin{column}{0.48\textwidth}
\textbf{Conduct of Business}
\begin{itemize}
\item Fair, clear, not misleading communications
\item Disclosure of fees and costs
\item Best execution for client orders
\item Prohibition of market manipulation, insider trading
\end{itemize}
\vspace{0.3cm}
\textbf{Transparency}
\begin{itemize}
\item Public disclosure of services
\item Conflicts of interest disclosure
\item Annual reports to competent authority
\end{itemize}
\end{column}
\end{columns}
\end{frame}

\begin{frame}{MiCA: Market Abuse Provisions}
\begin{itemize}
\item \textbf{Prohibited Conduct} (applies to crypto-assets under MiCA):
\begin{enumerate}
\item \textbf{Insider Dealing}:
\begin{itemize}
\item Trading based on material non-public information
\item Tipping (disclosing inside information)
\item Same rules as for traditional securities
\end{itemize}
\item \textbf{Market Manipulation}:
\begin{itemize}
\item Wash trading, spoofing, layering
\item Pump-and-dump schemes
\item Spreading false information
\end{itemize}
\item \textbf{Unlawful Disclosure}: Recommending transactions based on inside information
\end{enumerate}
\item \textbf{Enforcement}: National competent authorities, criminal penalties possible
\item \textbf{Surveillance}: CASPs must monitor and report suspicious transactions
\end{itemize}
\end{frame}

\begin{frame}{MiCA: White Paper Requirements}
\begin{itemize}
\item \textbf{Purpose}: Standardized disclosure document for crypto-asset offerings
\item \textbf{Required Information}:
\begin{itemize}
\item Issuer identification and information
\item Description of crypto-asset (purpose, technology, use cases)
\item Rights and obligations of holders
\item Underlying technology and consensus mechanism
\item Risks (comprehensive disclosure)
\item Sustainability impacts (energy consumption for PoW)
\item For ARTs/EMTs: Reserve composition, redemption rights
\end{itemize}
\item \textbf{Approval Process}:
\begin{itemize}
\item EMTs/ARTs: Competent authority approval required
\item Other crypto-assets: Notification to authority (20-day review)
\end{itemize}
\item \textbf{Liability}: Issuers liable for misleading or incomplete white papers
\end{itemize}
\end{frame}

\begin{frame}{MiCA vs Swiss Framework: Comparison}
\begin{table}
\centering
\small
\begin{tabular}{lll}
\toprule
\textbf{Aspect} & \textbf{Switzerland (FINMA)} & \textbf{EU (MiCA)} \\
\midrule
Approach & Principles-based & Rules-based \\
Legislation & Adapted existing + DLT Act & New comprehensive framework \\
Token Classification & 3 types (flexible) & 3 types (rigid) \\
Stablecoin Regulation & Case-by-case & Banking-like (EMTs) \\
Market Abuse & General law applies & Explicit crypto provisions \\
Passporting & No (Swiss-only) & Yes (EU-wide) \\
Scope & Narrower (financial services focus) & Broader (all CASPs) \\
DeFi Treatment & Unclear & Excluded (for now) \\
\bottomrule
\end{tabular}
\end{table}
\vspace{0.3cm}
\textbf{Both}: Clear frameworks attract institutional investment and innovation
\end{frame}

\begin{frame}{Impact on Industry}
\begin{itemize}
\item \textbf{Centralized Exchanges}:
\begin{itemize}
\item Must obtain CASP license (EU) or equivalent (Switzerland)
\item Enhanced custody and compliance costs
\item Barrier to entry increases (consolidation likely)
\end{itemize}
\item \textbf{Stablecoin Issuers}:
\begin{itemize}
\item Tether: May exit EU market or comply (costly)
\item Circle: Well-positioned (already reserves-backed)
\item New EU-based stablecoins may emerge
\end{itemize}
\item \textbf{DeFi Protocols}:
\begin{itemize}
\item MiCA: Currently excluded (no legal entity)
\item Future regulation likely (DeFi not regulated = loophole)
\item Some protocols may incorporate CASP entity for compliance
\end{itemize}
\item \textbf{Cross-Border Operations}: EU passport valuable, Swiss firms may seek EU presence
\end{itemize}
\end{frame}

\begin{frame}{Challenges and Criticisms}
\begin{columns}[T]
\begin{column}{0.48\textwidth}
\textbf{MiCA Criticisms}
\begin{itemize}
\item \textbf{Innovation Chilling}: Heavy compliance costs
\item \textbf{DeFi Gap}: Decentralized protocols not covered
\item \textbf{NFT Exemption}: Unclear boundaries (fractionalization)
\item \textbf{Self-Hosted Wallets}: Proposed KYC for large transfers (privacy concern)
\item \textbf{One-Size-Fits-All}: Diverse crypto assets treated uniformly
\end{itemize}
\end{column}
\begin{column}{0.48\textwidth}
\textbf{Swiss FINMA Challenges}
\begin{itemize}
\item \textbf{Limited Market}: Switzerland-only (no EU passport)
\item \textbf{DeFi Uncertainty}: Principles-based = case-by-case uncertainty
\item \textbf{Scaling Limits}: Regulatory capacity for rapid innovation
\item \textbf{Enforcement}: Cross-border crypto operations difficult to supervise
\end{itemize}
\end{column}
\end{columns}
\vspace{0.3cm}
\textbf{Industry Response}: Compliance costs as barrier to entry benefit incumbents
\end{frame}

\section{2024-2025 Updates}

\begin{frame}{MiCA Implementation Timeline (Realized)}
\begin{itemize}
\item \textbf{June 30, 2024}: Stablecoin provisions (EMTs, ARTs) entered into force
\begin{itemize}
\item Issuers must be authorized credit or e-money institutions
\item Reserve requirements and redemption rights mandatory
\item Existing stablecoins given grace period to comply
\end{itemize}
\item \textbf{December 30, 2024}: Full MiCA implementation
\begin{itemize}
\item All CASP provisions now in force
\item Existing providers must apply for authorization
\item 18-month grandfathering period begins
\end{itemize}
\item \textbf{Key Milestone}: First comprehensive crypto regulation globally now fully operational
\item \textbf{Scope}: 27 EU member states, 450 million people, single market for crypto
\end{itemize}
\end{frame}

\begin{frame}{Stablecoin Compliance Reality (2024)}
\begin{itemize}
\item \textbf{Circle (USDC)}:
\begin{itemize}
\item Obtained e-money license in France (July 2024)
\item First global stablecoin issuer MiCA-compliant
\item USDC and EURC fully authorized for EU market
\end{itemize}
\item \textbf{Tether (USDT)}:
\begin{itemize}
\item Did NOT obtain EU e-money license by deadline
\item Several EU exchanges delisted USDT (December 2024)
\item Coinbase Europe, Crypto.com removed USDT pairs
\item Tether exploring compliance options
\end{itemize}
\item \textbf{Market Impact}:
\begin{itemize}
\item USDC gaining EU market share
\item Euro-denominated stablecoins emerging (EURC, EURS)
\item Compliance as competitive advantage
\end{itemize}
\end{itemize}
\end{frame}

\begin{frame}{CASP Authorization Progress (2024-2025)}
\begin{itemize}
\item \textbf{Major Exchanges Seeking Authorization}:
\begin{itemize}
\item Binance: Applied in multiple EU jurisdictions
\item Kraken: Pursuing licenses in Germany, Ireland
\item Coinbase: Leveraging existing Irish e-money license
\item Bitstamp: Already licensed in Luxembourg
\end{itemize}
\item \textbf{Authorization Bottlenecks}:
\begin{itemize}
\item National regulators overwhelmed with applications
\item Varying interpretation across member states
\item Technical standards still being finalized (ESMA)
\end{itemize}
\item \textbf{Passporting Advantage}:
\begin{itemize}
\item Single authorization = access to entire EU market
\item Race to obtain first authorization
\item Ireland, France, Germany as preferred jurisdictions
\end{itemize}
\end{itemize}
\end{frame}

\begin{frame}{Swiss Developments 2024}
\begin{itemize}
\item \textbf{FINMA Guidance Updates}:
\begin{itemize}
\item Clarified DeFi treatment (case-by-case continues)
\item Staking services guidance (custody vs non-custody)
\item Enhanced AML requirements for crypto transactions
\end{itemize}
\item \textbf{SDX (SIX Digital Exchange)}:
\begin{itemize}
\item Expanded tokenized bond offerings
\item First regulated digital asset exchange operational
\item Institutional adoption growing
\end{itemize}
\item \textbf{Swiss-EU Coordination}:
\begin{itemize}
\item No equivalence agreement yet
\item Swiss firms seeking EU presence for MiCA passport
\item Dual licensing strategy emerging
\end{itemize}
\item \textbf{Crypto Valley}: Remains competitive despite MiCA creating EU single market
\end{itemize}
\end{frame}

\begin{frame}{MiCA 2.0: What's Next}
\begin{itemize}
\item \textbf{European Commission Review} (expected 2025):
\begin{itemize}
\item DeFi regulation (currently excluded)
\item NFT treatment (fractionalization, utility NFTs)
\item Staking and lending services
\end{itemize}
\item \textbf{Technical Standards} (ESMA, EBA ongoing):
\begin{itemize}
\item Custody requirements specification
\item Sustainability disclosures (PoW energy reporting)
\item Market surveillance systems
\end{itemize}
\item \textbf{Transfer of Funds Regulation (TFR)}:
\begin{itemize}
\item Travel Rule for crypto (sender/receiver info)
\item Applies to CASPs from December 2024
\item Self-hosted wallet verification above thresholds
\end{itemize}
\item \textbf{Global Influence}: MiCA as template for UK, Japan, Singapore frameworks
\end{itemize}
\end{frame}

\begin{frame}{Summary}
\begin{itemize}
\item \textbf{Switzerland}: Principles-based, early clarity, DLT Act innovation
\item \textbf{FINMA token classification}: Payment, Utility, Asset (flexible)
\item \textbf{DLT Trading Facility}: New license for tokenized securities exchanges
\item \textbf{EU MiCA}: Comprehensive rules-based framework, 27 countries
\item \textbf{MiCA categories}: E-Money Tokens (EMTs), Asset-Referenced Tokens (ARTs), Other
\item \textbf{Stablecoin regulation}: Banking-like requirements, reserve backing, redemption rights
\item \textbf{CASPs}: Licensing, capital, custody, market abuse rules
\item \textbf{2024 Milestone}: MiCA fully implemented December 30, 2024
\item \textbf{Stablecoin Reality}: Circle compliant, Tether delisted from EU exchanges
\item \textbf{Future}: MiCA 2.0 (DeFi, NFTs), Travel Rule enforcement, global template
\end{itemize}
\end{frame}

\end{document}
