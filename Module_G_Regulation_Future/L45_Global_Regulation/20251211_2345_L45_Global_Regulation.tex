\documentclass[8pt,aspectratio=169]{beamer}
\usetheme{Madrid}
\usepackage[utf8]{inputenc}
\usepackage{graphicx}
\usepackage{booktabs}
\usepackage{hyperref}
\usepackage{amsmath}

\newcommand{\bottomnote}[1]{\vfill\par\noindent\footnotesize\textit{#1}}

\title{L45: Global Regulatory Landscape}
\subtitle{Module G: Regulation \& Future}
\author{Blockchain \& Cryptocurrency Course}
\date{December 2025}

\begin{document}

\begin{frame}
\titlepage
\end{frame}

\begin{frame}{Learning Objectives}
\begin{itemize}
\item Understand why governments regulate cryptocurrencies
\item Compare regulatory approaches across major jurisdictions
\item Analyze the US fragmented regulatory landscape
\item Evaluate EU MiCA as comprehensive framework
\item Assess enforcement actions and their impact
\item Understand the significance of Bitcoin ETF approval
\end{itemize}
\end{frame}

\begin{frame}{Why Regulate Cryptocurrencies?}
\begin{columns}[T]
\begin{column}{0.48\textwidth}
\textbf{Regulatory Concerns}
\begin{itemize}
\item \textbf{Money Laundering}: Anonymous transactions facilitate illicit finance
\item \textbf{Consumer Protection}: Scams, rug pulls, exchange failures
\item \textbf{Market Manipulation}: Pump-and-dump, wash trading
\item \textbf{Tax Evasion}: Unreported capital gains
\item \textbf{Systemic Risk}: Contagion to traditional finance
\end{itemize}
\end{column}
\begin{column}{0.48\textwidth}
\textbf{Industry Arguments}
\begin{itemize}
\item \textbf{Legitimacy}: Legal certainty attracts institutional capital
\item \textbf{Innovation}: Clear rules enable compliant products
\item \textbf{Consumer Trust}: Regulated exchanges reduce fraud
\item \textbf{Financial Inclusion}: Regulated stablecoins for payments
\end{itemize}
\end{column}
\end{columns}
\vspace{0.2cm}
\textbf{Core Tension}: Innovation vs consumer protection vs financial stability
\end{frame}

\begin{frame}[t]{Global Regulatory Spectrum}
\begin{center}
\includegraphics[width=0.55\textwidth]{charts/01_regulatory_spectrum/chart.pdf}
\end{center}
\bottomnote{Most jurisdictions moving from restrictive toward regulated permissiveness}
\end{frame}

\begin{frame}{The Regulatory Spectrum}
\begin{table}
\centering
\small
\begin{tabular}{lll}
\toprule
\textbf{Hostile} & \textbf{Restrictive} & \textbf{Permissive} \\
\midrule
China (2021 ban) & India (30\% tax) & Switzerland (clear framework) \\
Algeria, Bangladesh & South Korea (strict KYC) & Singapore (licensing) \\
Nepal, Morocco & Brazil (no DeFi clarity) & UAE (crypto-friendly zones) \\
 & Russia (payment ban) & Portugal (tax-friendly) \\
 & Turkey (payment ban) & El Salvador (legal tender) \\
\bottomrule
\end{tabular}
\end{table}
\vspace{0.3cm}
\textbf{Trend}: Convergence toward licensing regimes with consumer protections
\end{frame}

\begin{frame}[t]{United States: Fragmented Jurisdiction}
\begin{center}
\includegraphics[width=0.55\textwidth]{charts/02_us_regulators/chart.pdf}
\end{center}
\bottomnote{Multiple agencies with overlapping and unclear jurisdiction}
\end{frame}

\begin{frame}{US Regulatory Agencies}
\begin{itemize}
\item \textbf{No Unified Framework}: Multiple agencies with overlapping jurisdiction
\item \textbf{Key Regulators}:
\begin{itemize}
\item \textbf{SEC}: Securities regulation (most tokens per Howey Test)
\item \textbf{CFTC}: Commodities (Bitcoin, ETH, derivatives)
\item \textbf{FinCEN}: AML/CTF enforcement
\item \textbf{OCC}: Banking charters for crypto custody
\item \textbf{IRS}: Tax treatment (crypto = property)
\item \textbf{State regulators}: Money transmitter licenses (NY BitLicense)
\end{itemize}
\item \textbf{Result}: Regulation by enforcement, legal uncertainty
\end{itemize}
\end{frame}

\begin{frame}{SEC vs CFTC: Jurisdiction Battle}
\begin{columns}[T]
\begin{column}{0.48\textwidth}
\textbf{SEC Position}
\begin{itemize}
\item Most tokens are \textbf{securities}
\item ICOs = unregistered offerings
\item Exchanges must register
\end{itemize}
\vspace{0.2cm}
\textbf{Howey Test}:
\begin{enumerate}
\item Investment of money
\item Common enterprise
\item Expectation of profits
\item From efforts of others
\end{enumerate}
\end{column}
\begin{column}{0.48\textwidth}
\textbf{CFTC Position}
\begin{itemize}
\item BTC and ETH are \textbf{commodities}
\item Jurisdiction over derivatives
\item Lighter regulatory approach
\end{itemize}
\vspace{0.2cm}
\textbf{Congressional Debate}:
\begin{itemize}
\item Multiple bills (FIT21, DCCPA)
\item Goal: Clarify jurisdiction
\item Status: Stalled until 2025
\end{itemize}
\end{column}
\end{columns}
\end{frame}

\begin{frame}[t]{Major Enforcement Actions}
\begin{center}
\includegraphics[width=0.55\textwidth]{charts/03_enforcement_fines/chart.pdf}
\end{center}
\bottomnote{Regulation by enforcement creates legal uncertainty for industry}
\end{frame}

\begin{frame}{Notable Enforcement Cases}
\begin{itemize}
\item \textbf{Ripple Labs (2020-2024)}: SEC sued for \$1.3B unregistered XRP sales
\begin{itemize}
\item Result: Institutional sales = securities, retail sales = not securities
\item Final penalty: \$125M (reduced from \$2B SEC request)
\end{itemize}
\item \textbf{Binance (2023)}: \$4.3B DOJ settlement, CZ resigned and served prison time
\item \textbf{Terraform Labs (2024)}: \$4.5B penalty for UST/LUNA fraud
\item \textbf{FTX (2022-2024)}: Criminal fraud charges, SBF convicted
\vspace{0.2cm}
\item \textbf{Pattern}: Enforcement first, rulemaking later
\end{itemize}
\end{frame}

\begin{frame}{European Union: MiCA Regulation}
\begin{itemize}
\item \textbf{MiCA (Markets in Crypto-Assets Regulation)}: Comprehensive EU framework
\item \textbf{Timeline}: Approved 2023, full implementation December 2024
\item \textbf{Scope}:
\begin{itemize}
\item Crypto-asset service providers (CASPs)
\item Stablecoin issuers (EMTs, ARTs)
\item NFTs excluded (unless fungible/fractionalized)
\end{itemize}
\item \textbf{Key Requirements}:
\begin{itemize}
\item Authorization and capital requirements for CASPs
\item Market abuse prohibitions
\item Consumer protection (disclosure, conflicts)
\item Stablecoin reserve requirements
\end{itemize}
\item \textbf{Impact}: Single framework across 27 EU countries
\end{itemize}
\end{frame}

\begin{frame}[t]{MiCA Implementation Timeline}
\begin{center}
\includegraphics[width=0.65\textwidth]{charts/05_mica_timeline/chart.pdf}
\end{center}
\bottomnote{First comprehensive crypto regulation covering entire trading bloc}
\end{frame}

\begin{frame}{MiCA: Stablecoin Provisions}
\begin{itemize}
\item \textbf{E-Money Tokens (EMTs)}: Pegged to single fiat (USDC, USDT)
\begin{itemize}
\item Issuers must be credit/e-money institutions
\item 1:1 reserve backing in segregated accounts
\item Redemption at par value anytime
\end{itemize}
\item \textbf{Asset-Referenced Tokens (ARTs)}: Basket or non-fiat pegged
\begin{itemize}
\item Stricter capital and governance requirements
\end{itemize}
\item \textbf{Significant Tokens}: Enhanced EBA oversight
\item \textbf{Algorithmic Stablecoins}: Effectively banned (post-Terra)
\vspace{0.2cm}
\item \textbf{Practical Impact}: Tether (USDT) compliance issues, Circle (USDC) compliant
\end{itemize}
\end{frame}

\begin{frame}[t]{2024 Milestone: Bitcoin ETF Approval}
\begin{center}
\includegraphics[width=0.55\textwidth]{charts/04_btc_etf_inflows/chart.pdf}
\end{center}
\bottomnote{SEC approved 11 spot Bitcoin ETFs on January 10, 2024}
\end{frame}

\begin{frame}{Bitcoin ETF: Regulatory Significance}
\begin{itemize}
\item \textbf{January 10, 2024}: SEC approves 11 spot Bitcoin ETFs
\begin{itemize}
\item First spot Bitcoin ETFs in US history (after decade of rejections)
\item Issuers: BlackRock (IBIT), Fidelity (FBTC), Grayscale (GBTC)
\end{itemize}
\item \textbf{Impact}:
\begin{itemize}
\item \$50B+ inflows in first year
\item Institutional access via traditional brokerage accounts
\item Bitcoin legitimized as investable asset class
\item BTC reached new ATH >\$100k (late 2024)
\end{itemize}
\item \textbf{July 2024}: SEC approves spot Ethereum ETFs
\item \textbf{Significance}: Major shift from SEC's hostile stance
\end{itemize}
\end{frame}

\begin{frame}{Asia-Pacific: Diverse Approaches}
\begin{itemize}
\item \textbf{Singapore (MAS)}:
\begin{itemize}
\item Payment Services Act licensing
\item Strict AML/CFT, retail protections
\end{itemize}
\item \textbf{Hong Kong (SFC)}:
\begin{itemize}
\item Mandatory licensing (2023)
\item Positioning as Asian crypto hub post-China ban
\end{itemize}
\item \textbf{Japan (FSA)}:
\begin{itemize}
\item Early framework (2017), strict custody standards
\end{itemize}
\item \textbf{China}: Complete ban (2021) - transactions, mining, exchanges
\begin{itemize}
\item Rationale: Capital controls, CBDC strategy
\item Result: Mining exodus to US, Kazakhstan, Russia
\end{itemize}
\end{itemize}
\end{frame}

\begin{frame}{Switzerland: Crypto Nation}
\begin{itemize}
\item \textbf{Crypto Valley (Zug)}: Global blockchain hub
\item \textbf{Legal Framework}:
\begin{itemize}
\item DLT Act (2021): Tailored regulation for digital assets
\item Token classification: Payment, Utility, Asset tokens
\item Securities law applies to asset tokens only
\end{itemize}
\item \textbf{FINMA Guidance}:
\begin{itemize}
\item Clear licensing categories
\item AML/CFT compliance for exchanges
\item No blanket prohibition on specific activities
\end{itemize}
\item \textbf{Advantages}: Regulatory clarity, innovation-friendly, strong rule of law
\end{itemize}
\end{frame}

\begin{frame}{AML/CFT: The Travel Rule}
\begin{itemize}
\item \textbf{FATF Recommendation 16}: Apply to Virtual Asset Service Providers
\item \textbf{Requirements}:
\begin{itemize}
\item Collect/transmit customer info for transactions >\$1,000
\item Originator and beneficiary details
\end{itemize}
\item \textbf{Implementation}:
\begin{itemize}
\item Centralized exchanges: Implemented (Coinbase, Kraken)
\item Cross-border: Technical solutions (TRP, TRUST, Notabene)
\item Self-hosted wallets: Controversial (EU proposed restrictions)
\end{itemize}
\item \textbf{Challenge}: DeFi has no intermediary to enforce rules
\item \textbf{Tension}: KYC requirements conflict with crypto privacy ethos
\end{itemize}
\end{frame}

\begin{frame}{DeFi: The Regulatory Challenge}
\begin{itemize}
\item \textbf{Problem}: Traditional regulation assumes intermediaries
\item \textbf{DeFi Reality}: Smart contracts, no central operator
\item \textbf{Key Questions}:
\begin{itemize}
\item Who is liable for smart contract exploits?
\item How to enforce AML without custodian?
\item Are DeFi protocols securities or commodities?
\end{itemize}
\item \textbf{Enforcement Actions}:
\begin{itemize}
\item Tornado Cash (2022): Treasury sanctioned mixer protocol
\item Uniswap Labs: SEC Wells Notice
\item Ooki DAO (2022): CFTC sued DAO as legal entity
\end{itemize}
\item \textbf{Debate}: Code is speech vs code is conduct
\end{itemize}
\end{frame}

\begin{frame}{2025 US Political Shift}
\begin{itemize}
\item \textbf{2024 Election Impact}:
\begin{itemize}
\item Crypto PACs spent \$100M+ on elections
\item Pro-crypto candidates elected to Congress
\end{itemize}
\item \textbf{New Administration (2025)}:
\begin{itemize}
\item Promise to make US ``crypto capital of the world''
\item SEC Chair Gensler resigned
\item Strategic Bitcoin Reserve proposal discussed
\end{itemize}
\item \textbf{Legislative Outlook}:
\begin{itemize}
\item FIT21 (Financial Innovation and Technology Act) revival
\item Stablecoin legislation priority
\item CFTC vs SEC jurisdiction clarity expected
\end{itemize}
\item \textbf{Shift}: From enforcement to rulemaking approach
\end{itemize}
\end{frame}

\begin{frame}{Regulatory Outlook: Future Directions}
\begin{itemize}
\item \textbf{Trend 1: Convergence}: Countries adopting similar frameworks (MiCA template)
\item \textbf{Trend 2: Stablecoin Focus}: Banking-like regulation for systemic stablecoins
\item \textbf{Trend 3: DeFi Reckoning}: Regulatory clarity (or crackdown) coming
\item \textbf{Trend 4: CBDC Competition}: Central banks competing with private stablecoins
\item \textbf{Trend 5: Global Coordination}: FATF, FSB harmonizing standards
\item \textbf{Trend 6: Licensing Regimes}: Most jurisdictions requiring VASP licenses
\vspace{0.2cm}
\item \textbf{Uncertainty Remains}: Technology evolves faster than regulation
\end{itemize}
\end{frame}

\begin{frame}{Summary}
\textbf{Key Takeaways:}
\begin{itemize}
\item \textbf{Global landscape}: Highly fragmented, evolving rapidly
\item \textbf{US}: Fragmented (SEC vs CFTC), regulation by enforcement
\item \textbf{EU}: Comprehensive MiCA framework, December 2024 full implementation
\item \textbf{Asia}: Diverse (Singapore/HK permissive, China ban, Japan conservative)
\item \textbf{Switzerland}: Clear framework, crypto-friendly (Crypto Valley)
\item \textbf{Bitcoin ETF (Jan 2024)}: Major legitimization milestone
\item \textbf{Enforcement}: \$13B+ in penalties (2022-2024)
\item \textbf{Trend}: Convergence toward licensing with consumer protections
\end{itemize}
\end{frame}

\begin{frame}{Questions for Reflection}
\begin{enumerate}
\item How does regulatory fragmentation affect crypto innovation in the US?
\item Why did the EU choose comprehensive legislation (MiCA) vs enforcement?
\item What are the trade-offs of China's complete ban approach?
\item Should DeFi protocols be regulated like traditional financial services?
\item How significant is the Bitcoin ETF approval for mainstream adoption?
\end{enumerate}
\end{frame}

\end{document}
