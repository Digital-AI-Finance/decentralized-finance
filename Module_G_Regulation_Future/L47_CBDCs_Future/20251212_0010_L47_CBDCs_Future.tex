\documentclass[8pt,aspectratio=169]{beamer}
\usetheme{Madrid}
\usepackage[utf8]{inputenc}
\usepackage{graphicx}
\usepackage{booktabs}
\usepackage{hyperref}
\usepackage{amsmath}

\newcommand{\bottomnote}[1]{\vfill\par\noindent\footnotesize\textit{#1}}

\title{L47: CBDCs and Future Trends}
\subtitle{Module G: Regulation \& Future}
\author{Blockchain \& Cryptocurrency Course}
\date{December 2025}

\begin{document}

\begin{frame}
\titlepage
\end{frame}

\begin{frame}{Learning Objectives}
\begin{itemize}
\item Understand Central Bank Digital Currencies (CBDCs)
\item Compare retail vs wholesale CBDC designs
\item Analyze privacy vs surveillance tradeoffs
\item Evaluate China e-CNY and Digital Euro progress
\item Identify key future trends in blockchain technology
\item Assess career opportunities in the blockchain space
\end{itemize}
\end{frame}

\begin{frame}{What is a CBDC?}
\begin{itemize}
\item \textbf{Definition}: Digital form of central bank money (fiat currency)
\item \textbf{Not Cryptocurrency}: Centrally issued and controlled
\item \textbf{Key Characteristics}:
\begin{itemize}
\item Legal tender status
\item Liability of central bank (not commercial bank)
\item Electronic/digital (not physical cash)
\item May use DLT (but not required)
\end{itemize}
\item \textbf{Motivation}: Respond to cash decline, private stablecoins, financial inclusion
\item \textbf{Status}: 130+ countries exploring CBDCs (90\% of global GDP)
\end{itemize}
\end{frame}

\begin{frame}[t]{Global CBDC Development Status}
\begin{center}
\includegraphics[width=0.55\textwidth]{charts/01_cbdc_global_status/chart.pdf}
\end{center}
\bottomnote{130+ countries exploring CBDCs, representing 90\% of global GDP}
\end{frame}

\begin{frame}{Retail vs Wholesale CBDCs}
\begin{table}
\centering
\small
\begin{tabular}{lll}
\toprule
\textbf{Aspect} & \textbf{Retail CBDC} & \textbf{Wholesale CBDC} \\
\midrule
Users & General public & Financial institutions \\
Use Case & Payments, store of value & Interbank settlement \\
Access & Widely accessible & Restricted to banks \\
Technology & Various (may use DLT) & Likely DLT (efficiency) \\
Privacy & Balance privacy vs AML & Less concern \\
Examples & e-CNY, Digital Euro & mBridge, Project Jasper \\
\bottomrule
\end{tabular}
\end{table}
\vspace{0.2cm}
\textbf{Focus}: Retail CBDCs have greater societal impact and complexity
\end{frame}

\begin{frame}{Privacy vs Surveillance Tradeoff}
\begin{columns}[T]
\begin{column}{0.48\textwidth}
\textbf{Privacy Concerns}
\begin{itemize}
\item Central bank sees all transactions
\item Government surveillance potential
\item Social credit system risks
\item No cash-like anonymity
\end{itemize}
\textbf{Privacy Technologies}
\begin{itemize}
\item Zero-knowledge proofs
\item Tiered privacy (small anonymous, large KYC)
\end{itemize}
\end{column}
\begin{column}{0.48\textwidth}
\textbf{AML/CFT Requirements}
\begin{itemize}
\item Full anonymity enables illicit finance
\item Regulatory pressure (FATF)
\item Tax enforcement needs
\end{itemize}
\textbf{Design Spectrum}
\begin{itemize}
\item \textbf{Full Surveillance}: China e-CNY
\item \textbf{Balanced}: Digital Euro
\item \textbf{Privacy-First}: Unlikely in practice
\end{itemize}
\end{column}
\end{columns}
\end{frame}

\begin{frame}[t]{China e-CNY Adoption}
\begin{center}
\includegraphics[width=0.55\textwidth]{charts/02_ecny_adoption/chart.pdf}
\end{center}
\bottomnote{Largest CBDC pilot globally, 260M+ wallets by late 2024}
\end{frame}

\begin{frame}{Case Study: China's e-CNY}
\begin{itemize}
\item \textbf{Status}: Largest CBDC pilot globally (2020-present)
\item \textbf{Architecture}: Two-tier (PBOC wholesale, banks retail)
\item \textbf{Technology}: Centralized with distributed database
\item \textbf{Features}:
\begin{itemize}
\item Dual offline payment (no internet required)
\item Programmability (smart contracts)
\item ``Controllable anonymity'' (PBOC sees all)
\end{itemize}
\item \textbf{2024 Stats}: 260M+ wallets, 7T+ yuan cumulative transactions
\item \textbf{Geopolitical Angle}: Challenge USD dominance
\item \textbf{Concerns}: Surveillance, limited daily usage vs Alipay/WeChat
\end{itemize}
\end{frame}

\begin{frame}{Case Study: Digital Euro}
\begin{itemize}
\item \textbf{Status}: Preparation phase (2024-2026)
\item \textbf{Motivation}: Preserve monetary sovereignty, counter stablecoins
\item \textbf{Design Principles}:
\begin{itemize}
\item Privacy-focused (stronger than e-CNY)
\item Offline capability (like cash)
\item Free for basic use
\item Intermediated model (banks distribute)
\end{itemize}
\item \textbf{Privacy Model}:
\begin{itemize}
\item Small transactions: Cash-like privacy
\item Large transactions: Full AML compliance
\end{itemize}
\item \textbf{Timeline}: Decision expected late 2025, rollout 2027-2028
\end{itemize}
\end{frame}

\begin{frame}[t]{Digital Currency Comparison}
\begin{center}
\includegraphics[width=0.55\textwidth]{charts/03_cbdc_comparison/chart.pdf}
\end{center}
\bottomnote{CBDCs may crowd out stablecoins but not cryptocurrencies (different use cases)}
\end{frame}

\begin{frame}{Cross-Border CBDC: mBridge}
\begin{itemize}
\item \textbf{Project mBridge}: Multi-CBDC platform for cross-border payments
\item \textbf{Participants}: China, Hong Kong, Thailand, UAE, Saudi Arabia
\item \textbf{Goal}: Replace SWIFT for cross-border settlements
\begin{itemize}
\item Instant settlement (vs 2-5 days)
\item Lower costs (no correspondent banking fees)
\item 24/7 operation
\end{itemize}
\item \textbf{Technology}: Permissioned blockchain
\item \textbf{Status}: MVP launched June 2024, live transactions completed
\item \textbf{Geopolitical Implications}: Bypass USD-dominated SWIFT
\item \textbf{BRICS Interest}: Alternative payment system for member nations
\end{itemize}
\end{frame}

\begin{frame}[t]{Future Trends: Current vs Projected}
\begin{center}
\includegraphics[width=0.55\textwidth]{charts/04_future_trends/chart.pdf}
\end{center}
\bottomnote{ZK proofs, liquid staking, and RWA tokenization showing strongest growth}
\end{frame}

\begin{frame}{Trend 1: Institutional Adoption Acceleration}
\begin{itemize}
\item \textbf{2024 Status}: Crypto assets mainstream in institutional portfolios
\item \textbf{Drivers}:
\begin{itemize}
\item Spot Bitcoin ETFs (approved US January 2024)
\item Ethereum ETFs (July 2024)
\item Regulatory clarity (MiCA, Swiss framework)
\item Custody solutions (Coinbase Prime, Fidelity Digital Assets)
\end{itemize}
\item \textbf{Institutional Products}:
\begin{itemize}
\item Tokenized securities (bonds, real estate, funds)
\item Crypto lending and prime brokerage
\item Derivatives (CME futures, options)
\end{itemize}
\item \textbf{Impact}: \$1T+ institutional capital in crypto by 2030
\end{itemize}
\end{frame}

\begin{frame}[t]{RWA Tokenization Market Growth}
\begin{center}
\includegraphics[width=0.55\textwidth]{charts/05_rwa_market/chart.pdf}
\end{center}
\bottomnote{Real world asset tokenization projected to reach \$16T by 2030 (BCG estimate)}
\end{frame}

\begin{frame}{Trend 2: RWA Tokenization}
\begin{itemize}
\item \textbf{RWA Tokenization}: Representing real assets on blockchain
\item \textbf{Asset Classes}:
\begin{itemize}
\item Real estate (fractional ownership)
\item Private equity and venture capital
\item Bonds (government, corporate)
\item Commodities (gold, carbon credits)
\end{itemize}
\item \textbf{Advantages}:
\begin{itemize}
\item Fractional ownership (lower barriers)
\item 24/7 trading (no market hours)
\item Programmable compliance
\end{itemize}
\item \textbf{Leaders}: Centrifuge, Ondo Finance, Securitize, tZERO
\end{itemize}
\end{frame}

\begin{frame}{Trend 3: AI + Blockchain Convergence}
\begin{itemize}
\item \textbf{AI for Blockchain}:
\begin{itemize}
\item Smart contract auditing
\item MEV optimization
\item DeFi risk modeling
\item On-chain analytics
\end{itemize}
\item \textbf{Blockchain for AI}:
\begin{itemize}
\item Decentralized AI training (Bittensor, Ocean Protocol)
\item Verifiable AI models (proof of training)
\item AI agent payments
\item Data marketplaces with access control
\end{itemize}
\item \textbf{Projects}: Fetch.ai, SingularityNET, Render Network
\end{itemize}
\end{frame}

\begin{frame}{Trend 4: Zero-Knowledge Proofs Everywhere}
\begin{itemize}
\item \textbf{ZK Technology Maturation}: From research to production
\item \textbf{Applications}:
\begin{enumerate}
\item \textbf{ZK-Rollups}: Scalability (StarkNet, zkSync, Polygon zkEVM)
\item \textbf{Privacy}: Private transactions (Aztec, Railgun)
\item \textbf{Identity}: Prove attributes without revealing data
\begin{itemize}
\item Age verification without birthdate
\item Credit score proofs without full history
\end{itemize}
\item \textbf{Compliance}: Prove regulatory compliance privately
\end{enumerate}
\item \textbf{Hardware}: ZK ASICs for faster proof generation
\item \textbf{Impact}: Privacy + scalability without tradeoffs
\end{itemize}
\end{frame}

\begin{frame}{Emerging Risks and Challenges}
\begin{enumerate}
\item \textbf{Quantum Computing Threat}:
\begin{itemize}
\item ECDSA signatures vulnerable (10-20 year timeline)
\item Mitigation: Post-quantum cryptography migration
\end{itemize}
\item \textbf{Regulatory Fragmentation}:
\begin{itemize}
\item Conflicting national regulations
\item Compliance complexity vs arbitrage
\end{itemize}
\item \textbf{Centralization Creep}:
\begin{itemize}
\item Validator concentration (Lido 30\%+ of staked ETH)
\item MEV centralization (Flashbots dominance)
\end{itemize}
\item \textbf{Systemic DeFi Risk}:
\begin{itemize}
\item Composability creates cascading failures
\end{itemize}
\end{enumerate}
\end{frame}

\begin{frame}{Career Paths in Blockchain (2025+)}
\begin{columns}[T]
\begin{column}{0.48\textwidth}
\textbf{Technical Roles}
\begin{itemize}
\item Smart contract developer
\item Blockchain protocol engineer
\item Security auditor
\item ZK cryptographer
\end{itemize}
\textbf{Finance/Economics}
\begin{itemize}
\item DeFi analyst
\item Tokenomics designer
\item Crypto trader/quant
\end{itemize}
\end{column}
\begin{column}{0.48\textwidth}
\textbf{Legal/Compliance}
\begin{itemize}
\item Crypto regulatory specialist
\item AML/CFT compliance officer
\item Web3 lawyer
\end{itemize}
\textbf{Business/Product}
\begin{itemize}
\item Web3 product manager
\item DAO operations
\item Community manager
\end{itemize}
\end{column}
\end{columns}
\vspace{0.2cm}
\textbf{Demand}: 50,000+ open blockchain jobs, growing 30\%+ annually
\end{frame}

\begin{frame}{Summary}
\textbf{Key Takeaways:}
\begin{itemize}
\item \textbf{CBDCs}: 130+ countries exploring, retail vs wholesale designs
\item \textbf{Privacy vs surveillance}: Key CBDC design tradeoff
\item \textbf{e-CNY (2024)}: 260M+ wallets, expanded nationwide
\item \textbf{Digital Euro}: Preparation phase, decision late 2025
\item \textbf{mBridge}: Cross-border CBDC platform, MVP launched 2024
\item \textbf{RWA tokenization}: Projected \$16T by 2030
\item \textbf{ZK proofs}: Privacy + scalability convergence
\item \textbf{AI + Blockchain}: Emerging synergies
\item \textbf{Career opportunities}: 50,000+ jobs across tech, finance, legal
\end{itemize}
\end{frame}

\begin{frame}{Questions for Reflection}
\begin{enumerate}
\item What are the key differences between retail and wholesale CBDCs?
\item How should CBDCs balance privacy and AML compliance?
\item Why might e-CNY adoption remain limited despite government push?
\item Which future trend (RWA, ZK, AI+Blockchain) has most potential?
\item How might mBridge affect the global financial system?
\end{enumerate}
\end{frame}

\end{document}
