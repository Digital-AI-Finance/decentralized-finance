\documentclass[8pt,aspectratio=169]{beamer}
\usetheme{Madrid}
\usepackage[utf8]{inputenc}
\usepackage{graphicx}
\usepackage{booktabs}
\usepackage{hyperref}
\usepackage{amsmath}

\title{L47: CBDCs and Future Trends}
\subtitle{Module G: Regulation \& Future}
\author{Blockchain \& Cryptocurrency Course}
\date{December 2025}

\begin{document}

\begin{frame}
\titlepage
\end{frame}

\section{Central Bank Digital Currencies (CBDCs)}

\begin{frame}{What is a CBDC?}
\begin{itemize}
\item \textbf{Definition}: Digital form of central bank money (fiat currency)
\item \textbf{Not Cryptocurrency}: Centrally issued and controlled by central bank
\item \textbf{Key Characteristics}:
\begin{itemize}
\item Legal tender status
\item Liability of central bank (not commercial bank)
\item Electronic/digital (not physical cash)
\item May use blockchain/DLT (but not required)
\end{itemize}
\item \textbf{Motivation}: Respond to decline in cash usage, private stablecoins, financial inclusion
\item \textbf{Status}: 130+ countries exploring CBDCs (90\% of global GDP)
\item \textbf{Operational}: Bahamas (Sand Dollar), Nigeria (eNaira), Jamaica (JAM-DEX)
\item \textbf{Pilots}: China (e-CNY), EU (Digital Euro), India (e-Rupee)
\end{itemize}
\end{frame}

\begin{frame}{Retail vs Wholesale CBDCs}
\begin{table}
\centering
\small
\begin{tabular}{lll}
\toprule
\textbf{Aspect} & \textbf{Retail CBDC} & \textbf{Wholesale CBDC} \\
\midrule
Users & General public & Financial institutions \\
Use Case & Payments, store of value & Interbank settlement \\
Access & Widely accessible & Restricted to banks \\
Amount & Small transactions & Large-value transfers \\
Technology & May use DLT & Likely DLT (efficiency) \\
Competition & Competes with bank deposits & Complements RTGS systems \\
Privacy & Balance privacy vs AML & Less privacy concern \\
Examples & e-CNY, Digital Euro & Project Ubin (Singapore) \\
 &  & Project Jasper (Canada) \\
\bottomrule
\end{tabular}
\end{table}
\vspace{0.3cm}
\textbf{Focus}: Retail CBDCs have greater societal impact and complexity
\end{frame}

\begin{frame}{Retail CBDC: Design Choices}
\begin{enumerate}
\item \textbf{Architecture}:
\begin{itemize}
\item \textbf{Direct}: Central bank manages all accounts (Sweden Riksbank model)
\item \textbf{Hybrid}: Central bank ledger, commercial banks interface with users (e-CNY model)
\item \textbf{Intermediated}: Commercial banks hold CBDC, central bank wholesale only
\end{itemize}
\item \textbf{Technology}:
\begin{itemize}
\item DLT/blockchain vs centralized database
\item Permissioned ledger (if DLT)
\item Offline capability (for unbanked areas)
\end{itemize}
\item \textbf{Access}:
\begin{itemize}
\item Account-based vs token-based
\item Identification requirements (KYC levels)
\item Limits on holdings (prevent bank disintermediation)
\end{itemize}
\item \textbf{Interest}: Pay interest on CBDC balances or not?
\end{enumerate}
\end{frame}

\begin{frame}{Privacy vs Surveillance Tradeoff}
\begin{columns}[T]
\begin{column}{0.48\textwidth}
\textbf{Privacy Concerns}
\begin{itemize}
\item Central bank sees all transactions
\item Potential for government surveillance
\item Social credit system risks (e.g., China)
\item Chilling effect on lawful activities
\item No cash-like anonymity
\end{itemize}
\vspace{0.3cm}
\textbf{Privacy-Enhancing Technologies}
\begin{itemize}
\item Zero-knowledge proofs (prove validity, hide details)
\item Tiered privacy (small transactions anonymous, large KYC)
\item Blind signatures (central bank can't link user to transaction)
\end{itemize}
\end{column}
\begin{column}{0.48\textwidth}
\textbf{AML/CFT Requirements}
\begin{itemize}
\item Full anonymity enables illicit finance
\item Regulatory pressure (FATF standards)
\item Tax enforcement needs
\item Counter-terrorism financing
\end{itemize}
\vspace{0.3cm}
\textbf{Design Spectrum}
\begin{itemize}
\item \textbf{Full Surveillance}: China e-CNY (central visibility)
\item \textbf{Balanced}: Digital Euro (privacy for small, KYC for large)
\item \textbf{Privacy-First}: Hypothetical (similar to cash, unlikely)
\end{itemize}
\end{column}
\end{columns}
\end{frame}

\begin{frame}{CBDC Risks and Challenges}
\begin{enumerate}
\item \textbf{Bank Disintermediation}:
\begin{itemize}
\item If CBDC pays interest, users move deposits from banks to CBDC
\item Banks lose funding $\rightarrow$ reduced lending $\rightarrow$ economic contraction
\item Mitigation: Caps on CBDC holdings, no/low interest
\end{itemize}
\item \textbf{Bank Runs}:
\begin{itemize}
\item Crisis triggers instant flight from bank deposits to CBDC (digital bank run)
\item Faster and larger than traditional bank runs
\item Mitigation: Holding limits, transfer limits
\end{itemize}
\item \textbf{Cybersecurity}:
\begin{itemize}
\item Central point of failure (entire monetary system)
\item DDoS, hacking, quantum computing threats
\end{itemize}
\item \textbf{Cross-Border Implications}:
\begin{itemize}
\item Currency substitution (dollarization/yuan-ization via CBDC)
\item Capital controls circumvention
\end{itemize}
\end{enumerate}
\end{frame}

\begin{frame}{Case Study: China's e-CNY (Digital Yuan)}
\begin{itemize}
\item \textbf{Status}: Largest CBDC pilot globally (2020-present)
\item \textbf{Architecture}: Two-tier (PBOC wholesale, banks retail)
\item \textbf{Technology}: Centralized with distributed database (not blockchain)
\item \textbf{Features}:
\begin{itemize}
\item Dual offline payment (no internet required)
\item Programmability (smart contracts)
\item Controllable anonymity (PBOC sees, commercial banks don't)
\end{itemize}
\item \textbf{Adoption Tactics}:
\begin{itemize}
\item Free e-CNY airdrops (lotteries)
\item Integration with AliPay, WeChat Pay
\item Salary payments in e-CNY (government workers)
\end{itemize}
\item \textbf{Geopolitical Angle}: Challenge USD dominance, cross-border CBDC settlement
\item \textbf{Concerns}: Surveillance (integration with social credit system)
\end{itemize}
\end{frame}

\begin{frame}{Case Study: Digital Euro}
\begin{itemize}
\item \textbf{Status}: Investigation phase (2021-2023), preparation phase (2024-2026)
\item \textbf{Motivation}: Preserve monetary sovereignty, counter private stablecoins (Libra/Diem scare)
\item \textbf{Design Principles}:
\begin{itemize}
\item Privacy-focused (stronger than e-CNY)
\item Offline capability (like cash)
\item Free for basic use (no transaction fees for users)
\item Intermediated model (banks distribute)
\end{itemize}
\item \textbf{Privacy Model}:
\begin{itemize}
\item ECB sees aggregate data only
\item Commercial banks handle KYC
\item Small transactions: Cash-like privacy
\item Large transactions: Full AML compliance
\end{itemize}
\item \textbf{Timeline}: Launch decision expected 2025, rollout 2027-2028
\item \textbf{Challenge}: Coordination across 20 Eurozone countries
\end{itemize}
\end{frame}

\begin{frame}{Cross-Border CBDC: mBridge Project}
\begin{itemize}
\item \textbf{Project mBridge}: Multi-CBDC platform for cross-border payments
\item \textbf{Participants}: China, Hong Kong, Thailand, UAE, Saudi Arabia (BIS Innovation Hub)
\item \textbf{Goal}: Replace SWIFT for cross-border settlements
\begin{itemize}
\item Instant settlement (vs 2-5 days)
\item Lower costs (no correspondent banking fees)
\item 24/7 operation
\end{itemize}
\item \textbf{Technology}: Permissioned blockchain (customized DLT)
\item \textbf{Mechanism}:
\begin{itemize}
\item Central banks issue CBDCs on shared ledger
\item Atomic swaps between currencies (no intermediary)
\item Smart contracts for compliance (AML checks)
\end{itemize}
\item \textbf{Geopolitical Implications}: Bypass USD-dominated SWIFT system
\item \textbf{Status}: Pilot phase, live transactions completed
\end{itemize}
\end{frame}

\begin{frame}{CBDCs vs Stablecoins vs Crypto}
\begin{table}
\centering
\small
\begin{tabular}{llll}
\toprule
\textbf{Property} & \textbf{CBDC} & \textbf{Stablecoin} & \textbf{Cryptocurrency} \\
\midrule
Issuer & Central bank & Private company & Decentralized protocol \\
Backing & Sovereign fiat & Reserves or algorithm & Consensus mechanism \\
Legal Tender & Yes & No & No \\
Volatility & None (= fiat) & Low (if properly backed) & High \\
Privacy & Variable (design choice) & Low (KYC required) & High (pseudonymous) \\
Programmability & Possible & Yes & Yes \\
Control & Centralized & Centralized & Decentralized \\
Use Case & Payments, settlement & DeFi, payments & Speculation, store of value \\
\bottomrule
\end{tabular}
\end{table}
\vspace{0.3cm}
\textbf{Competition}: CBDCs may crowd out stablecoins, not cryptocurrencies (different use cases)
\end{frame}

\section{Future Trends}

\begin{frame}{Trend 1: Institutional Adoption Acceleration}
\begin{itemize}
\item \textbf{2024 Status}: Crypto assets mainstream in institutional portfolios
\item \textbf{Drivers}:
\begin{itemize}
\item Spot Bitcoin ETFs (approved US 2024, Europe, Asia following)
\item Ethereum ETFs (post-Merge institutional interest)
\item Regulatory clarity (MiCA, Swiss framework)
\item Custody solutions (Coinbase Prime, Fidelity Digital Assets, BNY Mellon)
\end{itemize}
\item \textbf{Institutional Products}:
\begin{itemize}
\item Tokenized securities (bonds, real estate, funds)
\item Crypto lending and prime brokerage
\item Derivatives (CME Bitcoin futures, options)
\item Yield products (staking as a service)
\end{itemize}
\item \textbf{Impact}: \$1T+ institutional capital in crypto by 2030 (estimates)
\end{itemize}
\end{frame}

\begin{frame}{Trend 2: Tokenization of Real-World Assets (RWA)}
\begin{itemize}
\item \textbf{RWA Tokenization}: Representing real assets on blockchain
\item \textbf{Asset Classes}:
\begin{itemize}
\item Real estate (fractional ownership, REITs)
\item Private equity and venture capital
\item Bonds (government, corporate)
\item Commodities (gold, carbon credits)
\item Art and collectibles
\end{itemize}
\item \textbf{Advantages}:
\begin{itemize}
\item Fractional ownership (lower barriers to entry)
\item 24/7 trading (no market hours)
\item Programmable compliance (smart contracts enforce regulations)
\item Global liquidity pools
\end{itemize}
\item \textbf{Market Size}: \$10T+ tokenized assets by 2030 (BCG estimate)
\item \textbf{Leaders}: Centrifuge, Ondo Finance, Securitize, tZERO
\end{itemize}
\end{frame}

\begin{frame}{Trend 3: AI + Blockchain Convergence}
\begin{itemize}
\item \textbf{AI for Blockchain}:
\begin{itemize}
\item Smart contract auditing (automated vulnerability detection)
\item MEV optimization (machine learning for transaction ordering)
\item DeFi risk modeling (predictive analytics)
\item On-chain analytics (pattern detection, fraud identification)
\end{itemize}
\item \textbf{Blockchain for AI}:
\begin{itemize}
\item Decentralized AI training (Bittensor, Ocean Protocol)
\item Verifiable AI models (proof of training, model provenance)
\item AI agent payments (micropayments for AI services)
\item Data marketplaces (tokenized datasets with access control)
\end{itemize}
\item \textbf{Emerging Projects}:
\begin{itemize}
\item Fetch.ai: Autonomous economic agents
\item SingularityNET: Decentralized AI marketplace
\item Render Network: GPU compute for AI/rendering
\end{itemize}
\end{itemize}
\end{frame}

\begin{frame}{Trend 4: DePIN (Decentralized Physical Infrastructure Networks)}
\begin{itemize}
\item \textbf{DePIN}: Blockchain-incentivized physical infrastructure
\item \textbf{Categories}:
\begin{enumerate}
\item \textbf{Wireless Networks}:
\begin{itemize}
\item Helium: Decentralized LoRaWAN and 5G (IoT connectivity)
\item XNET: Decentralized mobile network
\end{itemize}
\item \textbf{Compute/Storage}:
\begin{itemize}
\item Filecoin: Decentralized storage
\item Akash: Decentralized cloud compute
\item Render Network: GPU rendering
\end{itemize}
\item \textbf{Energy}:
\begin{itemize}
\item Powerledger: P2P energy trading
\item LO3 Energy: Local energy markets
\end{itemize}
\item \textbf{Sensors/Mapping}:
\begin{itemize}
\item FOAM: Decentralized location services
\item Hivemapper: Crowdsourced mapping
\end{itemize}
\end{enumerate}
\item \textbf{Value Proposition}: Token incentives bootstrap network effects
\end{itemize}
\end{frame}

\begin{frame}{Trend 5: Account Abstraction (ERC-4337)}
\begin{itemize}
\item \textbf{Problem}: Current wallets (EOAs) have poor UX
\begin{itemize}
\item Seed phrases (lose it = lose funds)
\item Gas fees paid in native token (ETH)
\item No transaction batching or automation
\end{itemize}
\item \textbf{Account Abstraction}: Smart contract wallets as first-class citizens
\item \textbf{ERC-4337 Features}:
\begin{itemize}
\item \textbf{Social Recovery}: Multi-sig guardians can recover account
\item \textbf{Gas Abstraction}: Pay fees in any token (USDC, DAI) or sponsor transactions
\item \textbf{Batching}: Multiple operations in one transaction
\item \textbf{Automation}: Scheduled payments, limit orders
\item \textbf{Session Keys}: Temporary permissions for dApps (no approval fatigue)
\end{itemize}
\item \textbf{Impact}: UX comparable to Web2 (no seed phrases, no gas headaches)
\item \textbf{Adoption}: Deployed on Ethereum (2023), gaining traction
\end{itemize}
\end{frame}

\begin{frame}{Trend 6: Modular Blockchains}
\begin{itemize}
\item \textbf{Monolithic Blockchains}: Single chain handles execution, consensus, data availability
\begin{itemize}
\item Examples: Bitcoin, Ethereum L1
\item Limitation: Scalability bottleneck
\end{itemize}
\item \textbf{Modular Architecture}: Separate layers for different functions
\begin{enumerate}
\item \textbf{Execution Layer}: Process transactions (rollups)
\item \textbf{Consensus Layer}: Order and finalize blocks (Ethereum PoS)
\item \textbf{Data Availability Layer}: Store transaction data (Celestia, EigenDA)
\end{enumerate}
\item \textbf{Advantages}:
\begin{itemize}
\item Specialization (each layer optimized)
\item Scalability (parallel execution)
\item Flexibility (swap layers)
\end{itemize}
\item \textbf{Projects}: Celestia, Fuel, Eclipse, Sovereign SDK
\item \textbf{Vision}: Thousands of app-specific rollups sharing infrastructure
\end{itemize}
\end{frame}

\begin{frame}{Trend 7: Zero-Knowledge Proofs Everywhere}
\begin{itemize}
\item \textbf{ZK Technology Maturation}: From research to production
\item \textbf{Applications}:
\begin{enumerate}
\item \textbf{ZK-Rollups}: Scalability (StarkNet, zkSync, Polygon zkEVM)
\item \textbf{Privacy}: Private transactions (Zcash, Aztec, Railgun)
\item \textbf{Identity}: Prove attributes without revealing data
\begin{itemize}
\item Age verification (prove >18 without revealing birthdate)
\item Creditworthiness (prove credit score >X without revealing full history)
\end{itemize}
\item \textbf{Interoperability}: Cross-chain bridges with validity proofs
\item \textbf{Compliance}: Prove regulatory compliance without exposing data
\end{enumerate}
\item \textbf{Developer Tools}: Improved (Circom, Noir, o1js)
\item \textbf{Hardware Acceleration}: ZK ASICs for faster proof generation
\item \textbf{Impact}: Privacy + scalability without tradeoffs
\end{itemize}
\end{frame}

\begin{frame}{Trend 8: Regenerative Finance (ReFi)}
\begin{itemize}
\item \textbf{ReFi}: Using crypto/blockchain for environmental and social impact
\item \textbf{Use Cases}:
\begin{enumerate}
\item \textbf{Carbon Credits}:
\begin{itemize}
\item Tokenized carbon offsets (KlimaDAO, Toucan Protocol)
\item Transparent tracking, retirement on-chain
\item Liquid carbon markets
\end{itemize}
\item \textbf{Biodiversity Credits}:
\begin{itemize}
\item Tokenize conservation outcomes
\item Fund nature restoration via DeFi mechanisms
\end{itemize}
\item \textbf{Quadratic Funding}:
\begin{itemize}
\item Gitcoin Grants: Democratic funding for public goods
\item Matching pools amplify small donations
\end{itemize}
\item \textbf{Universal Basic Income (UBI)}:
\begin{itemize}
\item GoodDollar: Blockchain-based UBI distribution
\end{itemize}
\end{enumerate}
\item \textbf{Philosophy}: Align financial incentives with planetary regeneration
\end{itemize}
\end{frame}

\begin{frame}{Trend 9: Decentralized Science (DeSci)}
\begin{itemize}
\item \textbf{DeSci}: Blockchain for scientific research and collaboration
\item \textbf{Problems Addressed}:
\begin{itemize}
\item Publication paywalls (taxpayer-funded research locked behind fees)
\item Peer review inefficiency (slow, unpaid reviewers)
\item Funding bias (established labs favored over novel ideas)
\item Data sharing barriers (no incentives to share)
\end{itemize}
\item \textbf{Blockchain Solutions}:
\begin{itemize}
\item \textbf{IP-NFTs}: Intellectual property as tradeable NFTs (Molecule Protocol)
\item \textbf{DAOs for Research Funding}: Community-governed grants (VitaDAO for longevity research)
\item \textbf{Data Marketplaces}: Researchers compensated for data sharing (Ocean Protocol)
\item \textbf{Open Access Publishing}: Immutable, timestamped publications on-chain
\end{itemize}
\item \textbf{Projects}: Molecule, VitaDAO, ResearchHub, LabDAO
\end{itemize}
\end{frame}

\begin{frame}{Trend 10: Liquid Staking Derivatives (LSD)}
\begin{itemize}
\item \textbf{Problem}: Staked ETH (PoS) is illiquid (locked in validator)
\item \textbf{Solution}: Liquid staking tokens represent staked assets
\begin{itemize}
\item Lido: stETH (staked ETH)
\item Rocket Pool: rETH
\item Frax: frxETH
\end{itemize}
\item \textbf{Mechanism}:
\begin{enumerate}
\item User deposits ETH to protocol
\item Protocol stakes ETH in validators
\item User receives liquid staking token (stETH)
\item stETH tradeable, usable in DeFi (collateral, liquidity pools)
\item Earns staking yield while remaining liquid
\end{enumerate}
\item \textbf{Adoption}: \$40B+ in liquid staking (2024)
\item \textbf{Risk}: Centralization (Lido has 30\%+ of all staked ETH)
\item \textbf{Future}: Liquid staking for all PoS chains (Solana, Cosmos, Polkadot)
\end{itemize}
\end{frame}

\begin{frame}{Emerging Risks and Challenges}
\begin{enumerate}
\item \textbf{Quantum Computing Threat}:
\begin{itemize}
\item ECDSA signatures vulnerable to Shor's algorithm
\item Timeline: 10-20 years to quantum computers breaking crypto
\item Mitigation: Post-quantum cryptography research, migration plans
\end{itemize}
\item \textbf{Regulatory Fragmentation}:
\begin{itemize}
\item Conflicting national regulations (compliance complexity)
\item Stifling innovation vs jurisdictional arbitrage
\end{itemize}
\item \textbf{Centralization Creep}:
\begin{itemize}
\item Validator concentration (Lido, large staking pools)
\item MEV centralization (Flashbots dominance)
\item Infrastructure providers (Infura, Alchemy)
\end{itemize}
\item \textbf{Systemic DeFi Risk}:
\begin{itemize}
\item Composability creates cascading failures
\item Lack of circuit breakers in protocols
\end{itemize}
\end{enumerate}
\end{frame}

\begin{frame}{Career Paths in Blockchain (2025 and Beyond)}
\begin{columns}[T]
\begin{column}{0.48\textwidth}
\textbf{Technical Roles}
\begin{itemize}
\item Smart contract developer (Solidity, Rust)
\item Blockchain protocol engineer
\item Security auditor
\item ZK cryptographer
\item DevOps (node operations, infrastructure)
\end{itemize}
\vspace{0.3cm}
\textbf{Finance/Economics}
\begin{itemize}
\item DeFi analyst
\item Tokenomics designer
\item Crypto trader/quant
\item Institutional crypto advisor
\item DAO treasury manager
\end{itemize}
\end{column}
\begin{column}{0.48\textwidth}
\textbf{Legal/Compliance}
\begin{itemize}
\item Crypto regulatory specialist
\item AML/CFT compliance officer
\item Web3 lawyer
\item Policy analyst
\end{itemize}
\vspace{0.3cm}
\textbf{Business/Product}
\begin{itemize}
\item Web3 product manager
\item DAO operations
\item Community manager
\item Business development (partnerships)
\item Crypto marketing/growth
\end{itemize}
\end{column}
\end{columns}
\vspace{0.3cm}
\textbf{Demand}: 50,000+ open blockchain jobs (2024), growing 30\%+ annually
\end{frame}

\begin{frame}{Resources for Continued Learning}
\begin{itemize}
\item \textbf{Developer Resources}:
\begin{itemize}
\item Ethereum.org, Solidity docs, OpenZeppelin
\item CryptoZombies (Solidity tutorial)
\item Foundry, Hardhat (development frameworks)
\end{itemize}
\item \textbf{Research and News}:
\begin{itemize}
\item Vitalik Buterin's blog, Ethereum Research Forum
\item a16z Crypto Research, Messari, The Block
\item Bankless podcast, Unchained podcast
\end{itemize}
\item \textbf{Online Courses}:
\begin{itemize}
\item Coursera: Blockchain Specialization (University at Buffalo)
\item Udemy: Ethereum and Solidity courses
\item Alchemy University (free, developer-focused)
\end{itemize}
\item \textbf{Communities}: Twitter Crypto, Discord servers, local blockchain meetups
\end{itemize}
\end{frame}

\section{2024-2025 CBDC Developments}

\begin{frame}{Digital Euro: Preparation Phase (2024-2025)}
\begin{itemize}
\item \textbf{October 2023}: ECB launched 2-year preparation phase
\item \textbf{Key Developments 2024}:
\begin{itemize}
\item Finalized technical design specifications
\item Selected technology partners (5 vendors shortlisted)
\item Privacy framework: Offline payments with cash-like anonymity
\item Holding limits: Likely 3,000-5,000 EUR per person
\end{itemize}
\item \textbf{Legislative Progress}:
\begin{itemize}
\item European Commission proposed Digital Euro Act (June 2023)
\item Parliament review ongoing throughout 2024
\item Legal tender status a key debate point
\end{itemize}
\item \textbf{Timeline Update}:
\begin{itemize}
\item Decision on issuance: Late 2025 (ECB Governing Council)
\item If approved: Rollout 2027-2028
\item Will coexist with physical Euro (not replace cash)
\end{itemize}
\end{itemize}
\end{frame}

\begin{frame}{United States: FedNow and CBDC Debate}
\begin{itemize}
\item \textbf{FedNow (Launched July 2023)}:
\begin{itemize}
\item Instant payment system (not a CBDC)
\item 24/7/365 real-time settlement
\item 900+ banks enrolled by late 2024
\item Reduces need for retail CBDC (same-day payments already possible)
\end{itemize}
\item \textbf{US Digital Dollar Debate}:
\begin{itemize}
\item Fed research continues but no commitment to issuance
\item Republican opposition: Privacy concerns, ``surveillance currency''
\item 2024 Election: Trump campaign opposed to CBDC
\end{itemize}
\item \textbf{2025 Policy Shift}:
\begin{itemize}
\item Trump administration: Executive order against CBDC development
\item Focus shifts to stablecoin regulation instead
\item US likely last among G7 to launch retail CBDC (if ever)
\end{itemize}
\end{itemize}
\end{frame}

\begin{frame}{China e-CNY: Expansion Update (2024)}
\begin{itemize}
\item \textbf{Adoption Statistics (2024)}:
\begin{itemize}
\item 260+ million individual wallets
\item 7+ trillion yuan in cumulative transactions
\item Expanded to all provinces (from pilot cities)
\end{itemize}
\item \textbf{Use Case Expansion}:
\begin{itemize}
\item Government salary payments in e-CNY
\item Social welfare distribution
\item Hong Kong cross-border pilots
\item Belt and Road Initiative settlements
\end{itemize}
\item \textbf{Challenges}:
\begin{itemize}
\item Low daily usage (Alipay/WeChat dominance)
\item Privacy concerns persist internationally
\item Limited international adoption outside China sphere
\end{itemize}
\item \textbf{Geopolitical Impact}: Alternative to USD for China-aligned trade
\end{itemize}
\end{frame}

\begin{frame}{Global CBDC Landscape (Late 2024)}
\begin{itemize}
\item \textbf{Launched (Live)}:
\begin{itemize}
\item Bahamas (Sand Dollar), Jamaica (JAM-DEX), Nigeria (eNaira)
\item Eastern Caribbean (DCash)
\end{itemize}
\item \textbf{Advanced Pilots}:
\begin{itemize}
\item China (e-CNY), India (e-Rupee), Brazil (Drex)
\item Sweden (e-Krona testing paused)
\end{itemize}
\item \textbf{Research/Development}:
\begin{itemize}
\item EU (Digital Euro), UK (Digital Pound), Japan, South Korea
\item Russia (Digital Ruble - sanctions-driven)
\end{itemize}
\item \textbf{Skeptical/Opposed}:
\begin{itemize}
\item United States (2025 policy shift)
\item Switzerland (SNB prefers wholesale only)
\end{itemize}
\item \textbf{Trend}: Cross-border interoperability becoming key focus
\end{itemize}
\end{frame}

\begin{frame}{mBridge and BRICS Payment Systems (2024)}
\begin{itemize}
\item \textbf{mBridge Progress}:
\begin{itemize}
\item Minimum Viable Product (MVP) launched June 2024
\item China, Hong Kong, Thailand, UAE, Saudi Arabia
\item 20+ central banks as observers
\item Real cross-border transactions completed
\end{itemize}
\item \textbf{BRICS Payment Initiative}:
\begin{itemize}
\item Proposed at 2024 BRICS Summit (Kazan)
\item Goal: Alternative to SWIFT for member nations
\item Potential integration with mBridge
\end{itemize}
\item \textbf{Implications}:
\begin{itemize}
\item Reduced USD dependence for participating nations
\item Sanctions resistance (Russia, Iran interest)
\item Fragmentation of global payment infrastructure
\end{itemize}
\item \textbf{Challenges}: Political alignment required, technical interoperability
\end{itemize}
\end{frame}

\begin{frame}{Summary}
\begin{itemize}
\item \textbf{CBDCs}: 130+ countries exploring, retail vs wholesale designs
\item \textbf{Privacy vs surveillance}: Key CBDC design tradeoff
\item \textbf{e-CNY 2024}: 260M+ wallets, expanded nationwide, geopolitical tool
\item \textbf{Digital Euro}: Preparation phase, decision expected late 2025
\item \textbf{US Stance}: FedNow live, CBDC development paused (2025 policy)
\item \textbf{mBridge}: MVP launched 2024, BRICS alternative to SWIFT emerging
\item \textbf{Future trends}: Institutional adoption, RWA tokenization, AI+crypto, DePIN
\item \textbf{ZK proofs}: Privacy + scalability convergence
\item \textbf{Modular blockchains}: Separation of execution, consensus, data availability
\item \textbf{Career opportunities}: 50,000+ jobs, diverse roles across tech, finance, legal
\end{itemize}
\end{frame}

\end{document}
