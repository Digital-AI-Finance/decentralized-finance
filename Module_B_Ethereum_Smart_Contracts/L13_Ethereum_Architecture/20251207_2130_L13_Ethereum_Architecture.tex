\documentclass[8pt,aspectratio=169]{beamer}
\usetheme{Madrid}
\usepackage[utf8]{inputenc}
\usepackage{graphicx}
\usepackage{booktabs}
\usepackage{hyperref}
\usepackage{listings}
\lstset{basicstyle=\tiny\ttfamily,breaklines=true}

\title{L13: Ethereum Architecture}
\subtitle{Module B: Ethereum \& Smart Contracts}
\author{Blockchain \& Cryptocurrency Course}
\date{December 2025}

\begin{document}

\begin{frame}
\titlepage
\end{frame}

\begin{frame}{Learning Objectives}
By the end of this lesson, you will be able to:
\begin{itemize}
\item Explain the Ethereum Virtual Machine (EVM) architecture
\item Distinguish between Externally Owned Accounts (EOA) and Contract Accounts
\item Describe Ethereum as a state machine and understand the world state concept
\item Explain the Merkle Patricia Trie data structure
\item Compare Ethereum's account model to Bitcoin's UTXO model
\end{itemize}
\end{frame}

\begin{frame}{What is Ethereum?}
\begin{columns}[T]
\begin{column}{0.48\textwidth}
\textbf{Key Characteristics:}
\begin{itemize}
\item Decentralized computation platform
\item Turing-complete blockchain
\item Smart contract support
\item Native cryptocurrency: Ether (ETH)
\item Launched July 30, 2015
\end{itemize}
\end{column}
\begin{column}{0.48\textwidth}
\textbf{Beyond Bitcoin:}
\begin{itemize}
\item Bitcoin: Digital gold, value transfer
\item Ethereum: World computer, programmable logic
\item Enables decentralized applications (dApps)
\item Supports complex financial instruments
\item Foundation for DeFi, NFTs, DAOs
\end{itemize}
\end{column}
\end{columns}
\end{frame}

\begin{frame}{Ethereum Virtual Machine (EVM)}
\textbf{The EVM is a quasi-Turing complete state machine:}
\begin{itemize}
\item Stack-based virtual machine (256-bit word size)
\item Executes bytecode compiled from high-level languages (Solidity, Vyper)
\item Deterministic execution: same input always produces same output
\item Gas metering prevents infinite loops (making it ``quasi''-Turing complete)
\item Isolated environment: no access to network, filesystem, or other processes
\end{itemize}

\vspace{0.3cm}
\textbf{EVM Operations:}
\begin{itemize}
\item Arithmetic: ADD, MUL, SUB, DIV, MOD, EXP
\item Comparison: LT, GT, EQ
\item Bitwise: AND, OR, XOR, NOT
\item Storage: SLOAD, SSTORE
\item Context: CALLER, CALLVALUE, ADDRESS
\end{itemize}
\end{frame}

\begin{frame}{Account Types: EOA vs Contract}
\begin{columns}[T]
\begin{column}{0.48\textwidth}
\textbf{Externally Owned Account (EOA):}
\begin{itemize}
\item Controlled by private key
\item Has an address (derived from public key)
\item Can hold Ether balance
\item Can send transactions
\item No code storage
\item Creation is free
\end{itemize}

\vspace{0.2cm}
\textbf{EOA Address:}
\begin{itemize}
\item Last 20 bytes of Keccak-256 hash of public key
\item Example: 0x5aAeb6053F3E94C9b9A09f33669435E7Ef1BeAed
\end{itemize}
\end{column}
\begin{column}{0.48\textwidth}
\textbf{Contract Account:}
\begin{itemize}
\item Controlled by smart contract code
\item Has an address (deterministically derived)
\item Can hold Ether balance
\item Cannot initiate transactions
\item Stores code and state
\item Creation costs gas
\end{itemize}

\vspace{0.2cm}
\textbf{Contract Address:}
\begin{itemize}
\item Derived from creator address + nonce
\item Example: 0xd4e56740f876aef8c010b86a40d5f5675c82f28
\end{itemize}
\end{column}
\end{columns}
\end{frame}

\begin{frame}{Account State Components}
\textbf{Every account (EOA or Contract) has four fields:}

\vspace{0.3cm}
\begin{enumerate}
\item \textbf{Nonce:}
\begin{itemize}
\item EOA: Counter of transactions sent from this address
\item Contract: Counter of contracts created by this contract
\item Prevents replay attacks
\end{itemize}

\item \textbf{Balance:}
\begin{itemize}
\item Amount of Wei (10\textsuperscript{-18} ETH) owned by this address
\item 1 ETH = 1,000,000,000,000,000,000 Wei
\end{itemize}

\item \textbf{StorageRoot:}
\begin{itemize}
\item Hash of the root node of the account's storage trie
\item Empty for EOAs (0x56e81f171bcc55a6ff8345e692c0f86e5b48e01b996cadc001622fb5e363b421)
\end{itemize}

\item \textbf{CodeHash:}
\begin{itemize}
\item Hash of the EVM bytecode for this account
\item Empty for EOAs (hash of empty string)
\item Immutable once set for contracts
\end{itemize}
\end{enumerate}
\end{frame}

\begin{frame}{Ethereum as a State Machine}
\textbf{State Transition Function:}
\begin{itemize}
\item Ethereum transitions from state $S_t$ to state $S_{t+1}$ via valid transactions
\item $S_{t+1} = Y(S_t, T)$ where $T$ is a transaction
\item State includes all account balances, storage, and code
\item Deterministic: Given state $S_t$ and transaction $T$, $S_{t+1}$ is uniquely determined
\end{itemize}

\vspace{0.3cm}
\textbf{Transaction Execution:}
\begin{enumerate}
\item Validate transaction (signature, nonce, gas limit, balance)
\item Deduct gas cost from sender
\item Execute transaction (transfer ETH, execute code)
\item Update state (balances, storage, nonces)
\item Refund unused gas
\item Distribute transaction fees to miner/validator
\end{enumerate}
\end{frame}

\begin{frame}{World State}
\textbf{The World State is a mapping between addresses and account states:}

\vspace{0.2cm}
\begin{itemize}
\item Maps 160-bit addresses to account states
\item Stored as a Merkle Patricia Trie
\item Root hash included in every block header
\item Allows efficient verification of account state
\item Enables light clients to verify data without full state
\end{itemize}

\vspace{0.3cm}
\textbf{State Root:}
\begin{itemize}
\item 256-bit hash representing entire world state
\item Changes with every block
\item Uniquely identifies state at a specific block height
\item Example: 0xd7f8974fb5ac78d9ac099b9ad5018bedc2ce0a72dad1827a1709da30580f0544
\end{itemize}
\end{frame}

\begin{frame}{Merkle Patricia Trie (MPT)}
\textbf{Combines three data structures:}
\begin{enumerate}
\item \textbf{Merkle Tree:} Cryptographic verification via hashes
\item \textbf{Patricia Trie:} Efficient key-value storage with shared prefixes
\item \textbf{Radix Trie:} Optimized path compression
\end{enumerate}

\vspace{0.3cm}
\textbf{MPT Properties:}
\begin{itemize}
\item Deterministic: Same key-value pairs always produce same root hash
\item Efficient verification: Prove inclusion with $O(\log n)$ proof size
\item Efficient updates: Only modified paths need rehashing
\item Path compression: Reduces storage for sparse trees
\end{itemize}

\vspace{0.3cm}
\textbf{Node Types:}
\begin{itemize}
\item \textbf{Branch Node:} 16 children (hex digits 0-F) + optional value
\item \textbf{Extension Node:} Shared path prefix + pointer to next node
\item \textbf{Leaf Node:} Remaining key path + value
\end{itemize}
\end{frame}

\begin{frame}{MPT Example: World State}
\textbf{Simplified example with 3 accounts:}

\vspace{0.2cm}
\begin{itemize}
\item Address A: 0xa7f9...  Balance: 100 ETH, Nonce: 5
\item Address B: 0xa8e3...  Balance: 50 ETH, Nonce: 2
\item Address C: 0xb2c1...  Balance: 200 ETH, Nonce: 10
\end{itemize}

\vspace{0.3cm}
\textbf{Trie Structure:}
\begin{enumerate}
\item Root is a branch node with children at positions 0xa and 0xb
\item Path 0xa has extension node with shared prefix ``a''
\item Extension branches to two leaves: 0xa7f9... and 0xa8e3...
\item Path 0xb has leaf node at 0xb2c1...
\item Each leaf stores RLP-encoded account state (balance, nonce, storage root, code hash)
\end{enumerate}

\vspace{0.3cm}
\textbf{Root Hash:} Keccak-256 of root node uniquely identifies this state
\end{frame}

\begin{frame}{MPT in Ethereum: Four Tries}
\textbf{Every block references four MPT roots:}

\vspace{0.3cm}
\begin{enumerate}
\item \textbf{State Trie:}
\begin{itemize}
\item Maps address $\rightarrow$ account state
\item Global, updated with every block
\end{itemize}

\item \textbf{Storage Trie:}
\begin{itemize}
\item Maps storage key $\rightarrow$ value for each contract
\item One per contract account
\end{itemize}

\item \textbf{Transaction Trie:}
\begin{itemize}
\item Maps transaction index $\rightarrow$ transaction in current block
\item One per block
\end{itemize}

\item \textbf{Receipt Trie:}
\begin{itemize}
\item Maps transaction index $\rightarrow$ receipt (logs, gas used, status)
\item One per block
\end{itemize}
\end{enumerate}
\end{frame}

\begin{frame}{Ethereum vs Bitcoin: Account Model vs UTXO}
\begin{columns}[T]
\begin{column}{0.48\textwidth}
\textbf{Bitcoin UTXO Model:}
\begin{itemize}
\item Unspent Transaction Outputs
\item No accounts, only UTXOs
\item Transaction consumes inputs, creates outputs
\item Stateless verification
\item Parallel transaction validation
\item Simpler, more secure
\item Limited programmability
\end{itemize}

\vspace{0.2cm}
\textbf{Example:}
\begin{itemize}
\item Alice has UTXO of 5 BTC
\item Sends 2 BTC to Bob
\item Creates: 2 BTC to Bob, 3 BTC back to Alice
\item Original 5 BTC UTXO destroyed
\end{itemize}
\end{column}
\begin{column}{0.48\textwidth}
\textbf{Ethereum Account Model:}
\begin{itemize}
\item Persistent account balances
\item State stored globally
\item Transaction modifies balances
\item Stateful execution
\item Sequential nonce ordering
\item More complex
\item Full programmability
\end{itemize}

\vspace{0.2cm}
\textbf{Example:}
\begin{itemize}
\item Alice has account with 5 ETH
\item Sends 2 ETH to Bob
\item Alice balance: 5 - 2 = 3 ETH
\item Bob balance: +2 ETH
\item Accounts persist
\end{itemize}
\end{column}
\end{columns}
\end{frame}

\begin{frame}{Account Model Advantages}
\textbf{Why Ethereum chose accounts over UTXO:}

\vspace{0.3cm}
\begin{enumerate}
\item \textbf{Simplicity:}
\begin{itemize}
\item Intuitive balance model
\item Easier to reason about state
\item Simpler wallet implementation
\end{itemize}

\item \textbf{Space Efficiency:}
\begin{itemize}
\item No need to track unspent outputs
\item Smaller state size for complex contracts
\end{itemize}

\item \textbf{Fungibility:}
\begin{itemize}
\item All Ether is equivalent
\item No dust accumulation
\end{itemize}

\item \textbf{Smart Contract Design:}
\begin{itemize}
\item Natural fit for persistent contract storage
\item Easier to implement tokens and complex logic
\item Direct interaction with contract state
\end{itemize}
\end{enumerate}
\end{frame}

\begin{frame}{Account Model Challenges}
\textbf{Tradeoffs of the account model:}

\vspace{0.3cm}
\begin{enumerate}
\item \textbf{Replay Attack Prevention:}
\begin{itemize}
\item Requires nonce tracking
\item Sequential transaction ordering per account
\item Cannot parallelize transactions from same account
\end{itemize}

\item \textbf{State Growth:}
\begin{itemize}
\item All accounts stored indefinitely
\item State size grows continuously
\item Higher storage requirements for full nodes
\end{itemize}

\item \textbf{Privacy:}
\begin{itemize}
\item All account activity linked to single address
\item Harder to implement UTXO-style mixing
\item Requires explicit privacy techniques (e.g., Tornado Cash)
\end{itemize}
\end{enumerate}
\end{frame}

\begin{frame}{Block Structure}
\textbf{Ethereum block header contains:}
\begin{itemize}
\item \textbf{parentHash:} Hash of parent block
\item \textbf{ommersHash:} Hash of uncle blocks (pre-merge)
\item \textbf{beneficiary:} Address receiving block reward
\item \textbf{stateRoot:} Root hash of state trie
\item \textbf{transactionsRoot:} Root hash of transaction trie
\item \textbf{receiptsRoot:} Root hash of receipt trie
\item \textbf{logsBloom:} Bloom filter for logs (efficient event lookup)
\item \textbf{difficulty:} Proof-of-work difficulty (pre-merge, now 0)
\item \textbf{number:} Block height
\item \textbf{gasLimit:} Maximum gas allowed in this block
\item \textbf{gasUsed:} Total gas used by all transactions
\item \textbf{timestamp:} Unix timestamp of block creation
\item \textbf{extraData:} Arbitrary data (max 32 bytes)
\item \textbf{mixHash, nonce:} PoW validation (pre-merge)
\end{itemize}
\end{frame}

\begin{frame}{2024 Milestone: Dencun Upgrade (March 13, 2024)}
\textbf{Major Network Upgrade (Proto-Danksharding):}
\begin{itemize}
\item \textbf{EIP-4844}: Introduces ``blob'' transactions for Layer 2 data
\item \textbf{Problem Solved}: L2 rollups pay expensive calldata for data availability
\item \textbf{Solution}: New data type (blobs) with separate, cheaper gas market
\end{itemize}

\vspace{0.3cm}
\textbf{How Blobs Work:}
\begin{itemize}
\item Blobs: 128 KB data chunks, stored for ~18 days (not permanent)
\item Separate ``blob gas'' market (independent of execution gas)
\item L2s post transaction batches as blobs instead of calldata
\item Data availability guaranteed, but not permanently stored
\end{itemize}

\vspace{0.3cm}
\textbf{Impact:}
\begin{itemize}
\item L2 transaction fees reduced by 90\%+ (from \$0.50 to \$0.01)
\item Arbitrum, Optimism, Base, zkSync all adopted immediately
\item Paves way for full Danksharding (future upgrade)
\item Ethereum becomes true settlement/DA layer
\end{itemize}
\end{frame}

\begin{frame}{Key Takeaways}
\begin{enumerate}
\item \textbf{EVM:} Deterministic, stack-based virtual machine enabling Turing-complete smart contracts

\item \textbf{Two Account Types:} EOAs (user-controlled) and Contract Accounts (code-controlled)

\item \textbf{State Machine:} Ethereum transitions between states via transactions, with deterministic outcomes

\item \textbf{World State:} Mapping of all addresses to account states, stored as Merkle Patricia Trie

\item \textbf{MPT:} Efficient cryptographic data structure enabling state verification and light clients

\item \textbf{Account Model:} Simpler and more suitable for smart contracts than Bitcoin's UTXO model, but with tradeoffs in privacy and parallelization
\end{enumerate}
\end{frame}

\begin{frame}{Discussion Questions}
\begin{enumerate}
\item Why is the EVM only ``quasi''-Turing complete rather than fully Turing complete?

\item What security benefits does the nonce provide for both EOAs and contracts?

\item How does the Merkle Patricia Trie enable light clients to verify account balances without downloading the full state?

\item In what scenarios might Bitcoin's UTXO model be preferable to Ethereum's account model?

\item How does Ethereum's state growth challenge affect node operators, and what solutions are being explored?
\end{enumerate}
\end{frame}

\begin{frame}{Next Lesson Preview: L14 - Gas Mechanics}
\textbf{Coming up next:}
\begin{itemize}
\item Understanding gas as a computational unit
\item Gas price, gas limit, and transaction cost calculation
\item EIP-1559: Base fee and priority fee mechanism
\item Gas costs for different EVM operations
\item Optimization techniques for reducing gas consumption
\item Real-world examples of gas-inefficient code
\end{itemize}

\vspace{0.3cm}
\textbf{Preparation:}
\begin{itemize}
\item Review basic Ethereum transaction structure
\item Familiarize yourself with Wei/Gwei/ETH units
\item Optional: Browse Etherscan to see gas usage in real transactions
\end{itemize}
\end{frame}

\end{document}
