\documentclass[8pt,aspectratio=169]{beamer}
\usetheme{Madrid}
\usepackage[utf8]{inputenc}
\usepackage{graphicx}
\usepackage{booktabs}
\usepackage{hyperref}
\usepackage{amsmath}
\usepackage{listings}
\lstset{basicstyle=\tiny\ttfamily,breaklines=true}

\newcommand{\bottomnote}[1]{\vfill\footnotesize\textit{#1}}

\title{L13: Ethereum Architecture}
\subtitle{Module B: Ethereum \& Smart Contracts}
\author{Blockchain \& Cryptocurrency Course}
\date{December 2025}

\begin{document}

\begin{frame}
\titlepage
\end{frame}

\begin{frame}{Learning Objectives}
By the end of this lesson, you will be able to:
\begin{itemize}
\item Explain the Ethereum Virtual Machine (EVM) architecture
\item Distinguish between Externally Owned Accounts (EOA) and Contract Accounts
\item Describe Ethereum as a state machine and understand the world state concept
\item Explain the Merkle Patricia Trie data structure
\item Compare Ethereum's account model to Bitcoin's UTXO model
\end{itemize}
\end{frame}

\begin{frame}{What is Ethereum?}
\begin{columns}[T]
\begin{column}{0.48\textwidth}
\textbf{Key Characteristics:}
\begin{itemize}
\item Decentralized computation platform
\item Turing-complete blockchain
\item Smart contract support
\item Native cryptocurrency: Ether (ETH)
\item Launched July 30, 2015
\end{itemize}
\end{column}
\begin{column}{0.48\textwidth}
\textbf{Beyond Bitcoin:}
\begin{itemize}
\item Bitcoin: Digital gold, value transfer
\item Ethereum: World computer, programmable logic
\item Enables decentralized applications (dApps)
\item Foundation for DeFi, NFTs, DAOs
\end{itemize}
\end{column}
\end{columns}
\end{frame}

\begin{frame}[t]{Ethereum Virtual Machine (EVM)}
\begin{center}
\includegraphics[width=0.55\textwidth]{charts/01_evm_architecture/chart.pdf}
\end{center}
\bottomnote{The EVM is a quasi-Turing complete, stack-based virtual machine with gas metering.}
\end{frame}

\begin{frame}{EVM: Technical Details}
\textbf{The EVM is a quasi-Turing complete state machine:}
\begin{itemize}
\item Stack-based virtual machine (256-bit word size)
\item Executes bytecode compiled from Solidity, Vyper
\item Deterministic: same input always produces same output
\item Gas metering prevents infinite loops
\item Isolated: no network, filesystem, or process access
\end{itemize}

\vspace{0.3cm}
\textbf{EVM Components:}
\begin{itemize}
\item \textbf{Stack:} LIFO, max 1024 items, 256-bit each
\item \textbf{Memory:} Byte array, volatile (per call), linear cost
\item \textbf{Storage:} Key-value, persistent (on-chain), expensive
\item \textbf{Program Counter:} Current bytecode position
\item \textbf{Gas Counter:} Remaining computation budget
\end{itemize}
\end{frame}

\begin{frame}[t]{Account Types: EOA vs Contract}
\begin{center}
\includegraphics[width=0.55\textwidth]{charts/02_account_types/chart.pdf}
\end{center}
\bottomnote{EOAs are controlled by private keys; contracts are controlled by code.}
\end{frame}

\begin{frame}{Account State Components}
\textbf{Every account (EOA or Contract) has four fields:}

\vspace{0.3cm}
\begin{enumerate}
\item \textbf{Nonce:}
\begin{itemize}
\item EOA: Counter of transactions sent
\item Contract: Counter of contracts created
\item Prevents replay attacks
\end{itemize}

\item \textbf{Balance:}
\begin{itemize}
\item Amount of Wei (10\textsuperscript{-18} ETH) owned
\item 1 ETH = 1,000,000,000,000,000,000 Wei
\end{itemize}

\item \textbf{StorageRoot:}
\begin{itemize}
\item Hash of account's storage trie root
\item Empty for EOAs
\end{itemize}

\item \textbf{CodeHash:}
\begin{itemize}
\item Hash of EVM bytecode
\item Empty for EOAs, immutable for contracts
\end{itemize}
\end{enumerate}
\end{frame}

\begin{frame}[t]{Ethereum as a State Machine}
\begin{center}
\includegraphics[width=0.55\textwidth]{charts/03_state_machine/chart.pdf}
\end{center}
\bottomnote{State transitions are deterministic: given $S_t$ and $T$, $S_{t+1}$ is uniquely determined.}
\end{frame}

\begin{frame}{World State}
\textbf{The World State is a mapping between addresses and account states:}

\vspace{0.2cm}
\begin{itemize}
\item Maps 160-bit addresses to account states
\item Stored as a Merkle Patricia Trie
\item Root hash included in every block header
\item Allows efficient verification of account state
\item Enables light clients to verify data without full state
\end{itemize}

\vspace{0.3cm}
\textbf{State Root:}
\begin{itemize}
\item 256-bit hash representing entire world state
\item Changes with every block
\item Uniquely identifies state at a specific block height
\end{itemize}
\end{frame}

\begin{frame}[t]{Merkle Patricia Trie (MPT)}
\begin{center}
\includegraphics[width=0.55\textwidth]{charts/04_mpt_structure/chart.pdf}
\end{center}
\bottomnote{MPT combines Merkle verification, Patricia path compression, and radix optimization.}
\end{frame}

\begin{frame}{MPT Properties}
\textbf{Combines three data structures:}
\begin{enumerate}
\item \textbf{Merkle Tree:} Cryptographic verification via hashes
\item \textbf{Patricia Trie:} Efficient key-value storage with shared prefixes
\item \textbf{Radix Trie:} Optimized path compression
\end{enumerate}

\vspace{0.3cm}
\textbf{Key Properties:}
\begin{itemize}
\item Deterministic: Same key-value pairs $\rightarrow$ same root hash
\item Efficient verification: $O(\log n)$ proof size
\item Efficient updates: Only modified paths need rehashing
\end{itemize}

\vspace{0.3cm}
\textbf{Node Types:}
\begin{itemize}
\item \textbf{Branch:} 16 children (hex 0-F) + optional value
\item \textbf{Extension:} Shared path prefix + next node pointer
\item \textbf{Leaf:} Remaining key path + value
\end{itemize}
\end{frame}

\begin{frame}[t]{Four MPT Tries in Ethereum}
\begin{center}
\includegraphics[width=0.55\textwidth]{charts/05_four_tries/chart.pdf}
\end{center}
\bottomnote{Block header contains root hashes of all four tries for verification.}
\end{frame}

\begin{frame}[t]{Ethereum vs Bitcoin: Account Model vs UTXO}
\begin{center}
\includegraphics[width=0.55\textwidth]{charts/06_utxo_vs_account/chart.pdf}
\end{center}
\bottomnote{Account model enables smart contracts but requires nonce tracking.}
\end{frame}

\begin{frame}{Account Model: Trade-offs}
\textbf{Advantages:}
\begin{enumerate}
\item \textbf{Simplicity:} Intuitive balance model, easier wallets
\item \textbf{Space Efficiency:} No UTXO tracking overhead
\item \textbf{Fungibility:} All Ether equivalent, no dust
\item \textbf{Smart Contracts:} Natural fit for persistent storage
\end{enumerate}

\vspace{0.3cm}
\textbf{Challenges:}
\begin{enumerate}
\item \textbf{Replay Prevention:} Requires nonce tracking
\item \textbf{State Growth:} All accounts stored indefinitely
\item \textbf{Privacy:} All activity linked to single address
\item \textbf{Parallelization:} Sequential nonces limit parallel validation
\end{enumerate}
\end{frame}

\begin{frame}{Block Structure}
\textbf{Ethereum block header contains:}
\begin{itemize}
\item \textbf{parentHash:} Hash of parent block
\item \textbf{beneficiary:} Address receiving block reward
\item \textbf{stateRoot:} Root hash of state trie
\item \textbf{transactionsRoot:} Root hash of transaction trie
\item \textbf{receiptsRoot:} Root hash of receipt trie
\item \textbf{logsBloom:} Bloom filter for efficient log lookup
\item \textbf{number:} Block height
\item \textbf{gasLimit / gasUsed:} Gas constraints
\item \textbf{timestamp:} Unix timestamp
\item \textbf{extraData:} Arbitrary data (max 32 bytes)
\end{itemize}
\end{frame}

\begin{frame}[t]{2024 Milestone: Dencun Upgrade}
\begin{center}
\includegraphics[width=0.55\textwidth]{charts/07_dencun_impact/chart.pdf}
\end{center}
\bottomnote{EIP-4844 blob transactions reduce L2 data costs by 90\%+.}
\end{frame}

\begin{frame}{Dencun: Proto-Danksharding}
\textbf{Major Network Upgrade (March 13, 2024):}
\begin{itemize}
\item \textbf{EIP-4844}: Introduces ``blob'' transactions for Layer 2 data
\item \textbf{Problem}: L2 rollups pay expensive calldata for data availability
\item \textbf{Solution}: New data type (blobs) with separate, cheaper gas market
\end{itemize}

\vspace{0.3cm}
\textbf{How Blobs Work:}
\begin{itemize}
\item Blobs: 128 KB data chunks, stored for $\sim$18 days
\item Separate ``blob gas'' market (independent of execution gas)
\item L2s post transaction batches as blobs instead of calldata
\end{itemize}

\vspace{0.3cm}
\textbf{Impact:}
\begin{itemize}
\item L2 transaction fees reduced by 90\%+ (\$0.50 $\rightarrow$ \$0.01)
\item Arbitrum, Optimism, Base adopted immediately
\item Paves way for full Danksharding (future upgrade)
\end{itemize}
\end{frame}

\begin{frame}{Key Takeaways}
\begin{enumerate}
\item \textbf{EVM:} Deterministic, stack-based VM enabling Turing-complete smart contracts

\item \textbf{Two Account Types:} EOAs (user-controlled) and Contracts (code-controlled)

\item \textbf{State Machine:} Ethereum transitions between states via transactions

\item \textbf{World State:} Mapping of addresses to accounts, stored as MPT

\item \textbf{MPT:} Efficient cryptographic data structure for state verification

\item \textbf{Account Model:} Simpler than UTXO but with privacy/parallelization trade-offs
\end{enumerate}
\end{frame}

\begin{frame}{Discussion Questions}
\begin{enumerate}
\item Why is the EVM only ``quasi''-Turing complete?

\item What security benefits does the nonce provide?

\item How does MPT enable light client verification?

\item When might Bitcoin's UTXO model be preferable?

\item How does state growth affect node operators?
\end{enumerate}
\end{frame}

\begin{frame}{Next Lesson Preview: L14 - Gas Mechanics}
\textbf{Coming up next:}
\begin{itemize}
\item Understanding gas as a computational unit
\item Gas price, gas limit, and transaction cost calculation
\item EIP-1559: Base fee and priority fee mechanism
\item Gas costs for different EVM operations
\item Optimization techniques for reducing gas consumption
\end{itemize}

\vspace{0.3cm}
\textbf{Preparation:}
\begin{itemize}
\item Review basic Ethereum transaction structure
\item Familiarize yourself with Wei/Gwei/ETH units
\item Browse Etherscan to see gas usage in real transactions
\end{itemize}
\end{frame}

\end{document}
