\documentclass[8pt,aspectratio=169]{beamer}
\usetheme{Madrid}
\usepackage[utf8]{inputenc}
\usepackage{graphicx}
\usepackage{booktabs}
\usepackage{hyperref}
\usepackage{listings}
\lstset{basicstyle=\tiny\ttfamily,breaklines=true}

\title{L14: Gas Mechanics}
\subtitle{Module B: Ethereum \& Smart Contracts}
\author{Blockchain \& Cryptocurrency Course}
\date{December 2025}

\begin{document}

\begin{frame}
\titlepage
\end{frame}

\begin{frame}{Learning Objectives}
By the end of this lesson, you will be able to:
\begin{itemize}
\item Explain what gas is and why Ethereum uses it
\item Calculate transaction costs using gas price and gas limit
\item Understand EIP-1559's base fee and priority fee mechanism
\item Identify gas costs for different EVM operations
\item Apply optimization techniques to reduce gas consumption
\item Analyze real-world gas usage patterns
\end{itemize}
\end{frame}

\begin{frame}{What is Gas?}
\textbf{Gas is a unit of computational effort in Ethereum:}

\vspace{0.3cm}
\begin{itemize}
\item Measures the cost of executing operations on the EVM
\item Prevents infinite loops and spam attacks
\item Compensates validators for computation and storage
\item Decouples computational cost from ETH price volatility
\end{itemize}

\vspace{0.3cm}
\textbf{Key Concepts:}
\begin{itemize}
\item \textbf{Gas:} Abstract unit of work (e.g., 21,000 gas for simple transfer)
\item \textbf{Gas Price:} Amount of ETH per gas unit (measured in Gwei)
\item \textbf{Gas Limit:} Maximum gas user is willing to consume
\item \textbf{Transaction Fee:} Gas Used $\times$ Gas Price (in ETH)
\end{itemize}
\end{frame}

\begin{frame}{Why Gas Exists}
\textbf{Three Critical Functions:}

\vspace{0.3cm}
\begin{enumerate}
\item \textbf{Prevent Denial-of-Service Attacks:}
\begin{itemize}
\item Without gas, infinite loops could halt the network
\item Attackers would need to pay for computational resources
\item Example: \texttt{while(true) \{\}} would drain attacker's balance
\end{itemize}

\item \textbf{Incentivize Validators:}
\begin{itemize}
\item Validators earn transaction fees for including transactions
\item Higher gas price = higher priority in block inclusion
\item Market-based fee mechanism
\end{itemize}

\item \textbf{Resource Allocation:}
\begin{itemize}
\item Limited block gas limit (e.g., 30,000,000 gas per block)
\item Prioritizes transactions willing to pay more
\item Prevents block bloat
\end{itemize}
\end{enumerate}
\end{frame}

\begin{frame}{Ether Denominations}
\textbf{Ether units from smallest to largest:}

\vspace{0.3cm}
\begin{center}
\begin{tabular}{lll}
\toprule
\textbf{Unit} & \textbf{Wei Value} & \textbf{Typical Use} \\
\midrule
Wei & 1 & Smallest unit (like satoshi) \\
Kwei (Babbage) & 10\textsuperscript{3} & - \\
Mwei (Lovelace) & 10\textsuperscript{6} & - \\
Gwei (Shannon) & 10\textsuperscript{9} & Gas prices \\
Microether (Szabo) & 10\textsuperscript{12} & - \\
Milliether (Finney) & 10\textsuperscript{15} & - \\
Ether & 10\textsuperscript{18} & Main unit \\
\bottomrule
\end{tabular}
\end{center}

\vspace{0.3cm}
\textbf{Most Common:}
\begin{itemize}
\item \textbf{Gwei (Gigawei):} Standard unit for gas prices (1 Gwei = 10\textsuperscript{9} Wei)
\item \textbf{Ether:} User-facing unit (1 ETH = 10\textsuperscript{18} Wei)
\end{itemize}
\end{frame}

\begin{frame}{Gas Calculation: Pre-EIP-1559}
\textbf{Legacy Transaction Fee Model (before August 2021):}

\vspace{0.3cm}
\textbf{Formula:}
\[
\text{Transaction Fee} = \text{Gas Used} \times \text{Gas Price}
\]

\vspace{0.3cm}
\textbf{Example:}
\begin{itemize}
\item Gas Used: 21,000 (simple ETH transfer)
\item Gas Price: 50 Gwei (user-specified)
\item Transaction Fee: $21{,}000 \times 50 = 1{,}050{,}000$ Gwei = 0.00105 ETH
\end{itemize}

\vspace{0.3cm}
\textbf{Challenges:}
\begin{itemize}
\item Users had to manually estimate gas price
\item Overpaying was common to ensure inclusion
\item No refund if gas price was too high
\item Fee volatility during network congestion
\end{itemize}
\end{frame}

\begin{frame}{EIP-1559: London Hard Fork (August 2021)}
\textbf{Major overhaul of gas fee mechanism:}

\vspace{0.3cm}
\textbf{Key Changes:}
\begin{enumerate}
\item \textbf{Base Fee:}
\begin{itemize}
\item Algorithmically determined per block
\item Burned (removed from circulation)
\item Adjusts based on network congestion (target 50\% full blocks)
\end{itemize}

\item \textbf{Priority Fee (Tip):}
\begin{itemize}
\item User-specified tip to validator
\item Incentivizes block inclusion
\item Goes directly to validator
\end{itemize}

\item \textbf{Max Fee:}
\begin{itemize}
\item Maximum gas price user is willing to pay
\item Refund if actual cost is lower
\end{itemize}
\end{enumerate}
\end{frame}

\begin{frame}{EIP-1559 Fee Calculation}
\textbf{New Formula:}
\[
\text{Transaction Fee} = \text{Gas Used} \times (\text{Base Fee} + \text{Priority Fee})
\]

\textbf{With cap:}
\[
\text{Effective Gas Price} = \min(\text{Base Fee} + \text{Priority Fee}, \text{Max Fee})
\]

\vspace{0.3cm}
\textbf{Example:}
\begin{itemize}
\item Gas Used: 21,000
\item Base Fee: 30 Gwei (set by protocol)
\item Priority Fee: 2 Gwei (user tip)
\item Max Fee: 50 Gwei (user maximum)
\item Effective Gas Price: $\min(30 + 2, 50) = 32$ Gwei
\item Transaction Fee: $21{,}000 \times 32 = 672{,}000$ Gwei = 0.000672 ETH
\item Burned: $21{,}000 \times 30 = 630{,}000$ Gwei
\item To Validator: $21{,}000 \times 2 = 42{,}000$ Gwei
\end{itemize}
\end{frame}

\begin{frame}{Base Fee Adjustment Mechanism}
\textbf{Dynamic base fee targets 50\% full blocks:}

\vspace{0.3cm}
\textbf{Algorithm:}
\begin{itemize}
\item Target gas per block: 15,000,000 (50\% of 30M limit)
\item If block is more than 50\% full: Base fee increases by max 12.5\%
\item If block is less than 50\% full: Base fee decreases by max 12.5\%
\item Formula: $\text{BaseFee}_{new} = \text{BaseFee}_{old} \times \frac{\text{GasUsed} - \text{GasTarget}}{\text{GasTarget}} \times \frac{1}{8}$
\end{itemize}

\vspace{0.3cm}
\textbf{Example:}
\begin{itemize}
\item Current base fee: 100 Gwei
\item Block uses 20M gas (66\% full, above target)
\item Increase factor: $(20{,}000{,}000 - 15{,}000{,}000) / 15{,}000{,}000 / 8 = 0.0417$
\item New base fee: $100 \times (1 + 0.0417) = 104.17$ Gwei
\end{itemize}
\end{frame}

\begin{frame}{Gas Limit vs Gas Used}
\textbf{Understanding the difference:}

\vspace{0.3cm}
\begin{columns}[T]
\begin{column}{0.48\textwidth}
\textbf{Gas Limit:}
\begin{itemize}
\item Maximum gas transaction may consume
\item Set by user before sending transaction
\item Acts as safety cap
\item If exceeded, transaction reverts
\item Unused gas is refunded
\end{itemize}

\vspace{0.2cm}
\textbf{Example:}
\begin{itemize}
\item User sets gas limit: 100,000
\item Transaction uses: 65,000
\item Refund: 35,000 gas worth of ETH
\end{itemize}
\end{column}
\begin{column}{0.48\textwidth}
\textbf{Gas Used:}
\begin{itemize}
\item Actual gas consumed by transaction
\item Determined by operations executed
\item Cannot exceed gas limit
\item Used for fee calculation
\item Visible on Etherscan
\end{itemize}

\vspace{0.2cm}
\textbf{Common Values:}
\begin{itemize}
\item Simple transfer: 21,000
\item ERC-20 transfer: 45,000-65,000
\item Complex contract: 100,000-500,000+
\end{itemize}
\end{column}
\end{columns}
\end{frame}

\begin{frame}{Gas Costs by Operation}
\textbf{Every EVM opcode has a fixed gas cost:}

\vspace{0.3cm}
\begin{center}
\begin{tabular}{llr}
\toprule
\textbf{Operation} & \textbf{Description} & \textbf{Gas Cost} \\
\midrule
ADD, SUB, MUL & Arithmetic operations & 3 \\
DIV, MOD & Division/modulo & 5 \\
EXP & Exponentiation & 10 + 50/byte \\
SHA3 (Keccak-256) & Hash function & 30 + 6/word \\
SLOAD & Load from storage & 800 (warm) / 2100 (cold) \\
SSTORE & Write to storage & 20,000 (new) / 5,000 (update) \\
CALL & External contract call & 700 + value transfer costs \\
CREATE & Deploy contract & 32,000 + code size \\
LOG & Emit event & 375 + 375/topic + 8/byte \\
\bottomrule
\end{tabular}
\end{center}

\vspace{0.3cm}
\textbf{Most Expensive:} Storage operations (SSTORE, SLOAD)

\textbf{Cheapest:} Arithmetic and stack operations
\end{frame}

\begin{frame}{Storage: The Gas Guzzler}
\textbf{Why storage is expensive:}

\vspace{0.3cm}
\begin{itemize}
\item Persists data across all nodes forever
\item Requires disk I/O (slower than RAM)
\item State bloat affects all future nodes
\end{itemize}

\vspace{0.3cm}
\textbf{Storage Gas Costs (EIP-2929, EIP-2200):}
\begin{itemize}
\item \textbf{SSTORE (set to non-zero from zero):} 20,000 gas
\item \textbf{SSTORE (update non-zero):} 5,000 gas
\item \textbf{SSTORE (set to zero):} 5,000 gas + 15,000 refund
\item \textbf{SLOAD (cold access):} 2,100 gas (first access in transaction)
\item \textbf{SLOAD (warm access):} 100 gas (subsequent accesses)
\end{itemize}

\vspace{0.3cm}
\textbf{Example:}
\begin{itemize}
\item Storing one 256-bit word (32 bytes): 20,000 gas
\item At 30 Gwei base fee + 2 Gwei tip: 0.00064 ETH
\item At \$2000/ETH: \$1.28 to store 32 bytes!
\end{itemize}
\end{frame}

\begin{frame}[fragile]{Gas Optimization: Storage Patterns}
\textbf{Inefficient: Multiple SSTOREs}
\begin{lstlisting}
contract Inefficient {
    uint256 public value1;
    uint256 public value2;
    uint256 public value3;

    function updateAll(uint256 v1, uint256 v2, uint256 v3) public {
        value1 = v1;  // 20,000 gas (or 5,000 if updating)
        value2 = v2;  // 20,000 gas
        value3 = v3;  // 20,000 gas
    }
    // Total: 60,000 gas for 3 writes
}
\end{lstlisting}

\vspace{0.3cm}
\textbf{Efficient: Packed Storage}
\begin{lstlisting}
contract Efficient {
    uint256 public packedValues;  // Pack 3 uint85 values in one slot

    function updateAll(uint85 v1, uint85 v2, uint85 v3) public {
        packedValues = uint256(v1) | (uint256(v2) << 85) | (uint256(v3) << 170);
    }
    // Total: 20,000 gas for single write (3x cheaper!)
}
\end{lstlisting}
\end{frame}

\begin{frame}[fragile]{Gas Optimization: Memory vs Storage}
\textbf{Use memory for temporary data:}

\vspace{0.3cm}
\textbf{Inefficient: Storage for Temporary Array}
\begin{lstlisting}
contract Inefficient {
    uint256[] public tempArray;  // Storage

    function processData(uint256[] calldata input) public {
        delete tempArray;  // Gas refund, but still expensive
        for (uint i = 0; i < input.length; i++) {
            tempArray.push(input[i] * 2);  // SSTORE per iteration
        }
        // ... use tempArray ...
    }
}
\end{lstlisting}

\vspace{0.3cm}
\textbf{Efficient: Memory Array}
\begin{lstlisting}
contract Efficient {
    function processData(uint256[] calldata input) public {
        uint256[] memory tempArray = new uint256[](input.length);
        for (uint i = 0; i < input.length; i++) {
            tempArray[i] = input[i] * 2;  // Memory write (cheap)
        }
        // ... use tempArray ...
    }
}
\end{lstlisting}
\end{frame}

\begin{frame}[fragile]{Gas Optimization: Short-Circuiting}
\textbf{Exploit boolean evaluation order:}

\vspace{0.3cm}
\textbf{Inefficient: Expensive Check First}
\begin{lstlisting}
function transfer(address to, uint256 amount) public {
    require(balances[msg.sender] >= amount && to != address(0), "Invalid");
    // If to == address(0), still loads balances[msg.sender] (2100 gas SLOAD)
}
\end{lstlisting}

\vspace{0.3cm}
\textbf{Efficient: Cheap Check First}
\begin{lstlisting}
function transfer(address to, uint256 amount) public {
    require(to != address(0) && balances[msg.sender] >= amount, "Invalid");
    // If to == address(0), immediately fails (no SLOAD)
}
\end{lstlisting}

\vspace{0.3cm}
\textbf{Principle:} Place cheaper conditions first in logical AND (\&\&)

\textbf{Savings:} 2100 gas when early condition fails
\end{frame}

\begin{frame}[fragile]{Gas Optimization: Event Logs vs Storage}
\textbf{Events are much cheaper than storage:}

\vspace{0.3cm}
\begin{columns}[T]
\begin{column}{0.48\textwidth}
\textbf{Storage:}
\begin{itemize}
\item Accessible on-chain
\item 20,000 gas per new slot
\item 2,100 gas to read (cold)
\item Persistent, queryable
\item Required for contract logic
\end{itemize}
\end{column}
\begin{column}{0.48\textwidth}
\textbf{Events:}
\begin{itemize}
\item Not accessible on-chain
\item 375 gas + 375/topic + 8/byte
\item Cannot read from contracts
\item Stored in logs, queryable off-chain
\item Great for historical data
\end{itemize}
\end{column}
\end{columns}

\vspace{0.3cm}
\textbf{Example:}
\begin{lstlisting}
// Store transaction history in events (cheap)
event Transfer(address indexed from, address indexed to, uint256 amount);

function transfer(address to, uint256 amount) public {
    balances[msg.sender] -= amount;
    balances[to] += amount;
    emit Transfer(msg.sender, to, amount);  // ~1500 gas
    // DON'T store in array: transactionHistory.push(...) would be 20,000+ gas
}
\end{lstlisting}
\end{frame}

\begin{frame}[fragile]{Gas Refunds}
\textbf{Get partial refunds for freeing storage:}

\vspace{0.3cm}
\textbf{Refundable Actions:}
\begin{itemize}
\item \textbf{SSTORE to zero:} 15,000 gas refund (after paying 5,000 to clear)
\item \textbf{SELFDESTRUCT:} 24,000 gas refund (contract deletion)
\end{itemize}

\vspace{0.3cm}
\textbf{Refund Cap (EIP-3529):}
\begin{itemize}
\item Maximum refund: 20\% of gas used
\item Prevents gas token exploitation
\item Example: Use 100,000 gas $\rightarrow$ max refund 20,000 gas
\end{itemize}

\vspace{0.3cm}
\textbf{Example:}
\begin{lstlisting}
function clearStorage() public {
    delete largeMapping[key1];  // 5,000 gas cost + 15,000 refund
    delete largeMapping[key2];  // 5,000 gas cost + 15,000 refund
    // Gas used: 10,000
    // Potential refund: 30,000 (but capped at 20% = 2,000)
}
\end{lstlisting}
\end{frame}

\begin{frame}{Real-World Gas Usage Patterns}
\textbf{Typical gas costs on Ethereum mainnet:}

\vspace{0.3cm}
\begin{center}
\begin{tabular}{lr}
\toprule
\textbf{Transaction Type} & \textbf{Gas Used} \\
\midrule
Simple ETH transfer & 21,000 \\
ERC-20 token transfer & 45,000 - 65,000 \\
Uniswap V2 swap & 100,000 - 150,000 \\
Uniswap V3 swap & 120,000 - 185,000 \\
NFT mint (ERC-721) & 80,000 - 150,000 \\
OpenSea NFT purchase & 150,000 - 300,000 \\
Deploy simple contract & 200,000 - 500,000 \\
Deploy complex contract (e.g., Uniswap V3) & 4,000,000+ \\
\bottomrule
\end{tabular}
\end{center}

\vspace{0.3cm}
\textbf{At 30 Gwei + 2 Gwei priority fee (32 Gwei total):}
\begin{itemize}
\item Simple transfer: 0.000672 ETH (\$1.34 at \$2000/ETH)
\item Uniswap swap: 0.004 ETH (\$8 at \$2000/ETH)
\item Complex contract deploy: 0.128 ETH (\$256 at \$2000/ETH)
\end{itemize}
\end{frame}

\begin{frame}{EIP-4844: Blob Gas Market (Dencun Upgrade, March 2024)}
\textbf{New Gas Type for Layer 2 Data:}
\begin{itemize}
\item Ethereum now has TWO gas markets:
\begin{enumerate}
\item \textbf{Execution gas}: Traditional EVM operations (existing market)
\item \textbf{Blob gas}: Data availability for L2 rollups (new market)
\end{enumerate}
\item Markets operate independently with separate base fees
\end{itemize}

\vspace{0.3cm}
\textbf{Blob Gas Mechanics:}
\begin{itemize}
\item Target: 3 blobs per block (384 KB)
\item Maximum: 6 blobs per block (768 KB)
\item Blob gas price adjusts like EIP-1559 (targets 50\% capacity)
\item Typical blob gas price: 1-10 Gwei (much cheaper than calldata)
\end{itemize}

\vspace{0.3cm}
\textbf{Cost Comparison (posting 100KB of L2 data):}
\begin{itemize}
\item \textbf{Pre-Dencun (calldata)}: 100,000 bytes $\times$ 16 gas/byte = 1.6M gas = \$5-50
\item \textbf{Post-Dencun (blobs)}: ~131,072 blob gas = \$0.01-0.10
\item \textbf{Savings}: 90-99\% reduction in L2 fees
\end{itemize}
\end{frame}

\begin{frame}{Key Takeaways}
\begin{enumerate}
\item \textbf{Gas Purpose:} Prevents spam/DoS, compensates validators, allocates scarce block space

\item \textbf{EIP-1559:} Introduced base fee (burned) + priority fee (to validator) for predictable pricing

\item \textbf{Base Fee Dynamics:} Adjusts by up to 12.5\% per block to target 50\% full blocks

\item \textbf{Storage is Expensive:} SSTORE costs 20,000 gas (new) or 5,000 gas (update), use sparingly

\item \textbf{Optimization Strategies:} Pack storage, use memory for temp data, batch operations, emit events

\item \textbf{Real Costs:} Simple transfer costs \$1-2, complex DeFi interactions can cost \$10-50+
\end{enumerate}
\end{frame}

\begin{frame}{Discussion Questions}
\begin{enumerate}
\item Why does EIP-1559 burn the base fee instead of giving it to validators?

\item If Ethereum's block gas limit is 30M and average block time is 12 seconds, what is the theoretical maximum transactions per second for simple ETH transfers?

\item Under what circumstances might a user set a very high max fee per gas?

\item How do Layer 2 solutions (e.g., Optimism, Arbitrum) reduce gas costs?

\item What are the tradeoffs between storing data on-chain vs using events vs using off-chain storage (IPFS)?
\end{enumerate}
\end{frame}

\begin{frame}{Next Lesson Preview: L15 - Solidity Fundamentals}
\textbf{Coming up next:}
\begin{itemize}
\item Introduction to Solidity programming language
\item Data types: uint, address, string, arrays, mappings
\item Functions, visibility modifiers, state mutability
\item Events and error handling
\item Inheritance and interfaces
\item Writing your first smart contract (HelloWorld, Counter)
\end{itemize}

\vspace{0.3cm}
\textbf{Preparation:}
\begin{itemize}
\item Install MetaMask browser extension
\item Familiarize yourself with Remix IDE (remix.ethereum.org)
\item Review basic programming concepts (if-else, loops, functions)
\end{itemize}
\end{frame}

\end{document}
