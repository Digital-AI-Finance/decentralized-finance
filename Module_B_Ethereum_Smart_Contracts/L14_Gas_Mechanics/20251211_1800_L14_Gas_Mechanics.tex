\documentclass[8pt,aspectratio=169]{beamer}
\usetheme{Madrid}
\usepackage[utf8]{inputenc}
\usepackage{graphicx}
\usepackage{booktabs}
\usepackage{hyperref}
\usepackage{listings}
\lstset{basicstyle=\tiny\ttfamily,breaklines=true}

\title{L14: Gas Mechanics}
\subtitle{Module B: Ethereum \& Smart Contracts}
\author{Blockchain \& Cryptocurrency Course}
\date{December 2025}

\begin{document}

\begin{frame}
\titlepage
\end{frame}

\begin{frame}{Learning Objectives}
By the end of this lesson, you will be able to:
\begin{itemize}
\item Explain what gas is and why Ethereum uses it
\item Calculate transaction costs using gas price and gas limit
\item Understand EIP-1559's base fee and priority fee mechanism
\item Identify gas costs for different EVM operations
\item Apply optimization techniques to reduce gas consumption
\item Analyze real-world gas usage patterns
\end{itemize}
\end{frame}

\begin{frame}{What is Gas?}
\textbf{Gas is a unit of computational effort in Ethereum:}
\begin{itemize}
\item Measures the cost of executing operations on the EVM
\item Prevents infinite loops and spam attacks
\item Compensates validators for computation and storage
\end{itemize}
\begin{center}
\includegraphics[width=0.58\textwidth]{charts/01_gas_concept/chart.pdf}
\end{center}
\end{frame}

\begin{frame}{Why Gas Exists}
\textbf{Three Critical Functions:}

\begin{enumerate}
\item \textbf{Prevent Denial-of-Service Attacks:}
\begin{itemize}
\item Without gas, infinite loops could halt the network
\item Attackers would need to pay for computational resources
\end{itemize}

\item \textbf{Incentivize Validators:}
\begin{itemize}
\item Validators earn transaction fees for including transactions
\item Higher gas price = higher priority in block inclusion
\end{itemize}

\item \textbf{Resource Allocation:}
\begin{itemize}
\item Limited block gas limit (e.g., 30,000,000 gas per block)
\item Prioritizes transactions willing to pay more
\end{itemize}
\end{enumerate}
\end{frame}

\begin{frame}{Ether Denominations}
\textbf{Ether units from smallest to largest:}

\vspace{0.2cm}
\begin{center}
\begin{tabular}{lll}
\toprule
\textbf{Unit} & \textbf{Wei Value} & \textbf{Typical Use} \\
\midrule
Wei & 1 & Smallest unit (like satoshi) \\
Gwei (Shannon) & 10\textsuperscript{9} & Gas prices \\
Microether (Szabo) & 10\textsuperscript{12} & - \\
Milliether (Finney) & 10\textsuperscript{15} & - \\
Ether & 10\textsuperscript{18} & Main unit \\
\bottomrule
\end{tabular}
\end{center}

\vspace{0.2cm}
\textbf{Most Common:}
\begin{itemize}
\item \textbf{Gwei (Gigawei):} Standard unit for gas prices (1 Gwei = 10\textsuperscript{9} Wei)
\item \textbf{Ether:} User-facing unit (1 ETH = 10\textsuperscript{18} Wei)
\end{itemize}
\end{frame}

\begin{frame}{Gas Calculation: Pre-EIP-1559}
\textbf{Legacy Transaction Fee Model (before August 2021):}

\textbf{Formula:}
\[
\text{Transaction Fee} = \text{Gas Used} \times \text{Gas Price}
\]

\textbf{Example:}
\begin{itemize}
\item Gas Used: 21,000 (simple ETH transfer)
\item Gas Price: 50 Gwei (user-specified)
\item Transaction Fee: $21{,}000 \times 50 = 1{,}050{,}000$ Gwei = 0.00105 ETH
\end{itemize}

\textbf{Challenges:}
\begin{itemize}
\item Users had to manually estimate gas price
\item Overpaying was common to ensure inclusion
\item No refund if gas price was too high
\end{itemize}
\end{frame}

\begin{frame}{EIP-1559: London Hard Fork (August 2021)}
\textbf{Major overhaul of gas fee mechanism:}
\begin{itemize}
\item \textbf{Base Fee:} Algorithmically determined, burned (removed from circulation)
\item \textbf{Priority Fee (Tip):} User-specified tip to validator for inclusion
\item \textbf{Max Fee:} Maximum gas price user is willing to pay
\end{itemize}
\begin{center}
\includegraphics[width=0.60\textwidth]{charts/02_eip1559_breakdown/chart.pdf}
\end{center}
\end{frame}

\begin{frame}{EIP-1559 Fee Calculation}
\textbf{New Formula:}
\[
\text{Transaction Fee} = \text{Gas Used} \times (\text{Base Fee} + \text{Priority Fee})
\]

\textbf{With cap:}
\[
\text{Effective Gas Price} = \min(\text{Base Fee} + \text{Priority Fee}, \text{Max Fee})
\]

\textbf{Example:}
\begin{itemize}
\item Gas Used: 21,000
\item Base Fee: 30 Gwei (set by protocol)
\item Priority Fee: 2 Gwei (user tip)
\item Effective Gas Price: $\min(30 + 2, 50) = 32$ Gwei
\item Transaction Fee: $21{,}000 \times 32 = 672{,}000$ Gwei = 0.000672 ETH
\item Burned: $21{,}000 \times 30 = 630{,}000$ Gwei
\end{itemize}
\end{frame}

\begin{frame}{Base Fee Adjustment Mechanism}
\textbf{Dynamic base fee targets 50\% full blocks:}
\begin{itemize}
\item Target gas per block: 15,000,000 (50\% of 30M limit)
\item If block $>$ 50\% full: Base fee increases by max 12.5\%
\item If block $<$ 50\% full: Base fee decreases by max 12.5\%
\end{itemize}
\begin{center}
\includegraphics[width=0.55\textwidth]{charts/03_base_fee_adjustment/chart.pdf}
\end{center}
\end{frame}

\begin{frame}{Gas Limit vs Gas Used}
\textbf{Understanding the difference:}

\begin{columns}[T]
\begin{column}{0.48\textwidth}
\textbf{Gas Limit:}
\begin{itemize}
\item Maximum gas transaction may consume
\item Set by user before sending transaction
\item Acts as safety cap
\item Unused gas is refunded
\end{itemize}
\end{column}
\begin{column}{0.48\textwidth}
\textbf{Gas Used:}
\begin{itemize}
\item Actual gas consumed by transaction
\item Determined by operations executed
\item Used for fee calculation
\item Visible on Etherscan
\end{itemize}
\end{column}
\end{columns}

\vspace{0.2cm}
\textbf{Common Values:} Simple transfer: 21,000 | ERC-20 transfer: 45,000-65,000 | Complex contract: 100,000+
\end{frame}

\begin{frame}{Gas Costs by Operation}
\textbf{Every EVM opcode has a fixed gas cost:}
\begin{center}
\includegraphics[width=0.58\textwidth]{charts/04_gas_costs_operations/chart.pdf}
\end{center}
\textbf{Key Insight:} Storage operations (SSTORE, SLOAD) are by far the most expensive
\end{frame}

\begin{frame}{Gas Costs by Operation (Table)}
\begin{center}
\begin{tabular}{llr}
\toprule
\textbf{Operation} & \textbf{Description} & \textbf{Gas Cost} \\
\midrule
ADD, SUB, MUL & Arithmetic operations & 3 \\
DIV, MOD & Division/modulo & 5 \\
SHA3 (Keccak-256) & Hash function & 30 + 6/word \\
SLOAD & Load from storage & 100 (warm) / 2100 (cold) \\
SSTORE & Write to storage & 20,000 (new) / 5,000 (update) \\
CALL & External contract call & 700 + value transfer costs \\
CREATE & Deploy contract & 32,000 + code size \\
\bottomrule
\end{tabular}
\end{center}

\textbf{Cold vs Warm Access (EIP-2929):}
\begin{itemize}
\item \textbf{Cold:} First access to storage slot in transaction (expensive)
\item \textbf{Warm:} Subsequent accesses to same slot (cheaper)
\end{itemize}
\end{frame}

\begin{frame}{Storage: The Gas Guzzler}
\textbf{Why storage is expensive:}
\begin{itemize}
\item Persists data across all nodes forever
\item Requires disk I/O (slower than RAM)
\item State bloat affects all future nodes
\end{itemize}

\textbf{Storage Gas Costs (EIP-2929, EIP-2200):}
\begin{itemize}
\item \textbf{SSTORE (set to non-zero from zero):} 20,000 gas
\item \textbf{SSTORE (update non-zero):} 5,000 gas
\item \textbf{SSTORE (set to zero):} 5,000 gas + 15,000 refund
\item \textbf{SLOAD (cold access):} 2,100 gas (first access in transaction)
\item \textbf{SLOAD (warm access):} 100 gas (subsequent accesses)
\end{itemize}

\textbf{Example:} Storing one 256-bit word (32 bytes) at 32 Gwei: 0.00064 ETH = \$1.28 at \$2000/ETH
\end{frame}

\begin{frame}[fragile]{Gas Optimization: Storage Patterns}
\textbf{Inefficient: Multiple SSTOREs}
\begin{lstlisting}
contract Inefficient {
    uint256 public value1;
    uint256 public value2;
    uint256 public value3;

    function updateAll(uint256 v1, uint256 v2, uint256 v3) public {
        value1 = v1;  // 20,000 gas (or 5,000 if updating)
        value2 = v2;  // 20,000 gas
        value3 = v3;  // 20,000 gas
    }
    // Total: 60,000 gas for 3 writes
}
\end{lstlisting}

\textbf{Efficient: Packed Storage}
\begin{lstlisting}
contract Efficient {
    uint256 public packedValues;  // Pack 3 uint85 values in one slot

    function updateAll(uint85 v1, uint85 v2, uint85 v3) public {
        packedValues = uint256(v1) | (uint256(v2) << 85) | (uint256(v3) << 170);
    }
    // Total: 20,000 gas for single write (3x cheaper!)
}
\end{lstlisting}
\end{frame}

\begin{frame}[fragile]{Gas Optimization: Memory vs Storage}
\textbf{Use memory for temporary data:}

\textbf{Inefficient: Storage for Temporary Array}
\begin{lstlisting}
contract Inefficient {
    uint256[] public tempArray;  // Storage
    function processData(uint256[] calldata input) public {
        delete tempArray;
        for (uint i = 0; i < input.length; i++) {
            tempArray.push(input[i] * 2);  // SSTORE per iteration
        }
    }
}
\end{lstlisting}

\textbf{Efficient: Memory Array}
\begin{lstlisting}
contract Efficient {
    function processData(uint256[] calldata input) public {
        uint256[] memory tempArray = new uint256[](input.length);
        for (uint i = 0; i < input.length; i++) {
            tempArray[i] = input[i] * 2;  // Memory write (cheap)
        }
    }
}
\end{lstlisting}
\end{frame}

\begin{frame}[fragile]{Gas Optimization: Short-Circuiting}
\textbf{Exploit boolean evaluation order:}

\textbf{Inefficient: Expensive Check First}
\begin{lstlisting}
function transfer(address to, uint256 amount) public {
    require(balances[msg.sender] >= amount && to != address(0), "Invalid");
    // If to == address(0), still loads balances[msg.sender] (2100 gas SLOAD)
}
\end{lstlisting}

\textbf{Efficient: Cheap Check First}
\begin{lstlisting}
function transfer(address to, uint256 amount) public {
    require(to != address(0) && balances[msg.sender] >= amount, "Invalid");
    // If to == address(0), immediately fails (no SLOAD)
}
\end{lstlisting}

\textbf{Principle:} Place cheaper conditions first in logical AND (\&\&)

\textbf{Savings:} 2100 gas when early condition fails
\end{frame}

\begin{frame}[fragile]{Gas Optimization: Event Logs vs Storage}
\textbf{Events are much cheaper than storage:}

\begin{columns}[T]
\begin{column}{0.48\textwidth}
\textbf{Storage:} Accessible on-chain, 20,000 gas per new slot
\end{column}
\begin{column}{0.48\textwidth}
\textbf{Events:} Not accessible on-chain, 375 gas + 8/byte
\end{column}
\end{columns}

\begin{lstlisting}
event Transfer(address indexed from, address indexed to, uint256 amount);

function transfer(address to, uint256 amount) public {
    balances[msg.sender] -= amount;
    balances[to] += amount;
    emit Transfer(msg.sender, to, amount);  // ~1500 gas vs 20,000+ for storage
}
\end{lstlisting}
\end{frame}

\begin{frame}{Real-World Transaction Costs}
\textbf{Typical gas costs on Ethereum mainnet:}
\begin{center}
\includegraphics[width=0.58\textwidth]{charts/05_real_world_costs/chart.pdf}
\end{center}
\textbf{Note:} Costs vary significantly based on network congestion and gas prices
\end{frame}

\begin{frame}{Real-World Gas Usage (Table)}
\begin{center}
\begin{tabular}{lr}
\toprule
\textbf{Transaction Type} & \textbf{Gas Used} \\
\midrule
Simple ETH transfer & 21,000 \\
ERC-20 token transfer & 45,000 - 65,000 \\
Uniswap V2 swap & 100,000 - 150,000 \\
Uniswap V3 swap & 120,000 - 185,000 \\
NFT mint (ERC-721) & 80,000 - 150,000 \\
OpenSea NFT purchase & 150,000 - 300,000 \\
Deploy simple contract & 200,000 - 500,000 \\
\bottomrule
\end{tabular}
\end{center}

\textbf{At 32 Gwei (30 base + 2 tip):}
\begin{itemize}
\item Simple transfer: 0.000672 ETH (\$1.34 at \$2000/ETH)
\item Uniswap swap: 0.004 ETH (\$8 at \$2000/ETH)
\item Contract deploy: 0.016+ ETH (\$32+ at \$2000/ETH)
\end{itemize}
\end{frame}

\begin{frame}[fragile]{Gas Refunds}
\textbf{Refundable Actions:}
\begin{itemize}
\item \textbf{SSTORE to zero:} 15,000 gas refund (after paying 5,000 to clear)
\item \textbf{SELFDESTRUCT:} 24,000 gas refund (contract deletion)
\end{itemize}

\textbf{Refund Cap (EIP-3529):} Maximum refund: 20\% of gas used (prevents exploitation)

\begin{lstlisting}
function clearStorage() public {
    delete largeMapping[key1];  // 5,000 cost + 15,000 refund
    delete largeMapping[key2];  // 5,000 cost + 15,000 refund
    // Gas used: 10,000, Potential: 30,000 (capped at 2,000)
}
\end{lstlisting}
\end{frame}

\begin{frame}{EIP-4844: Blob Gas Market (Dencun Upgrade, March 2024)}
\textbf{Ethereum now has TWO gas markets:}
\begin{itemize}
\item \textbf{Execution gas}: Traditional EVM operations (existing market)
\item \textbf{Blob gas}: Data availability for L2 rollups (new market)
\item Markets operate independently with separate base fees
\end{itemize}

\textbf{Blob Gas Mechanics:}
\begin{itemize}
\item Target: 3 blobs per block (384 KB), Maximum: 6 blobs (768 KB)
\item Blob gas price adjusts like EIP-1559 (targets 50\% capacity)
\end{itemize}
\end{frame}

\begin{frame}{EIP-4844: Blob Gas Savings}
\textbf{Cost comparison for posting L2 data to Ethereum:}
\begin{center}
\includegraphics[width=0.58\textwidth]{charts/06_blob_gas_savings/chart.pdf}
\end{center}
\textbf{Impact:} 90-99\% reduction in L2 transaction costs since Dencun upgrade
\end{frame}

\begin{frame}{Key Takeaways}
\begin{enumerate}
\item \textbf{Gas Purpose:} Prevents spam/DoS, compensates validators, allocates block space
\item \textbf{EIP-1559:} Base fee (burned) + priority fee (to validator) for predictable pricing
\item \textbf{Base Fee Dynamics:} Adjusts up to 12.5\% per block to target 50\% full blocks
\item \textbf{Storage is Expensive:} SSTORE costs 20,000 gas (new) or 5,000 gas (update)
\item \textbf{Optimization:} Pack storage, use memory for temp data, batch operations
\item \textbf{EIP-4844:} Blob gas market reduced L2 costs by 90-99\%
\end{enumerate}
\end{frame}

\begin{frame}{Discussion Questions}
\begin{enumerate}
\item Why does EIP-1559 burn the base fee instead of giving it to validators?

\item If Ethereum's block gas limit is 30M and average block time is 12 seconds, what is the theoretical maximum transactions per second for simple ETH transfers?

\item Under what circumstances might a user set a very high max fee per gas?

\item How do Layer 2 solutions (e.g., Optimism, Arbitrum) reduce gas costs?

\item What are the tradeoffs between storing data on-chain vs using events vs off-chain storage?
\end{enumerate}
\end{frame}

\begin{frame}{Next Lesson Preview: L15 - Solidity Fundamentals}
\textbf{Coming up next:}
\begin{itemize}
\item Introduction to Solidity programming language
\item Data types: uint, address, string, arrays, mappings
\item Functions, visibility modifiers, state mutability
\item Events and error handling
\item Inheritance and interfaces
\end{itemize}

\textbf{Preparation:}
\begin{itemize}
\item Install MetaMask browser extension
\item Familiarize yourself with Remix IDE (remix.ethereum.org)
\item Review basic programming concepts (if-else, loops, functions)
\end{itemize}
\end{frame}

\end{document}
