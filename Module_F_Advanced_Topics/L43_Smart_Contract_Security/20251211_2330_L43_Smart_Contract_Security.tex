\documentclass[8pt,aspectratio=169]{beamer}
\usetheme{Madrid}
\usepackage[utf8]{inputenc}
\usepackage{graphicx}
\usepackage{booktabs}
\usepackage{hyperref}
\usepackage{amsmath}

\newcommand{\bottomnote}[1]{\vfill\par\noindent\footnotesize\textit{#1}}

\title{L43: Smart Contract Security}
\subtitle{Module F: Advanced Topics}
\author{Blockchain \& Cryptocurrency Course}
\date{December 2025}

\begin{document}

\begin{frame}
\titlepage
\end{frame}

\begin{frame}{Learning Objectives}
\begin{itemize}
\item Understand the stakes of smart contract security
\item Identify top vulnerability types (reentrancy, access control, oracle manipulation)
\item Analyze real-world exploits (The DAO, Parity, bridge hacks)
\item Apply security tools (Slither, Mythril, formal verification)
\item Implement defense in depth strategies
\end{itemize}
\end{frame}

\begin{frame}{The Stakes of Smart Contract Security}
\begin{itemize}
\item \textbf{Code is Law}: Smart contracts are immutable and self-executing
\item \textbf{High-Value Targets}: DeFi protocols hold billions in assets
\item \textbf{Irreversibility}: Bugs cannot be patched without upgradeable contracts
\item \textbf{Losses to Date}: \$8B+ stolen from smart contract exploits (2016-2024)
\item \textbf{Asymmetry}: One vulnerability can drain entire protocol
\item \textbf{Composability Risk}: Vulnerabilities cascade across protocols
\end{itemize}
\end{frame}

\begin{frame}[t]{Top Vulnerability Types}
\begin{center}
\includegraphics[width=0.55\textwidth]{charts/01_top_vulnerabilities/chart.pdf}
\end{center}
\bottomnote{Access control failures have become the leading cause of DeFi exploits}
\end{frame}

\begin{frame}[t]{Historical Losses}
\begin{center}
\includegraphics[width=0.55\textwidth]{charts/02_hack_losses_timeline/chart.pdf}
\end{center}
\bottomnote{2022 was peak year (\$3.1B); losses decreasing but still significant}
\end{frame}

\begin{frame}{Reentrancy: The DAO Hack (2016)}
\begin{itemize}
\item \textbf{The DAO}: Decentralized organization, \$150M raised
\item \textbf{Vulnerability}: External call before state update
\item \textbf{Attack Mechanism}:
\begin{enumerate}
\item Attacker deposits 1 ETH
\item Calls withdraw(1), contract sends 1 ETH
\item Attacker's fallback recursively calls withdraw()
\item Balance not yet updated, sends another 1 ETH
\item Loop continues until contract drained
\end{enumerate}
\item \textbf{Result}: \$60M drained
\item \textbf{Aftermath}: Ethereum hard fork (ETH vs ETC split)
\end{itemize}
\end{frame}

\begin{frame}{Reentrancy: Defense}
\textbf{Checks-Effects-Interactions Pattern:}
\begin{enumerate}
\item \textbf{Check}: Verify conditions (require statements)
\item \textbf{Effect}: Update state variables FIRST
\item \textbf{Interaction}: Make external calls LAST
\end{enumerate}

\vspace{0.3cm}
\textbf{Reentrancy Guard (Mutex):}
\begin{itemize}
\item Lock flag prevents recursive calls
\item OpenZeppelin ReentrancyGuard is industry standard
\end{itemize}

\vspace{0.3cm}
\textbf{Root Cause}: State updated after external call

\textbf{Prevention}: Always update state before external calls
\end{frame}

\begin{frame}{Access Control Vulnerabilities}
\begin{columns}[T]
\begin{column}{0.45\textwidth}
\textbf{Common Mistakes}
\begin{itemize}
\item Missing onlyOwner modifier
\item Default function visibility (public)
\item Unprotected selfdestruct
\item Constructor typo (pre-0.5.0)
\end{itemize}
\vspace{0.2cm}
\textbf{Parity Wallet Hack (2017)}
\begin{itemize}
\item initWallet() unprotected
\item Attacker became owner
\item Called selfdestruct
\item \$300M frozen forever
\end{itemize}
\end{column}
\begin{column}{0.45\textwidth}
\textbf{Best Practices}
\begin{itemize}
\item Use OpenZeppelin Ownable
\item Explicit visibility modifiers
\item Role-based access control
\item Multi-sig for critical functions
\end{itemize}
\vspace{0.2cm}
\textbf{2024 Reality}
\begin{itemize}
\item Access control is now top attack vector
\item Private key compromises
\item Admin privilege escalation
\end{itemize}
\end{column}
\end{columns}
\end{frame}

\begin{frame}{Oracle Manipulation}
\begin{itemize}
\item \textbf{Problem}: DeFi relies on external price data (oracles)
\item \textbf{Vulnerable Pattern}: Using single DEX price as oracle
\item \textbf{Attack Vector}:
\begin{enumerate}
\item Flash loan borrow massive funds
\item Manipulate DEX pool price (10x increase)
\item Protocol reads manipulated price
\item Exploit (borrow, liquidate, mint at wrong price)
\item Restore pool, repay flash loan, keep profit
\end{enumerate}
\item \textbf{Real Examples}: Harvest Finance (\$34M), Cream (\$130M)
\item \textbf{Defense}: Chainlink oracles, TWAP, multiple sources
\end{itemize}
\end{frame}

\begin{frame}[t]{Attack Vectors by Loss Amount}
\begin{center}
\includegraphics[width=0.50\textwidth]{charts/05_attack_vectors/chart.pdf}
\end{center}
\bottomnote{Bridge exploits account for largest share of total losses}
\end{frame}

\begin{frame}{Bridge Exploits: The New Frontier}
\begin{itemize}
\item \textbf{Why Bridges?}: Cross-chain transfers require lock-and-mint
\item \textbf{Attack Surface}: Validators, smart contracts, key management
\item \textbf{Notable Exploits}:
\begin{itemize}
\item Ronin Bridge (2022): \$625M - validator key compromise
\item Wormhole (2022): \$320M - signature verification bug
\item Nomad (2022): \$190M - initialization vulnerability
\end{itemize}
\item \textbf{Key Issues}:
\begin{itemize}
\item Centralized validator sets
\item Complex multi-chain logic
\item High-value targets (\$10B+ locked in bridges)
\end{itemize}
\item \textbf{Trend}: Bridge security is top priority for 2024-2025
\end{itemize}
\end{frame}

\begin{frame}[t]{Security Tools Comparison}
\begin{center}
\includegraphics[width=0.55\textwidth]{charts/04_security_tools/chart.pdf}
\end{center}
\bottomnote{Combine fast tools (Slither) with deep analysis (Mythril, Certora)}
\end{frame}

\begin{frame}{Security Tools Overview}
\begin{table}
\centering
\small
\begin{tabular}{lll}
\toprule
\textbf{Tool} & \textbf{Type} & \textbf{Detects} \\
\midrule
Slither & Static analyzer & Reentrancy, overflow, access \\
Mythril & Symbolic execution & Integer bugs, unchecked calls \\
Echidna & Fuzzer & Invariant violations \\
Certora & Formal verifier & Specification violations \\
\bottomrule
\end{tabular}
\end{table}
\vspace{0.3cm}
\begin{itemize}
\item \textbf{Slither}: Fast, easy CI/CD integration
\item \textbf{Mythril}: Deep analysis, finds complex bugs
\item \textbf{Echidna}: Property-based testing
\item \textbf{Certora}: Mathematical proofs
\end{itemize}
\end{frame}

\begin{frame}[t]{Defense in Depth}
\begin{center}
\includegraphics[width=0.55\textwidth]{charts/03_audit_funnel/chart.pdf}
\end{center}
\bottomnote{Each security layer catches bugs that previous layers missed}
\end{frame}

\begin{frame}{Audit Process}
\begin{enumerate}
\item \textbf{Internal Review}: Developer self-audit, peer review
\item \textbf{Automated Tools}: Slither, Mythril in CI/CD
\item \textbf{Manual Audit}: Security firm (2-4 weeks, \$50k-\$500k)
\begin{itemize}
\item Trail of Bits, OpenZeppelin, Quantstamp
\end{itemize}
\item \textbf{Formal Verification}: High-value contracts (Certora)
\item \textbf{Bug Bounty}: Community testing (Immunefi)
\item \textbf{Monitoring}: Real-time detection (Forta, OpenZeppelin Defender)
\item \textbf{Insurance}: Coverage for exploits (Nexus Mutual)
\end{enumerate}
\end{frame}

\begin{frame}{Bug Bounty Programs}
\begin{itemize}
\item \textbf{Purpose}: Incentivize white-hat hackers
\item \textbf{Platforms}: Immunefi, HackerOne, Code4rena
\item \textbf{Payouts}: \$1k (low) to \$10M+ (critical)
\item \textbf{Record}: Wormhole \$10M bounty (2022)
\item \textbf{Notable Programs}:
\begin{itemize}
\item MakerDAO: Up to \$10M
\item Ethereum Foundation: Up to \$250k
\item Compound: Up to \$500k
\end{itemize}
\item \textbf{ROI}: \$1 in bounties prevents \$100+ in losses
\item \textbf{Best Practice}: Continuous bounty, not just pre-launch
\end{itemize}
\end{frame}

\begin{frame}{Upgradeable Contracts}
\begin{itemize}
\item \textbf{Problem}: Immutable contracts cannot be patched
\item \textbf{Solution}: Proxy pattern (separate storage and logic)
\item \textbf{Transparent Proxy}:
\begin{itemize}
\item Proxy holds storage, delegates to implementation
\item Admin can upgrade implementation address
\end{itemize}
\item \textbf{UUPS (Universal Upgradeable Proxy)}:
\begin{itemize}
\item Upgrade logic in implementation (smaller proxy)
\end{itemize}
\item \textbf{Risk}: Admin key compromise $\to$ malicious upgrade
\item \textbf{Mitigation}: Multi-sig, timelock delays, immutable after maturity
\end{itemize}
\end{frame}

\begin{frame}{Security Best Practices}
\begin{enumerate}
\item Use latest Solidity (0.8.0+ for overflow protection)
\item Follow Checks-Effects-Interactions pattern
\item Apply reentrancy guards (OpenZeppelin)
\item Explicit visibility for all functions
\item Decentralized oracles (Chainlink) or TWAP
\item Pull over push for payments
\item Avoid loops over unbounded arrays
\item Run Slither + Mythril in CI/CD
\item Professional audit before mainnet
\item Bug bounty program
\item Real-time monitoring (Forta)
\end{enumerate}
\end{frame}

\begin{frame}{Summary}
\textbf{Key Takeaways:}
\begin{itemize}
\item Smart contract security is critical: \$8B+ cumulative losses
\item Top vulnerabilities: Access control, reentrancy, oracle manipulation, bridges
\item The DAO hack (\$60M) led to Ethereum hard fork
\item Bridge exploits are now largest loss category
\item Defense in depth: Automated tools + audits + formal verification + bounties
\item Tools: Slither (fast), Mythril (deep), Certora (formal proofs)
\item Upgradeable contracts: Allow patches but introduce admin key risk
\item 2024 trend: Real-time monitoring and ``Security as a Service''
\end{itemize}
\end{frame}

\begin{frame}{Questions for Reflection}
\begin{enumerate}
\item Why is the Checks-Effects-Interactions pattern important?
\item How do oracle manipulation attacks exploit DeFi protocols?
\item What makes bridge security particularly challenging?
\item Should audited protocols still have bug bounties?
\item What are the trade-offs of upgradeable contracts?
\end{enumerate}
\end{frame}

\end{document}
