\documentclass[8pt,aspectratio=169]{beamer}
\usetheme{Madrid}
\usepackage[utf8]{inputenc}
\usepackage{graphicx}
\usepackage{booktabs}
\usepackage{hyperref}
\usepackage{amsmath}

\newcommand{\bottomnote}[1]{\vfill\par\noindent\footnotesize\textit{#1}}

\title{L42: Flash Loans and Composability}
\subtitle{Module F: Advanced Topics}
\author{Blockchain \& Cryptocurrency Course}
\date{December 2025}

\begin{document}

\begin{frame}
\titlepage
\end{frame}

\begin{frame}{Learning Objectives}
\begin{itemize}
\item Understand flash loan mechanics and atomicity
\item Analyze legitimate flash loan use cases
\item Examine flash loan attack vectors and notable exploits
\item Explore DeFi composability (``money legos'')
\item Understand MEV and its relationship to flash loans
\end{itemize}
\end{frame}

\begin{frame}{What is a Flash Loan?}
\begin{itemize}
\item \textbf{Definition}: Uncollateralized loan borrowed and repaid within single transaction
\item \textbf{Key Property}: \textit{Atomicity} -- loan and repayment are all-or-nothing
\item \textbf{If repayment fails}: Entire transaction reverts, lender loses nothing
\item \textbf{No Collateral Required}: Enabled by smart contract execution model
\item \textbf{Loan Size}: Unlimited (constrained only by liquidity pool)
\item \textbf{Duration}: Fraction of a second (one block)
\item \textbf{Unique to DeFi}: Impossible in traditional finance
\end{itemize}
\end{frame}

\begin{frame}[t]{Flash Loan Transaction Flow}
\begin{center}
\includegraphics[width=0.65\textwidth]{charts/01_flash_loan_flow/chart.pdf}
\end{center}
\bottomnote{If any step fails, the entire transaction reverts -- zero risk to lender}
\end{frame}

\begin{frame}{Traditional Loans vs Flash Loans}
\begin{table}
\centering
\small
\begin{tabular}{lll}
\toprule
\textbf{Property} & \textbf{Traditional Loan} & \textbf{Flash Loan} \\
\midrule
Collateral & Required (often >100\%) & None \\
Duration & Days/months/years & Single transaction \\
Creditworthiness & Required (KYC) & Not required \\
Repayment Guarantee & Legal contracts & Smart contract atomicity \\
Risk to Lender & Default risk & Zero (reverts) \\
\bottomrule
\end{tabular}
\end{table}
\vspace{0.3cm}
\textbf{Paradigm Shift}: Code execution guarantees replace legal enforcement
\end{frame}

\begin{frame}[t]{Flash Loan Providers}
\begin{center}
\includegraphics[width=0.55\textwidth]{charts/02_providers_comparison/chart.pdf}
\end{center}
\bottomnote{Aave is largest provider; dYdX offers free flash loans}
\end{frame}

\begin{frame}{DeFi Composability}
\begin{itemize}
\item \textbf{Composability}: DeFi protocols are like ``money legos''
\item \textbf{Permissionless Integration}: Any contract can call any other contract
\item \textbf{Atomic Transactions}: Multiple protocol interactions in one transaction
\item \textbf{Examples of Composability}:
\begin{enumerate}
\item Swap on Uniswap $\to$ Deposit into Aave $\to$ Borrow
\item Flash loan $\to$ Liquidate $\to$ Swap collateral $\to$ Repay
\item Borrow from Compound $\to$ Yield farm on Curve
\end{enumerate}
\item \textbf{Risk}: Cascading failures, expanded attack surface
\end{itemize}
\end{frame}

\begin{frame}[t]{Legitimate Use Cases}
\begin{center}
\includegraphics[width=0.55\textwidth]{charts/05_use_cases/chart.pdf}
\end{center}
\bottomnote{All use cases democratize capital-intensive strategies to anyone}
\end{frame}

\begin{frame}{Flash Loan Use Case: Arbitrage}
\begin{itemize}
\item \textbf{Scenario}: ETH trades at \$2000 on Uniswap, \$2020 on SushiSwap
\item \textbf{Flash Loan Arbitrage}:
\begin{enumerate}
\item Borrow 1000 ETH via flash loan
\item Buy 1000 ETH on Uniswap (\$2,000,000)
\item Sell 1000 ETH on SushiSwap (\$2,020,000)
\item Repay flash loan + fee (\$2,001,000)
\item Profit: \$19,000 in one transaction
\end{enumerate}
\item \textbf{Capital Required}: Only gas fees (\$50-\$200)
\item \textbf{Democratization}: Anyone can be arbitrageur, not just whales
\end{itemize}
\end{frame}

\begin{frame}{Flash Loan Use Case: Collateral Swap}
\begin{itemize}
\item \textbf{Problem}: User has debt collateralized with Asset A, wants Asset B
\item \textbf{Flash Loan Solution}:
\begin{enumerate}
\item Borrow Asset A via flash loan
\item Repay existing debt
\item Withdraw original collateral
\item Deposit new collateral (Asset B)
\item Borrow Asset A again
\item Repay flash loan
\end{enumerate}
\item \textbf{Result}: Collateral swapped atomically, zero liquidation risk
\item \textbf{Fee}: Only flash loan fee (0.05-0.09\%)
\end{itemize}
\end{frame}

\begin{frame}{Flash Loan Attacks: Overview}
\begin{itemize}
\item \textbf{Dark Side}: Flash loans enable large-scale attacks with zero capital
\item \textbf{Attack Pattern}:
\begin{enumerate}
\item Borrow massive amount via flash loan
\item Manipulate protocol state (price oracle, governance)
\item Exploit manipulation for profit
\item Repay flash loan
\end{enumerate}
\item \textbf{Impact}: \$500M+ stolen via flash loan attacks (2020-2023)
\item \textbf{Key Insight}: Flash loans amplify existing vulnerabilities
\end{itemize}
\end{frame}

\begin{frame}[t]{Major Flash Loan Attacks}
\begin{center}
\includegraphics[width=0.55\textwidth]{charts/03_attack_timeline/chart.pdf}
\end{center}
\bottomnote{Oracle manipulation is most common attack vector}
\end{frame}

\begin{frame}{Attack Example: Oracle Manipulation}
\begin{itemize}
\item \textbf{Vulnerable Protocol}: Uses single DEX price as oracle
\item \textbf{Attack Steps}:
\begin{enumerate}
\item Borrow 10,000 ETH via flash loan
\item Buy all TOKEN on Uniswap (10x price increase)
\item Protocol oracle reads inflated price
\item Borrow stablecoins using overvalued TOKEN
\item Sell TOKEN back (price normalizes)
\item Repay flash loan + keep stablecoins
\end{enumerate}
\item \textbf{Real Example}: Harvest Finance (\$34M stolen)
\item \textbf{Mitigation}: Use TWAP or Chainlink oracles
\end{itemize}
\end{frame}

\begin{frame}{Defense Mechanisms}
\begin{enumerate}
\item \textbf{Decentralized Oracles}: Use Chainlink (not single DEX)
\item \textbf{Time-Weighted Average Price (TWAP)}: Average over blocks
\item \textbf{Reentrancy Guards}: Prevent recursive contract calls
\item \textbf{Governance Delays}: Timelock on parameter changes
\item \textbf{Vote Locking}: Require tokens locked before voting
\item \textbf{Circuit Breakers}: Pause if anomalous activity detected
\end{enumerate}
\end{frame}

\begin{frame}[t]{MEV and Flash Loans}
\begin{center}
\includegraphics[width=0.50\textwidth]{charts/04_mev_extraction/chart.pdf}
\end{center}
\bottomnote{Flash loans amplify both legitimate arbitrage and harmful MEV extraction}
\end{frame}

\begin{frame}{MEV Extraction Techniques}
\begin{itemize}
\item \textbf{MEV (Maximal Extractable Value)}: Profit from transaction ordering
\item \textbf{Flash Loans + MEV}: Amplify arbitrage and liquidation profits
\item \textbf{Techniques}:
\begin{itemize}
\item \textbf{Front-running}: Place transaction before victim's
\item \textbf{Back-running}: Place transaction after victim's
\item \textbf{Sandwich Attacks}: Front-run + back-run
\end{itemize}
\item \textbf{Flashbots}: Democratize MEV extraction, reduce gas wars
\item \textbf{MEV Volume}: \$600M+ extracted in 2023
\end{itemize}
\end{frame}

\begin{frame}{Economic Impact}
\begin{columns}[T]
\begin{column}{0.45\textwidth}
\textbf{Positive Effects}
\begin{itemize}
\item Democratize arbitrage
\item Increase market efficiency
\item Enable capital-efficient refinancing
\item Liquidation bots improve protocol health
\end{itemize}
\end{column}
\begin{column}{0.45\textwidth}
\textbf{Negative Effects}
\begin{itemize}
\item Enable zero-capital attacks
\item Amplify protocol vulnerabilities
\item MEV extraction harms users
\item Governance manipulation risk
\end{itemize}
\end{column}
\end{columns}
\vspace{0.4cm}
\textbf{Net Assessment}: Powerful tool that magnifies both good and bad protocol design
\end{frame}

\begin{frame}{Flash Loans and Regulation}
\begin{itemize}
\item \textbf{Legal Gray Area}: No clear regulatory framework
\item \textbf{Key Questions}:
\begin{itemize}
\item Are flash loan attacks theft or code exploitation?
\item Is the protocol or attacker liable?
\item Code is law vs legal enforcement
\end{itemize}
\item \textbf{Precedent}: Mango Markets attacker arrested (Oct 2022)
\begin{itemize}
\item Charged with market manipulation, not flash loan use
\end{itemize}
\item \textbf{Protocol Responsibility}: Bug bounties, audits, insurance
\end{itemize}
\end{frame}

\begin{frame}{Summary}
\textbf{Key Takeaways:}
\begin{itemize}
\item Flash loans: Uncollateralized loans repaid in single atomic transaction
\item Enabled by smart contract atomicity: Revert on failure = zero lender risk
\item Use cases: Arbitrage, collateral swaps, self-liquidation, refinancing
\item Composability: DeFi protocols as interoperable building blocks
\item Attacks: Oracle manipulation, governance takeover (\$500M+ stolen)
\item Flash loans amplify existing protocol vulnerabilities (not root cause)
\item Defenses: Decentralized oracles, TWAP, vote locking, circuit breakers
\end{itemize}
\end{frame}

\begin{frame}{Questions for Reflection}
\begin{enumerate}
\item Why can flash loans exist in DeFi but not in traditional finance?
\item How do flash loans democratize arbitrage opportunities?
\item What makes oracle manipulation the most common attack vector?
\item Should flash loan attackers be prosecuted if they exploit code bugs?
\item How can protocols defend against flash loan governance attacks?
\end{enumerate}
\end{frame}

\end{document}
