\documentclass[8pt,aspectratio=169]{beamer}
\usetheme{Madrid}
\usepackage[utf8]{inputenc}
\usepackage{graphicx}
\usepackage{booktabs}
\usepackage{hyperref}
\usepackage{amsmath}

\title{L42: Flash Loans and Composability}
\subtitle{Module F: Advanced Topics}
\author{Blockchain \& Cryptocurrency Course}
\date{December 2025}

\begin{document}

\begin{frame}
\titlepage
\end{frame}

\begin{frame}{What is a Flash Loan?}
\begin{itemize}
\item \textbf{Definition}: Uncollateralized loan that must be borrowed and repaid within a single transaction
\item \textbf{Key Property}: \textit{Atomicity} -- loan and repayment are atomic (all-or-nothing)
\item \textbf{If repayment fails}: Entire transaction reverts, lender loses nothing
\item \textbf{No Collateral Required}: Enabled by smart contract execution model
\item \textbf{Loan Size}: Unlimited (constrained only by liquidity pool)
\item \textbf{Duration}: Fraction of a second (one block)
\item \textbf{Unique to DeFi}: Impossible in traditional finance
\end{itemize}
\end{frame}

\begin{frame}{Traditional Loans vs Flash Loans}
\begin{table}
\centering
\small
\begin{tabular}{lll}
\toprule
\textbf{Property} & \textbf{Traditional Loan} & \textbf{Flash Loan} \\
\midrule
Collateral & Required (often >100\%) & None \\
Duration & Days/months/years & Single transaction (seconds) \\
Creditworthiness & Required (KYC, credit score) & Not required \\
Repayment Guarantee & Legal contracts & Smart contract atomicity \\
Risk to Lender & Default risk & Zero (transaction reverts) \\
Use Cases & Consumption, investment & Arbitrage, refinancing \\
\bottomrule
\end{tabular}
\end{table}
\vspace{0.3cm}
\textbf{Paradigm Shift}: Code execution guarantees replace legal enforcement
\end{frame}

\begin{frame}{Flash Loan Mechanism}
\begin{enumerate}
\item \textbf{Borrower calls flash loan function} on lending protocol (e.g., Aave)
\item \textbf{Protocol transfers tokens} to borrower's contract
\item \textbf{Borrower's contract executes arbitrary logic}:
\begin{itemize}
\item Trade on DEXs
\item Refinance debt positions
\item Liquidate collateral
\item Exploit arbitrage opportunities
\end{itemize}
\item \textbf{Borrower repays loan + fee} (typically 0.05-0.09\%)
\item \textbf{Protocol checks repayment}:
\begin{itemize}
\item If successful: Transaction confirmed
\item If failed: \texttt{revert()} -- entire transaction cancelled
\end{itemize}
\end{enumerate}
\end{frame}

\begin{frame}[fragile]{Flash Loan Code Example (Simplified)}
\small
\begin{verbatim}
// Borrower's contract
function executeFlashLoan() external {
    ILendingPool(aave).flashLoan(
        address(this),        // receiver
        DAI_ADDRESS,         // asset to borrow
        1000000 ether,       // amount (1M DAI)
        ""                   // params
    );
}

function executeOperation(
    address asset,
    uint256 amount,
    uint256 premium,
    address initiator
) external returns (bool) {
    // 1. Received 1M DAI
    // 2. Execute arbitrage/liquidation/refinancing
    doArbitrage();
    // 3. Approve repayment
    IERC20(asset).approve(msg.sender, amount + premium);
    return true;  // Success: repay. Failure: revert entire transaction
}
\end{verbatim}
\end{frame}

\begin{frame}{DeFi Composability}
\begin{itemize}
\item \textbf{Composability}: DeFi protocols are like ``money legos''
\item \textbf{Permissionless Integration}: Any contract can call any other contract
\item \textbf{Atomic Transactions}: Multiple protocol interactions in one transaction
\item \textbf{Examples of Composability}:
\begin{enumerate}
\item Swap on Uniswap $\rightarrow$ Deposit into Aave $\rightarrow$ Borrow against collateral
\item Flash loan $\rightarrow$ Liquidate position $\rightarrow$ Swap collateral $\rightarrow$ Repay
\item Borrow from Compound $\rightarrow$ Yield farm on Curve $\rightarrow$ Stake LP tokens
\end{enumerate}
\item \textbf{Innovation Engine}: New protocols build on existing primitives
\item \textbf{Risk}: Cascading failures, attack surface expansion
\end{itemize}
\end{frame}

\begin{frame}{Flash Loan Use Case 1: Arbitrage}
\begin{itemize}
\item \textbf{Scenario}: ETH trades at \$2000 on Uniswap, \$2020 on SushiSwap
\item \textbf{Traditional Arbitrage}: Requires capital upfront
\item \textbf{Flash Loan Arbitrage}:
\begin{enumerate}
\item Borrow 1000 ETH via flash loan
\item Buy 1000 ETH on Uniswap (\$2,000,000)
\item Sell 1000 ETH on SushiSwap (\$2,020,000)
\item Repay flash loan + fee (\$2,000,000 + \$1,000)
\item Profit: \$19,000 in one transaction
\end{enumerate}
\item \textbf{Capital Required}: Only gas fees (\$50-\$200)
\item \textbf{Market Impact}: Arbitrage opportunities closed instantly
\item \textbf{Democratization}: Anyone can be arbitrageur, not just whales
\end{itemize}
\end{frame}

\begin{frame}{Flash Loan Use Case 2: Collateral Swap}
\begin{itemize}
\item \textbf{Problem}: User has debt collateralized with Asset A, wants to switch to Asset B
\item \textbf{Traditional Solution}: Close position, pay interest, open new position (costly)
\item \textbf{Flash Loan Solution}:
\begin{enumerate}
\item Borrow Asset A via flash loan
\item Repay existing debt on Protocol 1
\item Withdraw original collateral
\item Deposit new collateral (Asset B)
\item Borrow Asset A on Protocol 2
\item Repay flash loan
\end{enumerate}
\item \textbf{Result}: Collateral swapped in one transaction, zero liquidation risk
\item \textbf{Fee}: Only flash loan fee (0.05-0.09\%)
\end{itemize}
\end{frame}

\begin{frame}{Flash Loan Use Case 3: Self-Liquidation}
\begin{itemize}
\item \textbf{Scenario}: User's collateralized position near liquidation threshold
\item \textbf{Problem}: External liquidation incurs 5-15\% penalty
\item \textbf{Flash Loan Self-Liquidation}:
\begin{enumerate}
\item Borrow funds via flash loan
\item Repay own debt
\item Withdraw collateral
\item Sell collateral to repay flash loan
\end{enumerate}
\item \textbf{Savings}: Avoid liquidation penalty, keep liquidation bonus
\item \textbf{Example}: Save 10\% penalty on \$100,000 position = \$10,000 saved
\end{itemize}
\end{frame}

\begin{frame}{Flash Loan Providers}
\begin{table}
\centering
\small
\begin{tabular}{llll}
\toprule
\textbf{Protocol} & \textbf{Fee} & \textbf{Assets} & \textbf{Max Liquidity} \\
\midrule
Aave & 0.09\% & 30+ tokens & \$10B+ \\
dYdX & 0\% & ETH, USDC, DAI & \$500M \\
Uniswap V3 & Variable & Any pool token & Pool-dependent \\
Balancer & 0.05\% & Pool tokens & Pool-dependent \\
\bottomrule
\end{tabular}
\end{table}
\vspace{0.3cm}
\begin{itemize}
\item \textbf{Aave}: Largest provider, most liquid
\item \textbf{dYdX}: Free flash loans (margin trading focused)
\item \textbf{Uniswap V3}: Flash swaps (borrow any token in pool)
\end{itemize}
\end{frame}

\begin{frame}{Flash Loan Attacks: Overview}
\begin{itemize}
\item \textbf{Dark Side}: Flash loans enable large-scale attacks with zero capital
\item \textbf{Attack Pattern}:
\begin{enumerate}
\item Borrow massive amount via flash loan
\item Manipulate protocol state (price oracle, governance, liquidity)
\item Exploit manipulation for profit
\item Repay flash loan
\end{enumerate}
\item \textbf{Impact}: \$500M+ stolen via flash loan attacks (2020-2023)
\item \textbf{Key Insight}: Flash loans amplify existing vulnerabilities
\item \textbf{Not the flash loan's fault}: Underlying protocol weakness is the root cause
\end{itemize}
\end{frame}

\begin{frame}{Flash Loan Attack Example: Oracle Manipulation}
\begin{itemize}
\item \textbf{Vulnerable Protocol}: Uses on-chain DEX price as oracle (e.g., single Uniswap pool)
\item \textbf{Attack Steps}:
\begin{enumerate}
\item Borrow 10,000 ETH via flash loan
\item Buy all TOKEN on Uniswap pool (manipulate price 10x higher)
\item Protocol oracle reads inflated price
\item Borrow stablecoins using overvalued TOKEN as collateral
\item Sell TOKEN back to pool (price normalizes)
\item Repay flash loan + keep borrowed stablecoins
\end{enumerate}
\item \textbf{Real Example}: Harvest Finance attack (Oct 2020) -- \$34M stolen
\item \textbf{Mitigation}: Use time-weighted average price (TWAP) or Chainlink oracles
\end{itemize}
\end{frame}

\begin{frame}{Flash Loan Attack Example: Governance Attack}
\begin{itemize}
\item \textbf{Vulnerable Protocol}: Snapshot-based governance (voting power = token balance at block)
\item \textbf{Attack Steps}:
\begin{enumerate}
\item Borrow 10M governance tokens via flash loan
\item Vote on malicious proposal in same transaction
\item Repay flash loan
\end{enumerate}
\item \textbf{Malicious Proposals}:
\begin{itemize}
\item Change protocol parameters (fees, interest rates)
\item Drain treasury
\item Upgrade contract to backdoored version
\end{itemize}
\item \textbf{Real Example}: BZx Protocol (Feb 2020) -- governance manipulation
\item \textbf{Mitigation}: Time-delayed voting, vote locking, delegation mechanisms
\end{itemize}
\end{frame}

\begin{frame}{Notable Flash Loan Attacks}
\begin{itemize}
\item \textbf{BZx (Feb 2020)}: \$350k stolen via oracle manipulation
\item \textbf{Harvest Finance (Oct 2020)}: \$34M via USDC/USDT pool manipulation
\item \textbf{Cream Finance (Oct 2021)}: \$130M via reentrancy + flash loan
\item \textbf{Beanstalk (Apr 2022)}: \$182M via governance takeover (flash loan borrowed governance tokens)
\item \textbf{Mango Markets (Oct 2022)}: \$110M via oracle manipulation (borrowed \$116M USDC to inflate MNGO price)
\item \textbf{Euler Finance (Mar 2023)}: \$200M via donation attack + flash loan
\vspace{0.3cm}
\item \textbf{Common Thread}: Flash loans exploit existing vulnerabilities at scale
\end{itemize}
\end{frame}

\begin{frame}{Defense Mechanisms}
\begin{enumerate}
\item \textbf{Decentralized Oracles}: Use Chainlink, Band Protocol (not single DEX)
\item \textbf{Time-Weighted Average Price (TWAP)}: Average price over multiple blocks
\item \textbf{Reentrancy Guards}: Prevent recursive contract calls
\item \textbf{Governance Delays}: Timelock on parameter changes (24-48h)
\item \textbf{Vote Locking}: Require tokens locked for X days before voting
\item \textbf{Flash Loan Detection}: Check if \texttt{tx.origin == msg.sender}
\item \textbf{Circuit Breakers}: Pause protocol if anomalous activity detected
\item \textbf{Liquidity Caps}: Limit max borrow amount per transaction
\end{enumerate}
\end{frame}

\begin{frame}{Economic Impact of Flash Loans}
\begin{columns}[T]
\begin{column}{0.48\textwidth}
\textbf{Positive Effects}
\begin{itemize}
\item Democratize arbitrage (no capital barrier)
\item Increase market efficiency
\item Enable capital-efficient refinancing
\item Innovation in DeFi tooling
\item Liquidation bots improve protocol health
\end{itemize}
\end{column}
\begin{column}{0.48\textwidth}
\textbf{Negative Effects}
\begin{itemize}
\item Enable zero-capital attacks
\item Amplify protocol vulnerabilities
\item MEV extraction (miner extractable value)
\item Governance manipulation risk
\item Increased attack surface
\end{itemize}
\end{column}
\end{columns}
\vspace{0.5cm}
\textbf{Net Assessment}: Powerful tool that magnifies both good and bad aspects of protocol design
\end{frame}

\begin{frame}{MEV and Flash Loans}
\begin{itemize}
\item \textbf{MEV (Maximal Extractable Value)}: Profit from transaction ordering
\item \textbf{Flash Loans + MEV}: Amplify arbitrage and liquidation profits
\item \textbf{Searchers}: Bots scanning mempool for MEV opportunities
\item \textbf{Techniques}:
\begin{itemize}
\item \textbf{Front-running}: Place transaction before victim's transaction
\item \textbf{Back-running}: Place transaction after victim's transaction
\item \textbf{Sandwich Attacks}: Front-run + back-run (manipulate price around victim)
\end{itemize}
\item \textbf{Flashbots}: Democratize MEV extraction, reduce gas wars
\item \textbf{MEV Volume}: \$600M+ extracted in 2023
\end{itemize}
\end{frame}

\begin{frame}{Building a Flash Loan Strategy}
\begin{enumerate}
\item \textbf{Identify Opportunity}:
\begin{itemize}
\item Monitor price discrepancies across DEXs
\item Track liquidatable positions
\item Analyze governance proposals
\end{itemize}
\item \textbf{Design Atomic Transaction}:
\begin{itemize}
\item Flash loan $\rightarrow$ Execute strategy $\rightarrow$ Repay
\item Account for fees, slippage, gas costs
\end{itemize}
\item \textbf{Implement Smart Contract}:
\begin{itemize}
\item Integrate with lending protocol (Aave, dYdX)
\item Add safety checks (minimum profit threshold)
\end{itemize}
\item \textbf{Simulate and Test}:
\begin{itemize}
\item Use Hardhat/Foundry for local testing
\item Fork mainnet for realistic simulation
\end{itemize}
\item \textbf{Deploy and Monitor}:
\begin{itemize}
\item Gas optimization critical (failed transactions cost gas)
\end{itemize}
\end{itemize}
\end{frame}

\begin{frame}{Flash Loans and Regulation}
\begin{itemize}
\item \textbf{Legal Gray Area}: No clear regulatory framework
\item \textbf{Questions}:
\begin{itemize}
\item Are flash loan attacks theft or exploitation of code?
\item Is the protocol or attacker liable?
\item Code is law vs legal enforcement
\end{itemize}
\item \textbf{Precedents}:
\begin{itemize}
\item Mango Markets attacker Avi Eisenberg arrested (Oct 2022)
\item Charged with market manipulation, not flash loan use
\end{itemize}
\item \textbf{Regulatory Uncertainty}: Chilling effect on DeFi innovation
\item \textbf{Protocol Responsibility}: Bug bounties, audits, insurance
\end{itemize}
\end{frame}

\begin{frame}{Future of Flash Loans}
\begin{itemize}
\item \textbf{Cross-Chain Flash Loans}: Borrow on Ethereum, use on Arbitrum/Polygon
\item \textbf{Flash Minting}: Mint unbacked stablecoins within transaction (e.g., MakerDAO)
\item \textbf{Flash Loan Aggregators}: Optimize across multiple providers
\item \textbf{Improved Defenses}: Protocol-level flash loan resistance
\item \textbf{Institutional Adoption}: Hedge funds using flash loans for arbitrage
\item \textbf{Layer 2 Integration}: Cheaper gas makes smaller arbitrage profitable
\item \textbf{Regulatory Clarity}: Potential restrictions on governance voting with borrowed tokens
\end{itemize}
\end{frame}

\begin{frame}{Summary}
\begin{itemize}
\item \textbf{Flash loans}: Uncollateralized loans repaid in single atomic transaction
\item \textbf{Enabled by smart contract atomicity}: Revert on failure = zero lender risk
\item \textbf{Use cases}: Arbitrage, collateral swaps, self-liquidation, refinancing
\item \textbf{Composability}: DeFi protocols as interoperable building blocks
\item \textbf{Attacks}: Oracle manipulation, governance takeover (\$500M+ stolen)
\item \textbf{Not inherently malicious}: Flash loans amplify existing protocol vulnerabilities
\item \textbf{Defenses}: Decentralized oracles, TWAP, vote locking, circuit breakers
\item \textbf{Impact}: Democratize capital-intensive strategies, increase market efficiency
\item \textbf{Future}: Cross-chain flash loans, better protocol defenses, regulatory clarity
\end{itemize}
\end{frame}

\end{document}
