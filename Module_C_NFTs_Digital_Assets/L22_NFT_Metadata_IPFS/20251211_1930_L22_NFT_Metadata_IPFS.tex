\documentclass[8pt,aspectratio=169]{beamer}
\usetheme{Madrid}
\usepackage[utf8]{inputenc}
\usepackage{graphicx}
\usepackage{booktabs}
\usepackage{hyperref}

\newcommand{\bottomnote}[1]{\vfill\hfill{\scriptsize\textit{#1}}}

\title{L22: NFT Metadata and IPFS}
\subtitle{Module C: NFTs \& Digital Assets}
\author{Blockchain \& Cryptocurrency Course}
\date{December 2025}

\begin{document}

\begin{frame}
\titlepage
\end{frame}

\begin{frame}{Learning Objectives}
By the end of this lesson, you will be able to:
\begin{itemize}
\item Understand the JSON metadata format standard for NFTs
\item Explain how IPFS content addressing works
\item Describe the role of pinning services in NFT permanence
\item Compare IPFS and Arweave for decentralized storage
\item Evaluate metadata permanence and availability challenges
\end{itemize}
\end{frame}

\begin{frame}[fragile]{NFT Metadata: The JSON Standard}
\textbf{Metadata Structure:} OpenSea standard (widely adopted)

\vspace{0.2cm}
\textbf{Core Fields:}
\begin{itemize}
\item \texttt{name} -- Display name of the NFT
\item \texttt{description} -- Human-readable description
\item \texttt{image} -- URI to primary visual asset (IPFS, HTTP)
\item \texttt{external\_url} -- Link to project website
\item \texttt{attributes} -- Array of trait objects
\end{itemize}

\vspace{0.2cm}
\textbf{Example Metadata JSON:}
\begin{verbatim}
{
  "name": "Bored Ape #1234",
  "image": "ipfs://QmXyZ.../1234.png",
  "attributes": [{"trait_type": "Background", "value": "Blue"}]
}
\end{verbatim}
\end{frame}

\begin{frame}{Attributes: Defining Rarity}
\textbf{Attributes Array:} Defines traits and properties

\vspace{0.2cm}
\textbf{Attribute Object Structure:}
\begin{itemize}
\item \texttt{trait\_type} -- Category name (e.g., ``Hat'', ``Eyes'')
\item \texttt{value} -- Specific value (e.g., ``Beanie'', ``Laser Eyes'')
\item \texttt{display\_type} (optional) -- How to render (number, date, boost)
\end{itemize}

\vspace{0.2cm}
\textbf{Rarity Calculation:}
\begin{itemize}
\item Each trait has a frequency distribution in the collection
\item Rarer traits (low frequency) increase NFT value
\item Tools like Rarity Sniper calculate rarity scores
\end{itemize}

\vspace{0.2cm}
\textbf{Example Rarity Analysis:}
\begin{itemize}
\item Background: Blue (20\% frequency) vs. Gold (2\% frequency)
\item Laser Eyes: 1\% frequency (highly rare)
\end{itemize}
\end{frame}

\begin{frame}{IPFS: The InterPlanetary File System}
\textbf{IPFS:} Decentralized peer-to-peer file storage protocol

\vspace{0.2cm}
\textbf{Key Concepts:}
\begin{itemize}
\item \textbf{Content Addressing:} Files identified by cryptographic hash
\item \textbf{CID (Content Identifier):} Unique hash of file content
\item \textbf{Immutability:} Same content always produces same CID
\item \textbf{Distributed:} Files stored across multiple nodes
\end{itemize}

\vspace{0.2cm}
\textbf{Example IPFS URI:}
\begin{itemize}
\item \texttt{ipfs://QmXyZ123abc...} (CID)
\item Gateway URL: \texttt{https://ipfs.io/ipfs/QmXyZ123abc...}
\end{itemize}
\end{frame}

\begin{frame}{Content Addressing vs Location Addressing}
\begin{center}
\includegraphics[width=0.75\textwidth]{charts/01_content_addressing/chart.pdf}
\end{center}
\bottomnote{IPFS: Same content always produces same CID, retrievable from any node}
\end{frame}

\begin{frame}{Pinning: Ensuring Availability}
\textbf{IPFS Garbage Collection:} Nodes delete unpinned files to save space

\vspace{0.2cm}
\textbf{Without Pinning:}
\begin{itemize}
\item File may be removed from all nodes (lost)
\item NFT metadata/images become unavailable
\item Broken image links in wallets and marketplaces
\end{itemize}

\vspace{0.2cm}
\textbf{Pinning Services:}
\begin{itemize}
\item \textbf{Pinata:} Popular commercial pinning service
\item \textbf{NFT.Storage:} Free pinning for NFT data (funded by Filecoin)
\item \textbf{Infura IPFS:} Enterprise-grade IPFS infrastructure
\item \textbf{Self-hosting:} Run your own IPFS node and pin files
\end{itemize}

\vspace{0.2cm}
\textbf{Critical Issue:} If all pinning stops, files may disappear from IPFS
\end{frame}

\begin{frame}{Pinning Services Comparison}
\begin{center}
\includegraphics[width=0.72\textwidth]{charts/02_pinning_services/chart.pdf}
\end{center}
\bottomnote{NFT.Storage offers free pinning backed by Filecoin; Pinata is most popular commercial option}
\end{frame}

\begin{frame}{IPFS Gateways}
\textbf{Gateway:} HTTP bridge to access IPFS content

\vspace{0.2cm}
\textbf{Public Gateways:}
\begin{itemize}
\item \texttt{https://ipfs.io/ipfs/[CID]}
\item \texttt{https://gateway.pinata.cloud/ipfs/[CID]}
\item \texttt{https://cloudflare-ipfs.com/ipfs/[CID]}
\end{itemize}

\vspace{0.2cm}
\textbf{Why Gateways Matter:}
\begin{itemize}
\item Browsers do not natively support \texttt{ipfs://} protocol
\item Wallets and marketplaces use gateways to display NFTs
\item Gateway downtime = NFTs appear broken
\end{itemize}

\vspace{0.2cm}
\textbf{Centralization Risk:}
\begin{itemize}
\item Most users rely on centralized gateways (ipfs.io, Cloudflare)
\item Solution: Native IPFS support in browsers (IPFS Companion extension)
\end{itemize}
\end{frame}

\begin{frame}{Arweave: Permanent Decentralized Storage}
\textbf{Arweave:} Blockchain designed for permanent data storage

\vspace{0.2cm}
\textbf{Key Differences from IPFS:}
\begin{itemize}
\item \textbf{Permanence:} One-time payment for perpetual storage
\item \textbf{Blockchain-based:} Data stored on Arweave blockchain
\item \textbf{Economic model:} Storage endowment fund pays miners forever
\item \textbf{No pinning needed:} Data guaranteed to persist
\end{itemize}

\vspace{0.2cm}
\textbf{Arweave URI:}
\begin{itemize}
\item \texttt{ar://[Transaction ID]}
\item Gateway: \texttt{https://arweave.net/[Transaction ID]}
\end{itemize}

\vspace{0.2cm}
\textbf{Use Cases:} High-value NFTs requiring guaranteed permanence
\end{frame}

\begin{frame}{IPFS vs Arweave Comparison}
\begin{center}
\includegraphics[width=0.72\textwidth]{charts/03_ipfs_vs_arweave/chart.pdf}
\end{center}
\bottomnote{IPFS for most projects; Arweave for premium NFTs requiring guaranteed permanence}
\end{frame}

\begin{frame}{Metadata Permanence Challenges}
\textbf{Real-World Issues:}

\vspace{0.2cm}
\begin{enumerate}
\item \textbf{Unpinned IPFS files:} Project abandons pinning, files lost
\item \textbf{Centralized metadata servers:} Company shuts down, NFTs break
\item \textbf{Mutable tokenURI:} Smart contract allows owner to change metadata
\item \textbf{Gateway failures:} IPFS gateways go offline, NFTs appear broken
\item \textbf{Link rot:} HTTP URLs stop working (404 errors)
\end{enumerate}

\vspace{0.2cm}
\textbf{Case Study: Nifty Gateway (2021):}
\begin{itemize}
\item Platform used centralized servers for metadata
\item Outage caused all NFTs to display broken images
\item Community backlash led to IPFS migration
\end{itemize}
\end{frame}

\begin{frame}{Metadata Permanence Risks}
\begin{center}
\includegraphics[width=0.72\textwidth]{charts/04_permanence_risks/chart.pdf}
\end{center}
\bottomnote{Server shutdown has highest severity; HTTP link rot is most likely to occur}
\end{frame}

\begin{frame}{Mutable vs. Immutable Metadata}
\textbf{Mutable Metadata:}
\begin{itemize}
\item Smart contract owner can update \texttt{tokenURI}
\item Allows bug fixes and metadata improvements
\item Risk: Owner could change artwork or traits (rug pull)
\end{itemize}

\vspace{0.2cm}
\textbf{Immutable Metadata:}
\begin{itemize}
\item \texttt{tokenURI} frozen after minting (contract locked)
\item Guarantees metadata cannot be altered
\item Standard for high-value collections (e.g., CryptoPunks)
\end{itemize}

\vspace{0.2cm}
\textbf{Verification:}
\begin{itemize}
\item Check smart contract code for \texttt{setTokenURI()} functions
\item Verify contract ownership is renounced (no admin control)
\item Use Etherscan to audit contract mutability
\end{itemize}
\end{frame}

\begin{frame}{NFT Storage Adoption}
\begin{center}
\includegraphics[width=0.72\textwidth]{charts/05_storage_adoption/chart.pdf}
\end{center}
\bottomnote{IPFS dominates NFT storage; Arweave growing for premium projects}
\end{frame}

\begin{frame}{Best Practices for NFT Metadata}
\textbf{For NFT Projects:}

\vspace{0.2cm}
\begin{enumerate}
\item \textbf{Use IPFS or Arweave:} Avoid centralized servers
\item \textbf{Pin all files:} Use reputable pinning services (Pinata, NFT.Storage)
\item \textbf{Freeze metadata:} Make \texttt{tokenURI} immutable after reveal
\item \textbf{Redundancy:} Pin to multiple services (IPFS + Arweave)
\item \textbf{Document storage:} Inform buyers where metadata is hosted
\end{enumerate}

\vspace{0.2cm}
\textbf{For NFT Buyers:}
\begin{itemize}
\item Verify metadata is on IPFS/Arweave (not HTTP)
\item Check if \texttt{tokenURI} is immutable
\item Confirm pinning service reputation
\end{itemize}
\end{frame}

\begin{frame}{Key Takeaways}
\begin{enumerate}
\item NFT metadata follows a JSON standard (name, description, image, attributes)
\item IPFS uses content addressing (CIDs) for decentralized, immutable storage
\item Pinning is critical for IPFS file permanence (unpinned files can be lost)
\item Arweave provides guaranteed permanent storage for a one-time fee
\item Metadata permanence challenges include unpinned files and mutable URIs
\item Best practice: Use IPFS/Arweave, pin to multiple services, freeze metadata
\end{enumerate}
\end{frame}

\begin{frame}{Discussion Questions}
\begin{enumerate}
\item What are the trade-offs between IPFS and Arweave for NFT metadata storage?
\item Should NFT metadata be immutable, or is mutability acceptable for bug fixes?
\item How can the NFT community ensure long-term metadata availability?
\item Is fully on-chain metadata the ideal, or are off-chain solutions sufficient?
\item What happens to NFT value if metadata becomes unavailable?
\end{enumerate}
\end{frame}

\begin{frame}{Next Lesson Preview}
\textbf{L23: NFT Marketplaces}

\vspace{0.2cm}
We will explore:
\begin{itemize}
\item OpenSea, Blur, and Rarible business models
\item Listing mechanics and order book systems
\item Marketplace fees and royalty enforcement
\item Wash trading and market manipulation detection
\item Aggregators and cross-marketplace trading
\end{itemize}

\vspace{0.2cm}
\textbf{Preparation:} Create a wallet and browse NFT collections on OpenSea
\end{frame}

\end{document}
