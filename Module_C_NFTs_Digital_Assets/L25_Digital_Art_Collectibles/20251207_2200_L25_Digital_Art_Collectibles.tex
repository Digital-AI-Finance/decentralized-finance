\documentclass[8pt,aspectratio=169]{beamer}
\usetheme{Madrid}
\usepackage[utf8]{inputenc}
\usepackage{graphicx}
\usepackage{booktabs}
\usepackage{hyperref}

\title{L25: Digital Art and Collectibles}
\subtitle{Module C: NFTs \& Digital Assets}
\author{Blockchain \& Cryptocurrency Course}
\date{December 2025}

\begin{document}

\begin{frame}
\titlepage
\end{frame}

\begin{frame}{Learning Objectives}
By the end of this lesson, you will be able to:
\begin{itemize}
\item Distinguish between 1/1 art, editions, and generative art NFTs
\item Understand generative art platforms and on-chain algorithms (Art Blocks)
\item Analyze PFP collections and their cultural significance (BAYC, CryptoPunks)
\item Apply valuation frameworks to digital art NFTs
\item Evaluate the impact of high-profile sales (Beeple \$69M)
\end{itemize}
\end{frame}

\begin{frame}{NFT Art Categories}
\textbf{Three Primary Categories:}

\vspace{0.3cm}
\begin{enumerate}
\item \textbf{1/1 Digital Art}
  \begin{itemize}
  \item Unique, single-edition artwork
  \item Artist-created digital paintings, photography, 3D art
  \item Platforms: SuperRare, Foundation, Nifty Gateway
  \end{itemize}

\item \textbf{Editions (Limited Prints)}
  \begin{itemize}
  \item Multiple identical copies (e.g., 25/100)
  \item More affordable than 1/1s
  \item Platforms: Manifold, Zora, MakersPlace
  \end{itemize}

\item \textbf{Generative Art}
  \begin{itemize}
  \item Algorithm-created unique outputs
  \item Code-based, parameters + randomness
  \item Platforms: Art Blocks, fx(hash)
  \end{itemize}
\end{enumerate}
\end{frame}

\begin{frame}{1/1 Digital Art: Unique Creations}
\textbf{Characteristics:}
\begin{itemize}
\item Single NFT per artwork (no duplicates)
\item Artist-signed and authenticated on-chain
\item Direct artist-to-collector relationship
\item Premium pricing (scarcity value)
\end{itemize}

\vspace{0.3cm}
\textbf{Leading 1/1 Platforms:}
\begin{itemize}
\item \textbf{SuperRare:} Curated marketplace, invitation-only artists
\item \textbf{Foundation:} Auction-based, community-curated
\item \textbf{Nifty Gateway:} Drops from celebrity artists (Beeple, Pak)
\end{itemize}

\vspace{0.3cm}
\textbf{Valuation Factors:}
\begin{itemize}
\item Artist reputation and exhibition history
\item Aesthetic quality and cultural relevance
\item Provenance (ownership history)
\item Scarcity within artist's portfolio
\end{itemize}
\end{frame}

\begin{frame}{Case Study: Beeple's ``Everydays'' (\$69M)}
\textbf{Event:} March 2021, Christie's auction

\vspace{0.3cm}
\textbf{Artwork Details:}
\begin{itemize}
\item Title: ``Everydays: The First 5000 Days''
\item Artist: Beeple (Mike Winkelmann)
\item Format: Collage of 5,000 daily digital artworks (2007-2021)
\item Sale price: \$69.3 million (42,329 ETH at time)
\item Buyer: MetaKovan (Vignesh Sundaresan)
\end{itemize}

\vspace{0.3cm}
\textbf{Significance:}
\begin{itemize}
\item Third-highest price for living artist (after Jeff Koons, David Hockney)
\item Legitimized NFTs in traditional art world
\item Catalyzed NFT bull market (2021 peak)
\item Demonstrated institutional adoption (Christie's, Sotheby's)
\end{itemize}
\end{frame}

\begin{frame}{Beeple Sale: Market Impact}
\textbf{Before Sale (Pre-March 2021):}
\begin{itemize}
\item NFT art seen as niche crypto subculture
\item Limited mainstream media coverage
\item Skepticism from traditional art institutions
\end{itemize}

\vspace{0.3cm}
\textbf{After Sale:}
\begin{itemize}
\item Global media frenzy (NYTimes, WSJ, CNN)
\item Traditional auction houses enter NFT market
\item Celebrity and brand NFT launches surge
\item NFT search interest increases 1000\%+
\end{itemize}

\vspace{0.3cm}
\textbf{Criticism:}
\begin{itemize}
\item Buyer (MetaKovan) owned Metapurse fund (potential conflict)
\item Artwork file stored off-chain (not IPFS initially)
\item Speculation on whether price reflected true artistic value
\end{itemize}
\end{frame}

\begin{frame}{Generative Art: Code as Artist}
\textbf{Definition:} Artwork generated by algorithmic code with controlled randomness

\vspace{0.3cm}
\textbf{How It Works:}
\begin{enumerate}
\item Artist writes generative algorithm (JavaScript, p5.js, etc.)
\item Algorithm uses random seed (often from transaction hash)
\item Each NFT mint produces unique output from same code
\item Result: Infinite variations within defined aesthetic parameters
\end{enumerate}

\vspace{0.3cm}
\textbf{Appeal:}
\begin{itemize}
\item Each piece is unique (seed-based randomness)
\item Collector participates in creation (mint reveals outcome)
\item On-chain code storage (true permanence)
\item Lower price than 1/1s (algorithmic scalability)
\end{itemize}
\end{frame}

\begin{frame}{Art Blocks: Premier Generative Platform}
\textbf{Founded:} 2020 by Erick Calderon (Snowfro)

\vspace{0.3cm}
\textbf{How Art Blocks Works:}
\begin{enumerate}
\item Artist submits generative algorithm (JavaScript)
\item Art Blocks curates and approves project
\item Collector mints NFT (pays ETH + gas)
\item Transaction hash seeds random number generator
\item Algorithm runs on-chain, generates unique output
\item Output stored as SVG or rendered image
\end{enumerate}

\vspace{0.3cm}
\textbf{Project Tiers:}
\begin{itemize}
\item \textbf{Curated:} Highly selective, premium projects (Fidenza, Ringers)
\item \textbf{Playground:} Emerging artists, experimental work
\item \textbf{Factory:} Open submissions, lower curation bar
\end{itemize}
\end{frame}

\begin{frame}{Iconic Generative Projects}
\textbf{Fidenza by Tyler Hobbs (2021):}
\begin{itemize}
\item 999 unique outputs, flow field algorithm
\item Mint price: 0.17 ETH, Floor price peak: 140 ETH (\$500k+)
\item Aesthetic: Colorful, organic curves and patterns
\end{itemize}

\vspace{0.3cm}
\textbf{Ringers by Dmitri Cherniak (2021):}
\begin{itemize}
\item 1,000 outputs, wrapped strings around pegs
\item Mint price: 0 ETH (free + gas), Floor peak: 25 ETH
\end{itemize}

\vspace{0.3cm}
\textbf{Chromie Squiggle (2020):}
\begin{itemize}
\item Art Blocks' first project (Snowfro)
\item 10,000 outputs, simple colorful curves
\item Cultural significance as ``genesis'' Art Blocks NFT
\end{itemize}

\vspace{0.3cm}
\textbf{Rarity:} Certain outputs extremely rare (algorithmic traits)
\end{frame}

\begin{frame}{PFP Collections: Profile Pictures as Identity}
\textbf{PFP:} Profile Picture NFTs, avatar-style collectibles

\vspace{0.3cm}
\textbf{Characteristics:}
\begin{itemize}
\item 10,000-item collections (standard size)
\item Trait-based variation (hat, eyes, background, etc.)
\item Social signaling (Twitter/Discord avatars)
\item Community membership and status
\end{itemize}

\vspace{0.3cm}
\textbf{Leading PFP Collections:}
\begin{itemize}
\item \textbf{CryptoPunks (2017):} 10,000 pixel art characters, Larva Labs
\item \textbf{Bored Ape Yacht Club (2021):} 10,000 apes, Yuga Labs
\item \textbf{Azuki (2022):} 10,000 anime-style characters
\item \textbf{Doodles (2021):} 10,000 pastel-colored characters
\end{itemize}

\vspace{0.3cm}
\textbf{Value Drivers:} Brand recognition, community strength, utility, celebrity holders
\end{frame}

\begin{frame}{CryptoPunks: The OG NFT Collection}
\textbf{Created:} June 2017 by Larva Labs (Matt Hall, John Watkinson)

\vspace{0.3cm}
\textbf{Technical Details:}
\begin{itemize}
\item 10,000 unique 24x24 pixel art characters
\item Stored on-chain (Ethereum contract)
\item Originally free to claim (gas only)
\item Pre-dates ERC-721 standard (custom contract)
\end{itemize}

\vspace{0.3cm}
\textbf{Rarity Breakdown:}
\begin{itemize}
\item 6,039 Male Punks
\item 3,840 Female Punks
\item 88 Zombie Punks
\item 24 Ape Punks
\item 9 Alien Punks (rarest, >10,000 ETH floor at peak)
\end{itemize}

\vspace{0.3cm}
\textbf{Cultural Impact:} Inspired entire PFP category, held by celebrities (Jay-Z, Snoop Dogg)
\end{frame}

\begin{frame}{Bored Ape Yacht Club: Community and Utility}
\textbf{Created:} April 2021 by Yuga Labs

\vspace{0.3cm}
\textbf{Key Innovations:}
\begin{itemize}
\item \textbf{IP rights:} Holders own commercial rights to their ape
\item \textbf{Community events:} Exclusive parties, metaverse experiences
\item \textbf{Airdrops:} Free Mutant Apes, ApeCoin tokens
\item \textbf{Brand extensions:} Merchandise, media deals (Hollywood projects)
\end{itemize}

\vspace{0.3cm}
\textbf{Valuation Milestones:}
\begin{itemize}
\item Mint price: 0.08 ETH (April 2021)
\item Floor price peak: 153 ETH (\$500k+, April 2022)
\item Current floor (2024): 20-30 ETH range
\end{itemize}

\vspace{0.3cm}
\textbf{Celebrity Holders:} Eminem, Stephen Curry, Paris Hilton, Justin Bieber
\end{frame}

\begin{frame}{PFP Valuation: Beyond Art}
\textbf{Traditional Art Valuation:} Aesthetics, artist reputation, scarcity

\vspace{0.3cm}
\textbf{PFP Valuation Factors:}
\begin{enumerate}
\item \textbf{Community strength:} Discord activity, holder engagement
\item \textbf{Brand recognition:} Mainstream awareness, celebrity adoption
\item \textbf{Utility:} IP rights, governance, ecosystem access
\item \textbf{Holder loyalty:} Low turnover, diamond hands mentality
\item \textbf{Team execution:} Roadmap delivery, innovation
\item \textbf{Cultural relevance:} Memes, social media presence
\end{enumerate}

\vspace{0.3cm}
\textbf{Key Insight:} PFPs valued as membership badges, not just art
\end{frame}

\begin{frame}{The Social Capital Thesis}
\textbf{Argument:} NFTs represent digital social capital and identity

\vspace{0.3cm}
\textbf{Mechanisms:}
\begin{itemize}
\item \textbf{Signaling:} PFP shows wealth, taste, community affiliation
\item \textbf{Access:} Token-gated Discord channels, events, alpha groups
\item \textbf{Status:} Owning rare/expensive NFTs confers prestige
\item \textbf{Network effects:} Value increases with community size
\end{itemize}

\vspace{0.3cm}
\textbf{Comparison to Luxury Goods:}
\begin{itemize}
\item Rolex watch: \$10k+ for timekeeping (social signal)
\item Bored Ape: \$100k+ for pixel art (social signal)
\item Both: Functional value << social signaling value
\end{itemize}

\vspace{0.3cm}
\textbf{Critique:} Assumes persistent community value, vulnerable to trend shifts
\end{frame}

\begin{frame}{Photography and Digital Media NFTs}
\textbf{Photography NFTs:}
\begin{itemize}
\item High-resolution digital photography as 1/1 or editions
\item Platforms: Quantum Art, Obscura DAO
\item Challenges: Authenticity (anyone can copy digital file)
\end{itemize}

\vspace{0.3cm}
\textbf{Video and Multimedia NFTs:}
\begin{itemize}
\item Short films, music videos, animations
\item Platforms: Zora, Catalog (music NFTs)
\item Storage: Large files require efficient IPFS/Arweave pinning
\end{itemize}

\vspace{0.3cm}
\textbf{Music NFTs:}
\begin{itemize}
\item Artists sell albums, singles, or stems as NFTs
\item Royalty splits programmed into smart contracts
\item Examples: 3LAU, Kings of Leon, Grimes
\end{itemize}
\end{frame}

\begin{frame}{Valuation Framework for Digital Art}
\textbf{Methodology:} Adapt traditional art valuation to NFTs

\vspace{0.3cm}
\textbf{Key Criteria:}
\begin{enumerate}
\item \textbf{Artist reputation:} Exhibition history, awards, following
\item \textbf{Scarcity:} 1/1 vs. edition size
\item \textbf{Provenance:} Ownership history (celebrity collectors)
\item \textbf{Cultural significance:} Historical impact, media coverage
\item \textbf{Technical quality:} Resolution, metadata storage (IPFS)
\item \textbf{Market comparables:} Similar artist sales
\end{enumerate}

\vspace{0.3cm}
\textbf{Example Valuation:}
\begin{itemize}
\item Established artist, 1/1 artwork, IPFS storage
\item Recent comparable sale: 5 ETH
\item Estimated range: 4-6 ETH
\end{itemize}
\end{frame}

\begin{frame}{Comparative Sales Analysis}
\textbf{Method:} Compare recent sales of similar NFTs

\vspace{0.3cm}
\textbf{Steps:}
\begin{enumerate}
\item Identify comparable NFTs (same artist, style, or category)
\item Collect sale prices (OpenSea, SuperRare history)
\item Adjust for market conditions (bull vs. bear market)
\item Calculate average or median sale price
\item Apply discount/premium for unique factors
\end{enumerate}

\vspace{0.3cm}
\textbf{Example: Generative Art Valuation:}
\begin{itemize}
\item Art Blocks Curated project, 1000 total supply
\item Recent sales: 2.5 ETH, 3.1 ETH, 2.8 ETH, 3.4 ETH
\item Average: 2.95 ETH
\item Rare trait (top 5\% rarity): Apply 2x multiplier = 5.9 ETH
\end{itemize}
\end{frame}

\begin{frame}{The Speculation vs. Appreciation Debate}
\textbf{Speculation Argument:}
\begin{itemize}
\item NFT prices driven by hype and FOMO, not intrinsic value
\item Extreme volatility (10x gains or 90\% losses)
\item Market manipulation (wash trading, celebrity pumps)
\item Bubble dynamics (2021 peak, 2022-23 collapse)
\end{itemize}

\vspace{0.3cm}
\textbf{Appreciation Argument:}
\begin{itemize}
\item Digital art has intrinsic aesthetic and cultural value
\item Provenance and scarcity create collectible value
\item Community and utility provide ongoing worth
\item Long-term holders (``collectors'') vs. flippers
\end{itemize}

\vspace{0.3cm}
\textbf{Reality:} Both elements present, varies by project and buyer motivation
\end{frame}

\begin{frame}{NFT Art Market Size and Trends}
\textbf{Historical Volume:}
\begin{itemize}
\item 2020: \$100M total NFT sales
\item 2021: \$25B (peak bull market)
\item 2022: \$10B (bear market decline)
\item 2023: \$6B (stabilization)
\item 2024: \$8B (modest recovery)
\end{itemize}

\vspace{0.3cm}
\textbf{Category Breakdown (2024 estimate):}
\begin{itemize}
\item PFP collections: 50\%
\item Generative art: 20\%
\item 1/1 art: 15\%
\item Gaming NFTs: 10\%
\item Other (photography, music): 5\%
\end{itemize}

\vspace{0.3cm}
\textbf{Trend:} Market consolidation around blue-chip collections (BAYC, Punks, Art Blocks)
\end{frame}

\begin{frame}{Blue-Chip NFT Collections}
\textbf{Blue-Chip:} Established collections with sustained demand and liquidity

\vspace{0.3cm}
\textbf{Criteria:}
\begin{itemize}
\item 2+ years of trading history
\item Consistent floor price support (resilience in bear markets)
\item Strong community and holder base
\item Cultural significance and brand recognition
\end{itemize}

\vspace{0.3cm}
\textbf{Current Blue-Chips (2024):}
\begin{itemize}
\item CryptoPunks (50+ ETH floor)
\item Bored Ape Yacht Club (20-30 ETH floor)
\item Autoglyphs (on-chain generative, 100+ ETH floor)
\item Art Blocks Curated projects (Fidenza, Ringers)
\item Azuki (2+ ETH floor)
\end{itemize}
\end{frame}

\begin{frame}{Long-Term Value Preservation}
\textbf{What Sustains NFT Value?}

\vspace{0.3cm}
\textbf{Permanent Factors:}
\begin{itemize}
\item Historical significance (``first'' or iconic status)
\item On-chain storage (true permanence)
\item Artist reputation growth over time
\item Cultural embedding (memes, media references)
\end{itemize}

\vspace{0.3cm}
\textbf{Fragile Factors:}
\begin{itemize}
\item Hype and trend cycles
\item Community activity (can fade)
\item Marketplace platform survival
\item Speculative demand
\end{itemize}

\vspace{0.3cm}
\textbf{Recommendation:} Focus on collections with strong permanent factors for long-term holding
\end{frame}

\begin{frame}{Key Takeaways}
\begin{enumerate}
\item NFT art includes 1/1 unique pieces, limited editions, and generative algorithmic art
\item Generative platforms (Art Blocks) create unique outputs from code using randomness
\item PFP collections (BAYC, CryptoPunks) derive value from community, brand, and social capital
\item Beeple's \$69M sale catalyzed mainstream NFT adoption but raised speculation concerns
\item Valuation frameworks combine traditional art criteria with on-chain metrics
\item Blue-chip NFTs show resilience through market cycles due to cultural significance
\end{enumerate}
\end{frame}

\begin{frame}{Discussion Questions}
\begin{enumerate}
\item Is digital art fundamentally different from physical art in terms of value and ownership?
\item What distinguishes a blue-chip NFT collection from a speculative project?
\item How should generative art be valued compared to artist-created 1/1 pieces?
\item Will PFP collections retain value long-term, or are they a temporary trend?
\item Should NFT marketplaces curate art quality, or remain open platforms?
\end{enumerate}
\end{frame}

\begin{frame}{Next Lesson Preview}
\textbf{L26: Gaming NFTs and Metaverse}

\vspace{0.3cm}
We will explore:
\begin{itemize}
\item Play-to-earn gaming model and tokenomics
\item Case study: Axie Infinity rise and collapse
\item Virtual land NFTs (Decentraland, The Sandbox)
\item Interoperability and cross-game asset portability
\item Sustainability challenges of blockchain gaming
\end{itemize}

\vspace{0.3cm}
\textbf{Preparation:} Explore Decentraland or The Sandbox metaverse platforms
\end{frame}

\end{document}
