\documentclass[8pt,aspectratio=169]{beamer}
\usetheme{Madrid}
\usepackage[utf8]{inputenc}
\usepackage{graphicx}
\usepackage{booktabs}
\usepackage{hyperref}

\newcommand{\bottomnote}[1]{\vfill\hfill{\scriptsize\textit{#1}}}

\title{L25: Digital Art and Collectibles}
\subtitle{Module C: NFTs \& Digital Assets}
\author{Blockchain \& Cryptocurrency Course}
\date{December 2025}

\begin{document}

\begin{frame}
\titlepage
\end{frame}

\begin{frame}{Learning Objectives}
By the end of this lesson, you will be able to:
\begin{itemize}
\item Distinguish between 1/1 art, editions, and generative art NFTs
\item Understand generative art platforms and on-chain algorithms (Art Blocks)
\item Analyze PFP collections and their cultural significance (BAYC, CryptoPunks)
\item Apply valuation frameworks to digital art NFTs
\item Evaluate the impact of high-profile sales (Beeple \$69M)
\end{itemize}
\end{frame}

\begin{frame}{NFT Art Categories}
\textbf{Three Primary Categories:}

\vspace{0.2cm}
\begin{enumerate}
\item \textbf{1/1 Digital Art} -- Unique, single-edition artwork (SuperRare, Foundation)
\item \textbf{Editions} -- Multiple identical copies (Manifold, Zora)
\item \textbf{Generative Art} -- Algorithm-created unique outputs (Art Blocks)
\end{enumerate}

\vspace{0.2cm}
\textbf{Key Differences:}
\begin{itemize}
\item Scarcity: 1/1 highest, editions moderate, generative varies
\item Price: 1/1 premium, editions accessible, generative mid-range
\item Collector participation: Generative highest (mint reveals)
\end{itemize}
\end{frame}

\begin{frame}{NFT Art Categories Comparison}
\begin{center}
\includegraphics[width=0.72\textwidth]{charts/01_nft_art_categories/chart.pdf}
\end{center}
\bottomnote{1/1 art commands highest prices; generative art offers collector participation}
\end{frame}

\begin{frame}{1/1 Digital Art: Unique Creations}
\textbf{Characteristics:}
\begin{itemize}
\item Single NFT per artwork (no duplicates)
\item Artist-signed and authenticated on-chain
\item Direct artist-to-collector relationship
\end{itemize}

\vspace{0.2cm}
\textbf{Leading 1/1 Platforms:}
\begin{itemize}
\item \textbf{SuperRare:} Curated marketplace, invitation-only artists
\item \textbf{Foundation:} Auction-based, community-curated
\item \textbf{Nifty Gateway:} Drops from celebrity artists (Beeple, Pak)
\end{itemize}

\vspace{0.2cm}
\textbf{Valuation Factors:}
\begin{itemize}
\item Artist reputation and exhibition history
\item Aesthetic quality and cultural relevance
\item Provenance (ownership history)
\end{itemize}
\end{frame}

\begin{frame}{Case Study: Beeple's ``Everydays'' (\$69M)}
\textbf{Event:} March 2021, Christie's auction

\vspace{0.2cm}
\textbf{Artwork Details:}
\begin{itemize}
\item Title: ``Everydays: The First 5000 Days''
\item Artist: Beeple (Mike Winkelmann)
\item Format: Collage of 5,000 daily digital artworks (2007-2021)
\item Sale price: \$69.3 million (42,329 ETH at time)
\end{itemize}

\vspace{0.2cm}
\textbf{Significance:}
\begin{itemize}
\item Third-highest price for living artist (after Koons, Hockney)
\item Legitimized NFTs in traditional art world
\item Catalyzed NFT bull market (2021 peak)
\end{itemize}
\end{frame}

\begin{frame}{High-Profile NFT Art Sales}
\begin{center}
\includegraphics[width=0.72\textwidth]{charts/02_high_profile_sales/chart.pdf}
\end{center}
\bottomnote{Beeple's \$69M sale remains the highest NFT art price; CryptoPunks dominate PFP sales}
\end{frame}

\begin{frame}{Generative Art: Code as Artist}
\textbf{Definition:} Artwork generated by algorithmic code with controlled randomness

\vspace{0.2cm}
\textbf{How It Works:}
\begin{enumerate}
\item Artist writes generative algorithm (JavaScript, p5.js)
\item Algorithm uses random seed (from transaction hash)
\item Each NFT mint produces unique output from same code
\end{enumerate}

\vspace{0.2cm}
\textbf{Appeal:}
\begin{itemize}
\item Each piece is unique (seed-based randomness)
\item Collector participates in creation (mint reveals outcome)
\item On-chain code storage (true permanence)
\item Lower price than 1/1s (algorithmic scalability)
\end{itemize}
\end{frame}

\begin{frame}{Art Blocks: Premier Generative Platform}
\textbf{Founded:} 2020 by Erick Calderon (Snowfro)

\vspace{0.2cm}
\textbf{How Art Blocks Works:}
\begin{enumerate}
\item Artist submits generative algorithm (JavaScript)
\item Collector mints NFT (pays ETH + gas)
\item Transaction hash seeds random number generator
\item Algorithm runs, generates unique output
\end{enumerate}

\vspace{0.2cm}
\textbf{Project Tiers:}
\begin{itemize}
\item \textbf{Curated:} Highly selective, premium (Fidenza, Ringers)
\item \textbf{Playground:} Emerging artists, experimental
\item \textbf{Factory:} Open submissions, lower curation bar
\end{itemize}
\end{frame}

\begin{frame}{Iconic Generative Projects}
\textbf{Fidenza by Tyler Hobbs (2021):}
\begin{itemize}
\item 999 unique outputs, flow field algorithm
\item Mint: 0.17 ETH, Floor peak: 140 ETH (\$500k+)
\end{itemize}

\vspace{0.2cm}
\textbf{Ringers by Dmitri Cherniak (2021):}
\begin{itemize}
\item 1,000 outputs, wrapped strings around pegs
\item Mint: 0 ETH (free + gas), Floor peak: 25 ETH
\end{itemize}

\vspace{0.2cm}
\textbf{Chromie Squiggle (2020):}
\begin{itemize}
\item Art Blocks' first project (Snowfro)
\item 10,000 outputs, ``genesis'' Art Blocks NFT
\end{itemize}

\vspace{0.2cm}
\textbf{Rarity:} Certain outputs extremely rare (algorithmic traits)
\end{frame}

\begin{frame}{PFP Collections: Profile Pictures as Identity}
\textbf{PFP:} Profile Picture NFTs, avatar-style collectibles

\vspace{0.2cm}
\textbf{Characteristics:}
\begin{itemize}
\item 10,000-item collections (standard size)
\item Trait-based variation (hat, eyes, background)
\item Social signaling (Twitter/Discord avatars)
\item Community membership and status
\end{itemize}

\vspace{0.2cm}
\textbf{Leading PFP Collections:}
\begin{itemize}
\item \textbf{CryptoPunks (2017):} 10,000 pixel art characters
\item \textbf{Bored Ape Yacht Club (2021):} 10,000 apes, Yuga Labs
\item \textbf{Azuki (2022):} 10,000 anime-style characters
\end{itemize}

\vspace{0.2cm}
\textbf{Value Drivers:} Brand recognition, community, utility, celebrities
\end{frame}

\begin{frame}{Blue-Chip PFP Collection Comparison}
\begin{center}
\includegraphics[width=0.72\textwidth]{charts/03_pfp_collection_metrics/chart.pdf}
\end{center}
\bottomnote{BAYC leads in utility and community; CryptoPunks maintain highest floor prices}
\end{frame}

\begin{frame}{CryptoPunks: The OG NFT Collection}
\textbf{Created:} June 2017 by Larva Labs

\vspace{0.2cm}
\textbf{Technical Details:}
\begin{itemize}
\item 10,000 unique 24x24 pixel art characters
\item Originally free to claim (gas only)
\item Pre-dates ERC-721 standard (custom contract)
\end{itemize}

\vspace{0.2cm}
\textbf{Rarity Breakdown:}
\begin{itemize}
\item 6,039 Male | 3,840 Female | 88 Zombie
\item 24 Ape | 9 Alien (rarest, >10,000 ETH peak)
\end{itemize}

\vspace{0.2cm}
\textbf{Cultural Impact:} Inspired entire PFP category, held by Jay-Z, Snoop Dogg
\end{frame}

\begin{frame}{Bored Ape Yacht Club: Community and Utility}
\textbf{Created:} April 2021 by Yuga Labs

\vspace{0.2cm}
\textbf{Key Innovations:}
\begin{itemize}
\item \textbf{IP rights:} Holders own commercial rights to their ape
\item \textbf{Community events:} Exclusive parties, metaverse experiences
\item \textbf{Airdrops:} Free Mutant Apes, ApeCoin tokens
\end{itemize}

\vspace{0.2cm}
\textbf{Valuation Milestones:}
\begin{itemize}
\item Mint price: 0.08 ETH (April 2021)
\item Floor peak: 153 ETH (\$500k+, April 2022)
\item Current floor (2024): 20-30 ETH range
\end{itemize}

\vspace{0.2cm}
\textbf{Celebrity Holders:} Eminem, Stephen Curry, Paris Hilton
\end{frame}

\begin{frame}{NFT Art Market Trends}
\begin{center}
\includegraphics[width=0.72\textwidth]{charts/04_art_market_trends/chart.pdf}
\end{center}
\bottomnote{2021 peak driven by Beeple sale and mainstream adoption; market stabilizing post-correction}
\end{frame}

\begin{frame}{The Social Capital Thesis}
\textbf{Argument:} NFTs represent digital social capital and identity

\vspace{0.2cm}
\textbf{Mechanisms:}
\begin{itemize}
\item \textbf{Signaling:} PFP shows wealth, taste, community affiliation
\item \textbf{Access:} Token-gated Discord channels, events, alpha groups
\item \textbf{Status:} Owning rare/expensive NFTs confers prestige
\item \textbf{Network effects:} Value increases with community size
\end{itemize}

\vspace{0.2cm}
\textbf{Comparison to Luxury Goods:}
\begin{itemize}
\item Rolex watch: \$10k+ for timekeeping (social signal)
\item Bored Ape: \$100k+ for pixel art (social signal)
\item Both: Functional value << social signaling value
\end{itemize}
\end{frame}

\begin{frame}{NFT Art Category Distribution}
\begin{center}
\includegraphics[width=0.72\textwidth]{charts/05_category_distribution/chart.pdf}
\end{center}
\bottomnote{PFP collections dominate volume; generative art growing among collectors}
\end{frame}

\begin{frame}{Valuation Framework for Digital Art}
\textbf{Key Criteria:}
\begin{enumerate}
\item \textbf{Artist reputation:} Exhibition history, awards, following
\item \textbf{Scarcity:} 1/1 vs. edition size
\item \textbf{Provenance:} Ownership history (celebrity collectors)
\item \textbf{Cultural significance:} Historical impact, media coverage
\item \textbf{Technical quality:} Metadata storage (IPFS)
\item \textbf{Market comparables:} Similar artist sales
\end{enumerate}

\vspace{0.2cm}
\textbf{Example Valuation:}
\begin{itemize}
\item Established artist, 1/1 artwork, IPFS storage
\item Recent comparable: 5 ETH $\rightarrow$ Range: 4-6 ETH
\end{itemize}
\end{frame}

\begin{frame}{Blue-Chip NFT Collections}
\textbf{Blue-Chip:} Established collections with sustained demand and liquidity

\vspace{0.2cm}
\textbf{Criteria:}
\begin{itemize}
\item 2+ years of trading history
\item Consistent floor price support (bear market resilience)
\item Strong community and holder base
\item Cultural significance and brand recognition
\end{itemize}

\vspace{0.2cm}
\textbf{Current Blue-Chips (2024):}
\begin{itemize}
\item CryptoPunks (50+ ETH floor)
\item Bored Ape Yacht Club (20-30 ETH floor)
\item Autoglyphs (on-chain generative, 100+ ETH floor)
\item Art Blocks Curated projects (Fidenza, Ringers)
\end{itemize}
\end{frame}

\begin{frame}{Key Takeaways}
\begin{enumerate}
\item NFT art includes 1/1 unique pieces, limited editions, and generative algorithmic art
\item Generative platforms (Art Blocks) create unique outputs from code using randomness
\item PFP collections (BAYC, CryptoPunks) derive value from community and social capital
\item Beeple's \$69M sale catalyzed mainstream NFT adoption but raised speculation concerns
\item Valuation combines traditional art criteria with on-chain metrics and community factors
\item Blue-chip NFTs show resilience through market cycles due to cultural significance
\end{enumerate}
\end{frame}

\begin{frame}{Discussion Questions}
\begin{enumerate}
\item Is digital art fundamentally different from physical art in terms of value and ownership?
\item What distinguishes a blue-chip NFT collection from a speculative project?
\item How should generative art be valued compared to artist-created 1/1 pieces?
\item Will PFP collections retain value long-term, or are they a temporary trend?
\item Should NFT marketplaces curate art quality, or remain open platforms?
\end{enumerate}
\end{frame}

\begin{frame}{Next Lesson Preview}
\textbf{L26: Gaming NFTs and Metaverse}

\vspace{0.3cm}
We will explore:
\begin{itemize}
\item Play-to-earn gaming model and tokenomics
\item Case study: Axie Infinity rise and collapse
\item Virtual land NFTs (Decentraland, The Sandbox)
\item Interoperability and cross-game asset portability
\item Sustainability challenges of blockchain gaming
\end{itemize}

\vspace{0.3cm}
\textbf{Preparation:} Explore Decentraland or The Sandbox metaverse platforms
\end{frame}

\end{document}
