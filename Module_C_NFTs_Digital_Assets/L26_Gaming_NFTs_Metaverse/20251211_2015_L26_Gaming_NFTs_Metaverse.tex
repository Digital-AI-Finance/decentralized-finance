\documentclass[8pt,aspectratio=169]{beamer}
\usetheme{Madrid}
\usepackage[utf8]{inputenc}
\usepackage{graphicx}
\usepackage{booktabs}
\usepackage{hyperref}

\newcommand{\bottomnote}[1]{\vfill\hfill{\scriptsize\textit{#1}}}

\title{L26: Gaming NFTs and Metaverse}
\subtitle{Module C: NFTs \& Digital Assets}
\author{Blockchain \& Cryptocurrency Course}
\date{December 2025}

\begin{document}

\begin{frame}
\titlepage
\end{frame}

\begin{frame}{Learning Objectives}
By the end of this lesson, you will be able to:
\begin{itemize}
\item Understand the play-to-earn gaming model and tokenomics
\item Analyze Axie Infinity's rise and collapse as a case study
\item Evaluate virtual land NFTs in metaverse platforms
\item Assess interoperability challenges for cross-game assets
\item Identify sustainability issues in blockchain gaming economies
\end{itemize}
\end{frame}

\begin{frame}{Blockchain Gaming: Core Concepts}
\textbf{Traditional Gaming:}
\begin{itemize}
\item Players pay for games and in-game items
\item Items locked to game (no true ownership)
\item Value accrues to game company, not players
\end{itemize}

\vspace{0.2cm}
\textbf{Blockchain Gaming:}
\begin{itemize}
\item In-game assets are NFTs (player-owned)
\item Items tradable on open marketplaces
\item Players can earn cryptocurrency while playing
\item Potential for cross-game asset portability
\end{itemize}

\vspace{0.2cm}
\textbf{Promise:} Players as stakeholders, value extraction to players
\end{frame}

\begin{frame}{Play-to-Earn (P2E) Model}
\textbf{Concept:} Players earn tokens/NFTs through gameplay

\vspace{0.2cm}
\textbf{Mechanics:}
\begin{enumerate}
\item Player acquires NFT characters/assets (upfront investment)
\item Player completes in-game tasks (battles, quests)
\item Game rewards player with tokens (cryptocurrency)
\item Player sells tokens on exchange
\end{enumerate}

\vspace{0.2cm}
\textbf{Economic Model:}
\begin{itemize}
\item New players buy NFTs (inflow of capital)
\item Existing players earn and sell tokens (outflow)
\item Sustainability requires continuous new player demand
\end{itemize}
\end{frame}

\begin{frame}{Play-to-Earn Economic Flow}
\begin{center}
\includegraphics[width=0.72\textwidth]{charts/01_p2e_economics/chart.pdf}
\end{center}
\bottomnote{P2E models resemble Ponzi dynamics: new player capital pays existing players}
\end{frame}

\begin{frame}{Axie Infinity: The P2E Pioneer}
\textbf{Game Overview:}
\begin{itemize}
\item Creature-battling game (similar to Pokemon)
\item Players collect, breed, and battle Axies (NFT creatures)
\item Launched: 2018, exploded in popularity 2021
\end{itemize}

\vspace{0.2cm}
\textbf{Tokenomics:}
\begin{itemize}
\item \textbf{Axie NFTs:} Characters needed to play (3 required)
\item \textbf{AXS:} Governance and staking token
\item \textbf{SLP (Smooth Love Potion):} In-game reward token
\end{itemize}

\vspace{0.2cm}
\textbf{Peak Stats (2021):}
\begin{itemize}
\item 2.8 million daily active users
\item Entry cost: \$600-1000 (3 Axies)
\end{itemize}
\end{frame}

\begin{frame}{Axie Infinity: Rise and Fall}
\begin{center}
\includegraphics[width=0.72\textwidth]{charts/02_axie_rise_fall/chart.pdf}
\end{center}
\bottomnote{SLP token collapsed 99\%+; users dropped from 2.8M to <10K}
\end{frame}

\begin{frame}{Axie Infinity: What Went Wrong?}
\textbf{Fundamental Flaws:}
\begin{enumerate}
\item \textbf{Ponzi Dynamics:} Relied on new player money to pay existing players
\item \textbf{Unlimited Inflation:} SLP had no supply cap (infinite minting)
\item \textbf{Weak Demand Sinks:} Breeding not enough to absorb SLP supply
\item \textbf{High Entry Barrier:} \$600-1000 deterred new players
\item \textbf{Gameplay Quality:} Repetitive, grind-focused (not fun)
\end{enumerate}

\vspace{0.2cm}
\textbf{Lesson:} P2E model requires genuine value creation, not just token redistribution
\end{frame}

\begin{frame}{Ronin Bridge Hack (March 2022)}
\textbf{Incident:} Largest DeFi hack in history

\vspace{0.2cm}
\textbf{Details:}
\begin{itemize}
\item \textbf{Target:} Ronin Network (Ethereum sidechain for Axie)
\item \textbf{Amount stolen:} 173,600 ETH + 25.5M USDC = \$625M
\item \textbf{Attack vector:} Social engineering of validators
\item \textbf{Discovery:} Hack unnoticed for 6 days
\end{itemize}

\vspace{0.2cm}
\textbf{Impact:}
\begin{itemize}
\item Player confidence shattered
\item Accelerated player exodus
\end{itemize}

\vspace{0.2cm}
\textbf{Lesson:} Centralized bridges are critical vulnerabilities
\end{frame}

\begin{frame}{Virtual Land NFTs: Metaverse Real Estate}
\textbf{Concept:} Owning digital land parcels as NFTs in virtual worlds

\vspace{0.2cm}
\textbf{Major Metaverse Platforms:}
\begin{itemize}
\item \textbf{Decentraland:} Ethereum-based, 90,000 parcels
\item \textbf{The Sandbox:} Polygon-based, 166,464 parcels
\item \textbf{Otherside (Yuga Labs):} BAYC metaverse, 100,000 parcels
\end{itemize}

\vspace{0.2cm}
\textbf{Value Proposition:}
\begin{itemize}
\item Build experiences (games, galleries, events)
\item Monetize through rentals or advertising
\item Speculate on location value
\end{itemize}

\vspace{0.2cm}
\textbf{Reality:} Low usage, most land undeveloped, speculation-driven
\end{frame}

\begin{frame}{Metaverse Platform Comparison}
\begin{center}
\includegraphics[width=0.72\textwidth]{charts/03_metaverse_land_comparison/chart.pdf}
\end{center}
\bottomnote{All platforms struggle with low daily active users; brand partnerships exceed organic activity}
\end{frame}

\begin{frame}{Decentraland: The OG Metaverse}
\textbf{Launched:} 2020, Ethereum-based

\vspace{0.2cm}
\textbf{Structure:}
\begin{itemize}
\item 90,000 LAND parcels (16x16 meter plots)
\item MANA token: Currency for buying land
\item DAO governance: Landowners vote on changes
\end{itemize}

\vspace{0.2cm}
\textbf{Peak Hype (2021-2022):}
\begin{itemize}
\item Prime land sold for \$2.4M
\item Brands: Samsung, Adidas, Sotheby's
\end{itemize}

\vspace{0.2cm}
\textbf{Current State (2024):}
\begin{itemize}
\item Daily active users: <1,000
\item Floor price: \$500-1,000 per parcel (down 90\%+)
\end{itemize}
\end{frame}

\begin{frame}{Interoperability: The Cross-Game Asset Dream}
\textbf{Vision:} NFT items usable across multiple games

\vspace{0.2cm}
\textbf{Example Scenarios:}
\begin{itemize}
\item Sword earned in Game A usable in Game B
\item Avatar skin portable across virtual worlds
\item Virtual real estate accessible from multiple platforms
\end{itemize}

\vspace{0.2cm}
\textbf{Technical Challenges:}
\begin{enumerate}
\item 3D model compatibility (different engines, formats)
\item Game balance (overpowered items)
\item Art style coherence
\item Legal/IP issues
\end{enumerate}

\vspace{0.2cm}
\textbf{Reality:} Very limited interoperability today
\end{frame}

\begin{frame}{Interoperability Barriers}
\begin{center}
\includegraphics[width=0.72\textwidth]{charts/04_interoperability_barriers/chart.pdf}
\end{center}
\bottomnote{Economic and technical barriers make true cross-game asset portability extremely difficult}
\end{frame}

\begin{frame}{Sustainability of Blockchain Gaming}
\textbf{Concerns:}
\begin{enumerate}
\item \textbf{Ponzi Tokenomics:} Most P2E models collapse without new players
\item \textbf{Gameplay Quality:} Focus on earning, not fun
\item \textbf{High Costs:} Gas fees, NFT entry barriers
\item \textbf{Regulatory Risk:} P2E may be classified as gambling
\end{enumerate}

\vspace{0.2cm}
\textbf{Potential Solutions:}
\begin{itemize}
\item Shift to ``play-and-earn'' (fun first, earn secondary)
\item Deflationary tokenomics (burning mechanisms)
\item Layer 2 solutions (lower gas fees)
\item Free-to-play with optional NFT purchases
\end{itemize}
\end{frame}

\begin{frame}{Gaming Model Comparison}
\begin{center}
\includegraphics[width=0.55\textwidth]{charts/05_gaming_sustainability/chart.pdf}
\end{center}
\bottomnote{Play-and-Earn models score better on sustainability factors than traditional P2E}
\end{frame}

\begin{frame}{Play-and-Earn: The New Paradigm}
\textbf{Play-and-Earn:} Prioritize gameplay quality, earnings as bonus

\vspace{0.2cm}
\textbf{Contrast with P2E:}
\begin{itemize}
\item \textbf{P2E:} Gameplay is work, earnings primary motivation
\item \textbf{Play-and-Earn:} Gameplay is fun, earnings enhance experience
\end{itemize}

\vspace{0.2cm}
\textbf{Examples:}
\begin{itemize}
\item \textbf{Illuvium:} AAA-quality open-world RPG with NFT creatures
\item \textbf{Ember Sword:} Free-to-play MMORPG with optional cosmetics
\item \textbf{Gods Unchained:} Free-to-play card game, sustainable
\end{itemize}

\vspace{0.2cm}
\textbf{Key Insight:} Fun gameplay attracts organic users, reducing Ponzi dynamics
\end{frame}

\begin{frame}{Blockchain Gaming Guilds}
\textbf{Guilds:} Organizations that lend NFT assets to players

\vspace{0.2cm}
\textbf{Scholarship Model:}
\begin{enumerate}
\item Guild purchases NFT game assets
\item Guild lends assets to players (scholars)
\item Players earn tokens through gameplay
\item Earnings split: 70\% player, 30\% guild
\end{enumerate}

\vspace{0.2cm}
\textbf{Major Guilds:}
\begin{itemize}
\item \textbf{Yield Guild Games (YGG):} Largest guild
\item \textbf{Merit Circle:} DAO-governed
\end{itemize}

\vspace{0.2cm}
\textbf{Decline:} Axie collapse decimated guild revenues
\end{frame}

\begin{frame}{Key Takeaways}
\begin{enumerate}
\item Play-to-earn (P2E) models enable earning through gameplay but often resemble Ponzi schemes
\item Axie Infinity collapsed due to unsustainable tokenomics and unlimited SLP inflation
\item Virtual land NFTs (Decentraland, Sandbox) are speculation-driven with low actual usage
\item Interoperability of cross-game assets is limited by technical and economic barriers
\item Sustainable blockchain gaming requires fun-first design and balanced token economies
\item Play-and-Earn (fun primary, earning secondary) shows more promise
\end{enumerate}
\end{frame}

\begin{frame}{Discussion Questions}
\begin{enumerate}
\item Can play-to-earn gaming models ever be sustainable without Ponzi dynamics?
\item What would make virtual land NFTs genuinely valuable beyond speculation?
\item Is true cross-game interoperability achievable, or a pipe dream?
\item Should blockchain games focus on financialization or gameplay quality?
\item How can regulators balance innovation with consumer protection in P2E gaming?
\end{enumerate}
\end{frame}

\begin{frame}{Next Lesson Preview}
\textbf{L27: Real-World Asset Tokenization}

\vspace{0.3cm}
We will explore:
\begin{itemize}
\item RWA tokenization concept and mechanics
\item Real estate tokenization platforms and legal frameworks
\item Securities tokenization and compliance (Reg D, Reg S)
\item Case study: BlackRock BUIDL fund (\$500M+ on-chain)
\item Market size: \$50B on-chain, projected \$18T by 2033
\end{itemize}

\vspace{0.3cm}
\textbf{Preparation:} Review traditional real estate investment structures (REITs)
\end{frame}

\end{document}
