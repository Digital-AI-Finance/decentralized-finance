\documentclass[8pt,aspectratio=169]{beamer}
\usetheme{Madrid}
\usepackage[utf8]{inputenc}
\usepackage{graphicx}
\usepackage{booktabs}
\usepackage{hyperref}

\title{L26: Gaming NFTs and Metaverse}
\subtitle{Module C: NFTs \& Digital Assets}
\author{Blockchain \& Cryptocurrency Course}
\date{December 2025}

\begin{document}

\begin{frame}
\titlepage
\end{frame}

\begin{frame}{Learning Objectives}
By the end of this lesson, you will be able to:
\begin{itemize}
\item Understand the play-to-earn gaming model and tokenomics
\item Analyze Axie Infinity's rise and collapse as a case study
\item Evaluate virtual land NFTs in metaverse platforms
\item Assess interoperability challenges for cross-game assets
\item Identify sustainability issues in blockchain gaming economies
\end{itemize}
\end{frame}

\begin{frame}{Blockchain Gaming: Core Concepts}
\textbf{Traditional Gaming:}
\begin{itemize}
\item Players pay for games and in-game items
\item Items locked to game (no true ownership)
\item No ability to trade or sell assets outside platform
\item Value accrues to game company, not players
\end{itemize}

\vspace{0.3cm}
\textbf{Blockchain Gaming:}
\begin{itemize}
\item In-game assets are NFTs (player-owned)
\item Items tradable on open marketplaces
\item Players can earn cryptocurrency while playing
\item Potential for cross-game asset portability
\end{itemize}

\vspace{0.3cm}
\textbf{Promise:} Players as stakeholders, value extraction to players
\end{frame}

\begin{frame}{Play-to-Earn (P2E) Model}
\textbf{Concept:} Players earn tokens/NFTs through gameplay

\vspace{0.3cm}
\textbf{Mechanics:}
\begin{enumerate}
\item Player acquires NFT characters/assets (upfront investment)
\item Player completes in-game tasks (battles, quests, farming)
\item Game rewards player with tokens (fungible cryptocurrency)
\item Player sells tokens on exchange or buys more NFTs
\end{enumerate}

\vspace{0.3cm}
\textbf{Economic Model:}
\begin{itemize}
\item New players buy NFTs (inflow of capital)
\item Existing players earn and sell tokens (outflow)
\item Sustainability requires continuous new player demand
\item Resembles Ponzi dynamics if not backed by real value
\end{itemize}
\end{frame}

\begin{frame}{Axie Infinity: The P2E Pioneer}
\textbf{Game Overview:}
\begin{itemize}
\item Creature-battling game (similar to Pokemon)
\item Players collect, breed, and battle Axies (NFT creatures)
\item Developed by Sky Mavis (Vietnam-based studio)
\item Launched: 2018, exploded in popularity 2021
\end{itemize}

\vspace{0.3cm}
\textbf{Tokenomics:}
\begin{itemize}
\item \textbf{Axie NFTs:} Characters needed to play (3 required)
\item \textbf{AXS (Axie Infinity Shards):} Governance and staking token
\item \textbf{SLP (Smooth Love Potion):} In-game reward token, breeding cost
\end{itemize}

\vspace{0.3cm}
\textbf{Peak Stats (2021):}
\begin{itemize}
\item 2.8 million daily active users
\item \$4B+ total trading volume
\item Entry cost: \$600-1000 (3 Axies)
\end{itemize}
\end{frame}

\begin{frame}{Axie Infinity: Rise (2020-2021)}
\textbf{Growth Drivers:}

\vspace{0.3cm}
\begin{enumerate}
\item \textbf{COVID-19 Pandemic:} Unemployment in developing countries (Philippines)
\item \textbf{Scholarship Programs:} Guilds lend Axies to players (split earnings)
\item \textbf{High SLP Earnings:} Players earned \$10-40/day (above local wages)
\item \textbf{Viral Growth:} Word-of-mouth in Southeast Asia
\item \textbf{Investor Hype:} Andreessen Horowitz (a16z) invested \$152M (2021)
\end{enumerate}

\vspace{0.3cm}
\textbf{Social Impact:}
\begin{itemize}
\item Filipinos quit jobs to play Axie full-time
\item Media coverage: ``The future of work''
\item Axie became top NFT game by volume
\end{itemize}
\end{frame}

\begin{frame}{Axie Infinity: Collapse (2022)}
\textbf{Tokenomics Breakdown:}

\vspace{0.3cm}
\begin{itemize}
\item \textbf{SLP Inflation:} Unlimited minting from gameplay
\item \textbf{Demand Collapse:} Breeding declined (SLP use case)
\item \textbf{Price Death Spiral:} SLP price fell 99\% (peak \$0.40 to \$0.004)
\item \textbf{Player Exodus:} Earnings dropped below minimum wage
\end{itemize}

\vspace{0.3cm}
\textbf{Timeline:}
\begin{itemize}
\item July 2021: 2.8M DAU, SLP at \$0.30
\item November 2021: SLP drops to \$0.10 (sell pressure)
\item March 2022: Ronin bridge hack (\$625M stolen)
\item June 2022: 500k DAU, SLP at \$0.01
\item 2024: <10k DAU, SLP at \$0.003
\end{itemize}

\vspace{0.3cm}
\textbf{Lesson:} Unsustainable tokenomics without real value creation
\end{frame}

\begin{frame}{Axie Infinity Case Study: What Went Wrong?}
\textbf{Fundamental Flaws:}

\vspace{0.3cm}
\begin{enumerate}
\item \textbf{Ponzi Dynamics:} Relied on new player money to pay existing players
\item \textbf{Unlimited Inflation:} SLP had no supply cap (infinite minting)
\item \textbf{Weak Demand Sinks:} Breeding not enough to absorb SLP supply
\item \textbf{High Entry Barrier:} \$600-1000 initial cost deterred new players
\item \textbf{Gameplay Quality:} Repetitive, grind-focused (not fun)
\end{enumerate}

\vspace{0.3cm}
\textbf{Counterfactual:}
\begin{itemize}
\item If SLP had fixed supply or burning mechanisms, price could stabilize
\item If game had intrinsic fun (not just earnings), retention would improve
\item If entry cost was lower, new player flow would continue
\end{itemize}

\vspace{0.3cm}
\textbf{Conclusion:} P2E model requires genuine value creation, not just token redistribution
\end{frame}

\begin{frame}{Ronin Bridge Hack (March 2022)}
\textbf{Incident:} Largest DeFi hack in history

\vspace{0.3cm}
\textbf{Details:}
\begin{itemize}
\item \textbf{Target:} Ronin Network (Ethereum sidechain for Axie)
\item \textbf{Amount stolen:} 173,600 ETH + 25.5M USDC = \$625M
\item \textbf{Attack vector:} Social engineering of validators
\item \textbf{Discovery:} Hack unnoticed for 6 days
\end{itemize}

\vspace{0.3cm}
\textbf{Impact on Axie:}
\begin{itemize}
\item Player confidence shattered
\item Sky Mavis raised \$150M to reimburse victims
\item Accelerated player exodus (already declining from SLP collapse)
\end{itemize}

\vspace{0.3cm}
\textbf{Lesson:} Centralized bridges (Ronin had 5-of-9 validator model) are critical vulnerabilities
\end{frame}

\begin{frame}{Virtual Land NFTs: Metaverse Real Estate}
\textbf{Concept:} Owning digital land parcels as NFTs in virtual worlds

\vspace{0.3cm}
\textbf{Major Metaverse Platforms:}
\begin{itemize}
\item \textbf{Decentraland:} Ethereum-based, 90,000 parcels
\item \textbf{The Sandbox:} Polygon-based, 166,464 parcels
\item \textbf{Otherside (Yuga Labs):} BAYC metaverse, 100,000 parcels
\item \textbf{Somnium Space:} VR-focused metaverse
\end{itemize}

\vspace{0.3cm}
\textbf{Value Proposition:}
\begin{itemize}
\item Build experiences (games, galleries, events)
\item Monetize through rentals or advertising
\item Speculate on location value (virtual prime real estate)
\end{itemize}

\vspace{0.3cm}
\textbf{Reality:} Low usage, most land undeveloped, speculation-driven prices
\end{frame}

\begin{frame}{Decentraland: The OG Metaverse}
\textbf{Launched:} 2020, Ethereum-based

\vspace{0.3cm}
\textbf{Structure:}
\begin{itemize}
\item 90,000 LAND parcels (16x16 meter plots)
\item Parcel coordinates: (-150, -150) to (150, 150)
\item MANA token: Currency for buying land and assets
\item DAO governance: Landowners vote on platform changes
\end{itemize}

\vspace{0.3cm}
\textbf{Peak Hype (2021-2022):}
\begin{itemize}
\item Prime land sold for \$2.4M (near central plaza)
\item Brands bought land: Samsung, Adidas, Sotheby's
\item Virtual concerts and events (Decentraland Fashion Week)
\end{itemize}

\vspace{0.3cm}
\textbf{Current State (2024):}
\begin{itemize}
\item Daily active users: <1,000 (low engagement)
\item Floor price: \$500-1,000 per parcel (down 90\%+ from peak)
\end{itemize}
\end{frame}

\begin{frame}{The Sandbox: User-Generated Metaverse}
\textbf{Model:} Minecraft-style voxel world with NFT land

\vspace{0.3cm}
\textbf{Key Features:}
\begin{itemize}
\item \textbf{VoxEdit:} Tool to create 3D voxel NFT assets
\item \textbf{Game Maker:} No-code tool to build games on land
\item \textbf{SAND token:} Platform currency
\item \textbf{166,464 LAND parcels:} Fixed supply
\end{itemize}

\vspace{0.3cm}
\textbf{Partnerships:}
\begin{itemize}
\item Snoop Dogg's virtual mansion (land sold for \$450k next to it)
\item Atari, The Walking Dead, Smurfs (branded experiences)
\end{itemize}

\vspace{0.3cm}
\textbf{Challenges:}
\begin{itemize}
\item Most experiences low-quality (limited Game Maker capabilities)
\item Low user retention (few return after initial visit)
\end{itemize}
\end{frame}

\begin{frame}{Otherside: Yuga Labs' Metaverse}
\textbf{Launched:} 2022 by Yuga Labs (creators of BAYC)

\vspace{0.3cm}
\textbf{Details:}
\begin{itemize}
\item 100,000 Otherdeed NFTs (land parcels)
\item Mint price: 305 APE (~\$5,800 at time)
\item Total mint revenue: \$560M (largest NFT land sale)
\item Powered by Improbable's M2 technology (scalable multiplayer)
\end{itemize}

\vspace{0.3cm}
\textbf{Unique Features:}
\begin{itemize}
\item Each parcel has unique terrain and resource attributes
\item Integration with BAYC, MAYC, and other Yuga IPs
\item Focus on gaming experiences (not just social hangouts)
\end{itemize}

\vspace{0.3cm}
\textbf{Status:} In development, limited public access (2024)
\end{frame}

\begin{frame}{Virtual Land Valuation}
\textbf{What Determines Land Value?}

\vspace{0.3cm}
\textbf{Traditional Real Estate Parallels:}
\begin{itemize}
\item \textbf{Location:} Near central plazas, popular areas
\item \textbf{Neighbors:} Next to celebrity or brand land
\item \textbf{Size:} Larger estates (combined parcels)
\item \textbf{Development:} Built experiences vs. empty land
\end{itemize}

\vspace{0.3cm}
\textbf{Metaverse-Specific Factors:}
\begin{itemize}
\item Platform active users (more users = more valuable)
\item Scarcity (fixed supply of parcels)
\item Utility (can you monetize the land?)
\end{itemize}

\vspace{0.3cm}
\textbf{Current Reality:}
\begin{itemize}
\item Low user counts undermine location value
\item Speculation (not utility) drives most sales
\end{itemize}
\end{frame}

\begin{frame}{Interoperability: The Cross-Game Asset Dream}
\textbf{Vision:} NFT items usable across multiple games

\vspace{0.3cm}
\textbf{Example Scenarios:}
\begin{itemize}
\item Sword earned in Game A usable in Game B
\item Avatar skin portable across virtual worlds
\item Virtual real estate accessible from multiple platforms
\end{itemize}

\vspace{0.3cm}
\textbf{Technical Challenges:}
\begin{enumerate}
\item \textbf{3D model compatibility:} Different engines, formats, rigging
\item \textbf{Game balance:} Overpowered items from another game
\item \textbf{Art style coherence:} Realistic item in cartoonish game
\item \textbf{Legal/IP issues:} Who owns rights to item appearance?
\end{enumerate}

\vspace{0.3cm}
\textbf{Reality:} Very limited interoperability, mostly within same developer ecosystem
\end{frame}

\begin{frame}{Why Interoperability is Hard}
\textbf{Technical Barriers:}
\begin{itemize}
\item Game engines differ (Unity, Unreal, custom)
\item Asset formats incompatible (FBX, OBJ, proprietary)
\item Physics and animation systems vary
\item Performance constraints (mobile vs. PC)
\end{itemize}

\vspace{0.3cm}
\textbf{Economic Barriers:}
\begin{itemize}
\item Game developers lose control over item creation
\item Revenue cannibalization (items bought elsewhere)
\item Balance disruption from external assets
\end{itemize}

\vspace{0.3cm}
\textbf{Limited Implementations:}
\begin{itemize}
\item Same-developer games (e.g., Yuga Labs ecosystem)
\item Simple metadata interoperability (names, IDs, not full 3D assets)
\end{itemize}
\end{frame}

\begin{frame}{Sustainability of Blockchain Gaming}
\textbf{Concerns:}

\vspace{0.3cm}
\begin{enumerate}
\item \textbf{Ponzi Tokenomics:} Most P2E models collapse without new player inflow
\item \textbf{Gameplay Quality:} Focus on earning, not fun (grind-heavy)
\item \textbf{High Costs:} Gas fees, NFT entry barriers deter casual players
\item \textbf{Regulatory Risk:} P2E games may be classified as gambling
\item \textbf{Environmental Impact:} Blockchain energy consumption (mitigated by PoS)
\end{enumerate}

\vspace{0.3cm}
\textbf{Potential Solutions:}
\begin{itemize}
\item Shift to ``play-and-earn'' (fun first, earn secondary)
\item Deflationary tokenomics (burning mechanisms)
\item Layer 2 solutions (lower gas fees)
\item Free-to-play with optional NFT purchases
\end{itemize}
\end{frame}

\begin{frame}{Play-and-Earn: The New Paradigm}
\textbf{Play-and-Earn:} Prioritize gameplay quality, earnings as bonus

\vspace{0.3cm}
\textbf{Contrast with P2E:}
\begin{itemize}
\item \textbf{P2E:} Gameplay is work, earnings primary motivation
\item \textbf{Play-and-Earn:} Gameplay is fun, earnings enhance experience
\end{itemize}

\vspace{0.3cm}
\textbf{Examples:}
\begin{itemize}
\item \textbf{Illuvium:} AAA-quality open-world RPG with NFT creatures
\item \textbf{Ember Sword:} Free-to-play MMORPG with optional NFT cosmetics
\item \textbf{Guild of Guardians:} Mobile dungeon crawler with NFT heroes
\end{itemize}

\vspace{0.3cm}
\textbf{Key Insight:} Fun gameplay attracts organic users, reducing Ponzi dynamics
\end{frame}

\begin{frame}{Free-to-Play with NFT Cosmetics}
\textbf{Model:} Game free to play, NFTs are optional cosmetics

\vspace{0.3cm}
\textbf{Advantages:}
\begin{itemize}
\item No entry barrier (wide audience)
\item NFTs provide status, not gameplay advantage
\item Revenue from voluntary purchases (sustainable)
\item Avoids pay-to-win criticism
\end{itemize}

\vspace{0.3cm}
\textbf{Challenges:}
\begin{itemize}
\item Lower NFT demand (cosmetic-only)
\item Must compete with traditional F2P games
\item NFT utility limited (resale value main draw)
\end{itemize}

\vspace{0.3cm}
\textbf{Example:} Fortnite skins as NFTs (hypothetical model)
\end{frame}

\begin{frame}{Blockchain Gaming Guilds}
\textbf{Guilds:} Organizations that lend NFT assets to players

\vspace{0.3cm}
\textbf{Scholarship Model:}
\begin{enumerate}
\item Guild purchases NFT game assets (e.g., Axies)
\item Guild lends assets to players (scholars) for free
\item Players earn tokens through gameplay
\item Earnings split: 70\% player, 30\% guild (typical)
\end{enumerate}

\vspace{0.3cm}
\textbf{Major Guilds:}
\begin{itemize}
\item \textbf{Yield Guild Games (YGG):} Largest guild, \$1B+ assets at peak
\item \textbf{Merit Circle:} DAO-governed gaming guild
\item \textbf{Avocado Guild:} Latin America-focused
\end{itemize}

\vspace{0.3cm}
\textbf{Decline:} Axie collapse decimated guild revenues and valuations
\end{frame}

\begin{frame}{Regulatory Challenges}
\textbf{Key Concerns:}

\vspace{0.3cm}
\begin{enumerate}
\item \textbf{Gambling Classification:} P2E games may violate gambling laws
\item \textbf{Securities Regulation:} In-game tokens may be unregistered securities
\item \textbf{Tax Implications:} Unrealized gains on NFT assets, earned token income
\item \textbf{Consumer Protection:} Minors playing P2E, predatory mechanics
\end{enumerate}

\vspace{0.3cm}
\textbf{Regional Approaches:}
\begin{itemize}
\item \textbf{Philippines:} Initially embraced, later scrutinized for tax evasion
\item \textbf{China:} Banned NFT gaming (part of broader crypto ban)
\item \textbf{EU/US:} Case-by-case analysis, no clear framework yet
\end{itemize}

\vspace{0.3cm}
\textbf{Industry Response:} Shift to ``play-and-earn'' to reduce gambling perception
\end{frame}

\begin{frame}{Success Factors for Blockchain Games}
\textbf{What Makes a Sustainable Blockchain Game?}

\vspace{0.3cm}
\begin{enumerate}
\item \textbf{Fun gameplay:} Intrinsically enjoyable, not just earnings
\item \textbf{Sustainable tokenomics:} Balanced supply/demand, deflationary sinks
\item \textbf{Low entry barrier:} Free-to-play or affordable NFTs
\item \textbf{True ownership value:} NFTs provide utility beyond speculation
\item \textbf{Scalability:} Layer 2 or alt-chain (low gas fees)
\item \textbf{Community engagement:} Active players, not just farmers
\end{enumerate}

\vspace{0.3cm}
\textbf{Examples of Resilience:}
\begin{itemize}
\item Gods Unchained: Free-to-play card game, sustainable model
\item Illuvium: High-quality RPG with deflationary token burning
\end{itemize}
\end{frame}

\begin{frame}{Key Takeaways}
\begin{enumerate}
\item Play-to-earn (P2E) models enable earning through gameplay but often resemble Ponzi schemes
\item Axie Infinity collapsed due to unsustainable tokenomics and unlimited SLP inflation
\item Virtual land NFTs (Decentraland, Sandbox) speculative, low actual usage and development
\item Interoperability of cross-game assets limited by technical and economic barriers
\item Sustainable blockchain gaming requires fun-first design and balanced token economies
\item Shift to play-and-earn (fun primary, earning secondary) shows more promise
\end{enumerate}
\end{frame}

\begin{frame}{Discussion Questions}
\begin{enumerate}
\item Can play-to-earn gaming models ever be sustainable without Ponzi dynamics?
\item What would make virtual land NFTs genuinely valuable beyond speculation?
\item Is true cross-game interoperability achievable, or a pipe dream?
\item Should blockchain games focus on financialization or gameplay quality?
\item How can regulators balance innovation with consumer protection in P2E gaming?
\end{enumerate}
\end{frame}

\begin{frame}{Next Lesson Preview}
\textbf{L27: Real-World Asset Tokenization}

\vspace{0.3cm}
We will explore:
\begin{itemize}
\item RWA tokenization concept and mechanics
\item Real estate tokenization platforms and legal frameworks
\item Securities tokenization and compliance (Reg D, Reg S)
\item Case study: BlackRock BUIDL fund (\$500M+ on-chain)
\item Market size: \$50B on-chain, projected \$18T by 2033
\end{itemize}

\vspace{0.3cm}
\textbf{Preparation:} Review traditional real estate investment structures (REITs)
\end{frame}

\end{document}
