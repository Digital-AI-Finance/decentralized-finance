\documentclass[8pt,aspectratio=169]{beamer}
\usetheme{Madrid}
\usepackage[utf8]{inputenc}
\usepackage{graphicx}
\usepackage{booktabs}
\usepackage{hyperref}

\title{L21: NFT Technology Deep Dive}
\subtitle{Module C: NFTs \& Digital Assets}
\author{Blockchain \& Cryptocurrency Course}
\date{December 2025}

\begin{document}

\begin{frame}
\titlepage
\end{frame}

\begin{frame}{Learning Objectives}
By the end of this lesson, you will be able to:
\begin{itemize}
\item Explain the difference between on-chain and off-chain NFT data
\item Understand the role of token URIs in NFT metadata
\item Describe the ERC-721 token standard internals
\item Analyze NFT provenance and ownership verification
\item Evaluate the technical limitations of current NFT implementations
\end{itemize}
\end{frame}

\begin{frame}{What is an NFT? Technical Definition}
\textbf{Non-Fungible Token (NFT):} A unique digital asset on a blockchain

\vspace{0.3cm}
\textbf{Key Technical Properties:}
\begin{itemize}
\item \textbf{Non-fungible:} Each token is unique and not interchangeable
\item \textbf{Token ID:} Unique identifier within a smart contract
\item \textbf{Ownership:} Blockchain-verified ownership record
\item \textbf{Transferability:} Can be bought, sold, or transferred
\item \textbf{Programmability:} Smart contract logic defines behavior
\end{itemize}

\vspace{0.3cm}
\textbf{Contrast with Fungible Tokens (ERC-20):}
\begin{itemize}
\item Fungible: 1 ETH = 1 ETH (interchangeable)
\item Non-fungible: Token \#1 $\neq$ Token \#2 (unique)
\end{itemize}
\end{frame}

\begin{frame}{ERC-721: The NFT Standard}
\textbf{ERC-721} introduced by Dieter Shirley (CryptoKitties) in 2017

\vspace{0.3cm}
\textbf{Core Functions:}
\begin{itemize}
\item \texttt{balanceOf(owner)} -- Returns number of tokens owned
\item \texttt{ownerOf(tokenId)} -- Returns owner of specific token
\item \texttt{transferFrom(from, to, tokenId)} -- Transfers ownership
\item \texttt{approve(to, tokenId)} -- Grants transfer permission
\item \texttt{tokenURI(tokenId)} -- Returns metadata URI
\end{itemize}

\vspace{0.3cm}
\textbf{Events:}
\begin{itemize}
\item \texttt{Transfer(from, to, tokenId)} -- Emitted on ownership change
\item \texttt{Approval(owner, approved, tokenId)} -- Emitted on approval
\end{itemize}
\end{frame}

\begin{frame}{On-Chain vs Off-Chain Data}
\textbf{On-Chain Data:}
\begin{itemize}
\item Stored directly on the blockchain
\item Immutable and permanently accessible
\item Expensive (gas costs for storage)
\item Examples: Token ID, owner address, contract logic
\item Typical size: Minimal (addresses, IDs, small integers)
\end{itemize}

\vspace{0.3cm}
\textbf{Off-Chain Data:}
\begin{itemize}
\item Stored externally (IPFS, Arweave, centralized servers)
\item Referenced by on-chain URI pointer
\item Cost-effective for large files (images, videos)
\item Examples: Artwork, metadata JSON, high-res images
\item Risk: External storage may become unavailable
\end{itemize}
\end{frame}

\begin{frame}{Token URI: The Metadata Bridge}
\textbf{tokenURI Function:} Links on-chain token to off-chain metadata

\vspace{0.3cm}
\textbf{Example URI Patterns:}
\begin{enumerate}
\item \textbf{IPFS:} \texttt{ipfs://QmXyZ.../metadata.json}
\item \textbf{HTTP:} \texttt{https://api.project.com/token/123}
\item \textbf{Data URI:} \texttt{data:application/json;base64,...}
\end{enumerate}

\vspace{0.3cm}
\textbf{Metadata JSON Structure:}
\begin{itemize}
\item \texttt{name} -- Token name
\item \texttt{description} -- Human-readable description
\item \texttt{image} -- URI to visual asset
\item \texttt{attributes} -- Array of traits (rarity properties)
\end{itemize}

\vspace{0.3cm}
\textbf{Critical Issue:} If metadata server goes down, NFT may become unrenderable
\end{frame}

\begin{frame}{NFT Rendering: From Contract to Display}
\textbf{Rendering Pipeline:}
\begin{enumerate}
\item Wallet/marketplace calls \texttt{tokenURI(tokenId)}
\item Smart contract returns URI string
\item Client fetches metadata JSON from URI
\item Parse JSON to extract \texttt{image} field
\item Fetch and display image from image URI
\item Display attributes/traits from metadata
\end{enumerate}

\vspace{0.3cm}
\textbf{Decentralization Spectrum:}
\begin{itemize}
\item \textbf{Fully On-Chain:} All data in contract (rare, expensive)
\item \textbf{IPFS/Arweave:} Decentralized storage (common)
\item \textbf{Centralized Server:} Project-controlled API (risky)
\end{itemize}
\end{frame}

\begin{frame}{Provenance: Tracking Ownership History}
\textbf{Provenance:} Complete ownership history of an NFT

\vspace{0.3cm}
\textbf{Blockchain Provides:}
\begin{itemize}
\item Full transaction history via \texttt{Transfer} events
\item Verifiable chain of custody from mint to present
\item Immutable record (cannot be forged or altered)
\item Creator verification (original minting address)
\end{itemize}

\vspace{0.3cm}
\textbf{Why Provenance Matters:}
\begin{itemize}
\item Proves authenticity and origin
\item Increases value (celebrity ownership history)
\item Detects wash trading (suspicious transfers)
\item Validates artist attribution
\end{itemize}
\end{frame}

\begin{frame}{ERC-721 Internals: State Variables}
\textbf{Typical ERC-721 Contract State:}

\vspace{0.3cm}
\begin{itemize}
\item \texttt{mapping(uint256 => address) private \_owners}
  \begin{itemize}
  \item Maps token ID to owner address
  \end{itemize}
\item \texttt{mapping(address => uint256) private \_balances}
  \begin{itemize}
  \item Maps owner address to token count
  \end{itemize}
\item \texttt{mapping(uint256 => address) private \_tokenApprovals}
  \begin{itemize}
  \item Maps token ID to approved spender
  \end{itemize}
\item \texttt{mapping(address => mapping(address => bool)) private \_operatorApprovals}
  \begin{itemize}
  \item Allows operators to manage all tokens for an owner
  \end{itemize}
\end{itemize}

\vspace{0.3cm}
\textbf{Gas Efficiency:} Mappings are efficient for sparse data (not all token IDs exist)
\end{frame}

\begin{frame}{Minting Process}
\textbf{Minting:} Creating a new NFT token

\vspace{0.3cm}
\textbf{Typical Minting Flow:}
\begin{enumerate}
\item User calls \texttt{mint()} function (often with payment)
\item Contract generates or assigns unique token ID
\item Contract sets \texttt{\_owners[tokenId] = msg.sender}
\item Contract increments \texttt{\_balances[msg.sender]}
\item Contract emits \texttt{Transfer(address(0), msg.sender, tokenId)}
\item (Optional) Contract sets metadata URI
\end{enumerate}

\vspace{0.3cm}
\textbf{Minting Patterns:}
\begin{itemize}
\item \textbf{Fixed Supply:} Limited collection (e.g., 10,000 tokens)
\item \textbf{Open Edition:} Unlimited minting in time window
\item \textbf{Lazy Minting:} Token created only when first purchased
\end{itemize}
\end{frame}

\begin{frame}{Transfer Mechanics}
\textbf{Transfer Function Implementation:}

\vspace{0.3cm}
\textbf{Key Steps:}
\begin{enumerate}
\item Verify sender is owner or approved operator
\item Check \texttt{from} address owns the token
\item Clear any existing approvals for the token
\item Update \texttt{\_owners[tokenId] = to}
\item Decrement \texttt{\_balances[from]}
\item Increment \texttt{\_balances[to]}
\item Emit \texttt{Transfer(from, to, tokenId)}
\end{enumerate}

\vspace{0.3cm}
\textbf{Safety Checks:}
\begin{itemize}
\item Cannot transfer to zero address (burning requires explicit function)
\item Recipient must be able to receive NFTs (ERC-721 receiver check)
\end{itemize}
\end{frame}

\begin{frame}{ERC-721 Extensions}
\textbf{Common Extensions Beyond Base Standard:}

\vspace{0.3cm}
\begin{itemize}
\item \textbf{ERC-721 Metadata:} Adds \texttt{name()}, \texttt{symbol()}, \texttt{tokenURI()}
\item \textbf{ERC-721 Enumerable:} Allows iteration over all tokens
  \begin{itemize}
  \item \texttt{totalSupply()} -- Total number of tokens
  \item \texttt{tokenByIndex(index)} -- Get token ID by index
  \item \texttt{tokenOfOwnerByIndex(owner, index)} -- Owner's tokens
  \end{itemize}
\item \textbf{ERC-721 Burnable:} Allows token destruction
  \begin{itemize}
  \item \texttt{burn(tokenId)} -- Permanently destroys token
  \end{itemize}
\item \textbf{ERC-721 Pausable:} Emergency stop mechanism
\item \textbf{ERC-721 Royalty (ERC-2981):} On-chain royalty information
\end{itemize}
\end{frame}

\begin{frame}{ERC-1155: Multi-Token Standard}
\textbf{ERC-1155:} Supports both fungible and non-fungible tokens in one contract

\vspace{0.3cm}
\textbf{Key Differences from ERC-721:}
\begin{itemize}
\item Single contract can manage multiple token types
\item Batch transfers (gas efficient for multiple tokens)
\item Semi-fungible tokens (fungible until uniqueness assigned)
\item Used heavily in gaming (items, currencies, NFTs)
\end{itemize}

\vspace{0.3cm}
\textbf{Example Use Cases:}
\begin{itemize}
\item Gaming: 100 fungible ``gold coins'' + unique weapon NFTs
\item Event tickets: 500 general admission (fungible) + 10 VIP (non-fungible)
\end{itemize}
\end{frame}

\begin{frame}{Technical Limitations of NFTs}
\textbf{Current NFT Technology Challenges:}

\vspace{0.3cm}
\begin{itemize}
\item \textbf{Off-chain dependency:} Most NFTs rely on external storage
\item \textbf{Metadata mutability:} Some projects use mutable \texttt{tokenURI}
\item \textbf{Gas costs:} On-chain storage extremely expensive
\item \textbf{Interoperability:} Limited cross-chain NFT movement
\item \textbf{Smart contract bugs:} Exploits can drain entire collections
\item \textbf{Centralization risks:} Project teams control metadata servers
\item \textbf{Scalability:} Ethereum mainnet slow and expensive for minting
\end{itemize}

\vspace{0.3cm}
\textbf{Layer 2 Solutions:} Polygon, Arbitrum, Optimism reduce costs
\end{frame}

\begin{frame}{Fully On-Chain NFTs}
\textbf{On-Chain Maximalism:} All data stored on blockchain

\vspace{0.3cm}
\textbf{Examples:}
\begin{itemize}
\item \textbf{Autoglyphs:} Generative art stored as code in contract
\item \textbf{Blitmap:} Pixel art encoded in contract storage
\item \textbf{Chain Runners:} SVG generation entirely on-chain
\end{itemize}

\vspace{0.3cm}
\textbf{Advantages:}
\begin{itemize}
\item True permanence (no external dependencies)
\item Maximum decentralization
\item Provable scarcity of both token and artwork
\end{itemize}

\vspace{0.3cm}
\textbf{Disadvantages:}
\begin{itemize}
\item Extremely high minting costs (gas for storage)
\item Limited to simple/generative art (no high-res photos)
\end{itemize}
\end{frame}

\begin{frame}{Security Considerations}
\textbf{Common NFT Smart Contract Vulnerabilities:}

\vspace{0.3cm}
\begin{itemize}
\item \textbf{Reentrancy:} Malicious contracts exploiting callback functions
\item \textbf{Integer overflow/underflow:} Arithmetic errors in token counting
\item \textbf{Access control:} Unauthorized minting or burning
\item \textbf{Front-running:} MEV bots exploiting mint transactions
\item \textbf{Approval phishing:} Tricking users into approving malicious contracts
\end{itemize}

\vspace{0.3cm}
\textbf{Best Practices:}
\begin{itemize}
\item Use audited libraries (OpenZeppelin)
\item External security audits before launch
\item Reentrancy guards on state-changing functions
\item Limit approval scopes (per-token vs. all tokens)
\end{itemize}
\end{frame}

\begin{frame}{2023-2024: Bitcoin Ordinals and Inscriptions}
\textbf{What are Ordinals?}
\begin{itemize}
\item NFT-like artifacts directly on Bitcoin blockchain
\item Uses Ordinal Theory: assigns numbers to individual satoshis
\item ``Inscriptions'': Data embedded in Bitcoin transaction witness data
\item Enabled by Taproot upgrade (November 2021)
\end{itemize}

\vspace{0.3cm}
\textbf{Key Differences from Ethereum NFTs:}
\begin{itemize}
\item Fully on-chain (no external metadata links)
\item No smart contracts (Bitcoin Script limitations)
\item Higher permanence guarantees
\item Higher inscription costs (\$10-100+ per inscription)
\end{itemize}

\vspace{0.3cm}
\textbf{BRC-20 Tokens:}
\begin{itemize}
\item Fungible tokens on Bitcoin using inscriptions
\item JSON-based token standard (deploy, mint, transfer)
\item \$1B+ market cap by late 2023
\item Controversy: Bitcoin ``bloat'' vs innovation debate
\end{itemize}
\end{frame}

\begin{frame}{Key Takeaways}
\begin{enumerate}
\item NFTs are blockchain tokens with unique identifiers governed by smart contracts (ERC-721/ERC-1155)
\item Most NFT data is off-chain (images, metadata) with on-chain pointers (tokenURI)
\item Provenance tracking provides verifiable ownership history and authenticity
\item Bitcoin Ordinals (2023): Brought NFTs to Bitcoin via inscriptions
\item Technical limitations include off-chain dependencies, gas costs, and centralization risks
\item Fully on-chain NFTs (Ordinals, Art Blocks) solve permanence but sacrifice complexity/cost
\end{enumerate}
\end{frame}

\begin{frame}{Discussion Questions}
\begin{enumerate}
\item Should ``true'' NFTs require all data to be on-chain, or is off-chain storage acceptable?
\item How does the ERC-721 standard balance gas efficiency with functionality?
\item What are the trade-offs between using IPFS vs. centralized servers for NFT metadata?
\item How does NFT provenance compare to traditional art provenance verification?
\item What innovations could solve the scalability and cost issues of on-chain NFTs?
\end{enumerate}
\end{frame}

\begin{frame}{Next Lesson Preview}
\textbf{L22: NFT Metadata and IPFS}

\vspace{0.3cm}
We will explore:
\begin{itemize}
\item JSON metadata format standards
\item IPFS content addressing and pinning
\item Arweave permanent storage
\item Metadata permanence and availability challenges
\item Best practices for decentralized NFT storage
\end{itemize}

\vspace{0.3cm}
\textbf{Preparation:} Review IPFS documentation and explore NFT metadata on OpenSea
\end{frame}

\end{document}
