\documentclass[8pt,aspectratio=169]{beamer}
\usetheme{Madrid}
\usepackage[utf8]{inputenc}
\usepackage{graphicx}
\usepackage{booktabs}
\usepackage{hyperref}

\newcommand{\bottomnote}[1]{\vfill\hfill{\scriptsize\textit{#1}}}

\title{L21: NFT Technology Deep Dive}
\subtitle{Module C: NFTs \& Digital Assets}
\author{Blockchain \& Cryptocurrency Course}
\date{December 2025}

\begin{document}

\begin{frame}
\titlepage
\end{frame}

\begin{frame}{Learning Objectives}
By the end of this lesson, you will be able to:
\begin{itemize}
\item Explain the difference between on-chain and off-chain NFT data
\item Understand the role of token URIs in NFT metadata
\item Describe the ERC-721 token standard internals
\item Analyze NFT provenance and ownership verification
\item Evaluate the technical limitations of current NFT implementations
\end{itemize}
\end{frame}

\begin{frame}{What is an NFT? Technical Definition}
\textbf{Non-Fungible Token (NFT):} A unique digital asset on a blockchain

\vspace{0.2cm}
\textbf{Key Technical Properties:}
\begin{itemize}
\item \textbf{Non-fungible:} Each token is unique and not interchangeable
\item \textbf{Token ID:} Unique identifier within a smart contract
\item \textbf{Ownership:} Blockchain-verified ownership record
\item \textbf{Programmability:} Smart contract logic defines behavior
\end{itemize}

\vspace{0.2cm}
\textbf{Contrast with Fungible Tokens (ERC-20):}
\begin{itemize}
\item Fungible: 1 ETH = 1 ETH (interchangeable)
\item Non-fungible: Token \#1 $\neq$ Token \#2 (unique)
\end{itemize}
\end{frame}

\begin{frame}{On-Chain vs Off-Chain Data}
\begin{center}
\includegraphics[width=0.72\textwidth]{charts/01_onchain_vs_offchain/chart.pdf}
\end{center}
\bottomnote{Critical trade-off: On-chain is permanent but expensive; off-chain is cheap but risky}
\end{frame}

\begin{frame}{ERC-721: The NFT Standard}
\textbf{ERC-721} introduced by Dieter Shirley (CryptoKitties) in 2017

\vspace{0.2cm}
\textbf{Core Functions:}
\begin{itemize}
\item \texttt{balanceOf(owner)} -- Returns number of tokens owned
\item \texttt{ownerOf(tokenId)} -- Returns owner of specific token
\item \texttt{transferFrom(from, to, tokenId)} -- Transfers ownership
\item \texttt{approve(to, tokenId)} -- Grants transfer permission
\item \texttt{tokenURI(tokenId)} -- Returns metadata URI
\end{itemize}

\vspace{0.2cm}
\textbf{Events:}
\begin{itemize}
\item \texttt{Transfer(from, to, tokenId)} -- Emitted on ownership change
\item \texttt{Approval(owner, approved, tokenId)} -- Emitted on approval
\end{itemize}
\end{frame}

\begin{frame}{ERC-721 State Structure}
\begin{center}
\includegraphics[width=0.75\textwidth]{charts/02_erc721_state_structure/chart.pdf}
\end{center}
\bottomnote{Mappings provide O(1) lookup efficiency for sparse token ID spaces}
\end{frame}

\begin{frame}{Token URI: The Metadata Bridge}
\textbf{tokenURI Function:} Links on-chain token to off-chain metadata

\vspace{0.2cm}
\textbf{Example URI Patterns:}
\begin{enumerate}
\item \textbf{IPFS:} \texttt{ipfs://QmXyZ.../metadata.json}
\item \textbf{HTTP:} \texttt{https://api.project.com/token/123}
\item \textbf{Data URI:} \texttt{data:application/json;base64,...}
\end{enumerate}

\vspace{0.2cm}
\textbf{Metadata JSON Structure:}
\begin{itemize}
\item \texttt{name} -- Token name
\item \texttt{description} -- Human-readable description
\item \texttt{image} -- URI to visual asset
\item \texttt{attributes} -- Array of traits (rarity properties)
\end{itemize}

\vspace{0.2cm}
\textbf{Critical Issue:} If metadata server goes down, NFT may become unrenderable
\end{frame}

\begin{frame}{NFT Rendering Pipeline}
\begin{center}
\includegraphics[width=0.78\textwidth]{charts/03_nft_rendering_pipeline/chart.pdf}
\end{center}
\bottomnote{Multiple points of failure: contract, IPFS gateway, image server}
\end{frame}

\begin{frame}{Provenance: Tracking Ownership History}
\textbf{Provenance:} Complete ownership history of an NFT

\vspace{0.2cm}
\textbf{Blockchain Provides:}
\begin{itemize}
\item Full transaction history via \texttt{Transfer} events
\item Verifiable chain of custody from mint to present
\item Immutable record (cannot be forged or altered)
\item Creator verification (original minting address)
\end{itemize}

\vspace{0.2cm}
\textbf{Why Provenance Matters:}
\begin{itemize}
\item Proves authenticity and origin
\item Increases value (celebrity ownership history)
\item Detects wash trading (suspicious transfers)
\end{itemize}
\end{frame}

\begin{frame}{NFT Provenance Chain}
\begin{center}
\includegraphics[width=0.75\textwidth]{charts/04_nft_provenance/chart.pdf}
\end{center}
\bottomnote{Every Transfer event is immutably recorded on the blockchain}
\end{frame}

\begin{frame}{NFT Market Evolution}
\begin{center}
\includegraphics[width=0.72\textwidth]{charts/05_nft_market_timeline/chart.pdf}
\end{center}
\bottomnote{Market peaked in 2021 but technology continues to evolve}
\end{frame}

\begin{frame}{ERC-721 Extensions}
\textbf{Common Extensions Beyond Base Standard:}

\vspace{0.2cm}
\begin{itemize}
\item \textbf{ERC-721 Metadata:} Adds \texttt{name()}, \texttt{symbol()}, \texttt{tokenURI()}
\item \textbf{ERC-721 Enumerable:} Allows iteration over all tokens
  \begin{itemize}
  \item \texttt{totalSupply()} -- Total number of tokens
  \item \texttt{tokenByIndex(index)} -- Get token ID by index
  \end{itemize}
\item \textbf{ERC-721 Burnable:} Allows token destruction
\item \textbf{ERC-721 Pausable:} Emergency stop mechanism
\item \textbf{ERC-2981:} On-chain royalty information
\end{itemize}
\end{frame}

\begin{frame}{ERC-1155: Multi-Token Standard}
\textbf{ERC-1155:} Supports both fungible and non-fungible tokens in one contract

\vspace{0.2cm}
\textbf{Key Differences from ERC-721:}
\begin{itemize}
\item Single contract can manage multiple token types
\item Batch transfers (gas efficient for multiple tokens)
\item Semi-fungible tokens (fungible until uniqueness assigned)
\item Used heavily in gaming (items, currencies, NFTs)
\end{itemize}

\vspace{0.2cm}
\textbf{Example Use Cases:}
\begin{itemize}
\item Gaming: 100 fungible ``gold coins'' + unique weapon NFTs
\item Event tickets: 500 general admission (fungible) + 10 VIP (non-fungible)
\end{itemize}
\end{frame}

\begin{frame}{Storage Options Comparison}
\begin{center}
\includegraphics[width=0.72\textwidth]{charts/06_storage_tradeoffs/chart.pdf}
\end{center}
\bottomnote{IPFS offers best balance for most NFT projects; on-chain for maximum permanence}
\end{frame}

\begin{frame}{Fully On-Chain NFTs}
\textbf{On-Chain Maximalism:} All data stored on blockchain

\vspace{0.2cm}
\textbf{Examples:}
\begin{itemize}
\item \textbf{Autoglyphs:} Generative art stored as code in contract
\item \textbf{Blitmap:} Pixel art encoded in contract storage
\item \textbf{Chain Runners:} SVG generation entirely on-chain
\end{itemize}

\vspace{0.2cm}
\textbf{Advantages:}
\begin{itemize}
\item True permanence (no external dependencies)
\item Maximum decentralization
\end{itemize}

\vspace{0.2cm}
\textbf{Disadvantages:}
\begin{itemize}
\item Extremely high minting costs (gas for storage)
\item Limited to simple/generative art (no high-res photos)
\end{itemize}
\end{frame}

\begin{frame}{2023-2024: Bitcoin Ordinals and Inscriptions}
\textbf{What are Ordinals?}
\begin{itemize}
\item NFT-like artifacts directly on Bitcoin blockchain
\item Uses Ordinal Theory: assigns numbers to individual satoshis
\item ``Inscriptions'': Data embedded in Bitcoin transaction witness
\item Enabled by Taproot upgrade (November 2021)
\end{itemize}

\vspace{0.2cm}
\textbf{Key Differences from Ethereum NFTs:}
\begin{itemize}
\item Fully on-chain (no external metadata links)
\item No smart contracts (Bitcoin Script limitations)
\item Higher permanence guarantees
\item Higher inscription costs (\$10-100+ per inscription)
\end{itemize}
\end{frame}

\begin{frame}{Key Takeaways}
\begin{enumerate}
\item NFTs are blockchain tokens with unique identifiers governed by smart contracts (ERC-721/ERC-1155)
\item Most NFT data is off-chain (images, metadata) with on-chain pointers (tokenURI)
\item Provenance tracking provides verifiable ownership history and authenticity
\item Bitcoin Ordinals (2023): Brought NFTs to Bitcoin via inscriptions
\item Technical limitations include off-chain dependencies, gas costs, and centralization risks
\item Fully on-chain NFTs solve permanence but sacrifice complexity/cost
\end{enumerate}
\end{frame}

\begin{frame}{Discussion Questions}
\begin{enumerate}
\item Should ``true'' NFTs require all data to be on-chain, or is off-chain storage acceptable?
\item How does the ERC-721 standard balance gas efficiency with functionality?
\item What are the trade-offs between using IPFS vs. centralized servers for NFT metadata?
\item How does NFT provenance compare to traditional art provenance verification?
\item What innovations could solve the scalability and cost issues of on-chain NFTs?
\end{enumerate}
\end{frame}

\begin{frame}{Next Lesson Preview}
\textbf{L22: NFT Metadata and IPFS}

\vspace{0.2cm}
We will explore:
\begin{itemize}
\item JSON metadata format standards
\item IPFS content addressing and pinning
\item Arweave permanent storage
\item Metadata permanence and availability challenges
\item Best practices for decentralized NFT storage
\end{itemize}

\vspace{0.2cm}
\textbf{Preparation:} Review IPFS documentation and explore NFT metadata on OpenSea
\end{frame}

\end{document}
