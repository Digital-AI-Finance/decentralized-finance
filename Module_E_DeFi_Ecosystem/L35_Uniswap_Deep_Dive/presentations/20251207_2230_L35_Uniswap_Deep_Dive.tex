\documentclass[8pt,aspectratio=169]{beamer}
\usetheme{Madrid}
\usepackage[utf8]{inputenc}
\usepackage{graphicx}
\usepackage{booktabs}
\usepackage{hyperref}
\usepackage{amsmath}

\title{L35: Uniswap Deep Dive}
\subtitle{Module E: DeFi Ecosystem}
\author{Blockchain \& Cryptocurrency}
\date{December 2025}

\begin{document}

\frame{\titlepage}

\begin{frame}{Learning Objectives}
\begin{itemize}
\item Trace Uniswap's evolution from V1 to V4
\item Understand concentrated liquidity mechanics (V3)
\item Analyze fee tier optimization
\item Explore UNI token governance and the fee switch debate
\item Case Study: Fee switch governance controversy
\end{itemize}
\end{frame}

\begin{frame}{Uniswap: Origins}
\textbf{Created by:} Hayden Adams (inspired by Vitalik Buterin's post)

\vspace{0.3cm}
\textbf{Launch Timeline:}
\begin{itemize}
\item \textbf{November 2018:} Uniswap V1 launches
\item \textbf{May 2020:} Uniswap V2 with ERC-20/ERC-20 pairs
\item \textbf{May 2021:} Uniswap V3 with concentrated liquidity
\item \textbf{June 2024:} Uniswap V4 announced (hooks and singleton contracts)
\end{itemize}

\vspace{0.3cm}
\textbf{Market Position:}
\begin{itemize}
\item Largest DEX by volume (\$50B+ monthly)
\item ~\$3.5B TVL (Dec 2024)
\item Deployed on Ethereum, Polygon, Arbitrum, Optimism, Base, BNB Chain
\end{itemize}

\vspace{0.3cm}
\textbf{Impact:} Pioneered AMM model, inspired hundreds of forks.
\end{frame}

\begin{frame}{Uniswap V1 (2018)}
\textbf{Key Features:}
\begin{itemize}
\item ETH/ERC-20 pairs only (no direct ERC-20/ERC-20)
\item Constant product formula: $x \cdot y = k$
\item 0.3\% trading fee (100\% to LPs)
\item Simple, audited, gas-efficient
\end{itemize}

\vspace{0.3cm}
\textbf{Limitations:}
\begin{itemize}
\item Multi-hop trades required for ERC-20/ERC-20 (e.g., DAI $\to$ ETH $\to$ USDC)
\item Double fees and slippage on multi-hop
\item No price oracles
\item Fixed fee tier
\end{itemize}

\vspace{0.3cm}
\textbf{Innovation:} Proved AMM model viable on Ethereum.
\end{frame}

\begin{frame}{Uniswap V2 (2020)}
\textbf{Major Improvements:}
\begin{itemize}
\item \textbf{ERC-20/ERC-20 pairs:} Direct trading without ETH intermediary
\item \textbf{Price oracles:} Time-weighted average price (TWAP) for other protocols
\item \textbf{Flash swaps:} Borrow any token, repay in same transaction
\item \textbf{Protocol fee switch:} Potential 0.05\% fee to UNI holders (not activated)
\end{itemize}

\vspace{0.3cm}
\textbf{TWAP Oracle:}
\begin{itemize}
\item Accumulates price over time
\item Resistant to flash loan manipulation
\item Used by lending protocols (Compound, Aave)
\end{itemize}

\vspace{0.3cm}
\textbf{Flash Swaps:}
\begin{itemize}
\item Borrow tokens, use in DeFi, repay + fee in one transaction
\item Enables arbitrage without upfront capital
\end{itemize}
\end{frame}

\begin{frame}{Uniswap V3 (2021): Concentrated Liquidity}
\textbf{Revolutionary Concept:} LPs choose specific price ranges for liquidity.

\vspace{0.3cm}
\textbf{How It Works:}
\begin{itemize}
\item Instead of infinite range (\$0 to $\infty$), LP sets bounds (e.g., \$1,900-\$2,100 for ETH)
\item Liquidity only active within range
\item Earns fees only when price in range
\item If price exits range, position becomes 100\% one token
\end{itemize}

\vspace{0.3cm}
\textbf{Example:}
\begin{itemize}
\item LP deposits 1 ETH + 2,000 USDC with range \$1,900-\$2,100
\item If ETH price = \$2,000: Position active, earns fees
\item If ETH price = \$2,200: Position fully in USDC, no fees
\item If ETH price = \$1,800: Position fully in ETH, no fees
\end{itemize}
\end{frame}

\begin{frame}{Concentrated Liquidity: Capital Efficiency}
\textbf{Advantage:} Achieve same liquidity depth with less capital.

\vspace{0.3cm}
\textbf{Comparison:}
\begin{itemize}
\item \textbf{V2 LP:} Deposits \$100,000 across entire price range (\$0 to $\infty$)
\item \textbf{V3 LP:} Deposits \$10,000 in tight range (\$1,900-\$2,100)
\item \textbf{Result:} V3 LP provides same liquidity in relevant range with 10x less capital
\end{itemize}

\vspace{0.3cm}
\textbf{Capital Efficiency Factor:}
\begin{itemize}
\item Depends on range width
\item Narrow range (e.g., \$1,980-\$2,020): Up to 200x efficiency
\item Wide range (e.g., \$1,000-\$4,000): ~2-5x efficiency
\end{itemize}

\vspace{0.3cm}
\textbf{Trade-off:} Higher returns but requires active management.
\end{frame}

\begin{frame}{Concentrated Liquidity: Active Management}
\textbf{Challenge:} Price may exit your range, requiring rebalancing.

\vspace{0.3cm}
\textbf{Strategies:}
\begin{enumerate}
\item \textbf{Passive (Wide Range)}
\begin{itemize}
\item Set very wide range (e.g., \$1,000-\$5,000)
\item Lower returns but minimal management
\end{itemize}

\item \textbf{Active (Narrow Range)}
\begin{itemize}
\item Set tight range around current price
\item High returns if price stable
\item Rebalance frequently as price moves
\end{itemize}

\item \textbf{Automated (Vault Services)}
\begin{itemize}
\item Use Gamma, Arrakis, Charm to auto-rebalance
\item Pay management fee (0.1-0.5\%)
\end{itemize}
\end{enumerate}

\vspace{0.3cm}
\textbf{Gas Costs:} Rebalancing requires withdrawing and redepositing (expensive on Ethereum L1).
\end{frame}

\begin{frame}{Uniswap V3: Fee Tiers}
\textbf{Innovation:} Multiple fee tiers for same pair.

\vspace{0.3cm}
\begin{tabular}{llp{5cm}}
\toprule
Fee Tier & Use Case & Example Pairs \\
\midrule
0.01\% & Stablecoins & USDC/USDT, DAI/USDC \\
0.05\% & Correlated assets & ETH/stETH, WBTC/tBTC \\
0.3\% & Standard pairs & ETH/USDC, WBTC/ETH \\
1\% & Exotic/volatile & Low-liquidity tokens \\
\bottomrule
\end{tabular}

\vspace{0.3cm}
\textbf{Rationale:}
\begin{itemize}
\item Stablecoins: Low price risk, low fees attract volume
\item Volatile pairs: High impermanent loss, higher fees compensate
\end{itemize}

\vspace{0.3cm}
\textbf{Impact:} Most liquidity in 0.05\% and 0.3\% tiers.
\end{frame}

\begin{frame}{Uniswap V3: Non-Fungible LP Positions}
\textbf{Change from V2:}
\begin{itemize}
\item V2: LP tokens are fungible ERC-20s
\item V3: LP positions are unique NFTs (ERC-721)
\end{itemize}

\vspace{0.3cm}
\textbf{Why NFTs?}
\begin{itemize}
\item Each position has custom price range
\item Cannot standardize (different ranges, different values)
\item NFT represents specific position parameters
\end{itemize}

\vspace{0.3cm}
\textbf{Implications:}
\begin{itemize}
\item \textbf{Positive:} Flexibility, capital efficiency
\item \textbf{Negative:} Harder to integrate with DeFi (less composability)
\end{itemize}

\vspace{0.3cm}
\textbf{Solution:} Wrapper protocols (Gamma, Arrakis) create fungible tokens representing V3 positions.
\end{frame}

\begin{frame}{Uniswap V4 (2024): Hooks}
\textbf{Key Innovation:} Customizable pool behavior via hooks.

\vspace{0.3cm}
\textbf{What are Hooks?}
\begin{itemize}
\item Smart contracts that execute at specific points in trade lifecycle
\item Before/after swap, before/after adding liquidity, etc.
\item Developers can add custom logic without forking Uniswap
\end{itemize}

\vspace{0.3cm}
\textbf{Example Use Cases:}
\begin{itemize}
\item Dynamic fees based on volatility
\item On-chain limit orders
\item TWAP execution
\item Auto-rebalancing liquidity
\item Custom oracles
\item MEV protection mechanisms
\end{itemize}

\vspace{0.3cm}
\textbf{Singleton Contract:} All pools in one contract (gas savings).
\end{frame}

\begin{frame}{UNI Token: Governance}
\textbf{Launch:} September 2020 (retroactive airdrop of 400 UNI to every user)

\vspace{0.3cm}
\textbf{Token Supply:}
\begin{itemize}
\item Total: 1 billion UNI
\item Community airdrop: 15\% (150M)
\item Team/investors/advisors: 40\% (4-year vesting)
\item Community treasury: 43\% (governed by UNI holders)
\end{itemize}

\vspace{0.3cm}
\textbf{Governance Rights:}
\begin{itemize}
\item Vote on protocol upgrades
\item Treasury allocation (billions in assets)
\item Fee switch activation (controversial)
\item Cross-chain deployment approvals
\end{itemize}

\vspace{0.3cm}
\textbf{Voting:} 1 UNI = 1 vote (can delegate).

\textbf{Proposal Threshold:} 2.5M UNI to propose, 40M quorum to pass.
\end{frame}

\begin{frame}{The Fee Switch Debate}
\textbf{Current State:}
\begin{itemize}
\item 0.3\% trading fee goes 100\% to LPs
\item Protocol fee switch can redirect 0.05\% (1/6th) to UNI holders
\end{itemize}

\vspace{0.3cm}
\textbf{Arguments For Activation:}
\begin{itemize}
\item Value accrual for UNI token (currently only governance utility)
\item Sustainable revenue for protocol development
\item Competitive with other DEXs (SushiSwap shares fees)
\item Treasury accumulation for grants/operations
\end{itemize}

\vspace{0.3cm}
\textbf{Arguments Against:}
\begin{itemize}
\item Reduces LP returns (may lose liquidity to competitors)
\item Regulatory risk (UNI could be deemed security if profit-sharing)
\item Current model works (don't fix what's not broken)
\item Complexity of distribution mechanism
\end{itemize}
\end{frame}

\begin{frame}{Case Study: Fee Switch Governance (2023)}
\textbf{Timeline:}
\begin{itemize}
\item \textbf{Feb 2023:} Community proposal to activate fee switch
\item \textbf{Debate:} Heated discussion on governance forum
\item \textbf{Vote:} Proposal failed to reach quorum
\item \textbf{Outcome:} Fee switch remains off (as of Dec 2024)
\end{itemize}

\vspace{0.3cm}
\textbf{Key Issues:}
\begin{enumerate}
\item \textbf{Regulatory Uncertainty:} SEC scrutiny of DeFi, especially revenue-sharing
\item \textbf{LP Exodus Risk:} Would LPs leave for SushiSwap, Curve, etc.?
\item \textbf{Distribution Complexity:} How to fairly distribute fees to UNI holders?
\item \textbf{Timing:} Bear market = bad time to reduce LP incentives
\end{enumerate}

\vspace{0.3cm}
\textbf{Current Status:} Delayed indefinitely, may revisit in bull market.
\end{frame}

\begin{frame}{UNI Token Utility (Current)}
\textbf{What UNI Holders Get:}
\begin{itemize}
\item \textbf{Governance:} Vote on protocol decisions
\item \textbf{Treasury Control:} Manage \$5B+ in assets
\item \textbf{Deployment Approval:} New chains (e.g., BNB Chain vote)
\end{itemize}

\vspace{0.3cm}
\textbf{What UNI Holders Don't Get (Yet):}
\begin{itemize}
\item Trading fee revenue
\item Staking rewards
\item Dividends or buybacks
\end{itemize}

\vspace{0.3cm}
\textbf{Value Proposition Debate:}
\begin{itemize}
\item \textbf{Bulls:} Governance over massive treasury is valuable
\item \textbf{Bears:} No cash flow = overvalued compared to fundamentals
\end{itemize}

\vspace{0.3cm}
\textbf{Market Cap:} ~\$5-7B (Dec 2024), making it top 20 crypto by market cap.
\end{frame}

\begin{frame}{Uniswap vs. Competitors}
\begin{tabular}{llll}
\toprule
DEX & Key Feature & Fee Model & TVL \\
\midrule
Uniswap V3 & Concentrated liquidity & 0.01-1\% to LPs & \$3.5B \\
SushiSwap & Fork with fee sharing & 0.25\% LP, 0.05\% xSUSHI & \$300M \\
Curve & Stablecoin optimized & 0.04\% + CRV rewards & \$2B \\
Balancer & Weighted pools & Variable, custom & \$800M \\
PancakeSwap & BSC native, low fees & 0.25\% LP, CAKE rewards & \$1.5B \\
\bottomrule
\end{tabular}

\vspace{0.3cm}
\textbf{Uniswap's Moat:}
\begin{itemize}
\item Brand recognition and trust
\item Deepest liquidity (lowest slippage)
\item Innovation leadership (V3, V4)
\item Integration with wallets and aggregators
\end{itemize}
\end{frame}

\begin{frame}{Uniswap's Multi-Chain Strategy}
\textbf{Deployments (as of Dec 2024):}
\begin{itemize}
\item Ethereum (L1)
\item Polygon (sidechain)
\item Arbitrum (L2 Optimistic Rollup)
\item Optimism (L2 Optimistic Rollup)
\item Base (Coinbase L2)
\item BNB Chain (approved via governance)
\item Celo (mobile-first L1)
\end{itemize}

\vspace{0.3cm}
\textbf{Benefits:}
\begin{itemize}
\item Lower gas fees on L2s (10-100x cheaper)
\item Tap into different user bases
\item Maintain dominance across ecosystems
\end{itemize}

\vspace{0.3cm}
\textbf{Challenge:} Liquidity fragmentation across chains.
\end{frame}

\begin{frame}{Uniswap Grants Program}
\textbf{Uniswap Foundation:} Manages community treasury.

\vspace{0.3cm}
\textbf{Grants Allocated:}
\begin{itemize}
\item \$20M+ in grants since 2021
\item Developer tooling (e.g., SDK improvements)
\item Research (MEV, concentrated liquidity optimization)
\item Community projects (analytics dashboards)
\item Cross-chain infrastructure
\end{itemize}

\vspace{0.3cm}
\textbf{Recent Major Grants:}
\begin{itemize}
\item \$1.5M to Flipside Crypto (analytics)
\item \$500k to GFX Labs (governance tooling)
\item \$300k to Crocswap (V3 UI alternative)
\end{itemize}

\vspace{0.3cm}
\textbf{Goal:} Sustain ecosystem growth without protocol fee revenue.
\end{frame}

\begin{frame}{Uniswap Labs (The Company)}
\textbf{Structure:}
\begin{itemize}
\item \textbf{Uniswap Protocol:} Decentralized, governed by UNI holders
\item \textbf{Uniswap Labs:} For-profit company that develops the protocol
\end{itemize}

\vspace{0.3cm}
\textbf{Funding:}
\begin{itemize}
\item \$165M Series B (2022) at \$1.66B valuation
\item Investors: Polychain, a16z, Paradigm
\end{itemize}

\vspace{0.3cm}
\textbf{Revenue Streams:}
\begin{itemize}
\item No protocol fees (yet)
\item Venture investment in tokens
\item Potential: Interface fees, Uniswap X orderflow
\end{itemize}

\vspace{0.3cm}
\textbf{Controversy:} Some community concern about company vs. protocol alignment.
\end{frame}

\begin{frame}{Regulatory Challenges}
\textbf{SEC Wells Notice (April 2024):}
\begin{itemize}
\item Uniswap Labs received Wells notice (precursor to enforcement)
\item Allegations: Operating unregistered securities exchange
\item Uniswap response: Protocol is decentralized, Labs just builds software
\end{itemize}

\vspace{0.3cm}
\textbf{Legal Arguments:}
\begin{itemize}
\item Protocol vs. Interface distinction
\item Smart contracts are speech (First Amendment)
\item Users, not Uniswap, choose which tokens to list
\end{itemize}

\vspace{0.3cm}
\textbf{Potential Outcomes:}
\begin{itemize}
\item Settlement (delisting certain tokens, geo-blocking)
\item Lawsuit (precedent-setting for DeFi)
\item Favorable ruling (legitimizes decentralized protocols)
\end{itemize}

\vspace{0.3cm}
\textbf{Status:} Ongoing (as of Dec 2024), industry watching closely.
\end{frame}

\begin{frame}{Summary}
\textbf{Key Takeaways:}
\begin{itemize}
\item Uniswap evolved from simple AMM (V1) to concentrated liquidity (V3) to customizable hooks (V4)
\item V3's concentrated liquidity provides 10-200x capital efficiency but requires active management
\item Fee tiers optimize for different asset types (stablecoins 0.01\%, standard 0.3\%)
\item UNI token governs protocol but lacks direct cash flow (fee switch debate)
\item Uniswap dominates DEX market but faces regulatory scrutiny
\item Multi-chain strategy expands reach but fragments liquidity
\end{itemize}

\vspace{0.3cm}
\textbf{Next Lecture:} Lab - Testnet Swap (hands-on experience with Uniswap).
\end{frame}

\begin{frame}{Questions for Reflection}
\begin{enumerate}
\item How does concentrated liquidity improve capital efficiency in V3?
\item Why might activating the fee switch harm Uniswap's competitive position?
\item What are the trade-offs of NFT-based LP positions in V3?
\item How do hooks in V4 enable new DeFi primitives?
\item Should Uniswap prioritize decentralization or regulatory compliance?
\end{enumerate}
\end{frame}

\end{document}
