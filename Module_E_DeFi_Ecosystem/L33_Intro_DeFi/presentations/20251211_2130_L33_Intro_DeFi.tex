\documentclass[8pt,aspectratio=169]{beamer}
\usetheme{Madrid}
\usepackage[utf8]{inputenc}
\usepackage{graphicx}
\usepackage{booktabs}
\usepackage{hyperref}
\usepackage{amsmath}

\newcommand{\bottomnote}[1]{\vfill\par\noindent\footnotesize\textit{#1}}

\title{L33: Introduction to DeFi}
\subtitle{Module E: DeFi Ecosystem}
\author{Blockchain \& Cryptocurrency}
\date{December 2025}

\begin{document}

\frame{\titlepage}

\begin{frame}{Learning Objectives}
\begin{itemize}
\item Define Decentralized Finance (DeFi) and its core principles
\item Understand Total Value Locked (TVL) as a key metric
\item Explore the DeFi technology stack
\item Analyze composability and its implications
\item Compare DeFi to Traditional Finance (TradFi)
\end{itemize}
\end{frame}

\begin{frame}{What is DeFi?}
\textbf{Definition:} Decentralized Finance (DeFi) refers to financial services built on blockchain networks, operating without traditional intermediaries.

\vspace{0.3cm}
\textbf{Core Principles:}
\begin{itemize}
\item \textbf{Permissionless:} Anyone can access without approval
\item \textbf{Transparent:} All transactions visible on blockchain
\item \textbf{Non-custodial:} Users control their own assets
\item \textbf{Composable:} Protocols integrate seamlessly (money legos)
\item \textbf{Programmable:} Smart contracts automate execution
\end{itemize}

\vspace{0.3cm}
\textbf{Vision:} Recreate traditional financial system with greater accessibility, transparency, and efficiency.
\end{frame}

\begin{frame}[t]{DeFi vs. Traditional Finance}
\begin{center}
\includegraphics[width=0.65\textwidth]{../charts/01_defi_vs_tradfi/chart.pdf}
\end{center}
\bottomnote{DeFi excels in accessibility and transparency; TradFi offers stability and user protection}
\end{frame}

\begin{frame}{DeFi vs. TradFi: Key Differences}
\begin{columns}[T]
\begin{column}{0.48\textwidth}
\textbf{Traditional Finance (TradFi)}
\begin{itemize}
\item Centralized intermediaries (banks)
\item KYC/AML requirements
\item Business hours, slow settlements
\item Geographic restrictions
\item High barriers to entry
\end{itemize}
\end{column}

\begin{column}{0.48\textwidth}
\textbf{Decentralized Finance (DeFi)}
\begin{itemize}
\item Smart contracts (no intermediaries)
\item Pseudonymous (wallet addresses)
\item 24/7 operation, instant settlement
\item Global access
\item Low barriers (internet + wallet)
\end{itemize}
\end{column}
\end{columns}

\vspace{0.3cm}
\textbf{Trade-off:} DeFi offers accessibility and transparency but carries smart contract risks and regulatory uncertainty.
\end{frame}

\begin{frame}[t]{The Rise of DeFi: TVL History}
\begin{center}
\includegraphics[width=0.65\textwidth]{../charts/02_tvl_history/chart.pdf}
\end{center}
\bottomnote{DeFi Summer 2020 marked explosive growth; 2022 bear market saw major correction}
\end{frame}

\begin{frame}{Total Value Locked (TVL)}
\textbf{Definition:} The total amount of assets deposited in DeFi protocols, measured in USD.

\vspace{0.3cm}
\textbf{What TVL Measures:}
\begin{itemize}
\item Capital deployed across lending, DEXs, staking, derivatives
\item Proxy for DeFi adoption and trust
\item Indicator of liquidity depth
\end{itemize}

\vspace{0.3cm}
\textbf{Current State (Late 2024):}
\begin{itemize}
\item Total DeFi TVL: ~\$85 billion (recovered from 2022 lows)
\item Ethereum: ~55\% of TVL
\item Layer 2s (Arbitrum, Base): ~12\% combined
\end{itemize}
\end{frame}

\begin{frame}{TVL Calculation Example}
\textbf{Hypothetical Lending Protocol:}

\vspace{0.3cm}
\textbf{Deposits:}
\begin{itemize}
\item 1,000 ETH at \$2,000/ETH = \$2,000,000
\item 500,000 USDC at \$1/USDC = \$500,000
\item 10 BTC at \$40,000/BTC = \$400,000
\end{itemize}

\vspace{0.3cm}
\textbf{Total TVL:}
\[
\text{TVL} = \$2{,}000{,}000 + \$500{,}000 + \$400{,}000 = \$2{,}900{,}000
\]

\vspace{0.3cm}
\textbf{Note:} TVL fluctuates with crypto prices; double-counting can inflate numbers.
\end{frame}

\begin{frame}[t]{The DeFi Technology Stack}
\begin{center}
\includegraphics[width=0.65\textwidth]{../charts/03_defi_stack/chart.pdf}
\end{center}
\bottomnote{Composability allows protocols to build on each other like ``money legos''}
\end{frame}

\begin{frame}{Core DeFi Primitives}
\textbf{1. Decentralized Exchanges (DEXs)}
\begin{itemize}
\item Token swapping without intermediaries (Uniswap, Curve)
\end{itemize}

\textbf{2. Lending \& Borrowing}
\begin{itemize}
\item Earn interest on deposits, borrow against collateral (Aave, Compound)
\end{itemize}

\textbf{3. Stablecoins}
\begin{itemize}
\item Price-stable cryptocurrencies (USDC, DAI)
\end{itemize}

\textbf{4. Derivatives}
\begin{itemize}
\item Futures, options, synthetic assets (dYdX, GMX)
\end{itemize}

\textbf{5. Yield Aggregators}
\begin{itemize}
\item Automated yield optimization (Yearn Finance)
\end{itemize}
\end{frame}

\begin{frame}[t]{Top DeFi Protocols by TVL}
\begin{center}
\includegraphics[width=0.65\textwidth]{../charts/04_protocol_tvl/chart.pdf}
\end{center}
\bottomnote{Liquid staking (Lido) and restaking (EigenLayer) dominate; lending and DEXs follow}
\end{frame}

\begin{frame}{Composability: Money Legos}
\textbf{Definition:} DeFi protocols can interact seamlessly, allowing complex strategies by combining simple primitives.

\vspace{0.3cm}
\textbf{Example Workflow:}
\begin{enumerate}
\item Deposit ETH in Aave, receive aETH (interest-bearing token)
\item Use aETH as collateral to borrow DAI
\item Swap DAI for USDC on Uniswap
\item Deposit USDC in Curve for yield farming
\end{enumerate}

\vspace{0.3cm}
\textbf{Benefits:} Capital efficiency, innovation from combinations

\textbf{Risks:} Complexity increases attack surface, protocol failure can cascade
\end{frame}

\begin{frame}[t]{DeFi TVL by Blockchain}
\vspace{-0.2cm}
\begin{center}
\includegraphics[width=0.45\textwidth]{../charts/05_chain_tvl_share/chart.pdf}
\end{center}
\bottomnote{Ethereum dominates; L2s growing rapidly}
\end{frame}

\begin{frame}{Smart Contract Risk}
\textbf{Definition:} Bugs, exploits, or design flaws in smart contract code.

\vspace{0.3cm}
\textbf{Common Vulnerabilities:}
\begin{itemize}
\item Reentrancy attacks (famous: DAO hack 2016)
\item Oracle manipulation
\item Front-running and MEV exploitation
\item Access control failures
\end{itemize}

\vspace{0.3cm}
\textbf{Mitigation:}
\begin{itemize}
\item Professional audits (Trail of Bits, OpenZeppelin)
\item Bug bounties, formal verification
\item Time-locks and multi-sig governance
\end{itemize}
\end{frame}

\begin{frame}{Oracle Problem}
\textbf{Challenge:} Smart contracts can't natively access off-chain data (e.g., ETH price).

\vspace{0.3cm}
\textbf{Solution: Oracles}
\begin{itemize}
\item Third-party services that feed external data on-chain
\item Example: Chainlink (decentralized oracle network)
\end{itemize}

\vspace{0.3cm}
\textbf{Oracle Types:}
\begin{itemize}
\item \textbf{Centralized:} Single trusted source (fast but risky)
\item \textbf{Decentralized:} Multiple providers aggregated (Chainlink)
\item \textbf{On-chain:} Data derived from blockchain state (Uniswap TWAP)
\end{itemize}

\textbf{Risk:} Oracle manipulation can drain DeFi protocols (flash loan attacks).
\end{frame}

\begin{frame}{Permissionless and Non-Custodial}
\textbf{Permissionless Access:}
\begin{itemize}
\item No identity verification, no geographic restrictions
\item Only need: internet connection + crypto wallet
\item Benefits: Financial inclusion, censorship resistance
\end{itemize}

\vspace{0.3cm}
\textbf{Non-Custodial Finance:}
\begin{itemize}
\item You hold private keys, smart contract holds funds during interaction
\item No third party can freeze or seize
\item \textbf{Positive:} True ownership, no counterparty risk
\item \textbf{Negative:} No recovery if you lose keys
\end{itemize}

\vspace{0.3cm}
\textbf{Mantra:} Not your keys, not your coins.
\end{frame}

\begin{frame}[t]{2022 CeFi Collapses: The Case for DeFi}
\begin{center}
\includegraphics[width=0.65\textwidth]{../charts/06_cefi_collapses/chart.pdf}
\end{center}
\bottomnote{CeFi custodial risk exposed; DeFi protocols like Aave and Uniswap operated normally}
\end{frame}

\begin{frame}{DeFi vs. CeFi Comparison}
\textbf{Centralized Crypto Platforms:} Coinbase, Binance, BlockFi, Celsius

\vspace{0.3cm}
\begin{columns}[T]
\begin{column}{0.48\textwidth}
\textbf{CeFi Advantages}
\begin{itemize}
\item User-friendly interfaces
\item Customer support
\item Fiat on/off ramps
\end{itemize}
\end{column}

\begin{column}{0.48\textwidth}
\textbf{CeFi Risks}
\begin{itemize}
\item Custodial (platform holds assets)
\item Counterparty risk (FTX collapse)
\item Can freeze accounts
\end{itemize}
\end{column}
\end{columns}

\vspace{0.3cm}
\textbf{2022 Lesson:} Multiple CeFi platforms collapsed (Celsius, FTX), highlighting custodial risk. DeFi protocols survived.
\end{frame}

\begin{frame}{2024 Innovation: Restaking (EigenLayer)}
\textbf{What is Restaking?}
\begin{itemize}
\item Reusing staked ETH to secure additional networks/services
\item Introduced by EigenLayer (major growth 2024)
\end{itemize}

\vspace{0.3cm}
\textbf{How It Works:}
\begin{enumerate}
\item Stake ETH with Ethereum validators (earn ~3-4\% APY)
\item Opt-in to restaking via EigenLayer
\item Earn extra yield from securing additional services (AVS)
\end{enumerate}

\vspace{0.3cm}
\textbf{Impact:}
\begin{itemize}
\item \$15B+ TVL in EigenLayer by late 2024
\item Liquid Restaking Tokens (LRTs): eETH, rsETH, ezETH
\item Criticism: Added systemic risk, complexity
\end{itemize}
\end{frame}

\begin{frame}{Future of DeFi}
\textbf{Emerging Trends:}
\begin{itemize}
\item \textbf{Real-World Assets (RWA):} Tokenizing bonds, real estate
\item \textbf{Undercollateralized Lending:} Credit scoring on-chain
\item \textbf{Cross-Chain DeFi:} Seamless interaction across blockchains
\item \textbf{Institutional Adoption:} Banks exploring DeFi rails
\item \textbf{Regulation:} Clearer frameworks emerging (MiCA in EU)
\end{itemize}

\vspace{0.3cm}
\textbf{Long-Term Vision:}
\begin{itemize}
\item DeFi as backend infrastructure for TradFi
\item 24/7 settlement for global finance
\item Financial inclusion for billions
\end{itemize}
\end{frame}

\begin{frame}{Summary}
\textbf{Key Takeaways:}
\begin{itemize}
\item DeFi recreates financial services on blockchain: permissionless, transparent, non-custodial
\item TVL measures capital deployed (~\$85B in late 2024)
\item Composability enables innovation but increases complexity
\item Smart contract risk and oracle manipulation are key concerns
\item Restaking (EigenLayer) emerged as major 2024 innovation
\item Ethereum dominates but L2s capturing increasing share
\end{itemize}

\vspace{0.3cm}
\textbf{Next Lecture:} AMM Mechanics - How automated market makers work.
\end{frame}

\begin{frame}{Questions for Reflection}
\begin{enumerate}
\item How does TVL differ from traditional finance metrics like AUM?
\item Why is composability both a strength and a risk in DeFi?
\item What are the trade-offs between DeFi and CeFi for retail users?
\item How do oracles solve the external data problem, and what risks remain?
\item What regulatory challenges does DeFi face in the next 5 years?
\end{enumerate}
\end{frame}

\end{document}
