\documentclass[8pt,aspectratio=169]{beamer}
\usetheme{Madrid}
\usepackage[utf8]{inputenc}
\usepackage{graphicx}
\usepackage{booktabs}
\usepackage{hyperref}
\usepackage{amsmath}

\title{L33: Introduction to DeFi}
\subtitle{Module E: DeFi Ecosystem}
\author{Blockchain \& Cryptocurrency}
\date{December 2025}

\begin{document}

\frame{\titlepage}

\begin{frame}{Learning Objectives}
\begin{itemize}
\item Define Decentralized Finance (DeFi) and its core principles
\item Understand Total Value Locked (TVL) as a key metric
\item Explore the DeFi technology stack
\item Analyze composability and its implications
\item Compare DeFi to Traditional Finance (TradFi)
\end{itemize}
\end{frame}

\begin{frame}{What is DeFi?}
\textbf{Definition:} Decentralized Finance (DeFi) refers to financial services built on blockchain networks, operating without traditional intermediaries.

\vspace{0.3cm}
\textbf{Core Principles:}
\begin{itemize}
\item \textbf{Permissionless:} Anyone can access without approval
\item \textbf{Transparent:} All transactions visible on blockchain
\item \textbf{Non-custodial:} Users control their own assets
\item \textbf{Composable:} Protocols integrate seamlessly (money legos)
\item \textbf{Programmable:} Smart contracts automate execution
\end{itemize}

\vspace{0.3cm}
\textbf{Vision:} Recreate traditional financial system with greater accessibility, transparency, and efficiency.
\end{frame}

\begin{frame}{DeFi vs. Traditional Finance}
\begin{columns}[T]
\begin{column}{0.48\textwidth}
\textbf{Traditional Finance (TradFi)}
\begin{itemize}
\item Centralized intermediaries (banks)
\item KYC/AML requirements
\item Business hours, slow settlements
\item Geographic restrictions
\item Opaque processes
\item Custodial (bank holds assets)
\item High barriers to entry
\end{itemize}
\end{column}

\begin{column}{0.48\textwidth}
\textbf{Decentralized Finance (DeFi)}
\begin{itemize}
\item Smart contracts (no intermediaries)
\item Pseudonymous (wallet addresses)
\item 24/7 operation, instant settlement
\item Global access
\item Transparent on-chain data
\item Non-custodial (user controls keys)
\item Low barriers (internet + wallet)
\end{itemize}
\end{column}
\end{columns}

\vspace{0.3cm}
\textbf{Trade-off:} DeFi offers accessibility and transparency but carries smart contract risks and regulatory uncertainty.
\end{frame}

\begin{frame}{The Rise of DeFi}
\textbf{Historical Timeline:}
\begin{itemize}
\item \textbf{2015:} Ethereum launches, enabling smart contracts
\item \textbf{2017:} MakerDAO creates DAI stablecoin
\item \textbf{2018:} Uniswap V1 introduces AMM model
\item \textbf{2019:} Compound launches money markets
\item \textbf{2020:} DeFi Summer - explosive TVL growth (yield farming)
\item \textbf{2021:} Peak TVL reaches \$180B+
\item \textbf{2022:} Bear market, TVL drops to \$40B
\item \textbf{2024:} Recovery to \$50-60B range
\end{itemize}

\vspace{0.3cm}
\textbf{Key Insight:} DeFi cycles with broader crypto market but maintains core innovation.
\end{frame}

\begin{frame}{Total Value Locked (TVL)}
\textbf{Definition:} The total amount of assets deposited in DeFi protocols, measured in USD.

\vspace{0.3cm}
\textbf{What TVL Measures:}
\begin{itemize}
\item Capital deployed across lending, DEXs, staking, derivatives
\item Proxy for DeFi adoption and trust
\item Indicator of liquidity depth
\end{itemize}

\vspace{0.3cm}
\textbf{Current State (December 2024):}
\begin{itemize}
\item Total DeFi TVL: ~\$50-55 billion
\item Ethereum: ~55\% of TVL
\item Binance Smart Chain: ~10\%
\item Solana, Avalanche, Polygon: ~5-8\% each
\end{itemize}

\vspace{0.3cm}
\textbf{Top Protocols by TVL:}
\begin{enumerate}
\item Lido (liquid staking): ~\$20B
\item MakerDAO (stablecoin): ~\$5B
\item Aave (lending): ~\$4B
\item Uniswap (DEX): ~\$3.5B
\end{enumerate}
\end{frame}

\begin{frame}{TVL Calculation Example}
\textbf{Hypothetical Lending Protocol:}

\vspace{0.3cm}
\textbf{Deposits:}
\begin{itemize}
\item 1,000 ETH at \$2,000/ETH = \$2,000,000
\item 500,000 USDC at \$1/USDC = \$500,000
\item 10 BTC at \$40,000/BTC = \$400,000
\end{itemize}

\vspace{0.3cm}
\textbf{Total TVL:}
\[
\text{TVL} = \$2{,}000{,}000 + \$500{,}000 + \$400{,}000 = \$2{,}900{,}000
\]

\vspace{0.3cm}
\textbf{Note:}
\begin{itemize}
\item TVL fluctuates with crypto prices
\item Double-counting issue: same assets deposited multiple times
\item True TVL vs. inflated TVL (some trackers adjust for this)
\end{itemize}
\end{frame}

\begin{frame}{The DeFi Stack (Layers)}
\textbf{Layer 1: Settlement Layer}
\begin{itemize}
\item Base blockchain (Ethereum, Solana, etc.)
\item Consensus and transaction finality
\end{itemize}

\textbf{Layer 2: Asset Layer}
\begin{itemize}
\item Native tokens (ETH, BTC) and stablecoins (USDC, DAI)
\end{itemize}

\textbf{Layer 3: Protocol Layer}
\begin{itemize}
\item Smart contracts for DeFi services (Uniswap, Aave, Compound)
\end{itemize}

\textbf{Layer 4: Application Layer}
\begin{itemize}
\item User interfaces (web apps, wallets)
\end{itemize}

\textbf{Layer 5: Aggregation Layer}
\begin{itemize}
\item Meta-protocols (1inch, Yearn) that route across multiple protocols
\end{itemize}
\end{frame}

\begin{frame}{Core DeFi Primitives}
\textbf{1. Decentralized Exchanges (DEXs)}
\begin{itemize}
\item Token swapping without intermediaries
\item Examples: Uniswap, SushiSwap, Curve
\end{itemize}

\textbf{2. Lending \& Borrowing}
\begin{itemize}
\item Earn interest on deposits, borrow against collateral
\item Examples: Aave, Compound, MakerDAO
\end{itemize}

\textbf{3. Stablecoins}
\begin{itemize}
\item Price-stable cryptocurrencies
\item Examples: USDC (fiat-backed), DAI (crypto-backed)
\end{itemize}

\textbf{4. Derivatives}
\begin{itemize}
\item Futures, options, synthetic assets
\item Examples: dYdX, Synthetix, GMX
\end{itemize}

\textbf{5. Yield Aggregators}
\begin{itemize}
\item Automated yield optimization
\item Examples: Yearn Finance, Beefy
\end{itemize}
\end{frame}

\begin{frame}{Composability: Money Legos}
\textbf{Definition:} DeFi protocols can interact seamlessly, allowing complex strategies by combining simple primitives.

\vspace{0.3cm}
\textbf{Example Workflow:}
\begin{enumerate}
\item Deposit ETH in Aave, receive aETH (interest-bearing token)
\item Use aETH as collateral to borrow DAI
\item Swap DAI for USDC on Uniswap
\item Deposit USDC in Curve for yield farming
\item Stake Curve LP tokens in Convex for boosted rewards
\end{enumerate}

\vspace{0.3cm}
\textbf{Benefits:}
\begin{itemize}
\item Capital efficiency (same asset used multiple times)
\item Innovation (new strategies emerge from combinations)
\item User choice (pick best yields across protocols)
\end{itemize}

\textbf{Risks:}
\begin{itemize}
\item Complexity increases attack surface
\item One protocol failure can cascade
\end{itemize}
\end{frame}

\begin{frame}{Smart Contract Risk}
\textbf{Definition:} Bugs, exploits, or design flaws in smart contract code.

\vspace{0.3cm}
\textbf{Common Vulnerabilities:}
\begin{itemize}
\item Reentrancy attacks (famous: DAO hack 2016)
\item Integer overflow/underflow
\item Oracle manipulation
\item Front-running and MEV exploitation
\item Access control failures
\end{itemize}

\vspace{0.3cm}
\textbf{Mitigation:}
\begin{itemize}
\item Professional audits (Trail of Bits, OpenZeppelin, etc.)
\item Bug bounties (incentivize white-hat hackers)
\item Formal verification (mathematical proofs of correctness)
\item Time-locks and multi-sig governance
\end{itemize}

\vspace{0.3cm}
\textbf{Reality:} Billions lost to hacks/exploits, but security improving over time.
\end{frame}

\begin{frame}{Oracle Problem}
\textbf{Challenge:} Smart contracts can't natively access off-chain data (e.g., ETH price, weather).

\vspace{0.3cm}
\textbf{Solution: Oracles}
\begin{itemize}
\item Third-party services that feed external data on-chain
\item Example: Chainlink (decentralized oracle network)
\end{itemize}

\vspace{0.3cm}
\textbf{Oracle Types:}
\begin{itemize}
\item \textbf{Centralized:} Single trusted source (fast but risky)
\item \textbf{Decentralized:} Multiple data providers aggregated (Chainlink)
\item \textbf{On-chain:} Data derived from blockchain state (Uniswap TWAP)
\end{itemize}

\vspace{0.3cm}
\textbf{Risk:} Oracle manipulation can drain DeFi protocols (flash loan attacks).

\textbf{Example Attack:} Manipulate price oracle, borrow max, liquidate yourself profitably.
\end{frame}

\begin{frame}{Permissionless Access}
\textbf{What it means:}
\begin{itemize}
\item No identity verification required
\item No geographic restrictions
\item No credit checks or approval process
\item Only need: internet connection + crypto wallet
\end{itemize}

\vspace{0.3cm}
\textbf{Benefits:}
\begin{itemize}
\item Financial inclusion (unbanked populations)
\item Censorship resistance
\item Privacy (pseudonymous transactions)
\item Fast onboarding
\end{itemize}

\vspace{0.3cm}
\textbf{Drawbacks:}
\begin{itemize}
\item Money laundering concerns
\item No consumer protections
\item No recourse for errors/scams
\item Regulatory uncertainty
\end{itemize}
\end{frame}

\begin{frame}{Non-Custodial Finance}
\textbf{Traditional Finance:}
\begin{itemize}
\item Bank holds your money
\item You trust bank to not freeze accounts
\item Bank can restrict access
\end{itemize}

\textbf{DeFi:}
\begin{itemize}
\item You hold private keys
\item Smart contract holds funds during interaction
\item No third party can freeze or seize
\end{itemize}

\vspace{0.3cm}
\textbf{Implications:}
\begin{itemize}
\item \textbf{Positive:} True ownership, no counterparty risk
\item \textbf{Negative:} No recovery if you lose keys
\item \textbf{Responsibility:} User must secure own assets
\end{itemize}

\vspace{0.3cm}
\textbf{Mantra:} Not your keys, not your coins.
\end{frame}

\begin{frame}{Transparency and Auditability}
\textbf{All DeFi transactions are public:}
\begin{itemize}
\item Every trade, deposit, borrow visible on-chain
\item Wallet balances and positions are transparent
\item Smart contract code is open-source (usually)
\end{itemize}

\vspace{0.3cm}
\textbf{Benefits:}
\begin{itemize}
\item Anyone can audit protocol reserves
\item Real-time risk monitoring
\item Trust through verification, not authority
\end{itemize}

\vspace{0.3cm}
\textbf{Trade-offs:}
\begin{itemize}
\item Privacy concerns (address linkage to identity)
\item Front-running opportunities (MEV)
\item Competitive intelligence (whales tracked)
\end{itemize}

\vspace{0.3cm}
\textbf{Tools:} Etherscan, Dune Analytics, DeBank for on-chain analysis.
\end{frame}

\begin{frame}{DeFi Use Cases}
\textbf{1. High-Yield Savings}
\begin{itemize}
\item Earn 3-10\% APY on stablecoins (vs. <1\% in banks)
\end{itemize}

\textbf{2. Borrowing without Credit Checks}
\begin{itemize}
\item Collateralized loans (over-collateralized)
\end{itemize}

\textbf{3. Cross-Border Payments}
\begin{itemize}
\item Fast, cheap transfers without intermediaries
\end{itemize}

\textbf{4. Trading 24/7}
\begin{itemize}
\item DEXs never close, no geographic restrictions
\end{itemize}

\textbf{5. Yield Farming}
\begin{itemize}
\item Provide liquidity, earn fees + token rewards
\end{itemize}

\textbf{6. Synthetic Assets}
\begin{itemize}
\item Gain exposure to stocks, commodities on-chain
\end{itemize}
\end{frame}

\begin{frame}{Challenges Facing DeFi}
\textbf{1. Scalability}
\begin{itemize}
\item Ethereum congestion leads to high gas fees
\item Layer 2s and alt-chains offer solutions
\end{itemize}

\textbf{2. User Experience}
\begin{itemize}
\item Complex interfaces, steep learning curve
\item Risk of user error (wrong address, lost keys)
\end{itemize}

\textbf{3. Regulatory Uncertainty}
\begin{itemize}
\item Unclear legal status in many jurisdictions
\item Potential for restrictive regulations
\end{itemize}

\textbf{4. Security Risks}
\begin{itemize}
\item Smart contract bugs, hacks, exploits
\item No insurance (generally)
\end{itemize}

\textbf{5. Centralization Concerns}
\begin{itemize}
\item Some protocols have admin keys
\item Governance token concentration
\end{itemize}
\end{frame}

\begin{frame}{DeFi vs. CeFi (Centralized Finance)}
\textbf{Centralized Crypto Platforms:} Coinbase, Binance, BlockFi, Celsius

\vspace{0.3cm}
\begin{columns}[T]
\begin{column}{0.48\textwidth}
\textbf{CeFi Advantages}
\begin{itemize}
\item User-friendly interfaces
\item Customer support
\item Insurance (sometimes)
\item Fiat on/off ramps
\item Higher yields (historically)
\end{itemize}
\end{column}

\begin{column}{0.48\textwidth}
\textbf{CeFi Risks}
\begin{itemize}
\item Custodial (platform holds assets)
\item Counterparty risk (FTX, Celsius collapses)
\item KYC requirements
\item Geographic restrictions
\item Can freeze accounts
\end{itemize}
\end{column}
\end{columns}

\vspace{0.3cm}
\textbf{2022 Lesson:} Multiple CeFi platforms collapsed (Celsius, Voyager, FTX), highlighting custodial risk. DeFi protocols (mostly) survived.
\end{frame}

\begin{frame}{Major DeFi Protocols Overview}
\begin{tabular}{lll}
\toprule
Protocol & Category & TVL (approx.) \\
\midrule
Lido & Liquid Staking & \$20B \\
MakerDAO & Stablecoin \& Lending & \$5B \\
Aave & Lending & \$4B \\
Uniswap & DEX & \$3.5B \\
Curve & Stablecoin DEX & \$2B \\
Compound & Lending & \$1.5B \\
Rocket Pool & Liquid Staking & \$1.2B \\
Convex & Yield Aggregator & \$1B \\
\bottomrule
\end{tabular}

\vspace{0.3cm}
\textbf{Note:} TVL values fluctuate with market conditions (Dec 2024 estimates).
\end{frame}

\begin{frame}{Ethereum Dominance in DeFi}
\textbf{Why Ethereum Leads:}
\begin{itemize}
\item First-mover advantage (smart contracts since 2015)
\item Largest developer ecosystem
\item Most battle-tested protocols
\item Highest liquidity and composability
\item Strong network effects
\end{itemize}

\vspace{0.3cm}
\textbf{Challenges:}
\begin{itemize}
\item High gas fees during congestion
\item Slower finality than newer chains
\end{itemize}

\vspace{0.3cm}
\textbf{Competitors:}
\begin{itemize}
\item \textbf{Binance Smart Chain:} Cheaper fees, more centralized
\item \textbf{Solana:} Fast, low-cost, but less battle-tested
\item \textbf{Avalanche, Polygon:} Ethereum-compatible, lower fees
\end{itemize}
\end{frame}

\begin{frame}{Future of DeFi}
\textbf{Emerging Trends:}
\begin{itemize}
\item \textbf{Real-World Assets (RWA):} Tokenizing bonds, real estate
\item \textbf{Undercollateralized Lending:} Credit scoring on-chain
\item \textbf{Cross-Chain DeFi:} Seamless interaction across blockchains
\item \textbf{Institutional Adoption:} Banks exploring DeFi rails
\item \textbf{Regulation:} Clearer frameworks emerging (MiCA in EU)
\end{itemize}

\vspace{0.3cm}
\textbf{Long-Term Vision:}
\begin{itemize}
\item DeFi as backend infrastructure for TradFi
\item 24/7 settlement for global finance
\item Financial inclusion for billions
\item Programmable money and automated compliance
\end{itemize}
\end{frame}

\begin{frame}{Summary}
\textbf{Key Takeaways:}
\begin{itemize}
\item DeFi recreates financial services on blockchain: permissionless, transparent, non-custodial
\item TVL measures capital deployed (~\$50B currently)
\item Composability enables innovation but increases complexity
\item Smart contract risk and oracle manipulation are key concerns
\item DeFi outperformed CeFi during 2022 crisis (decentralization mattered)
\item Ethereum dominates but faces competition from faster, cheaper chains
\end{itemize}

\vspace{0.3cm}
\textbf{Next Lecture:} AMM Mechanics - How automated market makers work (Uniswap model).
\end{frame}

\begin{frame}{Questions for Reflection}
\begin{enumerate}
\item How does TVL differ from traditional finance metrics like AUM?
\item Why is composability both a strength and a risk in DeFi?
\item What are the trade-offs between DeFi and CeFi for retail users?
\item How do oracles solve the external data problem, and what risks remain?
\item What regulatory challenges does DeFi face in the next 5 years?
\end{enumerate}
\end{frame}

\end{document}
