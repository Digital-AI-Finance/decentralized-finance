\documentclass[8pt,aspectratio=169]{beamer}
\usetheme{Madrid}
\usepackage[utf8]{inputenc}
\usepackage{graphicx}
\usepackage{booktabs}
\usepackage{hyperref}
\usepackage{amsmath}

\title{L37: Lending Protocols}
\subtitle{Module E: DeFi Ecosystem}
\author{Blockchain \& Cryptocurrency}
\date{December 2025}

\begin{document}

\frame{\titlepage}

\begin{frame}{Learning Objectives}
\begin{itemize}
\item Understand how DeFi lending protocols work (Aave, Compound)
\item Analyze overcollateralization and its necessity
\item Calculate health factors and liquidation thresholds
\item Explore interest rate models (utilization-based)
\item Compare DeFi lending to traditional finance
\end{itemize}
\end{frame}

\begin{frame}{Traditional Lending vs. DeFi Lending}
\begin{columns}[T]
\begin{column}{0.48\textwidth}
\textbf{Traditional Lending}
\begin{itemize}
\item Credit checks required
\item Identity verification (KYC)
\item Undercollateralized (borrow more than collateral)
\item Long approval process
\item Fixed terms and rates
\item Geographic restrictions
\item Intermediaries (banks)
\end{itemize}
\end{column}

\begin{column}{0.48\textwidth}
\textbf{DeFi Lending}
\begin{itemize}
\item No credit checks
\item Pseudonymous (wallet address)
\item Overcollateralized (deposit > borrow)
\item Instant approval
\item Variable rates (algorithmic)
\item Global access
\item Smart contracts (no intermediaries)
\end{itemize}
\end{column}
\end{columns}

\vspace{0.3cm}
\textbf{Key Trade-off:} DeFi accessibility vs. capital efficiency (must lock more than you borrow).
\end{frame}

\begin{frame}{How DeFi Lending Works}
\textbf{Core Mechanism:}

\vspace{0.3cm}
\begin{enumerate}
\item \textbf{Lenders (Suppliers)}
\begin{itemize}
\item Deposit assets (ETH, USDC, DAI, etc.) into protocol
\item Earn interest on deposits (APY varies by utilization)
\item Can withdraw anytime (if liquidity available)
\end{itemize}

\item \textbf{Borrowers}
\begin{itemize}
\item Deposit collateral (e.g., ETH)
\item Borrow up to a percentage of collateral value (e.g., 75\%)
\item Pay interest to lenders
\item Must maintain health factor above 1
\end{itemize}

\item \textbf{Protocol}
\begin{itemize}
\item Matches lenders and borrowers via smart contracts
\item Sets interest rates algorithmically
\item Manages liquidations
\end{itemize}
\end{enumerate}
\end{frame}

\begin{frame}{Example: Borrowing on Aave}
\textbf{User Journey:}

\vspace{0.3cm}
\textbf{Step 1: Supply Collateral}
\begin{itemize}
\item Deposit 10 ETH (worth \$20,000 at \$2,000/ETH)
\item Receive aETH tokens (interest-bearing, 1:1 with ETH)
\end{itemize}

\textbf{Step 2: Borrow}
\begin{itemize}
\item ETH has 80\% Loan-to-Value (LTV) ratio
\item Max borrow: $10 \times \$2{,}000 \times 0.80 = \$16{,}000$
\item User borrows \$10,000 USDC (safe margin)
\end{itemize}

\textbf{Step 3: Manage Position}
\begin{itemize}
\item Interest accrues on \$10,000 debt (e.g., 5\% APY)
\item User earns interest on 10 ETH deposit (e.g., 2\% APY)
\item Net cost: 5\% - 2\% = 3\% on borrowed amount
\end{itemize}
\end{frame}

\begin{frame}{Why Overcollateralization?}
\textbf{Problem:} No identity means no recourse if borrower defaults.

\vspace{0.3cm}
\textbf{Solution:} Require collateral worth more than loan.

\vspace{0.3cm}
\textbf{Mechanism:}
\begin{itemize}
\item If collateral value drops, protocol liquidates automatically
\item Liquidators repay debt, seize collateral (with bonus)
\item Lenders remain whole
\end{itemize}

\vspace{0.3cm}
\textbf{Trade-off:}
\begin{itemize}
\item \textbf{Capital inefficient:} Can't access full collateral value
\item \textbf{Safe for lenders:} Minimal default risk
\end{itemize}

\vspace{0.3cm}
\textbf{Example Use Cases:}
\begin{itemize}
\item Leverage (borrow USDC, buy more ETH for long exposure)
\item Short selling (borrow ETH, sell it, buy back cheaper)
\item Liquidity without selling (keep ETH, borrow USDC for expenses)
\end{itemize}
\end{frame}

\begin{frame}{Loan-to-Value (LTV) Ratio}
\textbf{Definition:}
\[
\text{LTV} = \frac{\text{Borrowed Value}}{\text{Collateral Value}} \times 100\%
\]

\vspace{0.3cm}
\textbf{Example LTV Limits (Aave):}
\begin{itemize}
\item ETH: 80\% (can borrow up to 80\% of ETH value)
\item WBTC: 75\%
\item USDC: 85\% (stablecoin, less volatile)
\item Low-cap tokens: 40-60\% (higher risk)
\end{itemize}

\vspace{0.3cm}
\textbf{Why Different LTVs?}
\begin{itemize}
\item Volatile assets = lower LTV (more buffer for price swings)
\item Stable assets = higher LTV (minimal price risk)
\end{itemize}

\vspace{0.3cm}
\textbf{Calculation:}
\begin{itemize}
\item Collateral: 10 ETH at \$2,000 = \$20,000
\item Borrowed: \$12,000 USDC
\item LTV: $\frac{12{,}000}{20{,}000} = 60\%$ (safe, under 80\% max)
\end{itemize}
\end{frame}

\begin{frame}{Liquidation Threshold}
\textbf{Definition:} LTV at which liquidation occurs (always higher than max LTV).

\vspace{0.3cm}
\textbf{Example (Aave):}
\begin{itemize}
\item ETH Max LTV: 80\%
\item ETH Liquidation Threshold: 83\%
\item Buffer: 3\% (gives borrower time to add collateral)
\end{itemize}

\vspace{0.3cm}
\textbf{Scenario:}
\begin{itemize}
\item User borrows at 80\% LTV
\item ETH price drops 5\%
\item New LTV: $80\% \times \frac{1}{0.95} \approx 84.2\%$ (exceeds 83\% threshold)
\item \textbf{Liquidation triggered}
\end{itemize}

\vspace{0.3cm}
\textbf{Liquidation Penalty:}
\begin{itemize}
\item Liquidator repays debt, receives collateral + bonus (e.g., 5\%)
\item Borrower loses penalty amount
\item Remaining collateral returned to borrower
\end{itemize}
\end{frame}

\begin{frame}{Health Factor}
\textbf{Definition:} Metric indicating position safety.

\vspace{0.3cm}
\textbf{Formula (Aave):}
\[
\text{Health Factor} = \frac{\text{Collateral Value} \times \text{Liquidation Threshold}}{\text{Borrowed Value}}
\]

\vspace{0.3cm}
\textbf{Interpretation:}
\begin{itemize}
\item Health Factor > 1: Safe (no liquidation)
\item Health Factor = 1: At liquidation threshold
\item Health Factor < 1: Liquidation occurs
\end{itemize}

\vspace{0.3cm}
\textbf{Example:}
\begin{itemize}
\item Collateral: \$20,000 (ETH)
\item Borrowed: \$12,000 (USDC)
\item Liquidation threshold: 83\%
\end{itemize}
\[
\text{HF} = \frac{20{,}000 \times 0.83}{12{,}000} = 1.38
\]

\textbf{Safe:} 38\% buffer before liquidation.
\end{frame}

\begin{frame}{Health Factor Example: Price Drop}
\textbf{Initial State:}
\begin{itemize}
\item 10 ETH at \$2,000 = \$20,000 collateral
\item \$12,000 USDC borrowed
\item Liquidation threshold: 83\%
\item Health Factor: $\frac{20{,}000 \times 0.83}{12{,}000} = 1.38$
\end{itemize}

\vspace{0.3cm}
\textbf{ETH drops to \$1,600:}
\begin{itemize}
\item New collateral value: $10 \times \$1{,}600 = \$16{,}000$
\item Borrowed still: \$12,000
\item New HF: $\frac{16{,}000 \times 0.83}{12{,}000} = 1.11$ (still safe)
\end{itemize}

\textbf{ETH drops to \$1,446:}
\begin{itemize}
\item New collateral value: $10 \times \$1{,}446 = \$14{,}460$
\item New HF: $\frac{14{,}460 \times 0.83}{12{,}000} = 1.00$ (liquidation threshold)
\item \textbf{Position liquidated}
\end{itemize}
\end{frame}

\begin{frame}{Liquidation Process}
\textbf{When Health Factor < 1:}

\vspace{0.3cm}
\begin{enumerate}
\item \textbf{Liquidator Bot Detects} unhealthy position
\item \textbf{Liquidator Repays} portion of debt (up to 50\%)
\item \textbf{Liquidator Receives} equivalent collateral + bonus (5-10\%)
\item \textbf{Borrower Loses} liquidation penalty
\item \textbf{Remaining Collateral} returned to borrower
\end{enumerate}

\vspace{0.3cm}
\textbf{Example:}
\begin{itemize}
\item Debt: \$12,000 USDC
\item Collateral: \$14,460 ETH (at \$1,446/ETH)
\item Liquidator repays 50\%: \$6,000
\item Liquidator receives: $\$6{,}000 \times 1.05 = \$6{,}300$ in ETH
\item Borrower left with: \$14,460 - \$6,300 = \$8,160 collateral and \$6,000 debt
\end{itemize}

\vspace{0.3cm}
\textbf{Penalty:} Borrower lost \$300 (5\% of liquidated amount).
\end{frame}

\begin{frame}{Interest Rate Models}
\textbf{Key Concept:} Rates adjust based on utilization.

\vspace{0.3cm}
\textbf{Utilization Rate:}
\[
U = \frac{\text{Total Borrowed}}{\text{Total Supplied}}
\]

\textbf{Interest Rates:}
\begin{itemize}
\item Low utilization (0-60\%): Low borrow rates (encourage borrowing)
\item Optimal utilization (~80\%): Moderate rates (balanced)
\item High utilization (>90\%): High rates (encourage repayment, discourage borrowing)
\end{itemize}

\vspace{0.3cm}
\textbf{Example (USDC on Aave):}
\begin{itemize}
\item 0\% utilization: 1\% borrow APY
\item 80\% utilization: 5\% borrow APY
\item 95\% utilization: 20\% borrow APY (sharp increase)
\end{itemize}

\vspace{0.3cm}
\textbf{Why?} Prevents bank runs (always keep some liquidity for withdrawals).
\end{frame}

\begin{frame}{Supply and Borrow APY Relationship}
\textbf{Lenders earn a portion of what borrowers pay.}

\vspace{0.3cm}
\textbf{Formula:}
\[
\text{Supply APY} = \text{Borrow APY} \times U \times (1 - \text{Reserve Factor})
\]

where Reserve Factor is protocol fee (e.g., 10\%).

\vspace{0.3cm}
\textbf{Example:}
\begin{itemize}
\item Borrow APY: 5\%
\item Utilization: 80\%
\item Reserve Factor: 10\%
\end{itemize}
\[
\text{Supply APY} = 5\% \times 0.80 \times 0.90 = 3.6\%
\]

\vspace{0.3cm}
\textbf{Observation:} Supply APY always lower than borrow APY (protocol takes cut, not everyone borrows).
\end{frame}

\begin{frame}{Aave Protocol Overview}
\textbf{Key Features:}
\begin{itemize}
\item \textbf{Flash Loans:} Borrow without collateral if repaid in same transaction
\item \textbf{Rate Switching:} Choose stable or variable interest rates
\item \textbf{Credit Delegation:} Lend your credit line to others
\item \textbf{Isolation Mode:} New assets isolated to reduce systemic risk
\item \textbf{E-Mode:} Higher LTV for correlated assets (e.g., ETH/stETH)
\end{itemize}

\vspace{0.3cm}
\textbf{Governance:}
\begin{itemize}
\item AAVE token holders vote on protocol changes
\item Risk parameters (LTV, liquidation thresholds)
\item Asset listings
\item Treasury management
\end{itemize}

\vspace{0.3cm}
\textbf{TVL:} ~\$4B (Dec 2024), deployed on Ethereum, Polygon, Avalanche, Arbitrum, Optimism.
\end{frame}

\begin{frame}{Compound Protocol Overview}
\textbf{Key Features:}
\begin{itemize}
\item \textbf{cTokens:} Interest-bearing tokens (cUSDC, cETH)
\item \textbf{Algorithmic Rates:} Fully automated interest rate model
\item \textbf{Governance:} COMP token holders control protocol
\item \textbf{Simplicity:} Fewer features than Aave, easier to understand
\end{itemize}

\vspace{0.3cm}
\textbf{cToken Mechanism:}
\begin{itemize}
\item Supply 100 USDC, receive ~5,000 cUSDC
\item cUSDC exchange rate increases over time (accrues interest)
\item Redeem cUSDC for USDC + interest anytime
\end{itemize}

\vspace{0.3cm}
\textbf{Historical Significance:}
\begin{itemize}
\item Pioneered DeFi lending (launched 2018)
\item Introduced COMP governance token (2020)
\item Sparked ``DeFi Summer'' via liquidity mining
\end{itemize}

\vspace{0.3cm}
\textbf{TVL:} ~\$1.5B (Dec 2024).
\end{frame}

\begin{frame}{Flash Loans (Aave)}
\textbf{Definition:} Borrow any amount without collateral, repay in same transaction.

\vspace{0.3cm}
\textbf{How It Works:}
\begin{enumerate}
\item Borrow \$1M USDC from Aave
\item Use \$1M for arbitrage, collateral swap, or liquidation
\item Repay \$1M + 0.09\% fee
\item All in one atomic transaction (either all succeeds or all reverts)
\end{enumerate}

\vspace{0.3cm}
\textbf{Use Cases:}
\begin{itemize}
\item \textbf{Arbitrage:} Exploit price differences across DEXs
\item \textbf{Collateral Swap:} Switch collateral without closing position
\item \textbf{Self-Liquidation:} Repay debt before liquidation penalty
\item \textbf{Exploits:} Unfortunately, used in many DeFi hacks
\end{itemize}

\vspace{0.3cm}
\textbf{Fee:} 0.09\% of borrowed amount.

\textbf{Risk:} If transaction fails, borrow is reversed (no loss to lenders).
\end{frame}

\begin{frame}{Flash Loan Attack Example}
\textbf{Historical Attack: bZx (2020)}

\vspace{0.3cm}
\textbf{Attack Steps:}
\begin{enumerate}
\item Flash loan 10,000 ETH from dYdX
\item Deposit 5,500 ETH into Compound as collateral
\item Borrow 112 WBTC from Compound
\item Swap 112 WBTC on Uniswap for 6,871 ETH (manipulating price)
\item Use pumped ETH price to borrow max from bZx
\item Repay flash loan, keep profit (~\$350k)
\end{enumerate}

\vspace{0.3cm}
\textbf{Root Cause:} bZx used Uniswap as sole price oracle (manipulable).

\vspace{0.3cm}
\textbf{Lesson:} Flash loans amplify oracle manipulation attacks. Protocols must use decentralized, manipulation-resistant oracles.
\end{frame}

\begin{frame}{Stable vs. Variable Interest Rates}
\textbf{Aave offers borrowers a choice:}

\vspace{0.3cm}
\begin{columns}[T]
\begin{column}{0.48\textwidth}
\textbf{Variable Rate}
\begin{itemize}
\item Changes with utilization
\item Usually lower (on average)
\item Unpredictable
\item Can spike during high demand
\end{itemize}

\vspace{0.2cm}
\textbf{Best for:}
\begin{itemize}
\item Short-term borrowing
\item Users who monitor actively
\end{itemize}
\end{column}

\begin{column}{0.48\textwidth}
\textbf{Stable Rate}
\begin{itemize}
\item Fixed for duration
\item Higher than variable (premium for stability)
\item Predictable
\item Can be rebalanced by protocol if too far from market
\end{itemize}

\vspace{0.2cm}
\textbf{Best for:}
\begin{itemize}
\item Long-term borrowing
\item Risk-averse users
\end{itemize}
\end{column}
\end{columns}

\vspace{0.3cm}
\textbf{Switching:} Borrowers can switch between rate types anytime.
\end{frame}

\begin{frame}{Risks in DeFi Lending}
\textbf{1. Smart Contract Risk}
\begin{itemize}
\item Bugs or exploits in protocol code
\item Historical hacks: Rari Capital, Cream Finance
\end{itemize}

\textbf{2. Liquidation Risk}
\begin{itemize}
\item Volatile markets can trigger liquidations
\item Network congestion may prevent adding collateral
\end{itemize}

\textbf{3. Oracle Risk}
\begin{itemize}
\item Price feed manipulation
\item Stale prices during network issues
\end{itemize}

\textbf{4. Governance Risk}
\begin{itemize}
\item Malicious proposals changing parameters
\item Low voter participation
\end{itemize}

\textbf{5. Liquidity Risk}
\begin{itemize}
\item High utilization prevents withdrawals
\item Bank run scenarios (rare but possible)
\end{itemize}
\end{frame}

\begin{frame}{DeFi Lending vs. CeFi Lending}
\begin{columns}[T]
\begin{column}{0.48\textwidth}
\textbf{DeFi (Aave, Compound)}
\begin{itemize}
\item Non-custodial
\item Transparent reserves
\item Overcollateralized
\item Instant settlement
\item Smart contract risk
\item No insurance (usually)
\item Permissionless
\end{itemize}
\end{column}

\begin{column}{0.48\textwidth}
\textbf{CeFi (BlockFi, Celsius)}
\begin{itemize}
\item Custodial
\item Opaque reserves
\item Can be undercollateralized
\item T+1 settlement
\item Counterparty risk
\item Some insurance
\item KYC required
\end{itemize}
\end{column}
\end{columns}

\vspace{0.3cm}
\textbf{2022 Lesson:} CeFi lenders (Celsius, Voyager, BlockFi) collapsed. DeFi protocols (Aave, Compound) survived intact, demonstrating transparency advantage.
\end{frame}

\begin{frame}{Future: Undercollateralized Lending}
\textbf{Current Limitation:} DeFi lending is capital inefficient (overcollateralization).

\vspace{0.3cm}
\textbf{Emerging Solutions:}
\begin{enumerate}
\item \textbf{On-Chain Credit Scores}
\begin{itemize}
\item Track repayment history, wallet activity
\item Examples: Credora, ARCx
\end{itemize}

\item \textbf{Real-World Identity}
\begin{itemize}
\item Link wallet to verified identity (KYC)
\item Legal recourse if default
\item Examples: Goldfinch, TrueFi
\end{itemize}

\item \textbf{Social Collateral}
\begin{itemize}
\item Community vouching, reputation staking
\item Example: Teller Protocol
\end{itemize}
\end{enumerate}

\vspace{0.3cm}
\textbf{Trade-off:} Undercollateralization requires identity or trust, reducing permissionlessness.
\end{frame}

\begin{frame}{Summary}
\textbf{Key Takeaways:}
\begin{itemize}
\item DeFi lending is permissionless but requires overcollateralization
\item LTV ratios vary by asset risk (ETH 80\%, low-cap tokens 40\%)
\item Health factor must stay above 1 to avoid liquidation
\item Interest rates adjust algorithmically based on utilization
\item Flash loans enable zero-collateral borrowing within one transaction
\item Aave and Compound dominate DeFi lending (\$5.5B+ combined TVL)
\item Risks: smart contract bugs, liquidations, oracle manipulation
\item Future: undercollateralized lending via on-chain credit scoring
\end{itemize}

\vspace{0.3cm}
\textbf{Next Lecture:} Stablecoin Mechanisms - How stablecoins maintain price stability.
\end{frame}

\begin{frame}{Questions for Reflection}
\begin{enumerate}
\item Calculate health factor for: \$30k collateral, \$20k borrowed, 85\% liquidation threshold.
\item Why is overcollateralization necessary in DeFi lending?
\item How do interest rates adjust to prevent bank runs?
\item What risks do flash loans pose to DeFi protocols?
\item Would you prefer variable or stable interest rates for a 1-year borrow?
\end{enumerate}
\end{frame}

\end{document}
