\documentclass[8pt,aspectratio=169]{beamer}
\usetheme{Madrid}
\usepackage[utf8]{inputenc}
\usepackage{graphicx}
\usepackage{booktabs}
\usepackage{hyperref}
\usepackage{amsmath}

\newcommand{\bottomnote}[1]{\vfill\par\noindent\footnotesize\textit{#1}}

\title{L37: Lending Protocols}
\subtitle{Module E: DeFi Ecosystem}
\author{Blockchain \& Cryptocurrency}
\date{December 2025}

\begin{document}

\frame{\titlepage}

\begin{frame}{Learning Objectives}
\begin{itemize}
\item Understand how DeFi lending protocols work (Aave, Compound)
\item Analyze overcollateralization and its necessity
\item Calculate health factors and liquidation thresholds
\item Explore interest rate models (utilization-based)
\item Compare DeFi lending to traditional finance
\end{itemize}
\end{frame}

\begin{frame}[t]{DeFi vs Traditional Lending}
\begin{center}
\includegraphics[width=0.65\textwidth]{../charts/01_lending_comparison/chart.pdf}
\end{center}
\bottomnote{DeFi trades capital efficiency for accessibility and speed}
\end{frame}

\begin{frame}{Traditional vs DeFi Lending}
\begin{columns}[T]
\begin{column}{0.48\textwidth}
\textbf{Traditional Lending}
\begin{itemize}
\item Credit checks, KYC required
\item Undercollateralized possible
\item Long approval process
\item Geographic restrictions
\end{itemize}
\end{column}

\begin{column}{0.48\textwidth}
\textbf{DeFi Lending}
\begin{itemize}
\item No credit checks
\item Overcollateralized only
\item Instant approval
\item Global access, 24/7
\end{itemize}
\end{column}
\end{columns}

\vspace{0.3cm}
\textbf{Key Trade-off:} DeFi accessibility vs. capital efficiency (must lock more than you borrow).
\end{frame}

\begin{frame}{How DeFi Lending Works}
\textbf{Core Mechanism:}

\vspace{0.2cm}
\begin{enumerate}
\item \textbf{Lenders (Suppliers):} Deposit assets, earn interest (APY varies)
\item \textbf{Borrowers:} Deposit collateral, borrow up to LTV\%, pay interest
\item \textbf{Protocol:} Matches via smart contracts, manages liquidations
\end{enumerate}

\vspace{0.3cm}
\textbf{Example: Borrowing on Aave}
\begin{itemize}
\item Deposit 10 ETH (\$20,000), receive aETH tokens
\item ETH has 80\% LTV: max borrow = \$16,000
\item Borrow \$10,000 USDC (safe margin)
\item Pay 5\% APY on debt, earn 2\% on collateral
\end{itemize}
\end{frame}

\begin{frame}[t]{LTV and Liquidation Thresholds}
\begin{center}
\includegraphics[width=0.65\textwidth]{../charts/02_ltv_thresholds/chart.pdf}
\end{center}
\bottomnote{Buffer between LTV and liquidation threshold protects against immediate liquidation}
\end{frame}

\begin{frame}{Loan-to-Value (LTV) Explained}
\textbf{Definition:}
\[
\text{LTV} = \frac{\text{Borrowed Value}}{\text{Collateral Value}} \times 100\%
\]

\vspace{0.3cm}
\textbf{Why Different LTVs?}
\begin{itemize}
\item Volatile assets = lower LTV (more buffer for price swings)
\item Stable assets = higher LTV (minimal price risk)
\end{itemize}

\vspace{0.3cm}
\textbf{Example Calculation:}
\begin{itemize}
\item Collateral: 10 ETH at \$2,000 = \$20,000
\item Borrowed: \$12,000 USDC
\item LTV: $\frac{12{,}000}{20{,}000} = 60\%$ (safe, under 80\% max)
\end{itemize}
\end{frame}

\begin{frame}[t]{Health Factor Dynamics}
\begin{center}
\includegraphics[width=0.65\textwidth]{../charts/03_health_factor/chart.pdf}
\end{center}
\bottomnote{Health factor must stay above 1; monitor closely during volatility}
\end{frame}

\begin{frame}{Health Factor Calculation}
\textbf{Formula (Aave):}
\[
\text{Health Factor} = \frac{\text{Collateral Value} \times \text{Liquidation Threshold}}{\text{Borrowed Value}}
\]

\vspace{0.3cm}
\textbf{Interpretation:}
\begin{itemize}
\item HF > 1.5: Safe (comfortable buffer)
\item HF 1.0-1.5: Warning zone
\item HF < 1: Liquidation occurs
\end{itemize}

\vspace{0.3cm}
\textbf{Example:}
\begin{itemize}
\item Collateral: \$20,000 (ETH), Borrowed: \$12,000
\item Liquidation threshold: 83\%
\item $\text{HF} = \frac{20{,}000 \times 0.83}{12{,}000} = 1.38$ (safe)
\end{itemize}
\end{frame}

\begin{frame}{Liquidation Process}
\textbf{When Health Factor < 1:}

\vspace{0.2cm}
\begin{enumerate}
\item \textbf{Liquidator Bot Detects} unhealthy position
\item \textbf{Liquidator Repays} portion of debt (up to 50\%)
\item \textbf{Liquidator Receives} equivalent collateral + 5-10\% bonus
\item \textbf{Borrower Loses} liquidation penalty
\end{enumerate}

\vspace{0.3cm}
\textbf{Example:}
\begin{itemize}
\item Debt: \$12,000, Collateral: \$14,460
\item Liquidator repays 50\% (\$6,000), receives \$6,300 in ETH
\item Borrower loses \$300 penalty (5\%)
\end{itemize}
\end{frame}

\begin{frame}[t]{Interest Rate Model}
\begin{center}
\includegraphics[width=0.65\textwidth]{../charts/04_interest_rate_model/chart.pdf}
\end{center}
\bottomnote{High utilization triggers steep rate increase to prevent liquidity crises}
\end{frame}

\begin{frame}{Utilization-Based Interest Rates}
\textbf{Utilization Rate:}
\[
U = \frac{\text{Total Borrowed}}{\text{Total Supplied}}
\]

\vspace{0.3cm}
\textbf{Rate Behavior:}
\begin{itemize}
\item Low utilization (0-60\%): Low rates (encourage borrowing)
\item Optimal (~80\%): Moderate rates (balanced)
\item High (>90\%): Very high rates (discourage borrowing)
\end{itemize}

\vspace{0.3cm}
\textbf{Supply APY Formula:}
\[
\text{Supply APY} = \text{Borrow APY} \times U \times (1 - \text{Reserve Factor})
\]

\textbf{Key Insight:} Supply APY always lower than borrow APY.
\end{frame}

\begin{frame}[t]{Lending Protocol Market Share}
\begin{center}
\includegraphics[width=0.65\textwidth]{../charts/05_lending_protocol_tvl/chart.pdf}
\end{center}
\bottomnote{Aave leads through multi-chain expansion; Compound pioneered the model}
\end{frame}

\begin{frame}{Aave vs Compound}
\begin{columns}[T]
\begin{column}{0.48\textwidth}
\textbf{Aave}
\begin{itemize}
\item Flash loans (no collateral)
\item Stable/variable rate choice
\item Credit delegation
\item E-Mode for correlated assets
\item ~\$12B TVL
\end{itemize}
\end{column}

\begin{column}{0.48\textwidth}
\textbf{Compound}
\begin{itemize}
\item cTokens (interest-bearing)
\item Simpler, fewer features
\item Pioneered DeFi lending
\item COMP governance token
\item ~\$2.5B TVL
\end{itemize}
\end{column}
\end{columns}

\vspace{0.3cm}
\textbf{Historical Note:} Compound launched 2018, Aave 2020. Compound's COMP mining sparked ``DeFi Summer'' 2020.
\end{frame}

\begin{frame}{Flash Loans}
\textbf{Definition:} Borrow any amount without collateral, repay in same transaction.

\vspace{0.3cm}
\textbf{How It Works:}
\begin{enumerate}
\item Borrow \$1M USDC from Aave
\item Use for arbitrage, collateral swap, or liquidation
\item Repay \$1M + 0.09\% fee
\item All atomic (succeeds or reverts entirely)
\end{enumerate}

\vspace{0.3cm}
\textbf{Use Cases:}
\begin{itemize}
\item Arbitrage across DEXs
\item Collateral swaps without closing position
\item Self-liquidation to avoid penalty
\end{itemize}

\textbf{Risk:} Used in many DeFi exploits (oracle manipulation attacks).
\end{frame}

\begin{frame}{Risks in DeFi Lending}
\textbf{1. Smart Contract Risk}
\begin{itemize}
\item Bugs or exploits (Rari Capital, Cream Finance hacks)
\end{itemize}

\textbf{2. Liquidation Risk}
\begin{itemize}
\item Volatile markets, network congestion prevents adding collateral
\end{itemize}

\textbf{3. Oracle Risk}
\begin{itemize}
\item Price feed manipulation, stale prices
\end{itemize}

\textbf{4. Liquidity Risk}
\begin{itemize}
\item High utilization prevents withdrawals
\end{itemize}

\vspace{0.2cm}
\textbf{2022 Lesson:} CeFi lenders (Celsius, BlockFi) collapsed; DeFi protocols survived.
\end{frame}

\begin{frame}{Future: Undercollateralized Lending}
\textbf{Current Limitation:} Overcollateralization is capital inefficient.

\vspace{0.3cm}
\textbf{Emerging Solutions:}
\begin{enumerate}
\item \textbf{On-Chain Credit Scores:} Track repayment history (Credora, ARCx)
\item \textbf{Real-World Identity:} KYC-linked, legal recourse (Goldfinch, TrueFi)
\item \textbf{Social Collateral:} Community vouching (Teller Protocol)
\end{enumerate}

\vspace{0.3cm}
\textbf{Trade-off:} Undercollateralization requires identity or trust, reducing permissionlessness.
\end{frame}

\begin{frame}{Summary}
\textbf{Key Takeaways:}
\begin{itemize}
\item DeFi lending: permissionless but requires overcollateralization
\item LTV ratios vary by asset risk (ETH 80\%, low-cap 45\%)
\item Health factor must stay above 1 to avoid liquidation
\item Interest rates adjust algorithmically based on utilization
\item Flash loans enable zero-collateral borrowing within one tx
\item Aave dominates (\$12B TVL); Compound pioneered the model
\item Future: undercollateralized lending via credit scoring
\end{itemize}

\vspace{0.3cm}
\textbf{Next Lecture:} Stablecoin Mechanisms.
\end{frame}

\begin{frame}{Questions for Reflection}
\begin{enumerate}
\item Calculate health factor: \$30k collateral, \$20k borrowed, 85\% liq threshold.
\item Why is overcollateralization necessary in DeFi lending?
\item How do interest rates adjust to prevent bank runs?
\item What risks do flash loans pose to DeFi protocols?
\item Variable or stable rates for 1-year borrow?
\end{enumerate}
\end{frame}

\end{document}
