\documentclass[8pt,aspectratio=169]{beamer}
\usetheme{Madrid}
\usepackage[utf8]{inputenc}
\usepackage{graphicx}
\usepackage{booktabs}
\usepackage{hyperref}
\usepackage{amsmath}

\title{L34: AMM Mechanics}
\subtitle{Module E: DeFi Ecosystem}
\author{Blockchain \& Cryptocurrency}
\date{December 2025}

\begin{document}

\frame{\titlepage}

\begin{frame}{Learning Objectives}
\begin{itemize}
\item Understand the constant product formula ($x \cdot y = k$)
\item Analyze how liquidity provision works in AMMs
\item Calculate impermanent loss and its implications
\item Understand slippage and price impact
\item Compare AMMs to traditional order book exchanges
\end{itemize}
\end{frame}

\begin{frame}{Traditional Order Book Exchanges}
\textbf{How They Work:}
\begin{itemize}
\item \textbf{Buyers} place bids at various prices
\item \textbf{Sellers} place asks at various prices
\item \textbf{Matching engine} pairs buy/sell orders
\item Trade executes when bid meets ask
\end{itemize}

\vspace{0.3cm}
\textbf{Example Order Book:}
\begin{columns}[T]
\begin{column}{0.3\textwidth}
\textbf{Bids (Buy)}
\begin{tabular}{rr}
Price & Size \\
\midrule
\$1,999 & 5 ETH \\
\$1,998 & 10 ETH \\
\$1,997 & 15 ETH \\
\end{tabular}
\end{column}
\begin{column}{0.3\textwidth}
\textbf{Asks (Sell)}
\begin{tabular}{rr}
Price & Size \\
\midrule
\$2,000 & 8 ETH \\
\$2,001 & 12 ETH \\
\$2,002 & 20 ETH \\
\end{tabular}
\end{column}
\end{columns}

\vspace{0.3cm}
\textbf{Challenges on Blockchain:}
\begin{itemize}
\item Gas costs for every order update
\item Slow block times (not real-time)
\item Front-running and MEV
\end{itemize}
\end{frame}

\begin{frame}{Automated Market Makers (AMMs)}
\textbf{Key Idea:} Replace order books with liquidity pools governed by mathematical formulas.

\vspace{0.3cm}
\textbf{How It Works:}
\begin{itemize}
\item Liquidity Providers (LPs) deposit token pairs into a pool
\item Algorithm sets price based on pool ratio
\item Users trade directly against the pool
\item No matching engine or counterparty needed
\end{itemize}

\vspace{0.3cm}
\textbf{Advantages:}
\begin{itemize}
\item Always available liquidity (no need to wait for orders)
\item Passive income for LPs (earn trading fees)
\item Simple smart contract implementation
\item Gas efficient (fewer transactions)
\end{itemize}

\vspace{0.3cm}
\textbf{Trade-off:} Price determined by formula, not market consensus.
\end{frame}

\begin{frame}{The Constant Product Formula}
\textbf{Uniswap V2 Model:}
\[
x \cdot y = k
\]

where:
\begin{itemize}
\item $x$ = quantity of token A in pool
\item $y$ = quantity of token B in pool
\item $k$ = constant product (invariant)
\end{itemize}

\vspace{0.3cm}
\textbf{Key Property:} The product $k$ remains constant before and after trades (ignoring fees).

\vspace{0.3cm}
\textbf{Example Pool:}
\begin{itemize}
\item 100 ETH and 200,000 USDC
\item $k = 100 \times 200{,}000 = 20{,}000{,}000$
\item Implied price: 1 ETH = 2,000 USDC
\end{itemize}
\end{frame}

\begin{frame}{Price Determination in AMMs}
\textbf{Price is the ratio of reserves:}
\[
P = \frac{y}{x}
\]

\textbf{Example:}
\begin{itemize}
\item Pool: 100 ETH, 200,000 USDC
\item Price of 1 ETH: $\frac{200{,}000}{100} = 2{,}000$ USDC
\end{itemize}

\vspace{0.3cm}
\textbf{After Trade:}
User buys 1 ETH with USDC:
\begin{itemize}
\item New reserves: 99 ETH, $y'$ USDC
\item Constant product: $99 \cdot y' = 20{,}000{,}000$
\item Solve: $y' = \frac{20{,}000{,}000}{99} \approx 202{,}020$ USDC
\item USDC added: $202{,}020 - 200{,}000 = 2{,}020$ USDC
\item \textbf{Cost:} 2,020 USDC for 1 ETH (effective price: \$2,020)
\end{itemize}

\vspace{0.3cm}
\textbf{Observation:} Price moves unfavorably with trade size (slippage).
\end{frame}

\begin{frame}{Trade Calculation Example}
\textbf{Pool State:}
\begin{itemize}
\item 100 ETH, 200,000 USDC
\item $k = 20{,}000{,}000$
\end{itemize}

\textbf{User wants to buy 10 ETH:}

\vspace{0.2cm}
\textbf{Step 1:} Calculate new ETH reserve
\[
x' = 100 - 10 = 90 \text{ ETH}
\]

\textbf{Step 2:} Calculate required USDC reserve
\[
y' = \frac{k}{x'} = \frac{20{,}000{,}000}{90} \approx 222{,}222 \text{ USDC}
\]

\textbf{Step 3:} USDC to pay
\[
\Delta y = 222{,}222 - 200{,}000 = 22{,}222 \text{ USDC}
\]

\textbf{Average price:} $\frac{22{,}222}{10} = 2{,}222$ USDC per ETH (vs. 2,000 initially).

\vspace{0.2cm}
\textbf{Price impact:} 11\% higher than starting price.
\end{frame}

\begin{frame}{Slippage}
\textbf{Definition:} The difference between expected price and executed price due to trade size.

\vspace{0.3cm}
\textbf{Why Slippage Occurs:}
\begin{itemize}
\item AMM formula moves price as reserves change
\item Larger trades = larger price impact
\item Smaller pools = more slippage
\end{itemize}

\vspace{0.3cm}
\textbf{Slippage Formula:}
\[
\text{Slippage} = \frac{\text{Executed Price} - \text{Initial Price}}{\text{Initial Price}} \times 100\%
\]

\textbf{Example from Previous Slide:}
\[
\text{Slippage} = \frac{2{,}222 - 2{,}000}{2{,}000} \times 100\% = 11\%
\]

\vspace{0.3cm}
\textbf{Slippage Tolerance:} Users set maximum acceptable slippage (e.g., 0.5\%, 1\%, 5\%). Transaction reverts if exceeded.
\end{frame}

\begin{frame}{Liquidity Provision}
\textbf{How to Become an LP:}
\begin{enumerate}
\item Deposit equal value of both tokens (e.g., 1 ETH + 2,000 USDC)
\item Receive LP tokens representing pool share
\item Earn trading fees proportional to share
\item Withdraw anytime (burn LP tokens, receive reserves back)
\end{enumerate}

\vspace{0.3cm}
\textbf{Example:}
\begin{itemize}
\item Pool has 100 ETH + 200,000 USDC
\item You deposit 10 ETH + 20,000 USDC
\item Total pool now: 110 ETH + 220,000 USDC
\item Your share: $\frac{10}{110} = 9.09\%$
\item You receive 9.09\% of LP tokens
\end{itemize}

\vspace{0.3cm}
\textbf{Fee Earnings:}
\begin{itemize}
\item Uniswap charges 0.3\% per trade
\item Fees added to pool reserves
\item LPs earn pro-rata share
\end{itemize}
\end{frame}

\begin{frame}{LP Token Mechanics}
\textbf{Purpose:} Represent ownership share of liquidity pool.

\vspace{0.3cm}
\textbf{Properties:}
\begin{itemize}
\item Fungible ERC-20 tokens
\item Can be transferred or sold
\item Redeemable for underlying assets
\item Value appreciates with fee accumulation
\end{itemize}

\vspace{0.3cm}
\textbf{Withdrawal Process:}
\begin{enumerate}
\item Burn LP tokens
\item Receive pro-rata share of current pool reserves
\item May be different ratio than deposit (due to trades)
\end{enumerate}

\vspace{0.3cm}
\textbf{Example:}
\begin{itemize}
\item You hold 9.09\% of LP tokens
\item Pool now has 105 ETH + 210,000 USDC (after trades and fees)
\item You withdraw: $0.0909 \times 105 = 9.54$ ETH and $0.0909 \times 210{,}000 = 19{,}089$ USDC
\end{itemize}
\end{frame}

\begin{frame}{Impermanent Loss: Concept}
\textbf{Definition:} The opportunity cost of providing liquidity compared to simply holding tokens.

\vspace{0.3cm}
\textbf{Occurs when:}
\begin{itemize}
\item Token prices diverge from deposit ratio
\item Arbitrageurs rebalance pool to match external prices
\item LPs end up with more of the depreciated token
\end{itemize}

\vspace{0.3cm}
\textbf{Why ``Impermanent''?}
\begin{itemize}
\item Loss is only realized upon withdrawal
\item If prices return to original ratio, loss disappears
\item Trading fees may offset the loss over time
\end{itemize}

\vspace{0.3cm}
\textbf{Key Insight:} LPs effectively become market makers who buy low and sell high (but miss out on holding gains).
\end{frame}

\begin{frame}{Impermanent Loss: Example}
\textbf{Initial Deposit:}
\begin{itemize}
\item 1 ETH + 2,000 USDC (ETH price = \$2,000)
\item Total value: \$4,000
\end{itemize}

\vspace{0.3cm}
\textbf{Scenario: ETH doubles to \$4,000}

\textbf{If you just held:}
\begin{itemize}
\item 1 ETH now worth \$4,000
\item 2,000 USDC still worth \$2,000
\item \textbf{Total: \$6,000}
\end{itemize}

\textbf{If you provided liquidity:}
\begin{itemize}
\item Arbitrageurs rebalance pool: $x \cdot y = k$
\item New reserves: 0.707 ETH + 2,828 USDC
\item Value: $(0.707 \times \$4{,}000) + \$2{,}828 = \$5{,}656$
\item \textbf{Impermanent Loss: \$6,000 - \$5,656 = \$344 (5.7\%)}
\end{itemize}
\end{frame}

\begin{frame}{Impermanent Loss: Formula}
\textbf{General Formula:}
\[
\text{IL} = \frac{2\sqrt{r}}{1 + r} - 1
\]
where $r$ is the price ratio change.

\vspace{0.3cm}
\textbf{Common Scenarios:}
\begin{itemize}
\item 1.25x price change: -0.6\% IL
\item 1.5x price change: -2.0\% IL
\item 2x price change: -5.7\% IL
\item 3x price change: -13.4\% IL
\item 4x price change: -20.0\% IL
\item 5x price change: -25.5\% IL
\end{itemize}

\vspace{0.3cm}
\textbf{Observation:} IL accelerates with larger price movements (non-linear).
\end{frame}

\begin{frame}{Mitigating Impermanent Loss}
\textbf{Strategies:}

\vspace{0.3cm}
\textbf{1. Choose Stable Pairs}
\begin{itemize}
\item Provide liquidity for correlated assets (e.g., USDC/DAI, ETH/stETH)
\item Minimal price divergence = minimal IL
\end{itemize}

\textbf{2. High Trading Volume Pools}
\begin{itemize}
\item More fees to offset IL
\item Example: ETH/USDC on Uniswap (high volume)
\end{itemize}

\textbf{3. Concentrated Liquidity (Uniswap V3)}
\begin{itemize}
\item Provide liquidity in narrow price range
\item Higher fee efficiency but more active management
\end{itemize}

\textbf{4. Liquidity Mining Rewards}
\begin{itemize}
\item Extra token incentives may exceed IL
\end{itemize}

\textbf{5. Short-Term Provision}
\begin{itemize}
\item Withdraw before large price movements
\end{itemize}
\end{frame}

\begin{frame}{Arbitrage in AMMs}
\textbf{How Arbitrage Works:}
\begin{enumerate}
\item External market price deviates from AMM price
\item Arbitrageur buys cheaper asset, sells expensive one
\item Profits from price difference
\item AMM pool rebalances to match external price
\end{enumerate}

\vspace{0.3cm}
\textbf{Example:}
\begin{itemize}
\item Centralized exchange (CEX): 1 ETH = \$2,100
\item Uniswap pool implies: 1 ETH = \$2,000
\item Arbitrageur: Buy ETH on Uniswap (\$2,000), sell on CEX (\$2,100)
\item Profit: \$100 per ETH
\item Pool adjusts: ETH reserve decreases, USDC increases, price rises
\end{itemize}

\vspace{0.3cm}
\textbf{Benefit:} Arbitrage keeps AMM prices aligned with global markets.

\textbf{Cost:} LPs experience impermanent loss from price adjustments.
\end{frame}

\begin{frame}{Fee Structures}
\textbf{Uniswap Fee Tiers:}
\begin{itemize}
\item \textbf{0.01\%:} Stablecoin pairs (low volatility)
\item \textbf{0.05\%:} Correlated pairs (e.g., ETH/stETH)
\item \textbf{0.3\%:} Most pairs (standard)
\item \textbf{1\%:} Exotic/volatile pairs
\end{itemize}

\vspace{0.3cm}
\textbf{Fee Distribution:}
\begin{itemize}
\item 100\% to LPs (Uniswap governance can enable protocol fee)
\item Fees compound in pool reserves
\item LPs earn proportional to liquidity share and duration
\end{itemize}

\vspace{0.3cm}
\textbf{APY Calculation:}
\[
\text{APY} \approx \frac{\text{Daily Fees} \times 365}{\text{Pool TVL}} - \text{Impermanent Loss}
\]

\textbf{Example:}
\begin{itemize}
\item Pool TVL: \$10M, Daily fees: \$10,000
\item APY: $\frac{10{,}000 \times 365}{10{,}000{,}000} = 36.5\%$ (before IL)
\end{itemize}
\end{frame}

\begin{frame}{AMM Variants: Curve (Stableswap)}
\textbf{Problem with Constant Product:}
\begin{itemize}
\item Inefficient for assets that should trade 1:1 (stablecoins)
\item High slippage even for small trades
\end{itemize}

\vspace{0.3cm}
\textbf{Curve's Solution: Stableswap Invariant}
\begin{itemize}
\item Hybrid of constant product and constant sum
\item Flat curve near 1:1 price (low slippage)
\item Reverts to constant product at extremes (prevents pool drain)
\end{itemize}

\vspace{0.3cm}
\textbf{Benefits:}
\begin{itemize}
\item Trade millions of USDC/DAI with <0.01\% slippage
\item Capital efficient for stablecoin swaps
\item Minimal impermanent loss (prices stay near 1:1)
\end{itemize}

\vspace{0.3cm}
\textbf{Use Case:} Dominant DEX for stablecoin trading.
\end{frame}

\begin{frame}{AMM Variants: Balancer (Weighted Pools)}
\textbf{Generalization:}
\[
\prod_{i} x_i^{w_i} = k
\]
where $w_i$ are weights (must sum to 1).

\vspace{0.3cm}
\textbf{Example: 80/20 Pool}
\begin{itemize}
\item 80\% token A, 20\% token B by value
\item Less impermanent loss than 50/50 pool
\item Still earn fees from trading
\end{itemize}

\vspace{0.3cm}
\textbf{Advantages:}
\begin{itemize}
\item Customizable exposure (e.g., 80\% ETH, 20\% USDC)
\item Multi-token pools (up to 8 tokens)
\item Index fund functionality
\end{itemize}

\textbf{Use Case:} LPs wanting concentrated exposure to one asset while earning fees.
\end{frame}

\begin{frame}{Capital Efficiency}
\textbf{Problem:} In constant product AMMs, most liquidity is unused.

\vspace{0.3cm}
\textbf{Example:}
\begin{itemize}
\item ETH/USDC pool with 100 ETH and 200,000 USDC
\item Most trades happen near current price (\$2,000)
\item Liquidity far from current price (e.g., \$1,000 or \$4,000) rarely used
\end{itemize}

\vspace{0.3cm}
\textbf{Solution: Concentrated Liquidity (Uniswap V3)}
\begin{itemize}
\item LPs choose specific price range
\item Liquidity only active within range
\item More capital efficient (same liquidity with less capital)
\end{itemize}

\vspace{0.3cm}
\textbf{Trade-off:}
\begin{itemize}
\item Higher returns if price stays in range
\item Zero fees if price moves outside range
\item Requires active management (rebalancing)
\end{itemize}
\end{frame}

\begin{frame}{Comparison: AMM vs. Order Book}
\begin{columns}[T]
\begin{column}{0.48\textwidth}
\textbf{AMM (Uniswap)}
\begin{itemize}
\item Always available liquidity
\item Passive LP income
\item Slippage on large trades
\item Price discovery via arbitrage
\item Impermanent loss risk
\item Simple to implement
\end{itemize}
\end{column}

\begin{column}{0.48\textwidth}
\textbf{Order Book (Binance)}
\begin{itemize}
\item Liquidity depends on makers
\item Active market making
\item Better for large trades (limit orders)
\item Real-time price discovery
\item No impermanent loss
\item Complex infrastructure
\end{itemize}
\end{column}
\end{columns}

\vspace{0.3cm}
\textbf{Trend:} Hybrid models emerging (e.g., dYdX order book on Cosmos, Uniswap X).
\end{frame}

\begin{frame}{MEV and Front-Running in AMMs}
\textbf{Maximal Extractable Value (MEV):}
\begin{itemize}
\item Profit from reordering/inserting/censoring transactions
\item Particularly prevalent in AMM trades
\end{itemize}

\vspace{0.3cm}
\textbf{Common MEV Strategies:}
\begin{enumerate}
\item \textbf{Front-Running:} See large buy, buy first, sell after price moves
\item \textbf{Sandwich Attacks:} Buy before user, sell after (profit from their slippage)
\item \textbf{Arbitrage:} Exploit price differences between AMMs/CEXs
\end{enumerate}

\vspace{0.3cm}
\textbf{Impact on Users:}
\begin{itemize}
\item Worse execution prices
\item Hidden cost (in addition to explicit fees)
\end{itemize}

\vspace{0.3cm}
\textbf{Mitigation:}
\begin{itemize}
\item Private mempools (Flashbots Protect)
\item MEV-aware routers
\item Batch auctions (CoW Swap)
\end{itemize}
\end{frame}

\begin{frame}{Summary}
\textbf{Key Takeaways:}
\begin{itemize}
\item AMMs use $x \cdot y = k$ to provide algorithmic liquidity
\item Price determined by reserve ratio, trades move price
\item Slippage increases with trade size and decreases with pool depth
\item LPs earn fees but face impermanent loss when prices diverge
\item IL formula: $\frac{2\sqrt{r}}{1 + r} - 1$ where $r$ is price change ratio
\item Arbitrage keeps AMM prices aligned with external markets
\item Variants (Curve, Balancer, Uniswap V3) optimize for specific use cases
\item MEV is a hidden cost for AMM traders
\end{itemize}

\vspace{0.3cm}
\textbf{Next Lecture:} Uniswap Deep Dive - Evolution from V1 to V4, concentrated liquidity, governance.
\end{frame}

\begin{frame}{Questions for Reflection}
\begin{enumerate}
\item Calculate the cost to buy 5 ETH from a pool with 100 ETH and 200,000 USDC.
\item Why does slippage increase non-linearly with trade size?
\item How do trading fees help offset impermanent loss for LPs?
\item Why is Curve more suitable for stablecoin trading than Uniswap V2?
\item What are the trade-offs of concentrated liquidity in Uniswap V3?
\end{enumerate}
\end{frame}

\end{document}
