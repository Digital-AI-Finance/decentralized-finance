\documentclass[8pt,aspectratio=169]{beamer}
\usetheme{Madrid}
\usepackage[utf8]{inputenc}
\usepackage{graphicx}
\usepackage{booktabs}
\usepackage{hyperref}
\usepackage{amsmath}

\newcommand{\bottomnote}[1]{\vfill\par\noindent\footnotesize\textit{#1}}

\title{L34: AMM Mechanics}
\subtitle{Module E: DeFi Ecosystem}
\author{Blockchain \& Cryptocurrency}
\date{December 2025}

\begin{document}

\frame{\titlepage}

\begin{frame}{Learning Objectives}
\begin{itemize}
\item Understand the constant product formula ($x \cdot y = k$)
\item Analyze how liquidity provision works in AMMs
\item Calculate impermanent loss and its implications
\item Understand slippage and price impact
\item Compare AMMs to traditional order book exchanges
\end{itemize}
\end{frame}

\begin{frame}{Traditional Order Book Exchanges}
\textbf{How They Work:}
\begin{itemize}
\item \textbf{Buyers} place bids, \textbf{sellers} place asks
\item \textbf{Matching engine} pairs buy/sell orders
\item Trade executes when bid meets ask
\end{itemize}

\vspace{0.3cm}
\textbf{Example Order Book:}
\begin{columns}[T]
\begin{column}{0.3\textwidth}
\textbf{Bids (Buy)}
\begin{tabular}{rr}
Price & Size \\
\midrule
\$1,999 & 5 ETH \\
\$1,998 & 10 ETH \\
\end{tabular}
\end{column}
\begin{column}{0.3\textwidth}
\textbf{Asks (Sell)}
\begin{tabular}{rr}
Price & Size \\
\midrule
\$2,000 & 8 ETH \\
\$2,001 & 12 ETH \\
\end{tabular}
\end{column}
\end{columns}

\vspace{0.3cm}
\textbf{Challenges on Blockchain:} Gas costs for order updates, slow block times, front-running.
\end{frame}

\begin{frame}{Automated Market Makers (AMMs)}
\textbf{Key Idea:} Replace order books with liquidity pools governed by mathematical formulas.

\vspace{0.3cm}
\textbf{How It Works:}
\begin{itemize}
\item Liquidity Providers (LPs) deposit token pairs into a pool
\item Algorithm sets price based on pool ratio
\item Users trade directly against the pool
\end{itemize}

\vspace{0.3cm}
\textbf{Advantages:}
\begin{itemize}
\item Always available liquidity (no need to wait for orders)
\item Passive income for LPs (earn trading fees)
\item Gas efficient (fewer transactions)
\end{itemize}

\textbf{Trade-off:} Price determined by formula, not market consensus.
\end{frame}

\begin{frame}[t]{The Constant Product Formula}
\begin{center}
\includegraphics[width=0.65\textwidth]{../charts/01_constant_product/chart.pdf}
\end{center}
\bottomnote{Trades move along the curve; price is the slope at any point}
\end{frame}

\begin{frame}{Constant Product: Mathematics}
\textbf{Uniswap V2 Model:}
\[
x \cdot y = k
\]
where $x$ = token A quantity, $y$ = token B quantity, $k$ = constant.

\vspace{0.3cm}
\textbf{Example Pool:}
\begin{itemize}
\item 100 ETH and 200,000 USDC
\item $k = 100 \times 200{,}000 = 20{,}000{,}000$
\item Price: $\frac{y}{x} = \frac{200{,}000}{100} = 2{,}000$ USDC per ETH
\end{itemize}

\vspace{0.3cm}
\textbf{After Buying 10 ETH:}
\begin{itemize}
\item New ETH: 90, New USDC: $\frac{20{,}000{,}000}{90} = 222{,}222$
\item Cost: 22,222 USDC for 10 ETH = \$2,222/ETH average
\end{itemize}
\end{frame}

\begin{frame}[t]{Slippage vs Trade Size}
\begin{center}
\includegraphics[width=0.65\textwidth]{../charts/02_slippage_vs_size/chart.pdf}
\end{center}
\bottomnote{Slippage increases non-linearly; larger pools reduce slippage}
\end{frame}

\begin{frame}{Slippage Explained}
\textbf{Definition:} The difference between expected price and executed price due to trade size.

\vspace{0.3cm}
\textbf{Why Slippage Occurs:}
\begin{itemize}
\item AMM formula moves price as reserves change
\item Larger trades = larger price impact
\item Smaller pools = more slippage
\end{itemize}

\vspace{0.3cm}
\textbf{Slippage Formula:}
\[
\text{Slippage} = \frac{\text{Executed Price} - \text{Initial Price}}{\text{Initial Price}} \times 100\%
\]

\vspace{0.3cm}
\textbf{Slippage Tolerance:} Users set maximum acceptable (e.g., 0.5\%, 1\%). Transaction reverts if exceeded.
\end{frame}

\begin{frame}{Liquidity Provision}
\textbf{How to Become an LP:}
\begin{enumerate}
\item Deposit equal value of both tokens (e.g., 1 ETH + 2,000 USDC)
\item Receive LP tokens representing pool share
\item Earn trading fees proportional to share
\item Withdraw anytime (burn LP tokens, receive reserves)
\end{enumerate}

\vspace{0.3cm}
\textbf{Example:}
\begin{itemize}
\item Pool has 100 ETH + 200,000 USDC
\item You deposit 10 ETH + 20,000 USDC
\item Your share: $\frac{10}{110} = 9.09\%$ of pool
\end{itemize}

\vspace{0.3cm}
\textbf{Fee Earnings:} Uniswap charges 0.3\% per trade; fees compound in pool reserves.
\end{frame}

\begin{frame}[t]{Impermanent Loss}
\begin{center}
\includegraphics[width=0.65\textwidth]{../charts/03_impermanent_loss/chart.pdf}
\end{center}
\bottomnote{IL increases with price divergence; symmetric whether price rises or falls}
\end{frame}

\begin{frame}{Impermanent Loss: Concept}
\textbf{Definition:} The opportunity cost of providing liquidity vs. simply holding tokens.

\vspace{0.3cm}
\textbf{Occurs when:}
\begin{itemize}
\item Token prices diverge from deposit ratio
\item Arbitrageurs rebalance pool to match external prices
\item LPs end up with more of the depreciated token
\end{itemize}

\vspace{0.3cm}
\textbf{Example: ETH doubles to \$4,000}
\begin{itemize}
\item Initial: 1 ETH + 2,000 USDC = \$4,000 total
\item If held: 1 ETH @ \$4,000 + 2,000 USDC = \$6,000
\item If LP: 0.707 ETH + 2,828 USDC = \$5,656
\item \textbf{IL: \$344 (5.7\%)}
\end{itemize}
\end{frame}

\begin{frame}{Impermanent Loss: Key Values}
\textbf{Formula:}
\[
\text{IL} = \frac{2\sqrt{r}}{1 + r} - 1
\]
where $r$ is the price ratio (final/initial).

\vspace{0.3cm}
\textbf{Common Scenarios:}
\begin{itemize}
\item 1.25x price change: -0.6\% IL
\item 1.5x price change: -2.0\% IL
\item 2x price change: -5.7\% IL
\item 3x price change: -13.4\% IL
\item 4x price change: -20.0\% IL
\end{itemize}

\vspace{0.3cm}
\textbf{Why ``Impermanent''?} Loss only realized on withdrawal; if prices return, loss disappears.
\end{frame}

\begin{frame}[t]{AMM Curve Variants}
\begin{center}
\includegraphics[width=0.65\textwidth]{../charts/04_amm_variants/chart.pdf}
\end{center}
\bottomnote{Different curves optimize for different use cases}
\end{frame}

\begin{frame}{AMM Variants Explained}
\textbf{Uniswap (Constant Product): $x \cdot y = k$}
\begin{itemize}
\item General purpose, works for any pair
\item Higher slippage near extremes
\end{itemize}

\textbf{Curve (Stableswap): Hybrid formula}
\begin{itemize}
\item Flat curve near 1:1 (low slippage for stables)
\item Dominates stablecoin swaps (USDC/DAI)
\end{itemize}

\textbf{Balancer (Weighted): $\prod x_i^{w_i} = k$}
\begin{itemize}
\item Custom weights (e.g., 80/20 instead of 50/50)
\item Index fund functionality
\end{itemize}
\end{frame}

\begin{frame}[t]{Fee Tiers in AMMs}
\begin{center}
\includegraphics[width=0.65\textwidth]{../charts/05_fee_structure/chart.pdf}
\end{center}
\bottomnote{Higher fees compensate LPs for impermanent loss risk in volatile pairs}
\end{frame}

\begin{frame}{Arbitrage in AMMs}
\textbf{How Arbitrage Works:}
\begin{enumerate}
\item External market price deviates from AMM price
\item Arbitrageur buys cheaper asset, sells expensive one
\item AMM pool rebalances to match external price
\end{enumerate}

\vspace{0.3cm}
\textbf{Example:}
\begin{itemize}
\item CEX: 1 ETH = \$2,100
\item Uniswap pool: 1 ETH = \$2,000
\item Arbitrageur: Buy on Uniswap, sell on CEX, profit \$100/ETH
\end{itemize}

\vspace{0.3cm}
\textbf{Benefit:} Keeps AMM prices aligned with global markets.

\textbf{Cost:} LPs experience impermanent loss from adjustments.
\end{frame}

\begin{frame}{MEV and Front-Running}
\textbf{Maximal Extractable Value (MEV):}
\begin{itemize}
\item Profit from reordering/inserting transactions
\item Particularly prevalent in AMM trades
\end{itemize}

\vspace{0.3cm}
\textbf{Common MEV Strategies:}
\begin{enumerate}
\item \textbf{Front-Running:} See large buy, buy first, sell after
\item \textbf{Sandwich Attacks:} Buy before user, sell after
\item \textbf{Arbitrage:} Exploit price differences
\end{enumerate}

\vspace{0.3cm}
\textbf{Mitigation:}
\begin{itemize}
\item Private mempools (Flashbots Protect)
\item Batch auctions (CoW Swap)
\end{itemize}
\end{frame}

\begin{frame}{Comparison: AMM vs Order Book}
\begin{columns}[T]
\begin{column}{0.48\textwidth}
\textbf{AMM (Uniswap)}
\begin{itemize}
\item Always available liquidity
\item Passive LP income
\item Slippage on large trades
\item Impermanent loss risk
\end{itemize}
\end{column}

\begin{column}{0.48\textwidth}
\textbf{Order Book (Binance)}
\begin{itemize}
\item Liquidity depends on makers
\item Active market making
\item Better for large trades
\item No impermanent loss
\end{itemize}
\end{column}
\end{columns}

\vspace{0.3cm}
\textbf{Trend:} Hybrid models emerging (e.g., dYdX order book on Cosmos).
\end{frame}

\begin{frame}{Summary}
\textbf{Key Takeaways:}
\begin{itemize}
\item AMMs use $x \cdot y = k$ to provide algorithmic liquidity
\item Price determined by reserve ratio; trades move price
\item Slippage increases non-linearly with trade size
\item LPs earn fees but face impermanent loss when prices diverge
\item IL formula: $\frac{2\sqrt{r}}{1 + r} - 1$ (up to 20\%+ for 4x moves)
\item Curve optimizes for stables; Balancer for weighted pools
\item MEV is a hidden cost for AMM traders
\end{itemize}

\vspace{0.3cm}
\textbf{Next Lecture:} Uniswap Deep Dive - V1 to V4 evolution, concentrated liquidity.
\end{frame}

\begin{frame}{Questions for Reflection}
\begin{enumerate}
\item Calculate the cost to buy 5 ETH from a pool with 100 ETH and 200,000 USDC.
\item Why does slippage increase non-linearly with trade size?
\item How do trading fees help offset impermanent loss for LPs?
\item Why is Curve more suitable for stablecoin trading than Uniswap V2?
\item What are the trade-offs of concentrated liquidity in Uniswap V3?
\end{enumerate}
\end{frame}

\end{document}
