\documentclass[8pt,aspectratio=169]{beamer}
\usetheme{Madrid}
\usepackage[utf8]{inputenc}
\usepackage{graphicx}
\usepackage{booktabs}
\usepackage{hyperref}
\usepackage{amsmath}

\newcommand{\bottomnote}[1]{\vfill\par\noindent\footnotesize\textit{#1}}

\title{L29: Token Economics Fundamentals}
\subtitle{Module D: Tokenomics}
\author{Blockchain \& Cryptocurrency}
\date{December 2025}

\begin{document}

\frame{\titlepage}

\begin{frame}{Learning Objectives}
\begin{itemize}
\item Understand different token types and their purposes
\item Analyze value accrual mechanisms in token economies
\item Evaluate token supply models and their implications
\item Apply tokenomics principles to real-world projects
\end{itemize}
\end{frame}

\begin{frame}{What is Tokenomics?}
\textbf{Definition:} The study of the economic systems governing the creation, distribution, and management of tokens in blockchain ecosystems.

\vspace{0.3cm}
\textbf{Key Components:}
\begin{itemize}
\item Token design and purpose
\item Supply and distribution mechanics
\item Incentive structures
\item Value capture mechanisms
\item Governance and utility
\end{itemize}

\vspace{0.3cm}
\textbf{Why It Matters:}
Tokenomics determines the long-term sustainability and success of crypto projects.
\end{frame}

\begin{frame}{Token Types: Overview}
\begin{columns}[T]
\begin{column}{0.48\textwidth}
\textbf{1. Utility Tokens}
\begin{itemize}
\item Access to platform services
\item Payment for network fees
\item Examples: BNB, LINK, FIL
\end{itemize}

\vspace{0.3cm}
\textbf{2. Security Tokens}
\begin{itemize}
\item Represent ownership/dividends
\item Subject to securities laws
\item Examples: tokenized stocks, bonds
\end{itemize}
\end{column}

\begin{column}{0.48\textwidth}
\textbf{3. Governance Tokens}
\begin{itemize}
\item Voting rights on protocol changes
\item DAO participation
\item Examples: UNI, AAVE, MKR
\end{itemize}

\vspace{0.3cm}
\textbf{4. Store of Value Tokens}
\begin{itemize}
\item Digital gold analogy
\item Limited supply
\item Example: BTC
\end{itemize}
\end{column}
\end{columns}
\end{frame}

\begin{frame}[t]{Token Types Characteristics}
\begin{center}
\includegraphics[width=0.65\textwidth]{../charts/01_token_types_comparison/chart.pdf}
\end{center}
\bottomnote{Security tokens have highest regulatory burden; governance tokens offer strongest voting rights}
\end{frame}

\begin{frame}{Utility Tokens: Deep Dive}
\textbf{Purpose:} Grant access to specific products or services within an ecosystem.

\vspace{0.3cm}
\textbf{Characteristics:}
\begin{itemize}
\item Not designed as investments
\item Required for platform interaction
\item Value linked to platform usage
\item May offer fee discounts
\end{itemize}

\vspace{0.3cm}
\textbf{Example: Binance Coin (BNB)}
\begin{itemize}
\item Trading fee discounts on Binance exchange
\item Gas fees on BNB Chain
\item Participation in token sales (Launchpad)
\item Quarterly token burns based on trading volume
\end{itemize}
\end{frame}

\begin{frame}{Security Tokens}
\textbf{Definition:} Tokens that represent investment contracts and are subject to securities regulations.

\vspace{0.3cm}
\textbf{Howey Test Criteria:}
\begin{enumerate}
\item Investment of money
\item In a common enterprise
\item With expectation of profits
\item Derived from efforts of others
\end{enumerate}

\vspace{0.3cm}
\textbf{Implications:}
\begin{itemize}
\item Must comply with SEC regulations (in US)
\item Require registration or exemption
\item Investor protections apply
\item Limited to accredited investors (often)
\end{itemize}
\end{frame}

\begin{frame}{Governance Tokens}
\textbf{Purpose:} Enable decentralized decision-making in protocols.

\vspace{0.3cm}
\textbf{Voting Rights:}
\begin{itemize}
\item Protocol parameter changes
\item Treasury allocation
\item Fee structure modifications
\item Smart contract upgrades
\end{itemize}

\vspace{0.3cm}
\textbf{Example: Uniswap (UNI)}
\begin{itemize}
\item Governance over protocol fee switch
\item Treasury management (billions in assets)
\item Grant program decisions
\item 1 UNI = 1 vote (with delegation)
\end{itemize}
\end{frame}

\begin{frame}{Value Accrual Mechanisms}
\textbf{How do tokens capture value from protocol success?}

\vspace{0.3cm}
\begin{enumerate}
\item \textbf{Fee Distribution}
\begin{itemize}
\item Protocol fees shared with token holders
\item Example: GMX distributes 30\% of trading fees
\end{itemize}

\item \textbf{Staking Rewards}
\begin{itemize}
\item Lock tokens to earn yield
\item Example: Ethereum validators earn ETH rewards
\end{itemize}

\item \textbf{Token Burns}
\begin{itemize}
\item Reduce circulating supply
\item Example: BNB quarterly burns, ETH EIP-1559
\end{itemize}
\end{enumerate}
\end{frame}

\begin{frame}[t]{Value Accrual Mechanisms Comparison}
\begin{center}
\includegraphics[width=0.65\textwidth]{../charts/02_value_accrual_mechanisms/chart.pdf}
\end{center}
\bottomnote{Fee distribution provides direct value; staking rewards offer best sustainability}
\end{frame}

\begin{frame}{Supply Models: Fixed Supply}
\textbf{Characteristics:}
\begin{itemize}
\item Predetermined maximum supply
\item No new tokens created after cap
\item Deflationary if tokens are burned
\end{itemize}

\vspace{0.3cm}
\textbf{Bitcoin Example:}
\begin{itemize}
\item Maximum supply: 21 million BTC
\item Current supply: ~19.5 million (as of 2024)
\item Halving every 210,000 blocks (~4 years)
\item Final BTC mined around year 2140
\end{itemize}

\vspace{0.3cm}
\textbf{Advantages:}
\begin{itemize}
\item Scarcity creates potential value appreciation
\item Predictable monetary policy
\item Protection against inflation
\end{itemize}
\end{frame}

\begin{frame}{Supply Models: Inflationary vs. Deflationary}
\textbf{Inflationary:}
\begin{itemize}
\item New tokens continuously created
\item Inflation rate may be fixed or decreasing
\item Incentivizes staking/participation
\end{itemize}

\vspace{0.3cm}
\textbf{Deflationary:}
\begin{itemize}
\item Circulating supply decreases over time
\item Tokens removed through burns
\item Creates scarcity pressure
\end{itemize}

\vspace{0.3cm}
\textbf{BNB Burn Mechanism:}
\begin{itemize}
\item Auto-Burn: based on BNB Chain gas fees
\item Target: burn until 100M BNB remains (from 200M initial)
\item Current supply: ~144M BNB (Dec 2024)
\end{itemize}
\end{frame}

\begin{frame}[t]{Supply Model Projections}
\begin{center}
\includegraphics[width=0.65\textwidth]{../charts/03_supply_models/chart.pdf}
\end{center}
\bottomnote{BTC approaches 21M cap; BNB burns toward 100M; ETH slightly deflationary post-merge}
\end{frame}

\begin{frame}{Emission Schedules}
\textbf{How tokens enter circulation over time:}

\vspace{0.3cm}
\textbf{1. Linear Emission}
\begin{itemize}
\item Constant rate of new tokens
\item Predictable but perpetual inflation
\end{itemize}

\textbf{2. Decreasing Emission}
\begin{itemize}
\item Halving events (Bitcoin model)
\item Gradually reducing inflation
\end{itemize}

\textbf{3. Exponential Decay}
\begin{itemize}
\item Rapid initial distribution
\item Asymptotically approaches max supply
\end{itemize}

\textbf{4. Algorithmic Adjustment}
\begin{itemize}
\item Based on network metrics
\item Example: Ethereum's variable issuance
\end{itemize}
\end{frame}

\begin{frame}[t]{Emission Schedule Comparison}
\begin{center}
\includegraphics[width=0.65\textwidth]{../charts/06_emission_schedules/chart.pdf}
\end{center}
\bottomnote{DeFi protocols often use front-loaded emission; Bitcoin uses halving model}
\end{frame}

\begin{frame}{Token Velocity Problem}
\textbf{Problem:} High token velocity (frequent buying/selling) can suppress token price.

\vspace{0.3cm}
\textbf{Equation of Exchange:}
\[
MV = PQ
\]
where:
\begin{itemize}
\item $M$ = Money supply (token supply)
\item $V$ = Velocity (transaction frequency)
\item $P$ = Price level
\item $Q$ = Quantity of goods/services
\end{itemize}

\vspace{0.3cm}
\textbf{Solutions:}
\begin{itemize}
\item Staking mechanisms (reduce velocity)
\item Vesting periods
\item Utility that requires holding (governance, fee discounts)
\end{itemize}
\end{frame}

\begin{frame}{Token Sinks vs. Faucets}
\begin{columns}[T]
\begin{column}{0.48\textwidth}
\textbf{Token Sinks (Removal)}
\begin{itemize}
\item Transaction fee burns
\item Protocol penalties (slashing)
\item Staking locks
\item Governance participation locks
\end{itemize}

\vspace{0.3cm}
\textbf{Effect:}
Reduce circulating supply, increase scarcity.
\end{column}

\begin{column}{0.48\textwidth}
\textbf{Token Faucets (Creation)}
\begin{itemize}
\item Block rewards
\item Liquidity mining rewards
\item Airdrops
\item Developer grants
\end{itemize}

\vspace{0.3cm}
\textbf{Effect:}
Increase supply, incentivize participation.
\end{column}
\end{columns}

\vspace{0.3cm}
\textbf{Balance:} Healthy tokenomics requires equilibrium between sinks and faucets.
\end{frame}

\begin{frame}[t]{Token Supply Dynamics}
\begin{center}
\includegraphics[width=0.65\textwidth]{../charts/04_token_sinks_faucets/chart.pdf}
\end{center}
\bottomnote{Healthy tokenomics balances faucets (creation) with sinks (removal)}
\end{frame}

\begin{frame}{Case Study: ETH Tokenomics Evolution}
\textbf{Pre-EIP-1559 (Before Aug 2021):}
\begin{itemize}
\item Inflationary: ~4.5\% annual issuance
\item All fees go to miners
\item Unpredictable fee market
\end{itemize}

\vspace{0.3cm}
\textbf{Post-EIP-1559:}
\begin{itemize}
\item Base fee burned (deflationary mechanism)
\item Priority tips to miners
\item More predictable fees
\end{itemize}

\vspace{0.3cm}
\textbf{Post-Merge (Sept 2022):}
\begin{itemize}
\item Issuance reduced ~90\% (from ~4.5\% to ~0.5\%)
\item Net deflationary during high activity
\item Staking yields: 3-5\% APR
\end{itemize}
\end{frame}

\begin{frame}[t]{ETH Supply Evolution}
\begin{center}
\includegraphics[width=0.65\textwidth]{../charts/05_eth_supply_evolution/chart.pdf}
\end{center}
\bottomnote{EIP-1559 introduced burns; The Merge cut issuance 90\%, achieving net deflation}
\end{frame}

\begin{frame}{Good vs. Bad Tokenomics}
\begin{columns}[T]
\begin{column}{0.48\textwidth}
\textbf{Good Tokenomics}
\begin{itemize}
\item Clear value accrual
\item Sustainable incentives
\item Low initial team/VC allocation
\item Long vesting periods
\item Transparent distribution
\item Real utility beyond speculation
\end{itemize}
\end{column}

\begin{column}{0.48\textwidth}
\textbf{Bad Tokenomics}
\begin{itemize}
\item No clear value capture
\item Ponzi-like dynamics
\item High team/insider allocation
\item Short/no vesting
\item Opaque distribution
\item Pure speculation
\end{itemize}
\end{column}
\end{columns}

\vspace{0.3cm}
\textbf{Red Flags:}
\begin{itemize}
\item Excessive token supply to insiders (>30\%)
\item Unsustainable yield promises (>100\% APY)
\end{itemize}
\end{frame}

\begin{frame}{Key Metrics to Evaluate}
\begin{enumerate}
\item \textbf{Circulating Supply vs. Total Supply}
\begin{itemize}
\item Large difference indicates future dilution
\end{itemize}

\item \textbf{Token Distribution}
\begin{itemize}
\item Top holder concentration
\item Team/VC percentage
\end{itemize}

\item \textbf{Inflation Rate}
\begin{itemize}
\item Current and projected annual inflation
\end{itemize}

\item \textbf{Value Accrual Mechanisms}
\begin{itemize}
\item Fee sharing, burns, staking rewards
\end{itemize}

\item \textbf{Unlock Schedule}
\begin{itemize}
\item When do vested tokens enter circulation?
\end{itemize}
\end{enumerate}
\end{frame}

\begin{frame}{Summary}
\textbf{Key Takeaways:}
\begin{itemize}
\item Token type determines purpose and regulatory treatment
\item Value accrual mechanisms connect token price to protocol success
\item Supply models (fixed, inflationary, deflationary) have trade-offs
\item Good tokenomics aligns incentives across all stakeholders
\item Token velocity must be managed to maintain value
\item Transparent distribution and vesting are crucial
\end{itemize}

\vspace{0.3cm}
\textbf{Next Lecture:} Distribution and Vesting - How tokens are allocated and released over time.
\end{frame}

\begin{frame}{Questions for Reflection}
\begin{enumerate}
\item How does Bitcoin's fixed supply model compare to Ethereum's flexible issuance?
\item What are the trade-offs between utility tokens and governance tokens?
\item How can a protocol reduce token velocity without harming liquidity?
\item Why might high team/VC allocations be problematic?
\item What role do token burns play in value accrual?
\end{enumerate}
\end{frame}

\end{document}
