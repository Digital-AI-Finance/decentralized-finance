\documentclass[8pt,aspectratio=169]{beamer}
\usetheme{Madrid}
\usepackage[utf8]{inputenc}
\usepackage{graphicx}
\usepackage{booktabs}
\usepackage{hyperref}
\usepackage{amsmath}

\newcommand{\bottomnote}[1]{\vfill\par\noindent\footnotesize\textit{#1}}

\title{L30: Distribution and Vesting}
\subtitle{Module D: Tokenomics}
\author{Blockchain \& Cryptocurrency}
\date{December 2025}

\begin{document}

\frame{\titlepage}

\begin{frame}{Learning Objectives}
\begin{itemize}
\item Understand token allocation categories and their purposes
\item Analyze vesting schedules and cliff periods
\item Evaluate the impact of unlock events on token price
\item Apply distribution analysis to real projects
\item Case Study: Solana token unlocks
\end{itemize}
\end{frame}

\begin{frame}{Token Distribution Overview}
\textbf{Definition:} The initial allocation of tokens among different stakeholder groups.

\vspace{0.2cm}
\textbf{Common Allocation Categories:}
\begin{itemize}
\item Team and Founders / Early Investors (Seed, Series A/B/C)
\item Advisors / Community Sale (ICO/IEO/IDO)
\item Ecosystem Development / Liquidity Provision
\item Foundation/Treasury / Airdrops
\end{itemize}

\vspace{0.2cm}
\textbf{Goal:} Balance early supporter rewards with long-term community alignment.
\end{frame}

\begin{frame}[t]{Typical Allocation Breakdown}
\vspace{-0.2cm}
\begin{center}
\includegraphics[width=0.42\textwidth]{../charts/01_typical_allocation/chart.pdf}
\end{center}
\bottomnote{Red Flag: Team + Investors > 50\% indicates high centralization risk}
\end{frame}

\begin{frame}{Team and Founder Allocation}
\textbf{Purpose:} Incentivize long-term commitment and alignment.

\vspace{0.3cm}
\textbf{Best Practices:}
\begin{itemize}
\item Allocation: 15-20\% of total supply
\item Vesting: 4 years minimum
\item Cliff: 1 year (no tokens until 12 months)
\item Linear unlock after cliff
\end{itemize}

\vspace{0.3cm}
\textbf{Why Long Vesting?}
\begin{itemize}
\item Prevents ``pump and dump'' by insiders
\item Demonstrates commitment to project
\item Aligns incentives with long-term success
\item Builds trust with community
\end{itemize}

\vspace{0.3cm}
\textbf{Example:} Team gets 20M tokens, 1-year cliff, then 416,667 tokens/month for 48 months.
\end{frame}

\begin{frame}{Investor Allocation Rounds}
\textbf{Typical Investment Rounds:}

\vspace{0.3cm}
\begin{tabular}{llll}
\toprule
Round & Allocation & Price & Vesting \\
\midrule
Seed & 5-10\% & Lowest & 2-3 years \\
Series A & 5-10\% & Low & 18-24 months \\
Series B & 5-10\% & Medium & 12-18 months \\
Public Sale & 10-15\% & Highest & 0-6 months \\
\bottomrule
\end{tabular}

\vspace{0.3cm}
\textbf{Key Considerations:}
\begin{itemize}
\item Earlier investors = longer vesting
\item Discount compensates for higher risk
\item Too many investors = fragmented governance
\end{itemize}
\end{frame}

\begin{frame}[t]{Investment Round Characteristics}
\begin{center}
\includegraphics[width=0.65\textwidth]{../charts/05_investor_rounds/chart.pdf}
\end{center}
\bottomnote{Earlier rounds get bigger discounts but longer lock-ups and higher risk}
\end{frame}

\begin{frame}{Community Sale Methods}
\textbf{1. ICO (Initial Coin Offering) - 2017 Era}
\begin{itemize}
\item Direct token sale to public
\item Minimal regulation (historically)
\end{itemize}

\textbf{2. IEO (Initial Exchange Offering)}
\begin{itemize}
\item Conducted on centralized exchange
\item Exchange vets project (some due diligence)
\end{itemize}

\textbf{3. IDO (Initial DEX Offering)}
\begin{itemize}
\item Launch on decentralized exchange
\item Immediate liquidity
\end{itemize}

\textbf{4. Fair Launch}
\begin{itemize}
\item No pre-sale or VC rounds
\item Everyone buys at same price (Example: YFI)
\end{itemize}
\end{frame}

\begin{frame}{What is Vesting?}
\textbf{Definition:} A schedule that controls when token holders can access their allocated tokens.

\vspace{0.3cm}
\textbf{Key Terms:}
\begin{itemize}
\item \textbf{Cliff Period:} Initial waiting period before any tokens unlock
\item \textbf{Vesting Period:} Total duration until all tokens are unlocked
\item \textbf{Linear Vesting:} Equal amounts unlock at regular intervals
\item \textbf{Unlock Event:} Specific date when tokens become transferable
\end{itemize}

\vspace{0.3cm}
\textbf{Example:}
\begin{itemize}
\item Total allocation: 1,000,000 tokens
\item Cliff: 12 months (0 tokens unlocked)
\item Vesting: 48 months total
\item After cliff: 27,778 tokens unlock per month for 36 months
\end{itemize}
\end{frame}

\begin{frame}[t]{Vesting Schedule Types}
\begin{center}
\includegraphics[width=0.65\textwidth]{../charts/02_vesting_schedule_types/chart.pdf}
\end{center}
\bottomnote{Back-loaded vesting provides strongest long-term alignment; accelerated is risky}
\end{frame}

\begin{frame}{Cliff Periods}
\textbf{Purpose:} Ensure minimum commitment before any tokens unlock.

\vspace{0.3cm}
\textbf{Typical Cliff Durations:}
\begin{itemize}
\item Team: 12 months
\item Investors: 6-12 months
\item Advisors: 6 months
\item Community: Often 0 (immediate unlock)
\end{itemize}

\vspace{0.3cm}
\textbf{Why Use Cliffs?}
\begin{itemize}
\item Prevents immediate dumping
\item Tests commitment of team/advisors
\item Provides time for project to mature
\item Reduces initial circulating supply
\end{itemize}

\vspace{0.3cm}
\textbf{Investor Perspective:} Cliffs protect against early team departures.
\end{frame}

\begin{frame}{Token Unlock Events}
\textbf{Definition:} Dates when large amounts of vested tokens become tradable.

\vspace{0.3cm}
\textbf{Market Impact:}
\begin{itemize}
\item Increase circulating supply
\item Potential selling pressure
\item Price often drops before/during unlock
\item Market anticipates and prices in
\end{itemize}

\vspace{0.3cm}
\textbf{Types of Unlocks:}
\begin{enumerate}
\item \textbf{Cliff Unlocks:} Large one-time release
\item \textbf{Linear Unlocks:} Continuous monthly/daily releases
\item \textbf{TGE (Token Generation Event):} Initial launch unlocks
\end{enumerate}

\vspace{0.3cm}
\textbf{Tracking:} Use tools like Token Unlocks, Messari, CoinGecko to monitor upcoming events.
\end{frame}

\begin{frame}[t]{Unlock Event Price Impact}
\begin{center}
\includegraphics[width=0.65\textwidth]{../charts/04_unlock_price_impact/chart.pdf}
\end{center}
\bottomnote{Market front-runs unlocks; price typically drops 15-20\% around major events}
\end{frame}

\begin{frame}{Case Study: Solana (SOL) Unlocks}
\textbf{Background:}
\begin{itemize}
\item Total supply: 500M SOL (inflating to ~700M by 2030)
\item Major unlocks from 2021-2025
\item Massive VC backing (a16z, Multicoin, etc.)
\end{itemize}

\vspace{0.3cm}
\textbf{Key Unlock Events:}
\begin{itemize}
\item \textbf{Jan 2023:} 13.8M SOL unlocked (seed investors)
\item \textbf{Mar 2023:} 17.8M SOL unlocked (early investors)
\item \textbf{Ongoing:} Monthly validator/foundation unlocks
\end{itemize}

\vspace{0.3cm}
\textbf{Price Impact:}
\begin{itemize}
\item SOL price dropped 15-20\% around major unlocks
\item Long-term recovery as adoption grew
\end{itemize}
\end{frame}

\begin{frame}[t]{Solana Distribution Breakdown}
\begin{center}
\includegraphics[width=0.65\textwidth]{../charts/03_solana_distribution/chart.pdf}
\end{center}
\bottomnote{Insiders ~35\% with long vesting helped manage dilution over time}
\end{frame}

\begin{frame}{Managing Unlock Pressure}
\textbf{Strategies to Mitigate Selling Pressure:}

\vspace{0.3cm}
\begin{enumerate}
\item \textbf{Staking Incentives}
\begin{itemize}
\item Offer high yields to lock unlocked tokens
\item Example: Solana staking ~7\% APY
\end{itemize}

\item \textbf{Gradual Unlocks}
\begin{itemize}
\item Daily/monthly instead of quarterly
\item Smooths supply shock
\end{itemize}

\item \textbf{Lockup Extensions}
\begin{itemize}
\item Voluntary additional vesting
\item Bonus tokens for extended locks
\end{itemize}

\item \textbf{Strong Fundamentals}
\begin{itemize}
\item Demand growth offsets supply increase
\end{itemize}
\end{enumerate}
\end{frame}

\begin{frame}{Airdrops as Distribution}
\textbf{Definition:} Free distribution of tokens to users based on specific criteria.

\vspace{0.3cm}
\textbf{Common Airdrop Types:}
\begin{itemize}
\item \textbf{Retroactive:} Reward past users (Uniswap UNI)
\item \textbf{Holder Airdrop:} To existing token holders
\item \textbf{Task-based:} Complete specific actions
\item \textbf{Governance:} For DAO participation
\end{itemize}

\vspace{0.3cm}
\textbf{Benefits:}
\begin{itemize}
\item Bootstrap community
\item Decentralize ownership
\item Reward early adopters
\end{itemize}

\vspace{0.3cm}
\textbf{Risks:}
\begin{itemize}
\item Airdrop hunters (not genuine users)
\item Immediate selling pressure
\end{itemize}
\end{frame}

\begin{frame}{Famous Airdrop: Uniswap (UNI)}
\textbf{September 2020 Airdrop:}
\begin{itemize}
\item 400 UNI per address that used Uniswap before Sept 1, 2020
\item ~250,000 addresses eligible
\item Total airdropped: ~150M UNI (15\% of supply)
\item Value at claim: ~\$1,200 per user
\end{itemize}

\vspace{0.3cm}
\textbf{Impact:}
\begin{itemize}
\item Instant governance decentralization
\item Massive publicity and user growth
\item Many users held long-term (strong community)
\item Set standard for future DeFi airdrops
\end{itemize}

\vspace{0.3cm}
\textbf{Key Insight:} Retroactive airdrops reward genuine users, not speculators.
\end{frame}

\begin{frame}{Red Flags in Distribution}
\textbf{Warning Signs:}
\begin{enumerate}
\item \textbf{Excessive Insider Allocation}
\begin{itemize}
\item Team + VCs > 50\%
\end{itemize}

\item \textbf{Short or No Vesting}
\begin{itemize}
\item Team vesting < 2 years
\item No cliff period
\end{itemize}

\item \textbf{Opaque Distribution}
\begin{itemize}
\item No published allocation details
\item Unknown wallet holders
\end{itemize}

\item \textbf{Centralized Control}
\begin{itemize}
\item Foundation holds >30\% indefinitely
\end{itemize}
\end{enumerate}
\end{frame}

\begin{frame}{2024 Trend: Points Programs}
\textbf{What are Points?}
\begin{itemize}
\item Off-chain loyalty system tracking user activity
\item Converted to tokens at future TGE
\item Replaced traditional airdrops as primary distribution mechanism
\end{itemize}

\vspace{0.3cm}
\textbf{How Points Work:}
\begin{enumerate}
\item Protocol tracks user actions (deposits, trades, referrals)
\item Users accumulate ``points'' proportional to activity
\item At TGE, points convert to tokens based on total points issued
\end{enumerate}

\vspace{0.3cm}
\textbf{Examples (2024):}
\begin{itemize}
\item \textbf{EigenLayer}: Points for restaking
\item \textbf{Blast}: Points + Gold for L2 deposits
\item \textbf{Ethena}: Points for USDe staking
\end{itemize}

\vspace{0.3cm}
\textbf{Criticism:} Opaque, favors whales, mercenary capital
\end{frame}

\begin{frame}{Summary}
\textbf{Key Takeaways:}
\begin{itemize}
\item Token distribution determines ownership concentration
\item Healthy projects: Team + VCs < 40\%, long vesting (3-4 years)
\item Cliff periods prevent immediate insider selling
\item Unlock events create predictable selling pressure
\item 2024 Trend: Points programs replacing traditional airdrops
\item Transparency in vesting builds trust
\item Always check Token Unlocks before investing
\end{itemize}

\vspace{0.3cm}
\textbf{Next Lecture:} Token Classification and Valuation - Regulatory frameworks and how to value tokens.
\end{frame}

\begin{frame}{Questions for Reflection}
\begin{enumerate}
\item Why do early investors typically have longer vesting than public sale participants?
\item How did Solana's unlock events impact its price trajectory?
\item What are the pros and cons of retroactive airdrops vs. task-based airdrops?
\item How can a project mitigate selling pressure during major unlocks?
\item What vesting schedule would you design for a new token launch?
\end{enumerate}
\end{frame}

\end{document}
