\documentclass[8pt,aspectratio=169]{beamer}
\usetheme{Madrid}
\usepackage[utf8]{inputenc}
\usepackage{graphicx}
\usepackage{booktabs}
\usepackage{hyperref}
\usepackage{amsmath}

\newcommand{\bottomnote}[1]{\vfill\par\noindent\footnotesize\textit{#1}}

\title{L31: Token Classification and Valuation}
\subtitle{Module D: Tokenomics}
\author{Blockchain \& Cryptocurrency}
\date{December 2025}

\begin{document}

\frame{\titlepage}

\begin{frame}{Learning Objectives}
\begin{itemize}
\item Understand the Howey Test and securities classification
\item Distinguish between utility and security tokens
\item Apply valuation frameworks to crypto assets
\item Analyze network value metrics (NVT, Metcalfe)
\item Case Study: SEC vs. Ripple
\end{itemize}
\end{frame}

\begin{frame}{Why Classification Matters}
\textbf{Regulatory Implications:}
\begin{itemize}
\item Securities require registration with SEC (in US)
\item Investor protections apply
\item Trading restrictions (accredited investors only)
\item Disclosure requirements
\end{itemize}

\vspace{0.3cm}
\textbf{Market Implications:}
\begin{itemize}
\item Exchange listings (securities can't list on most exchanges)
\item Global accessibility
\item Tax treatment
\end{itemize}

\vspace{0.3cm}
\textbf{Bottom Line:} Misclassification can lead to enforcement actions, delisting, and legal penalties.
\end{frame}

\begin{frame}{The Howey Test (1946)}
\textbf{Origin:} SEC v. W.J. Howey Co. (Supreme Court case)

\vspace{0.3cm}
\textbf{An investment is a security if it involves:}
\begin{enumerate}
\item \textbf{Investment of Money} - Investors provide capital
\item \textbf{In a Common Enterprise} - Pooled funds or shared outcome
\item \textbf{With Expectation of Profits} - Investors seek financial return
\item \textbf{Derived from Efforts of Others} - Profits depend on promoter/third party work
\end{enumerate}

\vspace{0.3cm}
\textbf{All four criteria must be met} for classification as a security.
\end{frame}

\begin{frame}[t]{Howey Test Applied to Tokens}
\begin{center}
\includegraphics[width=0.65\textwidth]{../charts/01_howey_test_criteria/chart.pdf}
\end{center}
\bottomnote{BTC and ETH largely avoid ``efforts of others'' due to decentralization}
\end{frame}

\begin{frame}{Utility vs. Security Tokens}
\begin{columns}[T]
\begin{column}{0.48\textwidth}
\textbf{Utility Token}
\begin{itemize}
\item Access to product/service
\item Not marketed as investment
\item Value from usage, not speculation
\item Examples: BNB, FIL
\end{itemize}

\vspace{0.3cm}
\textbf{Howey Test:}
\begin{itemize}
\item Fails ``efforts of others'' if decentralized
\end{itemize}
\end{column}

\begin{column}{0.48\textwidth}
\textbf{Security Token}
\begin{itemize}
\item Investment contract
\item Promise of profits
\item Centralized management
\item Examples: Tokenized stocks
\end{itemize}

\vspace{0.3cm}
\textbf{Howey Test:}
\begin{itemize}
\item Meets all four criteria
\end{itemize}
\end{column}
\end{columns}

\vspace{0.3cm}
\textbf{Reality:} Most tokens exist on a spectrum, not binary classification.
\end{frame}

\begin{frame}{Case Study: SEC vs. Ripple}
\textbf{Background:}
\begin{itemize}
\item Ripple Labs created XRP (2012)
\item Used for cross-border payments
\item Ripple holds ~50\% of XRP supply
\item \$1.3B raised from XRP sales (2013-2020)
\end{itemize}

\vspace{0.3cm}
\textbf{SEC Complaint (Dec 2020):}
\begin{itemize}
\item XRP is an unregistered security
\item Ripple raised funds through illegal securities offering
\end{itemize}

\vspace{0.3cm}
\textbf{Ripple's Defense:}
\begin{itemize}
\item XRP is a currency, not a security
\item Network is decentralized (1,000+ validators)
\end{itemize}
\end{frame}

\begin{frame}{Ripple Case: Ruling (July 2023)}
\textbf{Judge Torres Decision - Partial Victory for Ripple:}

\vspace{0.3cm}
\textbf{1. Institutional Sales = Securities}
\begin{itemize}
\item XRP sold to VCs/hedge funds = securities
\item Buyers had expectation of profits from Ripple's efforts
\end{itemize}

\textbf{2. Programmatic Sales (Exchanges) = NOT Securities}
\begin{itemize}
\item XRP sold on public exchanges = not securities
\item Buyers didn't know they were buying from Ripple
\end{itemize}

\textbf{3. Employee Compensation = NOT Securities}

\vspace{0.3cm}
\textbf{Impact:} First major ruling that distinguished sale context matters.
\end{frame}

\begin{frame}{Token Valuation Challenges}
\textbf{Why Traditional Valuation is Hard:}
\begin{itemize}
\item No cash flows (most tokens)
\item No earnings or revenue
\item No tangible assets
\item Highly speculative markets
\item Network effects hard to quantify
\end{itemize}

\vspace{0.3cm}
\textbf{Approaches:}
\begin{enumerate}
\item Network Value to Transactions (NVT)
\item Metcalfe's Law
\item Discounted Cash Flow (DCF) - for productive assets
\item Comparable Analysis
\item Cost of Production (mining)
\end{enumerate}
\end{frame}

\begin{frame}[t]{Valuation Methods Comparison}
\begin{center}
\includegraphics[width=0.65\textwidth]{../charts/02_valuation_methods/chart.pdf}
\end{center}
\bottomnote{NVT is most widely applicable; DCF only works for yield-generating tokens}
\end{frame}

\begin{frame}{Network Value to Transactions (NVT)}
\textbf{Formula:}
\[
\text{NVT Ratio} = \frac{\text{Market Cap}}{\text{Daily Transaction Volume}}
\]

\vspace{0.3cm}
\textbf{Interpretation:}
\begin{itemize}
\item Similar to P/E ratio in stocks
\item High NVT = overvalued relative to usage
\item Low NVT = undervalued or high utility
\end{itemize}

\vspace{0.3cm}
\textbf{Typical Ranges:}
\begin{itemize}
\item BTC: 50-100 (higher = store of value, not payment)
\item ETH: 20-40 (more transactional)
\item Payment tokens: <20 (high transaction volume)
\end{itemize}
\end{frame}

\begin{frame}[t]{NVT Ratio Comparison}
\begin{center}
\includegraphics[width=0.65\textwidth]{../charts/03_nvt_ratio_comparison/chart.pdf}
\end{center}
\bottomnote{BTC's high NVT reflects store-of-value use; lower NVT indicates payment utility}
\end{frame}

\begin{frame}{Metcalfe's Law}
\textbf{Concept:} Network value grows with the square of the number of users.

\vspace{0.3cm}
\textbf{Formula:}
\[
V \propto n^2
\]
where $V$ = network value, $n$ = number of users.

\vspace{0.3cm}
\textbf{Application to Crypto:}
\begin{itemize}
\item More users = exponentially more connections
\item Active addresses proxy for $n$
\item Studies show BTC/ETH follow Metcalfe's Law loosely
\end{itemize}

\vspace{0.3cm}
\textbf{Limitations:}
\begin{itemize}
\item Not all users create equal value
\item Doesn't account for quality of usage
\end{itemize}
\end{frame}

\begin{frame}[t]{Metcalfe's Law Visualization}
\begin{center}
\includegraphics[width=0.65\textwidth]{../charts/04_metcalfes_law/chart.pdf}
\end{center}
\bottomnote{Network effects create exponential value growth as user count increases}
\end{frame}

\begin{frame}{Discounted Cash Flow (DCF)}
\textbf{When to Use:} Tokens with cash flow generation (staking rewards, fee sharing).

\vspace{0.3cm}
\textbf{Formula:}
\[
\text{Value} = \sum_{t=1}^{n} \frac{\text{CF}_t}{(1 + r)^t}
\]
where $\text{CF}_t$ = cash flow in year $t$, $r$ = discount rate.

\vspace{0.3cm}
\textbf{Example: Staking Token}
\begin{itemize}
\item Expected annual staking reward: \$100
\item Discount rate: 10\%
\item Perpetual reward assumption
\end{itemize}

\[
\text{Value} = \frac{100}{0.10} = \$1,000
\]
\end{frame}

\begin{frame}{Market Cap vs. Fully Diluted Valuation}
\textbf{Market Cap:}
\[
\text{Market Cap} = \text{Price} \times \text{Circulating Supply}
\]

\textbf{Fully Diluted Valuation (FDV):}
\[
\text{FDV} = \text{Price} \times \text{Total Supply (Max)}
\]

\vspace{0.3cm}
\textbf{Example:}
\begin{itemize}
\item Token price: \$10, Circulating: 100M, Total: 1B
\item Market Cap: \$1B
\item FDV: \$10B
\end{itemize}

\vspace{0.3cm}
\textbf{Warning:} Large FDV/MC ratio indicates future dilution risk.
\end{frame}

\begin{frame}[t]{Market Cap vs FDV Comparison}
\begin{center}
\includegraphics[width=0.65\textwidth]{../charts/05_mcap_vs_fdv/chart.pdf}
\end{center}
\bottomnote{New projects often have 10x+ FDV/MC ratio indicating massive dilution ahead}
\end{frame}

\begin{frame}{Token Valuation Checklist}
\textbf{Step-by-Step Analysis:}

\vspace{0.3cm}
\begin{enumerate}
\item Determine token type and purpose (utility, governance, etc.)
\item Check regulatory status (security vs. non-security)
\item Calculate NVT ratio (compare to peers)
\item Analyze on-chain metrics (addresses, transactions, TVL)
\item Review tokenomics (supply, inflation, unlock schedule)
\item Compare market cap to fundamentals (revenue, usage)
\item Assess FDV vs. Market Cap (dilution risk)
\end{enumerate}

\vspace{0.3cm}
\textbf{Output:} Informed estimate of fair value range (not precise number).
\end{frame}

\begin{frame}{Summary}
\textbf{Key Takeaways:}
\begin{itemize}
\item Howey Test determines security classification (4 criteria)
\item Context of sale matters (Ripple case precedent)
\item NVT ratio helps assess value relative to transaction volume
\item Metcalfe's Law suggests network value grows with users squared
\item DCF applicable only to tokens with cash flows
\item Comparable analysis useful for relative valuation
\item Always compare Market Cap to FDV (dilution risk)
\item Crypto valuation is imprecise - use multiple methods
\end{itemize}
\end{frame}

\begin{frame}{Questions for Reflection}
\begin{enumerate}
\item Apply the Howey Test to a token you're familiar with. Is it a security?
\item Why did the Ripple ruling distinguish between institutional and exchange sales?
\item Calculate the NVT ratio for Bitcoin and Ethereum. What does it tell you?
\item What are the limitations of using Metcalfe's Law for token valuation?
\item How would you value a governance token with no direct cash flows?
\end{enumerate}
\end{frame}

\end{document}
