\documentclass[8pt,aspectratio=169]{beamer}
\usetheme{Madrid}
\usepackage[utf8]{inputenc}
\usepackage{graphicx}
\usepackage{booktabs}
\usepackage{hyperref}
\usepackage{amsmath}

\title{L31: Token Classification and Valuation}
\subtitle{Module D: Tokenomics}
\author{Blockchain \& Cryptocurrency}
\date{December 2025}

\begin{document}

\frame{\titlepage}

\begin{frame}{Learning Objectives}
\begin{itemize}
\item Understand the Howey Test and securities classification
\item Distinguish between utility and security tokens
\item Apply valuation frameworks to crypto assets
\item Analyze network value metrics (NVT, Metcalfe)
\item Case Study: SEC vs. Ripple
\end{itemize}
\end{frame}

\begin{frame}{Why Classification Matters}
\textbf{Regulatory Implications:}
\begin{itemize}
\item Securities require registration with SEC (in US)
\item Investor protections apply
\item Trading restrictions (accredited investors only)
\item Disclosure requirements
\item Legal liability for issuers
\end{itemize}

\vspace{0.3cm}
\textbf{Market Implications:}
\begin{itemize}
\item Exchange listings (securities can't list on most exchanges)
\item Global accessibility
\item Tax treatment
\item Institutional adoption
\end{itemize}

\vspace{0.3cm}
\textbf{Bottom Line:} Misclassification can lead to enforcement actions, delisting, and legal penalties.
\end{frame}

\begin{frame}{The Howey Test (1946)}
\textbf{Origin:} SEC v. W.J. Howey Co. (Supreme Court case)

\vspace{0.3cm}
\textbf{An investment is a security if it involves:}
\begin{enumerate}
\item \textbf{Investment of Money}
\begin{itemize}
\item Investors provide capital
\end{itemize}

\item \textbf{In a Common Enterprise}
\begin{itemize}
\item Pooled funds or shared outcome
\end{itemize}

\item \textbf{With Expectation of Profits}
\begin{itemize}
\item Investors seek financial return
\end{itemize}

\item \textbf{Derived from Efforts of Others}
\begin{itemize}
\item Profits depend on promoter/third party work
\end{itemize}
\end{enumerate}

\vspace{0.3cm}
\textbf{All four criteria must be met} for classification as a security.
\end{frame}

\begin{frame}{Applying Howey to Tokens}
\textbf{Typical ICO Analysis:}

\vspace{0.3cm}
\begin{enumerate}
\item \textbf{Investment of Money:} Yes
\begin{itemize}
\item Investors pay ETH/USD for tokens
\end{itemize}

\item \textbf{Common Enterprise:} Yes
\begin{itemize}
\item All token holders share in project success
\end{itemize}

\item \textbf{Expectation of Profits:} Usually Yes
\begin{itemize}
\item Marketing emphasizes potential returns
\item Tokens traded on secondary markets
\end{itemize}

\item \textbf{Efforts of Others:} Key Question
\begin{itemize}
\item If value depends on team's work = Security
\item If value comes from decentralized network = Maybe not
\end{itemize}
\end{enumerate}

\vspace{0.3cm}
\textbf{Gray Area:} Many tokens start as securities but may decentralize over time.
\end{frame}

\begin{frame}{Utility vs. Security Tokens}
\begin{columns}[T]
\begin{column}{0.48\textwidth}
\textbf{Utility Token}
\begin{itemize}
\item Access to product/service
\item Not marketed as investment
\item Value from usage, not speculation
\item Decentralized governance
\item Examples: BNB (exchange fees), FIL (storage)
\end{itemize}

\vspace{0.3cm}
\textbf{Howey Test:}
\begin{itemize}
\item Fails ``efforts of others'' if truly decentralized
\end{itemize}
\end{column}

\begin{column}{0.48\textwidth}
\textbf{Security Token}
\begin{itemize}
\item Investment contract
\item Promise of profits
\item Centralized management
\item Shares, equity, dividends
\item Examples: Tokenized stocks, some ICOs
\end{itemize}

\vspace{0.3cm}
\textbf{Howey Test:}
\begin{itemize}
\item Meets all four criteria
\end{itemize}
\end{column}
\end{columns}

\vspace{0.3cm}
\textbf{Reality:} Most tokens exist on a spectrum, not binary classification.
\end{frame}

\begin{frame}{SEC's Position on Crypto}
\textbf{Chair Gary Gensler's View (2021-present):}
\begin{itemize}
\item ``The vast majority of crypto tokens are securities''
\item Only BTC explicitly named as commodity (not security)
\item ETH unclear (formerly investigated, no enforcement)
\end{itemize}

\vspace{0.3cm}
\textbf{Enforcement Actions:}
\begin{itemize}
\item 2020: Ripple (XRP)
\item 2023: Coinbase (listing unregistered securities)
\item 2023: Binance (multiple violations)
\item Many ICO settlements (2018-2020)
\end{itemize}

\vspace{0.3cm}
\textbf{Industry Criticism:}
\begin{itemize}
\item ``Regulation by enforcement'' instead of clear rules
\item Calls for dedicated crypto legislation
\item Uncertainty harms innovation
\end{itemize}
\end{frame}

\begin{frame}{Case Study: SEC vs. Ripple}
\textbf{Background:}
\begin{itemize}
\item Ripple Labs created XRP (2012)
\item Used for cross-border payments
\item Ripple holds ~50\% of XRP supply
\item \$1.3B raised from XRP sales (2013-2020)
\end{itemize}

\vspace{0.3cm}
\textbf{SEC Complaint (Dec 2020):}
\begin{itemize}
\item XRP is an unregistered security
\item Ripple raised funds through illegal securities offering
\item Executives personally profited from sales
\end{itemize}

\vspace{0.3cm}
\textbf{Ripple's Defense:}
\begin{itemize}
\item XRP is a currency, not a security
\item Used for payments, not investment
\item Network is decentralized (1,000+ validators)
\item Similar to BTC/ETH (not securities)
\end{itemize}
\end{frame}

\begin{frame}{Ripple Case: Ruling (July 2023)}
\textbf{Judge Torres Decision - Partial Victory for Ripple:}

\vspace{0.3cm}
\textbf{1. Institutional Sales = Securities}
\begin{itemize}
\item XRP sold to VCs/hedge funds = securities
\item Buyers had expectation of profits from Ripple's efforts
\item Violates securities laws
\end{itemize}

\textbf{2. Programmatic Sales (Exchanges) = NOT Securities}
\begin{itemize}
\item XRP sold on public exchanges = not securities
\item Buyers didn't know they were buying from Ripple
\item No direct contract or promise
\end{itemize}

\textbf{3. Employee Compensation = NOT Securities}
\begin{itemize}
\item XRP given to employees = not securities
\end{itemize}

\vspace{0.3cm}
\textbf{Impact:} First major ruling that distinguished sale context matters.
\end{frame}

\begin{frame}{Ripple Case: Implications}
\textbf{For XRP:}
\begin{itemize}
\item Price rallied 70\%+ on ruling
\item Some exchanges relisted XRP
\item Still uncertainty (SEC appealed ruling)
\end{itemize}

\vspace{0.3cm}
\textbf{For Crypto Industry:}
\begin{itemize}
\item Context of sale matters, not just token characteristics
\item Programmatic sales may have safer path
\item Decentralization over time could help
\item Not full clarity (district court, not precedent for all courts)
\end{itemize}

\vspace{0.3cm}
\textbf{Ongoing Issues:}
\begin{itemize}
\item SEC appealed decision (ongoing as of Dec 2024)
\item Other tokens still face scrutiny
\item Need for congressional legislation
\end{itemize}
\end{frame}

\begin{frame}{Token Valuation Challenges}
\textbf{Why Traditional Valuation is Hard:}
\begin{itemize}
\item No cash flows (most tokens)
\item No earnings or revenue
\item No tangible assets
\item Highly speculative markets
\item Network effects hard to quantify
\item Regulatory uncertainty
\end{itemize}

\vspace{0.3cm}
\textbf{Approaches:}
\begin{enumerate}
\item Network Value to Transactions (NVT)
\item Metcalfe's Law
\item Discounted Cash Flow (DCF) - for productive assets
\item Comparable Analysis
\item Cost of Production (mining)
\end{enumerate}

\vspace{0.3cm}
\textbf{Reality:} Valuation is more art than science in crypto.
\end{frame}

\begin{frame}{Network Value to Transactions (NVT)}
\textbf{Formula:}
\[
\text{NVT Ratio} = \frac{\text{Market Cap}}{\text{Daily Transaction Volume}}
\]

\vspace{0.3cm}
\textbf{Interpretation:}
\begin{itemize}
\item Similar to P/E ratio in stocks
\item High NVT = overvalued relative to usage
\item Low NVT = undervalued or high utility
\end{itemize}

\vspace{0.3cm}
\textbf{Typical Ranges:}
\begin{itemize}
\item BTC: 50-100 (higher = store of value, not payment)
\item ETH: 20-40 (more transactional)
\item Payment tokens: <20 (high transaction volume)
\end{itemize}

\vspace{0.3cm}
\textbf{Limitations:}
\begin{itemize}
\item Wash trading inflates volume
\item Doesn't account for future growth
\item Different tokens have different purposes
\end{itemize}
\end{frame}

\begin{frame}{NVT Example Calculation}
\textbf{Hypothetical Token X:}
\begin{itemize}
\item Market Cap: \$1,000,000,000 (1 billion)
\item Daily Transaction Volume: \$50,000,000 (50 million)
\end{itemize}

\vspace{0.3cm}
\textbf{Calculation:}
\[
\text{NVT} = \frac{1{,}000{,}000{,}000}{50{,}000{,}000} = 20
\]

\vspace{0.3cm}
\textbf{Analysis:}
\begin{itemize}
\item NVT of 20 is moderate
\item Comparable to Ethereum's historical range
\item If NVT > 50: potentially overvalued
\item If NVT < 10: potentially undervalued or high transaction spam
\end{itemize}

\vspace{0.3cm}
\textbf{Action:} Compare to similar tokens and historical trends.
\end{frame}

\begin{frame}{Metcalfe's Law}
\textbf{Concept:} Network value grows with the square of the number of users.

\vspace{0.3cm}
\textbf{Formula:}
\[
V \propto n^2
\]
where $V$ = network value, $n$ = number of users.

\vspace{0.3cm}
\textbf{Application to Crypto:}
\begin{itemize}
\item More users = exponentially more connections
\item Active addresses proxy for $n$
\item Studies show BTC/ETH follow Metcalfe's Law loosely
\end{itemize}

\vspace{0.3cm}
\textbf{Example:}
\begin{itemize}
\item Network with 100 users: Value $\propto$ 10,000
\item Network with 1,000 users: Value $\propto$ 1,000,000 (100x growth)
\end{itemize}

\vspace{0.3cm}
\textbf{Limitations:}
\begin{itemize}
\item Not all users create equal value
\item Doesn't account for quality of usage
\item Empirical fit varies by token
\end{itemize}
\end{frame}

\begin{frame}{Discounted Cash Flow (DCF)}
\textbf{When to Use:} Tokens with cash flow generation (staking rewards, fee sharing).

\vspace{0.3cm}
\textbf{Formula:}
\[
\text{Value} = \sum_{t=1}^{n} \frac{\text{CF}_t}{(1 + r)^t}
\]
where $\text{CF}_t$ = cash flow in year $t$, $r$ = discount rate.

\vspace{0.3cm}
\textbf{Example: Staking Token}
\begin{itemize}
\item Expected annual staking reward: \$100
\item Discount rate: 10\%
\item Perpetual reward assumption
\end{itemize}

\[
\text{Value} = \frac{100}{0.10} = \$1,000
\]

\vspace{0.3cm}
\textbf{Challenges:}
\begin{itemize}
\item Predicting future cash flows in volatile markets
\item Choosing appropriate discount rate
\item Many tokens have no direct cash flows
\end{itemize}
\end{frame}

\begin{frame}{Comparable Analysis}
\textbf{Method:} Compare token to similar projects using multiples.

\vspace{0.3cm}
\textbf{Common Metrics:}
\begin{itemize}
\item Market Cap / TVL (DeFi protocols)
\item Market Cap / Daily Active Users
\item Market Cap / Revenue (if applicable)
\item Market Cap / Transaction Volume
\end{itemize}

\vspace{0.3cm}
\textbf{Example: DeFi Protocol Valuation}
\begin{itemize}
\item Comparable protocols: Aave (MC/TVL = 0.5), Compound (MC/TVL = 0.3)
\item Your protocol TVL: \$500M
\item Average multiple: 0.4
\item Estimated Market Cap: \$500M $\times$ 0.4 = \$200M
\end{itemize}

\vspace{0.3cm}
\textbf{Limitations:}
\begin{itemize}
\item Market multiples change rapidly
\item Protocols differ in features/risks
\item Circular reasoning if whole sector overvalued
\end{itemize}
\end{frame}

\begin{frame}{Cost of Production (Bitcoin)}
\textbf{Theory:} Long-term price gravitates toward mining cost.

\vspace{0.3cm}
\textbf{Components:}
\begin{itemize}
\item Hardware (ASIC miners)
\item Electricity ($\sim$\$0.03-\$0.10 per kWh)
\item Cooling and facility costs
\item Labor and maintenance
\end{itemize}

\vspace{0.3cm}
\textbf{Estimated BTC Mining Cost (2024):}
\begin{itemize}
\item Efficient miners: \$15,000-\$25,000 per BTC
\item Inefficient miners: \$30,000-\$40,000 per BTC
\end{itemize}

\vspace{0.3cm}
\textbf{Observations:}
\begin{itemize}
\item Price often exceeds cost during bull markets
\item Price can fall below cost temporarily (bear markets)
\item Difficulty adjusts to maintain profitability
\end{itemize}

\vspace{0.3cm}
\textbf{Note:} Doesn't apply to non-PoW tokens.
\end{frame}

\begin{frame}{Fundamental vs. Speculative Value}
\begin{columns}[T]
\begin{column}{0.48\textwidth}
\textbf{Fundamental Value}
\begin{itemize}
\item Network utility
\item Transaction demand
\item Staking yields
\item Fee generation
\item User growth
\item Developer activity
\end{itemize}

\vspace{0.3cm}
\textbf{Analysis:}
On-chain metrics, usage data, revenue.
\end{column}

\begin{column}{0.48\textwidth}
\textbf{Speculative Value}
\begin{itemize}
\item Market sentiment
\item Hype cycles
\item Social media trends
\item Influencer endorsements
\item FOMO/FUD dynamics
\item Meme appeal
\end{itemize}

\vspace{0.3cm}
\textbf{Analysis:}
Sentiment analysis, technical charts, volatility.
\end{column}
\end{columns}

\vspace{0.3cm}
\textbf{Reality:} Crypto prices driven by both, but speculation often dominates short-term.
\end{frame}

\begin{frame}{On-Chain Metrics for Valuation}
\textbf{Key Indicators:}

\vspace{0.3cm}
\begin{enumerate}
\item \textbf{Active Addresses}
\begin{itemize}
\item Daily/monthly unique addresses
\item Proxy for network adoption
\end{itemize}

\item \textbf{Transaction Count \& Volume}
\begin{itemize}
\item Higher usage = higher value (potentially)
\end{itemize}

\item \textbf{Hash Rate (PoW chains)}
\begin{itemize}
\item Security investment by miners
\end{itemize}

\item \textbf{Total Value Locked (DeFi)}
\begin{itemize}
\item Capital deployed in protocol
\end{itemize}

\item \textbf{Developer Activity}
\begin{itemize}
\item GitHub commits, active developers
\end{itemize}
\end{enumerate}

\vspace{0.3cm}
\textbf{Tools:} Glassnode, IntoTheBlock, Messari, Dune Analytics
\end{frame}

\begin{frame}{Market Cap vs. Fully Diluted Valuation}
\textbf{Market Cap:}
\[
\text{Market Cap} = \text{Price} \times \text{Circulating Supply}
\]

\textbf{Fully Diluted Valuation (FDV):}
\[
\text{FDV} = \text{Price} \times \text{Total Supply (Max)}
\]

\vspace{0.3cm}
\textbf{Example:}
\begin{itemize}
\item Token price: \$10
\item Circulating supply: 100M tokens
\item Total supply: 1B tokens
\end{itemize}

\begin{itemize}
\item Market Cap: \$10 $\times$ 100M = \$1B
\item FDV: \$10 $\times$ 1B = \$10B
\end{itemize}

\vspace{0.3cm}
\textbf{Warning:} Large FDV/MC ratio indicates future dilution risk.
\end{frame}

\begin{frame}{Token Valuation Checklist}
\textbf{Step-by-Step Analysis:}

\vspace{0.3cm}
\begin{enumerate}
\item Determine token type and purpose (utility, governance, etc.)
\item Check regulatory status (security vs. non-security)
\item Calculate NVT ratio (compare to peers)
\item Analyze on-chain metrics (addresses, transactions, TVL)
\item Review tokenomics (supply, inflation, unlock schedule)
\item Compare market cap to fundamentals (revenue, usage)
\item Assess FDV vs. Market Cap (dilution risk)
\item Consider qualitative factors (team, tech, community)
\end{enumerate}

\vspace{0.3cm}
\textbf{Output:} Informed estimate of fair value range (not precise number).
\end{frame}

\begin{frame}{Limitations of Crypto Valuation}
\textbf{Challenges:}
\begin{itemize}
\item Extreme volatility makes models unstable
\item Lack of historical data for many tokens
\item Regulatory uncertainty changes risk profile
\item Network effects hard to quantify
\item Technology risks (bugs, hacks, obsolescence)
\item Market manipulation (wash trading, pump-and-dump)
\end{itemize}

\vspace{0.3cm}
\textbf{Best Practice:}
\begin{itemize}
\item Use multiple valuation methods
\item Establish ranges, not point estimates
\item Focus on relative value (undervalued vs. overvalued compared to peers)
\item Combine quantitative metrics with qualitative judgment
\item Update models frequently as data evolves
\end{itemize}
\end{frame}

\begin{frame}{Summary}
\textbf{Key Takeaways:}
\begin{itemize}
\item Howey Test determines security classification (4 criteria)
\item Context of sale matters (Ripple case precedent)
\item NVT ratio helps assess value relative to transaction volume
\item Metcalfe's Law suggests network value grows with users squared
\item DCF applicable only to tokens with cash flows
\item Comparable analysis useful for relative valuation
\item Always compare Market Cap to FDV (dilution risk)
\item Crypto valuation is imprecise - use multiple methods
\end{itemize}

\vspace{0.3cm}
\textbf{Next Lecture:} Lab - Tokenomics Analysis (apply these frameworks to real projects).
\end{frame}

\begin{frame}{Questions for Reflection}
\begin{enumerate}
\item Apply the Howey Test to a token you're familiar with. Is it a security?
\item Why did the Ripple ruling distinguish between institutional and exchange sales?
\item Calculate the NVT ratio for Bitcoin and Ethereum. What does it tell you?
\item What are the limitations of using Metcalfe's Law for token valuation?
\item How would you value a governance token with no direct cash flows?
\end{enumerate}
\end{frame}

\end{document}
