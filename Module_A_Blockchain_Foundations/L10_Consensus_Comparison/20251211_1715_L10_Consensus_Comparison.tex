\documentclass[8pt,aspectratio=169]{beamer}
\usetheme{Madrid}
\usepackage[utf8]{inputenc}
\usepackage{graphicx}
\usepackage{booktabs}
\usepackage{hyperref}
\usepackage{amsmath}

\newcommand{\bottomnote}[1]{\vfill\footnotesize\textit{#1}}

\title{Consensus Mechanism Comparison}
\subtitle{BSc Blockchain, Crypto Economy \& NFTs}
\author{Course Instructor}
\date{Module A: Blockchain Foundations}

\begin{document}

\begin{frame}
\titlepage
\end{frame}

\begin{frame}{Learning Objectives}
By the end of this lesson, you will be able to:
\begin{itemize}
    \item Compare proof-of-work, proof-of-stake, delegated proof-of-stake, and PBFT
    \item Evaluate security models and threat assumptions
    \item Analyze scalability and throughput trade-offs
    \item Assess energy consumption and environmental impact
    \item Measure decentralization across consensus protocols
    \item Understand finality and confirmation time differences
    \item Select appropriate consensus mechanism for specific use cases
\end{itemize}
\end{frame}

\begin{frame}{Consensus Mechanisms Overview}
\textbf{What is Consensus?}
\begin{itemize}
    \item Agreement among distributed nodes on shared state
    \item Ensures all participants have same transaction history
    \item Prevents double-spending and conflicting updates
\end{itemize}

\vspace{0.3cm}
\textbf{Major Consensus Families:}
\begin{enumerate}
    \item \textbf{Proof-of-Work (PoW):} Bitcoin, Litecoin, Dogecoin
    \item \textbf{Proof-of-Stake (PoS):} Ethereum, Cardano, Polkadot
    \item \textbf{Delegated Proof-of-Stake (DPoS):} EOS, Tron, Cosmos
    \item \textbf{Practical BFT (PBFT):} Hyperledger Fabric, Zilliqa
    \item \textbf{Hybrid Models:} Decred (PoW + PoS), Algorand (Pure PoS + VRF)
\end{enumerate}
\end{frame}

\begin{frame}[t]{Consensus Mechanism Comparison}
\begin{center}
\includegraphics[width=0.55\textwidth]{charts/01_consensus_radar/chart.pdf}
\end{center}
\bottomnote{No single mechanism excels in all dimensions -- trade-offs are fundamental.}
\end{frame}

\begin{frame}{Proof of Work: Deep Dive}
\textbf{Mechanism:}
\begin{itemize}
    \item Miners compete to find valid block hash (difficulty target)
    \item First to find valid hash broadcasts block
\end{itemize}

\vspace{0.2cm}
\textbf{Security Model:}
\begin{itemize}
    \item Honest majority: $>$ 50\% hash rate honest
    \item Attack cost proportional to hash rate
    \item Probabilistic finality (deeper blocks = safer)
\end{itemize}

\vspace{0.2cm}
\textbf{Advantages:}
\begin{itemize}
    \item Proven security (Bitcoin: 15+ years, no successful attack)
    \item No trusted setup, permissionless, external security
\end{itemize}

\vspace{0.2cm}
\textbf{Disadvantages:}
\begin{itemize}
    \item High energy (150 TWh/year for Bitcoin)
    \item Low throughput (7 TPS), slow finality (1 hour)
\end{itemize}
\end{frame}

\begin{frame}{Proof of Stake: Deep Dive}
\textbf{Mechanism:}
\begin{itemize}
    \item Validators stake tokens as collateral
    \item Pseudo-random selection for block proposal (weighted by stake)
    \item Penalties for misbehavior (slashing)
\end{itemize}

\vspace{0.2cm}
\textbf{Security Model:}
\begin{itemize}
    \item Honest majority: $>$ 67\% stake honest (for finality)
    \item Attack cost = token price $\times$ stake amount
    \item Economic finality (slashing guarantees)
\end{itemize}

\vspace{0.2cm}
\textbf{Advantages:}
\begin{itemize}
    \item Energy-efficient (99\% reduction vs. PoW)
    \item Faster finality (12 min for Ethereum)
    \item Economic alignment (attackers lose stake)
\end{itemize}

\vspace{0.2cm}
\textbf{Disadvantages:}
\begin{itemize}
    \item Wealth concentration, high capital requirement
    \item Centralization risk (exchanges, staking pools)
\end{itemize}
\end{frame}

\begin{frame}{Delegated Proof of Stake (DPoS)}
\textbf{Mechanism:}
\begin{itemize}
    \item Token holders vote for delegates (block producers)
    \item Top N delegates (21 in EOS, 27 in Tron) produce blocks in rotation
    \item Delegates share rewards with voters
\end{itemize}

\vspace{0.2cm}
\textbf{Security Model:}
\begin{itemize}
    \item Honest majority: $>$ 50\% of delegates honest
    \item Reputation-based trust (delegates have identities)
\end{itemize}

\vspace{0.2cm}
\textbf{Advantages:}
\begin{itemize}
    \item High throughput (4,000 TPS for EOS)
    \item Fast finality (1-3 seconds), energy-efficient
\end{itemize}

\vspace{0.2cm}
\textbf{Disadvantages:}
\begin{itemize}
    \item High centralization (only 21-100 block producers)
    \item Voter apathy, plutocracy, cartel risk
\end{itemize}
\end{frame}

\begin{frame}{Practical Byzantine Fault Tolerance (PBFT)}
\textbf{Mechanism:}
\begin{itemize}
    \item Pre-selected committee of validators
    \item Three-phase consensus: pre-prepare, prepare, commit
    \item 2/3+ agreement required to finalize block
\end{itemize}

\vspace{0.2cm}
\textbf{Security Model:}
\begin{itemize}
    \item BFT: tolerates $<$ 1/3 malicious nodes
    \item Known validator set (permissioned)
\end{itemize}

\vspace{0.2cm}
\textbf{Advantages:}
\begin{itemize}
    \item Instant finality (no probabilistic confirmation)
    \item High throughput (1,000-10,000 TPS), energy-efficient
\end{itemize}

\vspace{0.2cm}
\textbf{Disadvantages:}
\begin{itemize}
    \item Requires permissioned network
    \item Poor scalability ($O(N^2)$ communication)
    \item Centralized, not censorship-resistant
\end{itemize}
\end{frame}

\begin{frame}[t]{Throughput Comparison}
\begin{center}
\includegraphics[width=0.55\textwidth]{charts/02_throughput_comparison/chart.pdf}
\end{center}
\bottomnote{Higher throughput typically requires sacrificing decentralization.}
\end{frame}

\begin{frame}{Consensus Comparison Table}
\small
\begin{tabular}{lllll}
\toprule
\textbf{Property} & \textbf{PoW} & \textbf{PoS} & \textbf{DPoS} & \textbf{PBFT} \\
\midrule
\textbf{Throughput} & 7-15 TPS & 30-100 TPS & 1,000-4,000 & 1,000-10,000 \\
\textbf{Finality} & Probabilistic & 10-15 min & 1-3 sec & Instant \\
\textbf{Energy} & Very High & Very Low & Very Low & Very Low \\
\textbf{Decentralization} & High & Medium & Low & Very Low \\
\textbf{Permissionless} & Yes & Yes & Yes & No \\
\textbf{Attack Cost} & Hash rate & Stake value & Vote buying & Compromise 1/3 \\
\bottomrule
\end{tabular}

\vspace{0.4cm}
\textbf{Key Insight:}
\begin{itemize}
    \item No consensus mechanism is universally superior
    \item Trade-offs exist between decentralization, scalability, and finality
    \item Choice depends on use case requirements
\end{itemize}
\end{frame}

\begin{frame}[t]{Finality Comparison}
\begin{center}
\includegraphics[width=0.60\textwidth]{charts/03_finality_timeline/chart.pdf}
\end{center}
\bottomnote{Finality type affects settlement guarantees and application design.}
\end{frame}

\begin{frame}{Security Analysis: Attack Vectors}
\textbf{Proof of Work:}
\begin{itemize}
    \item \textbf{51\% Attack:} control $>$ 50\% hash rate
    \item Cost: hardware + electricity (billions for Bitcoin)
\end{itemize}

\vspace{0.3cm}
\textbf{Proof of Stake:}
\begin{itemize}
    \item \textbf{33\% Attack (liveness):} prevent finality with 33\% stake
    \item \textbf{67\% Attack (safety):} finalize conflicting blocks
    \item Mitigation: slashing destroys attacker's stake
\end{itemize}

\vspace{0.3cm}
\textbf{Delegated Proof of Stake:}
\begin{itemize}
    \item \textbf{Delegate Cartel:} majority of delegates collude
    \item \textbf{Vote Buying:} bribe token holders for votes
\end{itemize}

\vspace{0.3cm}
\textbf{PBFT:}
\begin{itemize}
    \item \textbf{Byzantine Generals:} $>$ 1/3 validators malicious
    \item Cost depends on permission model (regulatory/legal)
\end{itemize}
\end{frame}

\begin{frame}[t]{Attack Cost Comparison}
\begin{center}
\includegraphics[width=0.55\textwidth]{charts/06_attack_cost_comparison/chart.pdf}
\end{center}
\bottomnote{Economic security varies dramatically across consensus mechanisms.}
\end{frame}

\begin{frame}[t]{Energy Consumption Analysis}
\begin{center}
\includegraphics[width=0.55\textwidth]{charts/04_energy_comparison/chart.pdf}
\end{center}
\bottomnote{The Merge reduced Ethereum's energy by 99.95\% -- from Argentina's usage to negligible.}
\end{frame}

\begin{frame}{Energy and Environmental Context}
\textbf{Bitcoin Energy Usage:}
\begin{itemize}
    \item 150 TWh/year -- comparable to Argentina
    \item 70 Mt CO2/year carbon footprint
\end{itemize}

\vspace{0.3cm}
\textbf{PoS Reduction:}
\begin{itemize}
    \item Ethereum post-Merge: 0.01 TWh/year (99.95\% reduction)
    \item All PoS chains combined: $<$ 0.1 TWh/year
\end{itemize}

\vspace{0.3cm}
\textbf{Context:}
\begin{itemize}
    \item Traditional banking: $\sim$260 TWh/year (estimated)
    \item Data centers globally: $\sim$200-300 TWh/year
\end{itemize}

\vspace{0.3cm}
\textbf{Environmental Debate:}
\begin{itemize}
    \item PoW advocates: energy secures network, incentivizes renewables
    \item Critics: wasteful expenditure for limited throughput
    \item Industry trend: shift toward PoS driven by environmental concerns
\end{itemize}
\end{frame}

\begin{frame}[t]{Decentralization Spectrum}
\begin{center}
\includegraphics[width=0.60\textwidth]{charts/05_decentralization_spectrum/chart.pdf}
\end{center}
\bottomnote{Nakamoto coefficient measures minimum entities to control 51\%/33\% of the network.}
\end{frame}

\begin{frame}{Decentralization Metrics}
\textbf{How to Measure Decentralization?}
\begin{enumerate}
    \item \textbf{Nakamoto Coefficient:}
    \begin{itemize}
        \item Minimum entities needed to control 51\% (PoW) or 33\% (PoS)
        \item Bitcoin pools: $\sim$4 | Ethereum: $>$1000 | EOS: 11
    \end{itemize}

    \item \textbf{Node Distribution:}
    \begin{itemize}
        \item Bitcoin: $\sim$15,000 nodes | Ethereum: $\sim$7,000 nodes
        \item Permissioned chains: 10-100 nodes
    \end{itemize}

    \item \textbf{Client Diversity:}
    \begin{itemize}
        \item Multiple implementations reduce single-point-of-failure
        \item Ethereum: 5+ clients | Monolithic chains: 1 client
    \end{itemize}

    \item \textbf{Wealth Distribution:}
    \begin{itemize}
        \item Gini coefficient for token holdings
        \item PoS risk: concentrated wealth = concentrated power
    \end{itemize}
\end{enumerate}
\end{frame}

\begin{frame}{Use Case Selection Matrix}
\textbf{Proof of Work:}
\begin{itemize}
    \item Maximum decentralization, censorship resistance critical
    \item Examples: digital gold (Bitcoin), privacy coins (Monero)
\end{itemize}

\vspace{0.2cm}
\textbf{Proof of Stake:}
\begin{itemize}
    \item Balance decentralization and scalability, environmental sustainability
    \item Examples: DeFi platforms (Ethereum), general-purpose chains
\end{itemize}

\vspace{0.2cm}
\textbf{Delegated Proof of Stake:}
\begin{itemize}
    \item High throughput, fast finality essential
    \item Examples: gaming, social media dApps (EOS, Steemit)
\end{itemize}

\vspace{0.2cm}
\textbf{PBFT:}
\begin{itemize}
    \item Permissioned acceptable, enterprise/consortium use
    \item Examples: supply chain (Hyperledger), interbank settlement
\end{itemize}
\end{frame}

\begin{frame}{Emerging Consensus Mechanisms}
\textbf{Proof of History (Solana):}
\begin{itemize}
    \item Verifiable delay function creates timestamp proof
    \item Enables parallel processing, 65,000 TPS
    \item Concern: hardware requirements, network outages
\end{itemize}

\vspace{0.3cm}
\textbf{Pure Proof of Stake (Algorand):}
\begin{itemize}
    \item VRF for leader selection, instant finality
    \item Low barrier (any amount stakeable)
\end{itemize}

\vspace{0.3cm}
\textbf{Proof of Authority (PoA):}
\begin{itemize}
    \item Validators approved by reputation/identity
    \item Used in testnets (Goerli, Sepolia)
\end{itemize}

\vspace{0.3cm}
\textbf{Hybrid Models:}
\begin{itemize}
    \item Decred (PoW + PoS), Tendermint (BFT + PoS)
    \item Mitigate weaknesses of individual mechanisms
\end{itemize}
\end{frame}

\begin{frame}{Governance and Upgradability}
\textbf{PoW Governance:}
\begin{itemize}
    \item Off-chain (BIPs, rough consensus)
    \item Hard forks contentious (Bitcoin Cash, SV splits)
\end{itemize}

\vspace{0.3cm}
\textbf{PoS Governance:}
\begin{itemize}
    \item On-chain potential (Tezos, Polkadot)
    \item Stake-weighted voting on protocol upgrades
\end{itemize}

\vspace{0.3cm}
\textbf{DPoS Governance:}
\begin{itemize}
    \item Delegates propose and vote on changes
    \item Rapid upgrades possible, risk of centralized decisions
\end{itemize}

\vspace{0.3cm}
\textbf{PBFT Governance:}
\begin{itemize}
    \item Consortium governance among known entities
    \item Fastest upgrade cycles
\end{itemize}
\end{frame}

\begin{frame}{Key Takeaways}
\begin{itemize}
    \item Consensus mechanisms trade off decentralization, scalability, and finality
    \item PoW: maximum decentralization, high energy, low throughput
    \item PoS: balanced approach, energy-efficient, moderate throughput
    \item DPoS: high throughput, fast finality, lower decentralization
    \item PBFT: instant finality, permissioned, centralized
    \item No single consensus is optimal for all use cases
    \item Selection depends on application requirements: security, speed, openness
\end{itemize}

\vspace{0.4cm}
\textbf{Design Philosophy:}

Choose consensus based on priorities: censorship resistance (PoW), sustainability (PoS), throughput (DPoS), enterprise needs (PBFT). Understand trade-offs explicitly.
\end{frame}

\begin{frame}{Discussion Questions}
\begin{enumerate}
    \item Why does PBFT achieve instant finality while PoW only offers probabilistic finality?
    \item How does energy consumption relate to security in proof-of-work systems?
    \item What are the risks of delegating block production to a small set of validators?
    \item Can a highly scalable blockchain also be highly decentralized?
    \item How might quantum computing impact different consensus mechanisms?
    \item What role does governance play in consensus mechanism selection?
\end{enumerate}
\end{frame}

\begin{frame}{Next Lesson Preview: L11 Blockchain Scalability Trilemma}
\textbf{Topics to be covered:}
\begin{itemize}
    \item The scalability trilemma: security, decentralization, scalability
    \item Layer 1 scalability limits (block size, block time, state growth)
    \item Throughput comparisons (TPS benchmarks)
    \item Vertical vs. horizontal scaling approaches
    \item Emerging solutions: sharding, Layer 2, sidechains
\end{itemize}

\vspace{0.5cm}
\textbf{Preparation:}
\begin{itemize}
    \item Review consensus mechanism trade-offs from this lesson
    \item Explore current blockchain TPS statistics (L2Beat)
    \item Consider why traditional databases achieve millions of TPS
\end{itemize}
\end{frame}

\end{document}
