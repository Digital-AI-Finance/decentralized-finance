\documentclass[8pt,aspectratio=169]{beamer}
\usetheme{Madrid}
\usepackage[utf8]{inputenc}
\usepackage{graphicx}
\usepackage{booktabs}
\usepackage{hyperref}
\usepackage{amsmath}

\title{Consensus Mechanism Comparison}
\subtitle{BSc Blockchain, Crypto Economy \& NFTs}
\author{Course Instructor}
\date{Module A: Blockchain Foundations}

\begin{document}

\begin{frame}
\titlepage
\end{frame}

\begin{frame}{Learning Objectives}
By the end of this lesson, you will be able to:
\begin{itemize}
    \item Compare proof-of-work, proof-of-stake, delegated proof-of-stake, and practical Byzantine fault tolerance
    \item Evaluate security models and threat assumptions of different consensus mechanisms
    \item Analyze scalability and throughput trade-offs
    \item Assess energy consumption and environmental impact
    \item Measure decentralization across consensus protocols
    \item Understand finality and confirmation time differences
    \item Select appropriate consensus mechanism for specific use cases
\end{itemize}
\end{frame}

\begin{frame}{Consensus Mechanisms Overview}
\textbf{What is Consensus?}
\begin{itemize}
    \item Agreement among distributed nodes on shared state
    \item Ensures all participants have same transaction history
    \item Prevents double-spending and conflicting updates
    \item Core challenge: achieving agreement despite failures and malicious actors
\end{itemize}

\vspace{0.3cm}
\textbf{Major Consensus Families:}

\vspace{0.3cm}
\begin{enumerate}
    \item \textbf{Proof-of-Work (PoW):} Bitcoin, Litecoin, Dogecoin
    \item \textbf{Proof-of-Stake (PoS):} Ethereum, Cardano, Polkadot
    \item \textbf{Delegated Proof-of-Stake (DPoS):} EOS, Tron, Cosmos
    \item \textbf{Practical Byzantine Fault Tolerance (PBFT):} Hyperledger Fabric, Zilliqa
    \item \textbf{Hybrid Models:} Decred (PoW + PoS), Algorand (Pure PoS + VRF)
\end{enumerate}

\vspace{0.3cm}
\textbf{Selection Criteria:}
\begin{itemize}
    \item Security requirements
    \item Scalability needs
    \item Decentralization goals
    \item Energy constraints
    \item Finality requirements
\end{itemize}
\end{frame}

\begin{frame}{Proof of Work: Deep Dive}
\textbf{Mechanism:}
\begin{itemize}
    \item Miners compete to find valid block hash
    \item Hash must meet difficulty target (leading zeros)
    \item First to find valid hash broadcasts block
    \item Other miners verify and continue on new block
\end{itemize}

\vspace{0.3cm}
\textbf{Security Model:}
\begin{itemize}
    \item Honest majority assumption: > 50\% hash rate honest
    \item Attack cost proportional to hash rate
    \item Probabilistic finality (deeper blocks = safer)
\end{itemize}

\vspace{0.3cm}
\textbf{Advantages:}
\begin{itemize}
    \item Proven security (Bitcoin: 15+ years, no successful attack)
    \item No trusted setup required
    \item Permissionless participation
    \item External security (energy cost)
\end{itemize}

\vspace{0.3cm}
\textbf{Disadvantages:}
\begin{itemize}
    \item High energy consumption (150 TWh/year for Bitcoin)
    \item Low throughput (7 TPS for Bitcoin)
    \item Slow finality (1 hour for 6 confirmations)
    \item Mining centralization (ASIC manufacturers, cheap electricity regions)
\end{itemize}
\end{frame}

\begin{frame}{Proof of Stake: Deep Dive}
\textbf{Mechanism:}
\begin{itemize}
    \item Validators stake native tokens as collateral
    \item Pseudo-random selection for block proposal (weighted by stake)
    \item Validators attest to blocks
    \item Penalties for misbehavior (slashing)
\end{itemize}

\vspace{0.3cm}
\textbf{Security Model:}
\begin{itemize}
    \item Honest majority assumption: > 67\% stake honest (for finality)
    \item Attack cost proportional to token price $\times$ stake amount
    \item Economic finality (slashing guarantees)
\end{itemize}

\vspace{0.3cm}
\textbf{Advantages:}
\begin{itemize}
    \item Energy-efficient (99\% reduction vs. PoW)
    \item Faster finality (12 minutes for Ethereum)
    \item Economic alignment (attackers lose stake)
    \item Scalable to higher throughput (with sharding)
\end{itemize}

\vspace{0.3cm}
\textbf{Disadvantages:}
\begin{itemize}
    \item Wealth concentration (rich get richer)
    \item Nothing-at-stake problem (mitigated by slashing)
    \item High capital requirement (32 ETH = \$64,000)
    \item Centralization risk (exchanges, staking pools)
\end{itemize}
\end{frame}

\begin{frame}{Delegated Proof of Stake (DPoS)}
\textbf{Mechanism:}
\begin{itemize}
    \item Token holders vote for delegates (block producers)
    \item Top N delegates (e.g., 21 in EOS, 27 in Tron) produce blocks in rotation
    \item Delegates share rewards with voters
    \item Poor-performing delegates voted out
\end{itemize}

\vspace{0.3cm}
\textbf{Security Model:}
\begin{itemize}
    \item Honest majority assumption: > 50\% of delegates honest
    \item Reputation-based trust (delegates have identities)
    \item Attack requires majority of delegates colluding
\end{itemize}

\vspace{0.3cm}
\textbf{Advantages:}
\begin{itemize}
    \item High throughput (4,000 TPS for EOS, 2,000 for Tron)
    \item Fast finality (1-3 seconds)
    \item Energy-efficient
    \item Democratic governance (token holder voting)
\end{itemize}

\vspace{0.3cm}
\textbf{Disadvantages:}
\begin{itemize}
    \item High centralization (only 21-100 block producers)
    \item Voter apathy (low participation rates)
    \item Plutocracy (large holders dominate voting)
    \item Cartel risk (delegates collude)
    \item Susceptible to Sybil attacks via vote buying
\end{itemize}
\end{frame}

\begin{frame}{Practical Byzantine Fault Tolerance (PBFT)}
\textbf{Mechanism:}
\begin{itemize}
    \item Pre-selected committee of validators
    \item Three-phase consensus: pre-prepare, prepare, commit
    \item 2/3+ agreement required to finalize block
    \item Deterministic finality (no forks)
\end{itemize}

\vspace{0.3cm}
\textbf{Security Model:}
\begin{itemize}
    \item Byzantine fault tolerance: tolerates < 1/3 malicious nodes
    \item Known validator set (permissioned)
    \item Assumes synchronous or partially synchronous network
\end{itemize}

\vspace{0.3cm}
\textbf{Advantages:}
\begin{itemize}
    \item Instant finality (no probabilistic confirmation)
    \item High throughput (1,000-10,000 TPS)
    \item Energy-efficient
    \item Well-studied algorithm (since 1999)
\end{itemize}

\vspace{0.3cm}
\textbf{Disadvantages:}
\begin{itemize}
    \item Requires permissioned network (known participants)
    \item Poor scalability with validator count (communication complexity $O(N^2)$)
    \item Centralized (typically 4-100 validators)
    \item Not censorship-resistant (validators can be coerced)
    \item Unsuitable for public blockchains
\end{itemize}
\end{frame}

\begin{frame}{Consensus Comparison Table}
\small
\begin{tabular}{lllll}
\toprule
\textbf{Property} & \textbf{PoW} & \textbf{PoS} & \textbf{DPoS} & \textbf{PBFT} \\
\midrule
\textbf{Throughput} & 7-15 TPS & 30-100 TPS & 1,000-4,000 & 1,000-10,000 \\
\textbf{Finality} & Probabilistic & 10-15 min & 1-3 sec & Instant \\
\textbf{Energy} & Very High & Very Low & Very Low & Very Low \\
\textbf{Decentralization} & High & Medium & Low & Very Low \\
\textbf{Permissionless} & Yes & Yes & Yes & No \\
\textbf{Attack Cost} & Hash rate & Stake value & Vote buying & Compromise 1/3 \\
\textbf{Sybil Resistance} & Hash power & Stake weight & Vote count & Membership \\
\textbf{Examples} & Bitcoin & Ethereum & EOS, Tron & Hyperledger \\
\bottomrule
\end{tabular}

\vspace{0.4cm}
\textbf{Key Insight:}
\begin{itemize}
    \item No consensus mechanism is universally superior
    \item Trade-offs exist between decentralization, scalability, and finality
    \item Choice depends on use case requirements
\end{itemize}
\end{frame}

\begin{frame}{Security Analysis: Attack Vectors}
\textbf{Proof of Work:}
\begin{itemize}
    \item \textbf{51\% Attack:} control > 50\% hash rate
    \item Cost: hardware + electricity (billions for Bitcoin)
    \item Mitigation: high economic cost, hardware becomes worthless post-attack
\end{itemize}

\vspace{0.3cm}
\textbf{Proof of Stake:}
\begin{itemize}
    \item \textbf{33\% Attack (liveness):} prevent finality with 33\% stake
    \item \textbf{67\% Attack (safety):} finalize conflicting blocks
    \item Cost: 33-67\% of staked tokens
    \item Mitigation: slashing destroys attacker's stake
\end{itemize}

\vspace{0.3cm}
\textbf{Delegated Proof of Stake:}
\begin{itemize}
    \item \textbf{Delegate Cartel:} majority of delegates collude
    \item \textbf{Vote Buying:} bribe token holders for votes
    \item Cost: lower than PoW/PoS (only 21 delegates to compromise)
    \item Mitigation: delegate rotation, reputation systems
\end{itemize}

\vspace{0.3cm}
\textbf{PBFT:}
\begin{itemize}
    \item \textbf{Byzantine Generals:} > 1/3 validators malicious
    \item Cost: depends on permission model (often regulatory/legal)
    \item Mitigation: careful validator selection, monitoring
\end{itemize}
\end{frame}

\begin{frame}{Scalability Comparison}
\textbf{Throughput (Transactions Per Second):}

\vspace{0.3cm}
\begin{tabular}{lrr}
\toprule
\textbf{System} & \textbf{TPS} & \textbf{Block Time} \\
\midrule
Bitcoin (PoW) & 7 & 10 min \\
Ethereum (PoS) & 30 & 12 sec \\
Litecoin (PoW) & 56 & 2.5 min \\
Cardano (PoS) & 250 & 20 sec \\
EOS (DPoS) & 4,000 & 0.5 sec \\
Solana (PoH + PoS) & 65,000 & 0.4 sec \\
Hyperledger Fabric (PBFT) & 10,000 & configurable \\
Visa (centralized) & 24,000 & instant \\
\bottomrule
\end{tabular}

\vspace{0.4cm}
\textbf{Observations:}
\begin{itemize}
    \item PoW: lowest throughput (security prioritized)
    \item PoS: moderate throughput (10x improvement over PoW)
    \item DPoS/PBFT: high throughput (100-1000x improvement)
    \item Trade-off: throughput $\uparrow$, decentralization $\downarrow$
\end{itemize}

\vspace{0.3cm}
\textbf{Scalability Solutions:}
\begin{itemize}
    \item Layer 2: Lightning (Bitcoin), Rollups (Ethereum)
    \item Sharding: Ethereum 2.0 roadmap
    \item Sidechains: Polygon, Arbitrum
\end{itemize}
\end{frame}

\begin{frame}{Finality Comparison}
\textbf{Probabilistic Finality (PoW):}
\begin{itemize}
    \item Confidence increases with each confirmation
    \item Never 100\% final (theoretically reversible)
    \item Bitcoin: 6 confirmations (~1 hour) = ``final enough''
    \item Risk decreases exponentially with depth
\end{itemize}

\vspace{0.3cm}
\textbf{Economic Finality (PoS):}
\begin{itemize}
    \item Casper FFG checkpoints
    \item Ethereum: 2 epochs (~12.8 minutes) = finalized
    \item Reversal requires massive slashing (> 33\% stake destroyed)
    \item Practical irreversibility
\end{itemize}

\vspace{0.3cm}
\textbf{Instant Finality (DPoS/PBFT):}
\begin{itemize}
    \item Single-round commitment
    \item No forks, no reorganizations
    \item EOS: 1-3 seconds
    \item PBFT: immediate upon commit phase
    \item Critical for applications requiring immediate settlement
\end{itemize}

\vspace{0.3cm}
\textbf{Use Case Implications:}
\begin{itemize}
    \item High-value transfers: prefer economic/instant finality (PoS, PBFT)
    \item Decentralized applications: balance finality speed vs. decentralization
    \item Microtransactions: instant finality essential (DPoS, Layer 2)
\end{itemize}
\end{frame}

\begin{frame}{Energy Consumption Analysis}
\textbf{Annual Energy Usage (2024 estimates):}

\vspace{0.3cm}
\begin{tabular}{lrr}
\toprule
\textbf{Blockchain} & \textbf{Energy (TWh/year)} & \textbf{CO2 (Mt/year)} \\
\midrule
Bitcoin (PoW) & 150 & 70 \\
Ethereum (pre-Merge PoW) & 94 & 44 \\
Ethereum (post-Merge PoS) & 0.01 & 0.005 \\
Litecoin (PoW) & 0.5 & 0.2 \\
Cardano (PoS) & 0.006 & 0.003 \\
Polkadot (PoS) & 0.007 & 0.003 \\
All PoS chains combined & < 0.1 & < 0.05 \\
\bottomrule
\end{tabular}

\vspace{0.4cm}
\textbf{Context:}
\begin{itemize}
    \item Bitcoin: comparable to Argentina (~150 TWh/year)
    \item Ethereum PoS: 99.95\% reduction vs. PoW
    \item All PoS chains: less than a single data center
    \item Traditional banking: ~260 TWh/year (estimated)
\end{itemize}

\vspace{0.3cm}
\textbf{Environmental Debate:}
\begin{itemize}
    \item PoW advocates: energy secures network, incentivizes renewables
    \item Critics: wasteful energy expenditure for limited throughput
    \item Shift toward PoS driven partly by environmental concerns
\end{itemize}
\end{frame}

\begin{frame}{Decentralization Metrics}
\textbf{How to Measure Decentralization?}

\vspace{0.3cm}
\begin{enumerate}
    \item \textbf{Nakamoto Coefficient:}
    \begin{itemize}
        \item Minimum entities needed to control 51\% (PoW) or 33\% (PoS)
        \item Higher = more decentralized
        \item Bitcoin mining pools: ~4 entities (low)
        \item Ethereum validators: > 1000 entities (high)
        \item EOS: 11 delegates (very low)
    \end{itemize}

    \item \textbf{Node Distribution:}
    \begin{itemize}
        \item Geographic distribution
        \item Bitcoin: ~15,000 reachable nodes (global)
        \item Ethereum: ~7,000 nodes (global)
        \item Permissioned chains: 10-100 nodes (concentrated)
    \end{itemize}

    \item \textbf{Client Diversity:}
    \begin{itemize}
        \item Multiple independent implementations
        \item Reduces single-point-of-failure risk
        \item Ethereum: 5+ clients (Geth, Nethermind, Besu, Erigon, Reth)
        \item Monolithic chains: 1 client (risky)
    \end{itemize}

    \item \textbf{Wealth Distribution:}
    \begin{itemize}
        \item Gini coefficient for token holdings
        \item Lower = more equitable
        \item PoS risk: concentrated wealth = concentrated power
    \end{itemize}
\end{enumerate}
\end{frame}

\begin{frame}{Decentralization Ranking}
\textbf{Relative Decentralization (High to Low):}

\vspace{0.3cm}
\begin{enumerate}
    \item \textbf{Bitcoin (PoW):}
    \begin{itemize}
        \item 15,000+ nodes globally
        \item Anyone can mine (though ASICs dominate)
        \item No pre-mine, fair launch
        \item Concern: mining pool centralization
    \end{itemize}

    \item \textbf{Ethereum (PoS):}
    \begin{itemize}
        \item 7,000+ nodes, 900,000+ validators
        \item Permissionless staking
        \item Concern: Lido (30\% stake), exchange concentration
    \end{itemize}

    \item \textbf{Cardano/Polkadot (PoS):}
    \begin{itemize}
        \item 3,000-5,000 nodes
        \item Thousands of validators
        \item Concern: early investor token concentration
    \end{itemize}

    \item \textbf{EOS/Tron (DPoS):}
    \begin{itemize}
        \item 21-27 block producers
        \item Voter apathy (< 30\% participation)
        \item Significant centralization
    \end{itemize}

    \item \textbf{Hyperledger Fabric (PBFT):}
    \begin{itemize}
        \item Permissioned (10-100 validators)
        \item Known entities only
        \item Highly centralized by design
    \end{itemize}
\end{enumerate}
\end{frame}

\begin{frame}{Use Case Selection Matrix}
\textbf{When to Use Each Consensus:}

\vspace{0.3cm}
\textbf{Proof of Work:}
\begin{itemize}
    \item Maximum decentralization required
    \item Censorship resistance critical (e.g., money, store of value)
    \item Willing to sacrifice throughput and energy
    \item Examples: digital gold (Bitcoin), privacy coins (Monero)
\end{itemize}

\vspace{0.3cm}
\textbf{Proof of Stake:}
\begin{itemize}
    \item Balance decentralization and scalability
    \item Smart contract platforms
    \item Environmental sustainability important
    \item Examples: DeFi platforms (Ethereum), general-purpose chains (Cardano)
\end{itemize}

\vspace{0.3cm}
\textbf{Delegated Proof of Stake:}
\begin{itemize}
    \item High throughput essential
    \item Fast finality needed (gaming, social media)
    \item Willing to accept centralization trade-off
    \item Examples: dApps platforms (EOS), content platforms (Steemit)
\end{itemize}

\vspace{0.3cm}
\textbf{PBFT:}
\begin{itemize}
    \item Permissioned network acceptable
    \item Enterprise/consortium use case
    \item Regulatory compliance required
    \item Examples: supply chain (Hyperledger), interbank settlement
\end{itemize}
\end{frame}

\begin{frame}{Emerging Consensus Mechanisms}
\textbf{Proof of History (Solana):}
\begin{itemize}
    \item Verifiable delay function creates timestamp proof
    \item Enables parallel transaction processing
    \item Achieves 65,000 TPS
    \item Concern: hardware requirements, network outages
\end{itemize}

\vspace{0.3cm}
\textbf{Pure Proof of Stake (Algorand):}
\begin{itemize}
    \item VRF (Verifiable Random Function) for leader selection
    \item Instant finality, high throughput
    \item Low barrier to entry (any amount stakeable)
    \item Concern: early token distribution concentration
\end{itemize}

\vspace{0.3cm}
\textbf{Proof of Authority (PoA):}
\begin{itemize}
    \item Validators approved based on reputation/identity
    \item Used in testnets (Goerli, Sepolia)
    \item Fast, efficient, but fully centralized
    \item Enterprise/private chain use case
\end{itemize}

\vspace{0.3cm}
\textbf{Proof of Burn:}
\begin{itemize}
    \item Destroy tokens to earn mining rights
    \item Rarely used (Counterparty, Slimcoin)
    \item Theoretical alternative to PoW energy waste
\end{itemize}
\end{frame}

\begin{frame}{Hybrid Consensus Models}
\textbf{Decred (PoW + PoS):}
\begin{itemize}
    \item PoW miners propose blocks
    \item PoS voters approve/reject blocks
    \item Combines security of both mechanisms
    \item Governance via PoS voting
\end{itemize}

\vspace{0.3cm}
\textbf{Ethereum (Casper FFG + LMD GHOST):}
\begin{itemize}
    \item LMD GHOST: fork choice (short-term)
    \item Casper FFG: finality gadget (long-term)
    \item Hybrid approach balances speed and security
\end{itemize}

\vspace{0.3cm}
\textbf{Tendermint (BFT + PoS):}
\begin{itemize}
    \item BFT consensus protocol
    \item Validator set selected via PoS
    \item Used in Cosmos ecosystem
    \item Instant finality + economic security
\end{itemize}

\vspace{0.3cm}
\textbf{Advantages of Hybrids:}
\begin{itemize}
    \item Mitigate weaknesses of individual mechanisms
    \item Flexible governance and security models
    \item Innovation in consensus design
\end{itemize}
\end{frame}

\begin{frame}{Governance and Upgradability}
\textbf{Consensus and Governance Relationship:}

\vspace{0.3cm}
\textbf{Proof of Work:}
\begin{itemize}
    \item Off-chain governance (rough consensus, BIPs)
    \item Hard forks contentious (Bitcoin Cash, Bitcoin SV splits)
    \item Miners signal readiness but do not decide
    \item Users ultimately choose (run upgraded nodes)
\end{itemize}

\vspace{0.3cm}
\textbf{Proof of Stake:}
\begin{itemize}
    \item On-chain governance potential (Tezos, Polkadot)
    \item Validators vote on protocol upgrades
    \item Ethereum: still off-chain governance (EIPs)
    \item Stake-weighted voting
\end{itemize}

\vspace{0.3cm}
\textbf{Delegated Proof of Stake:}
\begin{itemize}
    \item Delegates propose and vote on changes
    \item Rapid upgrades possible
    \item Risk: centralized decision-making
\end{itemize}

\vspace{0.3cm}
\textbf{PBFT:}
\begin{itemize}
    \item Consortium governance
    \item Upgrades coordinated among known entities
    \item Fastest upgrade cycles
\end{itemize}
\end{frame}

\begin{frame}{Key Takeaways}
\begin{itemize}
    \item Consensus mechanisms trade off decentralization, scalability, and finality
    \item PoW: maximum decentralization, high energy, low throughput
    \item PoS: balanced approach, energy-efficient, moderate throughput
    \item DPoS: high throughput, fast finality, lower decentralization
    \item PBFT: instant finality, permissioned, centralized
    \item No single consensus is optimal for all use cases
    \item Emerging mechanisms explore novel trade-offs (Proof of History, VRF-based selection)
    \item Selection depends on application requirements: security, speed, openness
\end{itemize}

\vspace{0.4cm}
\textbf{Design Philosophy:}

Choose consensus based on priorities: censorship resistance (PoW), sustainability (PoS), throughput (DPoS), enterprise needs (PBFT). Understand trade-offs explicitly.
\end{frame}

\begin{frame}{Discussion Questions}
\begin{enumerate}
    \item Why does PBFT achieve instant finality while PoW only offers probabilistic finality?

    \item How does energy consumption relate to security in proof-of-work systems?

    \item What are the risks of delegating block production to a small set of validators?

    \item Can a highly scalable blockchain also be highly decentralized? Why or why not?

    \item How might quantum computing impact different consensus mechanisms?

    \item What role does governance play in consensus mechanism selection?

    \item Is there a fundamental limit to blockchain scalability within a single consensus mechanism?
\end{enumerate}
\end{frame}

\begin{frame}{Next Lesson Preview: L11 Blockchain Scalability Trilemma}
\textbf{Topics to be covered:}
\begin{itemize}
    \item The scalability trilemma: security, decentralization, scalability
    \item Layer 1 scalability limits (block size, block time, state growth)
    \item Throughput comparisons (TPS benchmarks)
    \item Vertical vs. horizontal scaling approaches
    \item Trade-offs in blockchain design
    \item Emerging solutions: sharding, Layer 2, sidechains
    \item Real-world performance analysis
\end{itemize}

\vspace{0.5cm}
\textbf{Preparation:}
\begin{itemize}
    \item Review consensus mechanism trade-offs from this lesson
    \item Explore current blockchain TPS statistics (e.g., Blockchain.com, L2Beat)
    \item Consider why traditional databases achieve millions of TPS
\end{itemize}
\end{frame}

\end{document}
