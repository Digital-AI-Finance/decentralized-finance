\documentclass[8pt,aspectratio=169]{beamer}
\usetheme{Madrid}
\usepackage[utf8]{inputenc}
\usepackage{graphicx}
\usepackage{booktabs}
\usepackage{hyperref}
\usepackage{amsmath}

\title{Bitcoin Protocol Deep Dive}
\subtitle{BSc Blockchain, Crypto Economy \& NFTs}
\author{Course Instructor}
\date{Module A: Blockchain Foundations}

\begin{document}

\begin{frame}
\titlepage
\end{frame}

\begin{frame}{Learning Objectives}
By the end of this lesson, you will be able to:
\begin{itemize}
    \item Explain the UTXO (Unspent Transaction Output) model
    \item Describe the structure of a Bitcoin transaction
    \item Understand transaction inputs and outputs
    \item Recognize different Bitcoin Script types
    \item Trace the lifecycle of a transaction from creation to confirmation
    \item Distinguish between legacy and SegWit transaction formats
\end{itemize}
\end{frame}

\begin{frame}{The UTXO Model: Core Concept}
\textbf{What is a UTXO?}
\begin{itemize}
    \item Unspent Transaction Output = a chunk of bitcoin that can be spent
    \item Bitcoin does not track account balances (unlike Ethereum)
    \item Instead, tracks individual ``coins'' (UTXOs)
\end{itemize}

\begin{center}
\includegraphics[width=0.60\textwidth]{charts/01_utxo_model/chart.pdf}
\end{center}
\end{frame}

\begin{frame}{UTXO Properties and Cash Analogy}
\textbf{Analogy: Physical Cash}
\begin{itemize}
    \item UTXOs are like bills in your wallet
    \item You do not have ``100 EUR balance'' -- you have five 20 EUR bills
    \item To pay 30 EUR: give one 20 EUR bill + one 10 EUR bill
    \item To pay 25 EUR with a 50 EUR bill: receive 25 EUR change
\end{itemize}

\vspace{0.3cm}
\textbf{Key Principles:}
\begin{itemize}
    \item Each UTXO can only be spent once (consumed entirely)
    \item Spending a UTXO creates new UTXOs
    \item Blockchain tracks which UTXOs are unspent
    \item Your wallet balance = sum of all UTXOs you can spend
\end{itemize}
\end{frame}

\begin{frame}{UTXO Model vs. Account Model}
\begin{center}
\includegraphics[width=0.60\textwidth]{charts/02_utxo_vs_account/chart.pdf}
\end{center}

\textbf{Why Bitcoin Uses UTXO:} Easier to verify (check UTXO existence), no global state needed, natural double-spend prevention.
\end{frame}

\begin{frame}{Bitcoin Transaction Structure}
\begin{center}
\includegraphics[width=0.58\textwidth]{charts/03_transaction_structure/chart.pdf}
\end{center}

\vspace{0.2cm}
\textbf{Transaction Hash (txid):}
\begin{itemize}
    \item Double SHA-256 of entire transaction -- unique identifier
    \item Used to reference transaction in inputs
\end{itemize}
\end{frame}

\begin{frame}{Transaction Inputs}
\textbf{Each Input Contains:}
\begin{enumerate}
    \item \textbf{Previous Transaction Hash:} txid of transaction containing UTXO
    \item \textbf{Output Index:} which output from previous transaction (0, 1, 2, ...)
    \item \textbf{ScriptSig (Unlocking Script):} provides signature and public key
    \item \textbf{Sequence Number:} originally for transaction replacement
\end{enumerate}

\vspace{0.3cm}
\textbf{Example Input:}
\begin{itemize}
    \item Previous tx: \texttt{5a3c7b...} (Alice received 3 BTC)
    \item Output index: 0 (first output of that transaction)
    \item ScriptSig: signature proving Alice owns the UTXO
\end{itemize}

\vspace{0.3cm}
\textbf{Multiple Inputs:}
\begin{itemize}
    \item Transaction can have many inputs (combining UTXOs)
    \item Each input must be signed separately
\end{itemize}
\end{frame}

\begin{frame}{Transaction Outputs}
\textbf{Each Output Contains:}
\begin{enumerate}
    \item \textbf{Value:} Amount of satoshis (1 BTC = 100,000,000 satoshis)
    \item \textbf{ScriptPubKey (Locking Script):} conditions to spend this output
\end{enumerate}

\vspace{0.3cm}
\textbf{Change Outputs:}
\begin{itemize}
    \item When input value > payment amount, create change output
    \item Change goes back to sender (usually new address for privacy)
    \item Example: Spend 5 BTC UTXO to send 3 BTC $\rightarrow$ 3 BTC to recipient + 1.999 BTC change
\end{itemize}

\vspace{0.3cm}
\textbf{Transaction Fee:}
\begin{itemize}
    \item Fee = Sum of inputs $-$ Sum of outputs
    \item Not explicitly stated in transaction (implicit)
    \item Miner collects the difference
\end{itemize}
\end{frame}

\begin{frame}{Bitcoin Script: A Stack-Based Language}
\textbf{What is Bitcoin Script?}
\begin{itemize}
    \item Simple, stack-based programming language
    \item Not Turing-complete (no loops, limited expressiveness)
    \item Executed during transaction validation
    \item Determines whether transaction is valid
\end{itemize}

\vspace{0.3cm}
\textbf{How It Works:}
\begin{enumerate}
    \item Combine ScriptSig (from input) + ScriptPubKey (from previous output)
    \item Execute script operations left to right
    \item Use a stack (LIFO data structure)
    \item Transaction valid if final stack value is TRUE
\end{enumerate}

\vspace{0.3cm}
\textbf{Basic Operations:}
\begin{itemize}
    \item \texttt{OP\_DUP}: duplicate top stack item
    \item \texttt{OP\_HASH160}: hash with SHA-256 then RIPEMD-160
    \item \texttt{OP\_EQUALVERIFY}: check equality, fail if not
    \item \texttt{OP\_CHECKSIG}: verify signature against public key
\end{itemize}
\end{frame}

\begin{frame}{P2PKH Script Execution}
\begin{center}
\includegraphics[width=0.65\textwidth]{charts/04_script_execution/chart.pdf}
\end{center}

\textbf{Why Public Key Hash?} Shorter addresses (20 vs 33 bytes), extra security (quantum resistance until spending).
\end{frame}

\begin{frame}[fragile]{P2SH: Pay-to-Script-Hash}
\textbf{Purpose:}
\begin{itemize}
    \item Allows complex spending conditions (multi-signature, time-locks)
    \item Hides complexity until spending time
    \item Sender only needs recipient's P2SH address
\end{itemize}

\vspace{0.3cm}
\textbf{ScriptPubKey:}
\begin{verbatim}
OP_HASH160 <ScriptHash> OP_EQUAL
\end{verbatim}

\textbf{ScriptSig:}
\begin{verbatim}
<Signature1> <Signature2> ... <RedeemScript>
\end{verbatim}

\vspace{0.3cm}
\textbf{Verification Process:}
\begin{enumerate}
    \item Hash the redeem script
    \item Verify hash matches ScriptHash in ScriptPubKey
    \item Execute redeem script with provided signatures
    \item Transaction valid if redeem script evaluates to TRUE
\end{enumerate}
\end{frame}

\begin{frame}{Transaction Lifecycle}
\begin{center}
\includegraphics[width=0.70\textwidth]{charts/05_transaction_lifecycle/chart.pdf}
\end{center}

\vspace{0.2cm}
\textbf{Key Stages:}
\begin{itemize}
    \item Creation $\rightarrow$ Signing $\rightarrow$ Broadcast $\rightarrow$ Mempool $\rightarrow$ Mining $\rightarrow$ Confirmation
\end{itemize}
\end{frame}

\begin{frame}{Transaction Lifecycle: Details}
\textbf{1. Creation and Signing:}
\begin{itemize}
    \item Wallet selects UTXOs, constructs inputs/outputs, calculates fee
    \item Signs each input with corresponding private key
\end{itemize}

\vspace{0.3cm}
\textbf{2. Broadcast and Mempool:}
\begin{itemize}
    \item Transaction sent to connected nodes, validated, added to mempool
    \item Nodes relay to peers (propagation across network)
    \item Transactions sorted by fee rate in mempool
\end{itemize}

\vspace{0.3cm}
\textbf{3. Mining and Confirmation:}
\begin{itemize}
    \item Miner includes transaction in candidate block
    \item Block mined, broadcast, validated by network
    \item Each new block adds one confirmation
    \item 6 confirmations typically considered final ($\sim$1 hour)
\end{itemize}
\end{frame}

\begin{frame}{SegWit: Segregated Witness}
\begin{center}
\includegraphics[width=0.65\textwidth]{charts/06_segwit_comparison/chart.pdf}
\end{center}

\vspace{0.2cm}
\textbf{Problem with Legacy:} Signature in TX hash $\rightarrow$ malleability\\
\textbf{SegWit Solution (BIP 141, 2017):} Separate witness data from TX ID
\end{frame}

\begin{frame}{SegWit Benefits}
\textbf{Block Capacity Increase:}
\begin{itemize}
    \item Legacy: 1 MB block size limit
    \item SegWit: measured in ``weight units'' (max 4 million)
    \item Witness data: 1 byte = 1 weight unit; Non-witness: 1 byte = 4 weight units
    \item Effective capacity: $\sim$2-2.7 MB per block
\end{itemize}

\vspace{0.3cm}
\textbf{Lower Transaction Fees:}
\begin{itemize}
    \item Witness data discounted by 75\%
    \item Same transaction costs less with SegWit
\end{itemize}

\vspace{0.3cm}
\textbf{Address Formats:}
\begin{itemize}
    \item P2WPKH (native SegWit): starts with ``bc1q'' (Bech32 encoding)
    \item P2SH-wrapped SegWit: starts with ``3'' (backward compatible)
\end{itemize}

\vspace{0.3cm}
\textbf{Enables Lightning Network:} Fixes malleability for secure payment channels
\end{frame}

\begin{frame}{Taproot: SegWit v1 (Activated 2021)}
\textbf{Key Improvements:}
\begin{itemize}
    \item \textbf{Schnorr Signatures:} More efficient, enable signature aggregation
    \item \textbf{MAST:} Complex scripts hidden until execution, only reveal used branch
    \item \textbf{Privacy:} All transactions look similar on-chain
\end{itemize}

\vspace{0.3cm}
\textbf{Benefits:}
\begin{itemize}
    \item Multi-sig indistinguishable from single-sig
    \item Complex smart contracts look like simple payments
    \item Smaller transaction size for complex scripts
\end{itemize}

\vspace{0.3cm}
\textbf{Address Format:}
\begin{itemize}
    \item Starts with ``bc1p'' (Bech32m encoding)
    \item Example: \texttt{bc1p5d7rjq7g6rdk2yhzks9smlaqtedr4dekq08ge8...}
\end{itemize}
\end{frame}

\begin{frame}{Transaction Validation Rules}
\textbf{Syntax Validation:}
\begin{itemize}
    \item Transaction size within limits
    \item Output values non-negative, do not exceed input values
    \item No duplicate inputs (double-spend within transaction)
\end{itemize}

\vspace{0.3cm}
\textbf{Semantic Validation:}
\begin{itemize}
    \item All referenced UTXOs exist and are unspent
    \item Signatures valid for all inputs
    \item Script execution succeeds for all inputs
    \item Locktime constraints satisfied
\end{itemize}

\vspace{0.3cm}
\textbf{Rejection Reasons:}
\begin{itemize}
    \item Invalid signature $\rightarrow$ likely fraud attempt
    \item Double-spend $\rightarrow$ UTXO already spent
    \item Dust output $\rightarrow$ output value too small (spam prevention)
\end{itemize}
\end{frame}

\begin{frame}{Transaction Fees: Economics}
\begin{center}
\includegraphics[width=0.55\textwidth]{charts/07_fee_market/chart.pdf}
\end{center}

\vspace{0.2cm}
\textbf{Fee Market Dynamics:}
\begin{itemize}
    \item Block space is scarce ($\sim$4 MB weight per 10 minutes)
    \item Users compete for inclusion via fees
    \item Fee estimation based on mempool state and target confirmation time
\end{itemize}
\end{frame}

\begin{frame}{Replace-by-Fee (RBF) and CPFP}
\textbf{RBF (BIP 125):}
\begin{itemize}
    \item Create replacement transaction with same inputs
    \item Increase fee by at least 1 satoshi per byte
    \item Signal RBF by setting sequence number < 0xfffffffe
    \item Use cases: fee bump, output modification, cancel transaction
\end{itemize}

\vspace{0.3cm}
\textbf{Child-Pays-for-Parent (CPFP):}
\begin{itemize}
    \item Recipient creates high-fee transaction spending unconfirmed output
    \item Miners must include parent to mine child
    \item Combined fee rate makes both transactions attractive
\end{itemize}

\vspace{0.3cm}
\textbf{Comparison:}
\begin{itemize}
    \item RBF: sender bumps fee (requires signaling)
    \item CPFP: receiver bumps fee (works for any transaction)
\end{itemize}
\end{frame}

\begin{frame}{Coinbase Transactions and Block Rewards}
\begin{center}
\includegraphics[width=0.60\textwidth]{charts/08_halving_schedule/chart.pdf}
\end{center}

\vspace{0.2cm}
\textbf{Coinbase Properties:}
\begin{itemize}
    \item First transaction in every block, creates new bitcoins
    \item Miner collects block reward + all transaction fees
    \item Must wait 100 confirmations before spending (maturity rule)
\end{itemize}
\end{frame}

\begin{frame}{2024 Milestone: Bitcoin ETFs and Institutional Adoption}
\textbf{January 10, 2024: Spot Bitcoin ETF Approval}
\begin{itemize}
    \item SEC approved 11 spot Bitcoin ETFs (first time in US)
    \item Major issuers: BlackRock (IBIT), Fidelity (FBTC), Grayscale (GBTC)
    \item Accumulated \$50B+ in assets under management by end of 2024
\end{itemize}

\vspace{0.3cm}
\textbf{Market Impact:}
\begin{itemize}
    \item Institutional legitimization of Bitcoin as asset class
    \item Daily trading volume rivals major commodity ETFs
    \item Custody handled by regulated institutions
\end{itemize}

\vspace{0.3cm}
\textbf{Transaction Implications:}
\begin{itemize}
    \item ETF creation/redemption uses large on-chain transactions
    \item Institutional custody drives UTXO consolidation
    \item Increased demand for block space during high activity
\end{itemize}
\end{frame}

\begin{frame}{Key Takeaways}
\begin{itemize}
    \item Bitcoin uses the UTXO model: transactions consume old outputs and create new ones
    \item Each transaction has inputs (UTXOs being spent) and outputs (new UTXOs)
    \item Bitcoin Script enables flexible spending conditions without Turing completeness
    \item P2PKH (legacy), P2SH (multi-sig), SegWit, and Taproot offer increasing efficiency
    \item Transaction lifecycle: creation $\rightarrow$ signing $\rightarrow$ broadcast $\rightarrow$ mempool $\rightarrow$ mining $\rightarrow$ confirmation
    \item Fees determined by market competition for block space
    \item SegWit and Taproot improve scalability and privacy
\end{itemize}

\vspace{0.3cm}
\textbf{Design Philosophy:}
Bitcoin prioritizes security and decentralization over transaction throughput.
\end{frame}

\begin{frame}{Discussion Questions}
\begin{enumerate}
    \item Why does Bitcoin use the UTXO model instead of the account model?
    \item How does the fee market incentivize miners to include transactions?
    \item What are the trade-offs between legacy addresses, SegWit, and Taproot?
    \item How does transaction malleability affect second-layer solutions?
    \item Why is the coinbase maturity rule (100 confirmations) necessary?
    \item How could you design a transaction that can only be spent after a certain date?
\end{enumerate}
\end{frame}

\begin{frame}{Next Lesson Preview: L07 Proof of Work Consensus}
\textbf{Topics to be covered:}
\begin{itemize}
    \item Mining mechanics and the proof-of-work algorithm
    \item Nonce searching and difficulty adjustment
    \item Block header structure and hash rate
    \item 51\% attacks and mining centralization risks
    \item Energy consumption and environmental concerns
\end{itemize}

\vspace{0.5cm}
\textbf{Preparation:}
\begin{itemize}
    \item Review hash function properties (pre-image resistance)
    \item Explore Bitcoin mining pools and hash rate distribution
    \item Consider the economics of mining profitability
\end{itemize}
\end{frame}

\end{document}
