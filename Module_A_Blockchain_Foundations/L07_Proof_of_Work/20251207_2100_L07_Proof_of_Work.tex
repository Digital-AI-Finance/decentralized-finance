\documentclass[8pt,aspectratio=169]{beamer}
\usetheme{Madrid}
\usepackage[utf8]{inputenc}
\usepackage{graphicx}
\usepackage{booktabs}
\usepackage{hyperref}
\usepackage{amsmath}

\title{Proof of Work Consensus}
\subtitle{BSc Blockchain, Crypto Economy \& NFTs}
\author{Course Instructor}
\date{Module A: Blockchain Foundations}

\begin{document}

\begin{frame}
\titlepage
\end{frame}

\begin{frame}{Learning Objectives}
By the end of this lesson, you will be able to:
\begin{itemize}
    \item Explain the proof-of-work consensus mechanism
    \item Describe the mining process and nonce searching
    \item Understand difficulty adjustment and its purpose
    \item Calculate mining profitability and hash rate economics
    \item Recognize the security guarantees and vulnerabilities of PoW
    \item Evaluate the 51\% attack threat model
    \item Discuss energy consumption and environmental impact
\end{itemize}
\end{frame}

\begin{frame}{The Byzantine Generals Problem}
\textbf{Distributed Consensus Challenge:}

\vspace{0.3cm}
Byzantine generals surround a city and must coordinate attack or retreat:
\begin{itemize}
    \item Generals communicate via messengers
    \item Some generals may be traitors (malicious)
    \item Traitors send conflicting messages
    \item How to reach consensus despite traitors?
\end{itemize}

\vspace{0.3cm}
\textbf{Blockchain Analogy:}
\begin{itemize}
    \item Nodes = generals
    \item Transaction ordering = attack/retreat decision
    \item Malicious nodes = traitors
    \item Network delays and partitions = unreliable messengers
\end{itemize}

\vspace{0.3cm}
\textbf{Proof-of-Work Solution:}
\begin{itemize}
    \item Make message creation costly (computational work)
    \item Honest majority by hash power (not node count)
    \item Longest chain rule resolves conflicts
    \item Economic incentives align honest behavior
\end{itemize}
\end{frame}

\begin{frame}{What is Proof of Work?}
\textbf{Core Concept:}
\begin{itemize}
    \item Find a nonce such that hash of block header meets difficulty target
    \item Target: hash must be below a specific value (equivalently, N leading zeros)
    \item No shortcut: must try nonces randomly until one works
    \item Verification is instant: anyone can check hash validity
\end{itemize}

\vspace{0.3cm}
\textbf{Mathematical Formulation:}
\[
\text{SHA-256}(\text{SHA-256}(\text{BlockHeader})) < \text{Target}
\]

\vspace{0.3cm}
\textbf{Key Properties:}
\begin{enumerate}
    \item \textbf{Asymmetry:} Hard to find, easy to verify
    \item \textbf{Probabilistic:} Expected time to find solution, no guarantee
    \item \textbf{Adjustable difficulty:} Target changes to maintain block time
    \item \textbf{Progress-free:} Past attempts do not help future attempts
\end{enumerate}

\vspace{0.3cm}
\textbf{Analogy:}
\begin{itemize}
    \item Rolling dice until you get 10 sixes in a row
    \item Each roll is independent
    \item Expected number of attempts: $6^{10}$ (very large)
    \item Verification: just look at the result
\end{itemize}
\end{frame}

\begin{frame}{Bitcoin Block Header Structure}
\textbf{Block Header (80 bytes):}

\vspace{0.3cm}
\begin{enumerate}
    \item \textbf{Version} (4 bytes): Protocol version
    \item \textbf{Previous Block Hash} (32 bytes): Hash of previous block
    \item \textbf{Merkle Root} (32 bytes): Root of transaction Merkle tree
    \item \textbf{Timestamp} (4 bytes): Current time (Unix epoch)
    \item \textbf{Difficulty Target} (4 bytes): Compact representation of target
    \item \textbf{Nonce} (4 bytes): Random value to vary hash
\end{enumerate}

\vspace{0.3cm}
\textbf{Mining Process:}
\begin{enumerate}
    \item Construct block with transactions
    \item Compute Merkle root
    \item Set timestamp and difficulty
    \item Try nonce = 0, compute hash
    \item If hash < target: success (broadcast block)
    \item If hash $\geq$ target: increment nonce, repeat
\end{enumerate}

\vspace{0.3cm}
\textbf{Nonce Space Exhaustion:}
\begin{itemize}
    \item 4 bytes = $2^{32}$ = 4.3 billion possible nonces
    \item Modern ASICs exceed this in milliseconds
    \item Solution: modify coinbase transaction (extra nonce), recompute Merkle root
\end{itemize}
\end{frame}

\begin{frame}{Merkle Trees: Efficient Transaction Commitment}
\textbf{Purpose:}
\begin{itemize}
    \item Commit to all transactions in block with single hash (32 bytes)
    \item Enable efficient transaction verification (SPV clients)
    \item Modify single transaction -> Merkle root changes
\end{itemize}

\vspace{0.3cm}
\textbf{Construction:}
\begin{enumerate}
    \item Hash each transaction: $H(tx_1), H(tx_2), \ldots, H(tx_n)$
    \item Pair hashes and hash again: $H(H(tx_1) \| H(tx_2))$
    \item Repeat until single root hash remains
\end{enumerate}

\vspace{0.3cm}
\textbf{Properties:}
\begin{itemize}
    \item Tree height: $\log_2(n)$ for $n$ transactions
    \item Proof size: $\log_2(n)$ hashes to prove transaction inclusion
    \item Example: 1000 transactions -> 10 hashes (~320 bytes proof)
\end{itemize}

\vspace{0.3cm}
\textbf{Extra Nonce Trick:}
\begin{itemize}
    \item Miners modify coinbase transaction (includes extra nonce field)
    \item Recompute Merkle root (different root for each extra nonce)
    \item Expands search space beyond $2^{32}$ nonces
\end{itemize}
\end{frame}

\begin{frame}{Difficulty Target and Adjustment}
\textbf{Difficulty Target:}
\begin{itemize}
    \item 256-bit number representing maximum valid hash
    \item Lower target = harder mining (fewer valid hashes)
    \item Difficulty = how hard current target is relative to maximum
\end{itemize}

\vspace{0.3cm}
\textbf{Target Representation:}
\begin{itemize}
    \item Compact format: 4 bytes (exponent-mantissa encoding)
    \item Example: \texttt{0x1b0404cb} = $0x0404cb \times 2^{8 \times (0x1b - 3)}$
    \item Full 256-bit target reconstructed during validation
\end{itemize}

\vspace{0.3cm}
\textbf{Difficulty Adjustment (Every 2016 Blocks):}
\[
\text{New Target} = \text{Old Target} \times \frac{\text{Actual Time}}{\text{Expected Time}}
\]

\begin{itemize}
    \item Expected time: 2016 blocks $\times$ 10 minutes = 20,160 minutes (2 weeks)
    \item Actual time: measured from timestamps
    \item Clamped to prevent extreme changes: $[T/4, T \times 4]$
\end{itemize}

\vspace{0.3cm}
\textbf{Purpose:}
\begin{itemize}
    \item Maintain ~10 minute average block time
    \item Adapt to changing total hash rate
    \item Self-stabilizing system
\end{itemize}
\end{frame}

\begin{frame}{Hash Rate and Mining Economics}
\textbf{Hash Rate:}
\begin{itemize}
    \item Number of hashes computed per second
    \item Units: H/s (hashes), KH/s, MH/s, GH/s, TH/s, PH/s, EH/s
    \item Bitcoin network (2024): ~500 EH/s (500 quintillion hashes per second)
\end{itemize}

\vspace{0.3cm}
\textbf{Mining Probability:}
\[
P(\text{find block in 10 min}) = \frac{\text{Your Hash Rate}}{\text{Network Hash Rate}}
\]

\vspace{0.3cm}
\textbf{Example:}
\begin{itemize}
    \item Miner hash rate: 100 TH/s
    \item Network hash rate: 500 EH/s = 500,000,000 TH/s
    \item Probability: $\frac{100}{500,000,000} = 0.0000002 = 0.00002\%$
    \item Expected blocks per year: $0.0000002 \times 52,560 \approx 0.01$ blocks
    \item Expected time to find block: 100 years
\end{itemize}

\vspace{0.3cm}
\textbf{Solution: Mining Pools}
\begin{itemize}
    \item Aggregate hash rate from many miners
    \item Share block rewards proportionally
    \item Reduce payout variance
\end{itemize}
\end{frame}

\begin{frame}{Mining Profitability}
\textbf{Revenue:}
\[
\text{Daily Revenue} = \frac{\text{Your Hash Rate}}{\text{Network Hash Rate}} \times 144 \text{ blocks/day} \times (\text{Block Reward} + \text{Avg Fees})
\]

\vspace{0.3cm}
\textbf{Costs:}
\begin{itemize}
    \item \textbf{Hardware:} ASIC miner cost (e.g., Antminer S19 Pro: ~\$2000-5000)
    \item \textbf{Electricity:} power consumption $\times$ electricity rate
    \item \textbf{Cooling:} additional power for air conditioning
    \item \textbf{Maintenance:} repairs, facility costs
\end{itemize}

\vspace{0.3cm}
\textbf{Example Calculation (Antminer S19 Pro):}
\begin{itemize}
    \item Hash rate: 110 TH/s
    \item Power consumption: 3250 W = 78 kWh/day
    \item Electricity cost: \$0.05/kWh $\times$ 78 = \$3.90/day
    \item Revenue (BTC = \$40,000): $\frac{110}{500,000,000} \times 144 \times 6.25 \times 40,000 \approx \$7.92/day$
    \item Profit: \$7.92 - \$3.90 = \$4.02/day
    \item Payback period (hardware cost \$3000): $\frac{3000}{4.02} \approx 746$ days (~2 years)
\end{itemize}

\vspace{0.3cm}
\textbf{Risk Factors:}
\begin{itemize}
    \item BTC price volatility
    \item Difficulty increases (hash rate growth)
    \item Hardware obsolescence
    \item Electricity price changes
\end{itemize}
\end{frame}

\begin{frame}{ASIC Mining Hardware Evolution}
\textbf{CPU Mining (2009-2010):}
\begin{itemize}
    \item Early Bitcoin mining on personal computers
    \item Hash rate: ~1-10 MH/s per CPU
    \item Quickly became unprofitable
\end{itemize}

\vspace{0.3cm}
\textbf{GPU Mining (2010-2013):}
\begin{itemize}
    \item Graphics cards (NVIDIA, AMD)
    \item Hash rate: ~100-1000 MH/s per GPU
    \item Parallel processing advantage
\end{itemize}

\vspace{0.3cm}
\textbf{FPGA Mining (2011-2013):}
\begin{itemize}
    \item Field-Programmable Gate Arrays
    \item Hash rate: ~100-1000 MH/s
    \item More efficient than GPUs
\end{itemize}

\vspace{0.3cm}
\textbf{ASIC Mining (2013-Present):}
\begin{itemize}
    \item Application-Specific Integrated Circuits
    \item Designed solely for SHA-256 hashing
    \item Hash rate: 1-200 TH/s (2024 models)
    \item 1000x more efficient than GPUs
    \item Dominates Bitcoin mining
\end{itemize}

\vspace{0.3cm}
\textbf{Implications:}
\begin{itemize}
    \item Mining centralization (large-scale operations)
    \item High barrier to entry
    \item Geographic concentration in low-electricity regions
\end{itemize}
\end{frame}

\begin{frame}{Mining Pools}
\textbf{Why Pools Exist:}
\begin{itemize}
    \item Solo mining: high variance (might wait years for block)
    \item Pooled mining: steady income (proportional to hash rate)
    \item Risk mitigation for small miners
\end{itemize}

\vspace{0.3cm}
\textbf{Pool Operation:}
\begin{enumerate}
    \item Pool coordinator distributes mining tasks (shares)
    \item Miners submit partial solutions (lower difficulty)
    \item Pool tracks contribution of each miner
    \item When pool finds block, reward distributed proportionally
    \item Pool takes fee (1-3\%)
\end{enumerate}

\vspace{0.3cm}
\textbf{Payout Schemes:}
\begin{itemize}
    \item \textbf{PPS (Pay-Per-Share):} fixed payment per share (lowest variance)
    \item \textbf{PPLNS (Pay-Per-Last-N-Shares):} share revenue from recent blocks
    \item \textbf{FPPS (Full PPS):} PPS + transaction fees
\end{itemize}

\vspace{0.3cm}
\textbf{Centralization Concern:}
\begin{itemize}
    \item Top 5 pools control ~70\% of hash rate
    \item Pools do not own hardware (miners can switch pools)
    \item Risk: pool operator could censor transactions
    \item Mitigation: decentralized pool protocols (P2Pool, Stratum V2)
\end{itemize}
\end{frame}

\begin{frame}{Block Rewards and the Halving Schedule}
\textbf{Block Reward Components:}
\[
\text{Total Reward} = \text{Block Subsidy} + \text{Transaction Fees}
\]

\vspace{0.3cm}
\textbf{Block Subsidy (New Bitcoins):}
\begin{itemize}
    \item Initial reward (2009): 50 BTC per block
    \item Halves every 210,000 blocks (~4 years)
    \item Current (2024): 6.25 BTC
    \item Next halving (2024): 3.125 BTC
    \item Asymptotic limit: 21 million BTC
\end{itemize}

\vspace{0.3cm}
\textbf{Halving Timeline:}

\vspace{0.2cm}
\begin{tabular}{llr}
\toprule
\textbf{Period} & \textbf{Reward} & \textbf{Cumulative Supply} \\
\midrule
2009-2012 & 50 BTC   & 10.5M BTC \\
2012-2016 & 25 BTC   & 15.75M BTC \\
2016-2020 & 12.5 BTC & 18.375M BTC \\
2020-2024 & 6.25 BTC & 19.6875M BTC \\
2024-2028 & 3.125 BTC & 20.34375M BTC \\
\bottomrule
\end{tabular}

\vspace{0.3cm}
\textbf{Implication:}
\begin{itemize}
    \item Transaction fees must eventually sustain mining
    \item Security model shifts over time
\end{itemize}
\end{frame}

\begin{frame}{Transaction Fees as Mining Incentive}
\textbf{Current State (2024):}
\begin{itemize}
    \item Block subsidy: 6.25 BTC (~\$250,000 at \$40,000/BTC)
    \item Transaction fees: 0.1-1 BTC per block (~\$4,000-40,000)
    \item Fees: ~2-15\% of total reward
\end{itemize}

\vspace{0.3cm}
\textbf{Future Scenario (2140):}
\begin{itemize}
    \item Block subsidy: 0 BTC (last bitcoin mined)
    \item Transaction fees: 100\% of mining revenue
    \item Security depends entirely on fee market
\end{itemize}

\vspace{0.3cm}
\textbf{Challenges:}
\begin{itemize}
    \item Will fees be sufficient to secure the network?
    \item Fee volatility: low during quiet periods, high during congestion
    \item Miner revenue stability concerns
\end{itemize}

\vspace{0.3cm}
\textbf{Potential Solutions:}
\begin{itemize}
    \item Layer 2 solutions (Lightning) move small transactions off-chain
    \item Base layer becomes settlement layer (high-value transactions)
    \item Higher fee-per-transaction compensates for lower transaction count
    \item Debate ongoing in Bitcoin community
\end{itemize}
\end{frame}

\begin{frame}{The 51\% Attack}
\textbf{Threat Model:}
\begin{itemize}
    \item Attacker controls > 50\% of network hash rate
    \item Can mine blocks faster than honest miners
    \item Longest chain rule allows attacker to dominate
\end{itemize}

\vspace{0.3cm}
\textbf{What Attacker CAN Do:}
\begin{itemize}
    \item \textbf{Double-spend:} reverse own transactions
    \begin{itemize}
        \item Send transaction to merchant (gets product)
        \item Mine secret chain without transaction
        \item Broadcast longer chain (reverses payment)
    \end{itemize}
    \item \textbf{Censor transactions:} refuse to include specific transactions
    \item \textbf{Block other miners:} prevent competitors from earning rewards
\end{itemize}

\vspace{0.3cm}
\textbf{What Attacker CANNOT Do:}
\begin{itemize}
    \item Steal bitcoins from others (requires private keys)
    \item Create bitcoins out of thin air (violates consensus rules)
    \item Change transaction history beyond attack start (infeasible to rewrite years of blocks)
\end{itemize}
\end{frame}

\begin{frame}{51\% Attack Economics}
\textbf{Cost of Attack:}
\[
\text{Cost} = \text{Hash Rate} \times \text{Duration} \times \text{Electricity Cost} + \text{Hardware Cost}
\]

\vspace{0.3cm}
\textbf{Example (Bitcoin):}
\begin{itemize}
    \item Network hash rate: 500 EH/s
    \item 51\% attack: need 255 EH/s
    \item Hardware: $\frac{255,000,000 \text{ TH/s}}{110 \text{ TH/s}} \approx 2.3$ million Antminer S19 Pro
    \item Hardware cost: 2.3M $\times$ \$3000 = \$6.9 billion
    \item Electricity (1 hour): 2.3M $\times$ 3.25 kW $\times$ \$0.05/kWh = \$373,750
    \item Total (1 week attack): \$6.9B + \$62.6M = ~\$7 billion
\end{itemize}

\vspace{0.3cm}
\textbf{Consequences:}
\begin{itemize}
    \item Attack becomes public knowledge immediately
    \item Bitcoin price crashes (attacker's hardware becomes worthless)
    \item Community may hard fork to new algorithm (bricks attacker's ASICs)
    \item Rational attacker: cost > benefit for major cryptocurrencies
\end{itemize}

\vspace{0.3cm}
\textbf{Vulnerable Chains:}
\begin{itemize}
    \item Small PoW chains (low hash rate)
    \item Shared mining algorithms (rent hash power from NiceHash)
    \item Historical attacks: Bitcoin Gold, Ethereum Classic, Verge
\end{itemize}
\end{frame}

\begin{frame}{Selfish Mining}
\textbf{Attack Strategy:}
\begin{itemize}
    \item Miner finds block but does not broadcast immediately
    \item Continues mining on top of secret block
    \item If honest miner finds block: race to propagate
    \item Attacker reveals secret chain if it is longer
\end{itemize}

\vspace{0.3cm}
\textbf{Potential Profit:}
\begin{itemize}
    \item Attacker can earn > fair share of rewards with < 50\% hash rate
    \item Theoretical threshold: ~33\% hash rate (with optimal strategy)
    \item Wastes honest miners' work (reduces network security)
\end{itemize}

\vspace{0.3cm}
\textbf{Mitigation:}
\begin{itemize}
    \item Random block propagation delays
    \item Penalize late-arriving blocks
    \item Timestamp-based block acceptance rules
    \item Not observed in practice (rational miners prioritize short-term honesty)
\end{itemize}

\vspace{0.3cm}
\textbf{Open Question:}
\begin{itemize}
    \item Selfish mining debate ongoing since 2013
    \item Real-world evidence limited
    \item Game theory suggests instability at certain hash rate thresholds
\end{itemize}
\end{frame}

\begin{frame}{Energy Consumption and Environmental Impact}
\textbf{Bitcoin Energy Usage (2024):}
\begin{itemize}
    \item Estimated annual consumption: ~150 TWh (terawatt-hours)
    \item Comparable to countries: Argentina, Netherlands
    \item Percentage of global electricity: ~0.6\%
\end{itemize}

\vspace{0.3cm}
\textbf{Sources of Energy:}
\begin{itemize}
    \item Renewable energy: ~40-60\% (hydroelectric, solar, wind)
    \item Fossil fuels: ~40-60\% (coal, natural gas)
    \item Nuclear: ~5-10\%
    \item Geographic concentration: areas with cheap electricity (Iceland, China, Kazakhstan, USA)
\end{itemize}

\vspace{0.3cm}
\textbf{Environmental Concerns:}
\begin{itemize}
    \item Carbon emissions from fossil fuel usage
    \item Electronic waste from obsolete mining hardware
    \item Water usage for cooling in some regions
\end{itemize}

\vspace{0.3cm}
\textbf{Counterarguments:}
\begin{itemize}
    \item Incentivizes renewable energy development (monetizes stranded energy)
    \item Facilitates grid balancing (flexible load)
    \item Energy usage proportional to security value
    \item Traditional banking system also consumes significant energy
\end{itemize}
\end{frame}

\begin{frame}{Proof-of-Work Alternatives: ASIC Resistance}
\textbf{Motivation:}
\begin{itemize}
    \item Prevent mining centralization
    \item Enable consumer hardware mining (GPUs, CPUs)
    \item Increase decentralization
\end{itemize}

\vspace{0.3cm}
\textbf{ASIC-Resistant Algorithms:}

\vspace{0.2cm}
\begin{itemize}
    \item \textbf{Scrypt (Litecoin):} memory-hard hashing
    \begin{itemize}
        \item Requires significant RAM
        \item ASIC eventually developed (2014)
    \end{itemize}

    \item \textbf{Ethash (Ethereum, pre-merge):} memory-hard with large DAG
    \begin{itemize}
        \item GPU-friendly, ASIC-resistant initially
        \item ASICs developed but less dominant than Bitcoin
    \end{itemize}

    \item \textbf{RandomX (Monero):} CPU-optimized
    \begin{itemize}
        \item Frequently updated to thwart ASICs
        \item Best performance on general-purpose CPUs
    \end{itemize}
\end{itemize}

\vspace{0.3cm}
\textbf{Trade-offs:}
\begin{itemize}
    \item ASIC resistance -> lower security per watt
    \item Easier for botnets to attack (commodity hardware)
    \item Algorithm changes create hard fork risks
    \item Debate: specialization increases security investment
\end{itemize}
\end{frame}

\begin{frame}{Key Takeaways}
\begin{itemize}
    \item Proof-of-work provides Sybil resistance via computational cost
    \item Mining searches for nonces to produce valid block hashes
    \item Difficulty adjusts every 2016 blocks to maintain 10-minute block time
    \item Mining profitability depends on hash rate, electricity cost, and BTC price
    \item 51\% attacks are economically infeasible for large PoW chains
    \item Block rewards halve every 4 years, shifting incentives toward transaction fees
    \item Energy consumption is a significant concern but incentivizes renewable energy
    \item Mining centralization and pool dominance pose governance risks
\end{itemize}

\vspace{0.4cm}
\textbf{Core Insight:}

Proof-of-work converts energy into cryptographic security. The cost of attacking the network is proportional to the cumulative computational work invested by honest miners.
\end{frame}

\begin{frame}{Discussion Questions}
\begin{enumerate}
    \item Why is proof-of-work described as ``progress-free''?

    \item How does difficulty adjustment make Bitcoin resilient to hash rate fluctuations?

    \item What would happen if block rewards fell to zero but transaction fees remained low?

    \item Is ASIC mining centralization a threat to Bitcoin's decentralization?

    \item How does mining pool concentration differ from miner concentration?

    \item Can proof-of-work be justified from an environmental perspective?

    \item Why has no successful 51\% attack occurred on Bitcoin?
\end{enumerate}
\end{frame}

\begin{frame}{Next Lesson Preview: L08 Lab - Wallet Setup}
\textbf{Lab activities:}
\begin{itemize}
    \item Install and configure MetaMask wallet
    \item Understand seed phrase security and backup
    \item Connect to Ethereum testnet (Sepolia or Goerli)
    \item Obtain testnet ETH from faucets
    \item Execute first testnet transaction
    \item Explore wallet features and settings
    \item Best practices for wallet security
\end{itemize}

\vspace{0.5cm}
\textbf{Preparation:}
\begin{itemize}
    \item Install a modern web browser (Chrome, Firefox, Brave)
    \item Review public-private key concepts from Lesson 5
    \item Prepare a secure location for seed phrase backup
\end{itemize}
\end{frame}

\end{document}
