\documentclass[8pt,aspectratio=169]{beamer}
\usetheme{Madrid}
\usepackage[utf8]{inputenc}
\usepackage{graphicx}
\usepackage{booktabs}
\usepackage{hyperref}
\usepackage{amsmath}

\title{Proof of Work Consensus}
\subtitle{BSc Blockchain, Crypto Economy \& NFTs}
\author{Course Instructor}
\date{Module A: Blockchain Foundations}

\begin{document}

\begin{frame}
\titlepage
\end{frame}

\begin{frame}{Learning Objectives}
By the end of this lesson, you will be able to:
\begin{itemize}
    \item Explain the proof-of-work consensus mechanism
    \item Describe the mining process and nonce searching
    \item Understand difficulty adjustment and its purpose
    \item Calculate mining profitability and hash rate economics
    \item Recognize the security guarantees and vulnerabilities of PoW
    \item Evaluate the 51\% attack threat model
    \item Discuss energy consumption and environmental impact
\end{itemize}
\end{frame}

\begin{frame}{What is Proof of Work?}
\begin{center}
\includegraphics[width=0.60\textwidth]{charts/01_pow_concept/chart.pdf}
\end{center}

\textbf{Core Concept:} Find a nonce such that the hash of block header is below the target. No shortcut: must try nonces randomly until one works.
\end{frame}

\begin{frame}{Proof-of-Work Properties}
\textbf{Key Properties:}
\begin{enumerate}
    \item \textbf{Asymmetry:} Hard to find, easy to verify
    \item \textbf{Probabilistic:} Expected time to find solution, no guarantee
    \item \textbf{Adjustable difficulty:} Target changes to maintain block time
    \item \textbf{Progress-free:} Past attempts do not help future attempts
\end{enumerate}

\vspace{0.3cm}
\textbf{Mathematical Formulation:}
\[
\text{SHA-256}(\text{SHA-256}(\text{BlockHeader})) < \text{Target}
\]

\vspace{0.3cm}
\textbf{Analogy:} Rolling dice until you get 10 sixes in a row. Each roll is independent. Expected number of attempts: $6^{10}$. Verification: just look at the result.
\end{frame}

\begin{frame}{Bitcoin Block Header Structure}
\begin{center}
\includegraphics[width=0.65\textwidth]{charts/02_block_header/chart.pdf}
\end{center}

\textbf{Mining Process:} Construct block $\rightarrow$ Set timestamp/difficulty $\rightarrow$ Try nonces $\rightarrow$ If hash $<$ target: success, else repeat.
\end{frame}

\begin{frame}{Nonce Space and Extra Nonce}
\textbf{Nonce Space Exhaustion:}
\begin{itemize}
    \item 4 bytes = $2^{32}$ = 4.3 billion possible nonces
    \item Modern ASICs exceed this in milliseconds
    \item Solution: modify coinbase transaction (extra nonce), recompute Merkle root
\end{itemize}

\vspace{0.3cm}
\textbf{Merkle Trees:}
\begin{itemize}
    \item Commit to all transactions with single 32-byte hash
    \item Tree height: $\log_2(n)$ for $n$ transactions
    \item Proof size: $\log_2(n)$ hashes to prove inclusion
    \item Example: 1000 transactions $\rightarrow$ 10 hashes ($\sim$320 bytes proof)
\end{itemize}

\vspace{0.3cm}
\textbf{Extra Nonce Trick:}
\begin{itemize}
    \item Miners modify coinbase transaction (includes extra nonce field)
    \item Recompute Merkle root (different root for each extra nonce)
    \item Expands search space beyond $2^{32}$ nonces
\end{itemize}
\end{frame}

\begin{frame}{Difficulty Adjustment}
\begin{center}
\includegraphics[width=0.55\textwidth]{charts/03_difficulty_adjustment/chart.pdf}
\end{center}

\textbf{Adjustment Rule (Every 2016 Blocks):}
\[
\text{New Target} = \text{Old Target} \times \frac{\text{Actual Time}}{\text{Expected Time (20,160 min)}}
\]
Clamped to $[T/4, T \times 4]$ to prevent extreme changes.
\end{frame}

\begin{frame}{Hash Rate and Mining Probability}
\textbf{Hash Rate:}
\begin{itemize}
    \item Number of hashes computed per second
    \item Units: H/s, KH/s, MH/s, GH/s, TH/s, PH/s, EH/s
    \item Bitcoin network (late 2024): $\sim$700 EH/s (all-time high)
\end{itemize}

\vspace{0.3cm}
\textbf{Mining Probability:}
\[
P(\text{find block}) = \frac{\text{Your Hash Rate}}{\text{Network Hash Rate}}
\]

\vspace{0.3cm}
\textbf{Example:}
\begin{itemize}
    \item Miner: 100 TH/s, Network: 500 EH/s
    \item Probability: $0.0000002 = 0.00002\%$
    \item Expected time to find block: $\sim$100 years solo mining
    \item Solution: Join mining pools for steady income
\end{itemize}
\end{frame}

\begin{frame}{Mining Hardware Evolution}
\begin{center}
\includegraphics[width=0.60\textwidth]{charts/04_mining_hardware/chart.pdf}
\end{center}

\textbf{Implications:} Mining centralization, high barrier to entry, geographic concentration in low-electricity regions.
\end{frame}

\begin{frame}{Mining Profitability}
\textbf{Revenue:}
\[
\text{Daily Revenue} = \frac{\text{Hash Rate}}{\text{Network Hash Rate}} \times 144 \times (\text{Reward} + \text{Fees})
\]

\vspace{0.3cm}
\textbf{Example (Antminer S19 Pro):}
\begin{itemize}
    \item Hash rate: 110 TH/s, Power: 3250 W = 78 kWh/day
    \item Electricity: \$0.05/kWh $\times$ 78 = \$3.90/day
    \item Revenue (@\$40k BTC): $\sim$\$7.92/day
    \item Profit: \$4.02/day, Payback: $\sim$2 years
\end{itemize}

\vspace{0.3cm}
\textbf{Risk Factors:} BTC price volatility, difficulty increases, hardware obsolescence, electricity cost changes.
\end{frame}

\begin{frame}{Mining Pools}
\begin{center}
\includegraphics[width=0.45\textwidth]{charts/05_mining_pools/chart.pdf}
\end{center}

\textbf{Payout Schemes:} PPS (fixed/share), PPLNS (share recent blocks), FPPS (PPS + fees).
\end{frame}

\begin{frame}{Mining Pool Operation}
\textbf{Why Pools Exist:}
\begin{itemize}
    \item Solo mining: high variance (might wait years for block)
    \item Pooled mining: steady income (proportional to hash rate)
\end{itemize}

\vspace{0.3cm}
\textbf{Pool Operation:}
\begin{enumerate}
    \item Pool coordinator distributes mining tasks (shares)
    \item Miners submit partial solutions (lower difficulty)
    \item Pool tracks contribution of each miner
    \item When pool finds block, reward distributed proportionally
    \item Pool takes fee (1-3\%)
\end{enumerate}

\vspace{0.3cm}
\textbf{Centralization Concern:}
\begin{itemize}
    \item Pools do not own hardware (miners can switch pools)
    \item Mitigation: decentralized pool protocols (P2Pool, Stratum V2)
\end{itemize}
\end{frame}

\begin{frame}{Block Rewards and Halving}
\textbf{Block Reward Components:}
\[
\text{Total Reward} = \text{Block Subsidy} + \text{Transaction Fees}
\]

\vspace{0.2cm}
\begin{tabular}{llr}
\toprule
\textbf{Period} & \textbf{Reward} & \textbf{Cumulative Supply} \\
\midrule
2009-2012 & 50 BTC   & 10.5M BTC \\
2012-2016 & 25 BTC   & 15.75M BTC \\
2016-2020 & 12.5 BTC & 18.375M BTC \\
2020-2024 & 6.25 BTC & 19.6875M BTC \\
2024-2028 (current) & 3.125 BTC & 20.34M BTC \\
\bottomrule
\end{tabular}

\vspace{0.3cm}
\textbf{Future:} Transaction fees must eventually sustain mining as block subsidy approaches zero ($\sim$2140).
\end{frame}

\begin{frame}{The 51\% Attack}
\begin{center}
\includegraphics[width=0.60\textwidth]{charts/06_attack_51percent/chart.pdf}
\end{center}

\textbf{CAN do:} Double-spend, censor transactions.\\
\textbf{CANNOT do:} Steal coins without keys, create coins out of thin air.
\end{frame}

\begin{frame}{51\% Attack Economics}
\textbf{Cost of Attack (Bitcoin):}
\begin{itemize}
    \item Need 255 EH/s ($>$50\% of network)
    \item Hardware: $\sim$2.3 million ASICs = \$6.9 billion
    \item Electricity (1 week): $\sim$\$63 million
    \item Total: $\sim$\$7 billion
\end{itemize}

\vspace{0.3cm}
\textbf{Consequences:}
\begin{itemize}
    \item Attack becomes public immediately
    \item Bitcoin price crashes (attacker's hardware worthless)
    \item Community may hard fork (bricks attacker's ASICs)
\end{itemize}

\vspace{0.3cm}
\textbf{Vulnerable Chains:} Small PoW chains, shared mining algorithms. Historical: Bitcoin Gold, Ethereum Classic, Verge.
\end{frame}

\begin{frame}{Energy Consumption}
\begin{center}
\includegraphics[width=0.55\textwidth]{charts/07_energy_comparison/chart.pdf}
\end{center}

\textbf{Counterarguments:} Incentivizes renewable development, facilitates grid balancing, security proportional to value.
\end{frame}

\begin{frame}{ASIC Resistance Alternatives}
\textbf{ASIC-Resistant Algorithms:}
\begin{itemize}
    \item \textbf{Scrypt (Litecoin):} Memory-hard hashing (ASICs developed 2014)
    \item \textbf{Ethash (Ethereum pre-merge):} Memory-hard with large DAG
    \item \textbf{RandomX (Monero):} CPU-optimized, frequently updated
\end{itemize}

\vspace{0.3cm}
\textbf{Trade-offs:}
\begin{itemize}
    \item ASIC resistance $\rightarrow$ lower security per watt
    \item Easier for botnets to attack (commodity hardware)
    \item Algorithm changes create hard fork risks
    \item Debate: specialization increases security investment
\end{itemize}
\end{frame}

\begin{frame}{Key Takeaways}
\begin{itemize}
    \item Proof-of-work provides Sybil resistance via computational cost
    \item Mining searches for nonces to produce valid block hashes
    \item Difficulty adjusts every 2016 blocks to maintain 10-minute block time
    \item Mining profitability depends on hash rate, electricity cost, and BTC price
    \item 51\% attacks are economically infeasible for large PoW chains
    \item Block rewards halve every 4 years, shifting incentives toward fees
    \item Energy consumption is significant but incentivizes renewable energy
\end{itemize}

\vspace{0.3cm}
\textbf{Core Insight:} Proof-of-work converts energy into cryptographic security. The cost of attacking equals cumulative computational work by honest miners.
\end{frame}

\begin{frame}{Discussion Questions}
\begin{enumerate}
    \item Why is proof-of-work described as ``progress-free''?
    \item How does difficulty adjustment make Bitcoin resilient to hash rate fluctuations?
    \item What would happen if block rewards fell to zero but fees remained low?
    \item Is ASIC mining centralization a threat to Bitcoin's decentralization?
    \item How does mining pool concentration differ from miner concentration?
    \item Can proof-of-work be justified from an environmental perspective?
\end{enumerate}
\end{frame}

\begin{frame}{Next Lesson Preview: L08 Lab - Wallet Setup}
\textbf{Lab activities:}
\begin{itemize}
    \item Install and configure MetaMask wallet
    \item Understand seed phrase security and backup
    \item Connect to Ethereum testnet (Sepolia)
    \item Obtain testnet ETH from faucets
    \item Execute first testnet transaction
\end{itemize}

\vspace{0.5cm}
\textbf{Preparation:}
\begin{itemize}
    \item Install a modern web browser (Chrome, Firefox, Brave)
    \item Review public-private key concepts from Lesson 5
    \item Prepare a secure location for seed phrase backup
\end{itemize}
\end{frame}

\end{document}
