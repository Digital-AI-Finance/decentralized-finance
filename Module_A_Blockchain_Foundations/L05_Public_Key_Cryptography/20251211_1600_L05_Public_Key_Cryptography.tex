\documentclass[8pt,aspectratio=169]{beamer}
\usetheme{Madrid}
\usepackage{graphicx}
\usepackage{booktabs}
\usepackage{adjustbox}
\usepackage{multicol}
\usepackage{amsmath}
\usepackage{amssymb}

% Color definitions
\definecolor{mlblue}{RGB}{0,102,204}
\definecolor{mlpurple}{RGB}{51,51,178}
\definecolor{mllavender}{RGB}{173,173,224}
\definecolor{mllavender2}{RGB}{193,193,232}
\definecolor{mllavender3}{RGB}{204,204,235}
\definecolor{mllavender4}{RGB}{214,214,239}
\definecolor{mlorange}{RGB}{255, 127, 14}
\definecolor{mlgreen}{RGB}{44, 160, 44}
\definecolor{mlred}{RGB}{214, 39, 40}
\definecolor{mlgray}{RGB}{127, 127, 127}

% Apply custom colors to Madrid theme
\setbeamercolor{palette primary}{bg=mllavender3,fg=mlpurple}
\setbeamercolor{palette secondary}{bg=mllavender2,fg=mlpurple}
\setbeamercolor{palette tertiary}{bg=mllavender,fg=white}
\setbeamercolor{palette quaternary}{bg=mlpurple,fg=white}
\setbeamercolor{structure}{fg=mlpurple}
\setbeamercolor{section in toc}{fg=mlpurple}
\setbeamercolor{subsection in toc}{fg=mlblue}
\setbeamercolor{title}{fg=mlpurple}
\setbeamercolor{frametitle}{fg=mlpurple,bg=mllavender3}
\setbeamercolor{block title}{bg=mllavender2,fg=mlpurple}
\setbeamercolor{block body}{bg=mllavender4,fg=black}

\setbeamertemplate{navigation symbols}{}
\setbeamertemplate{itemize items}[circle]
\setbeamertemplate{enumerate items}[default]
\setbeamersize{text margin left=5mm,text margin right=5mm}

\title{Lesson 5: Public Key Cryptography}
\subtitle{Module A: Blockchain Foundations}
\author{BSc Blockchain \& Cryptocurrency}
\institute{University Course}
\date{2025}

\begin{document}

\begin{frame}[plain]
\titlepage
\end{frame}

\begin{frame}[t]{Learning Objectives}
By the end of this lesson, you will be able to:

\begin{enumerate}
\item Explain the fundamental difference between symmetric and asymmetric cryptography
\item Understand how public-private key pairs enable secure communication
\item Describe the Elliptic Curve Digital Signature Algorithm (ECDSA)
\item Generate and verify digital signatures
\item Derive blockchain addresses from public keys
\item Recognize key security best practices for cryptocurrency wallets
\end{enumerate}

\vspace{1em}
\textbf{Prerequisites:} L03 - Cryptographic Hash Functions
\end{frame}

\section{Cryptography Fundamentals}

\begin{frame}[t]{Symmetric vs. Asymmetric Cryptography}
\begin{center}
\includegraphics[width=0.78\textwidth]{charts/01_symmetric_vs_asymmetric/chart.pdf}
\end{center}

\vspace{0.3em}
\small
\textbf{Key Insight:} Blockchains use asymmetric cryptography because it solves the key distribution problem.
\end{frame}

\begin{frame}[t]{Symmetric vs. Asymmetric: Comparison}
\begin{columns}[T]
\column{0.48\textwidth}
\textbf{Symmetric Cryptography}
\begin{itemize}
\item Same key for encryption and decryption
\item Fast and efficient
\item Key distribution problem
\item Examples: AES, DES, 3DES
\end{itemize}

\vspace{0.3cm}
\textbf{Use Case:}
\begin{itemize}
\item Encrypting large data volumes
\item Disk encryption
\item VPN tunnels
\end{itemize}

\column{0.48\textwidth}
\textbf{Asymmetric Cryptography}
\begin{itemize}
\item Different keys: public and private
\item Slower than symmetric
\item No key distribution problem
\item Examples: RSA, ECC, ECDSA
\end{itemize}

\vspace{0.3cm}
\textbf{Use Case:}
\begin{itemize}
\item Digital signatures
\item Key exchange
\item Blockchain transactions
\end{itemize}
\end{columns}

\vspace{0.4cm}
\textbf{Key Insight:} Blockchains rely exclusively on asymmetric cryptography for identity and authorization
\end{frame}

\begin{frame}[t]{Public-Private Key Pairs}
\textbf{Core Concept:}
\begin{itemize}
\item Mathematically related pair of keys
\item Public key: freely shared, used to verify signatures
\item Private key: kept secret, used to create signatures
\item One-way mathematical relationship (hard to derive private from public)
\end{itemize}

\vspace{0.3cm}
\textbf{Mathematical Property:}

If Alice has key pair $(PK_A, SK_A)$:
\begin{itemize}
\item Message encrypted with $PK_A$ can only be decrypted with $SK_A$
\item Signature created with $SK_A$ can only be verified with $PK_A$
\end{itemize}

\vspace{0.3cm}
\textbf{Blockchain Usage:}
\begin{itemize}
\item Private key = your ``password'' to control funds
\item Public key (or address) = your account identifier
\item Losing private key = losing access to funds permanently
\item No password reset mechanism exists
\end{itemize}
\end{frame}

\section{Elliptic Curve Cryptography}

\begin{frame}[t]{RSA vs. Elliptic Curve Cryptography}
\begin{columns}[T]
\column{0.48\textwidth}
\textbf{RSA (Rivest-Shamir-Adleman)}
\begin{itemize}
\item Based on factoring large primes
\item Key sizes: 2048-4096 bits
\item Computationally intensive
\item Well-established (1977)
\item Used in TLS/SSL certificates
\end{itemize}

\vspace{0.3cm}
\textbf{Security:}
\begin{itemize}
\item 2048-bit key = 112-bit security
\item 3072-bit key = 128-bit security
\end{itemize}

\column{0.48\textwidth}
\textbf{Elliptic Curve (ECC)}
\begin{itemize}
\item Based on discrete logarithm problem
\item Key sizes: 256 bits
\item More efficient computation
\item Newer (1985)
\item Used in Bitcoin, Ethereum
\end{itemize}

\vspace{0.3cm}
\textbf{Security:}
\begin{itemize}
\item 256-bit key = 128-bit security
\item Equivalent to 3072-bit RSA
\end{itemize}
\end{columns}

\vspace{0.4cm}
\textbf{Why Blockchains Use ECC:} Smaller key sizes = less storage, faster verification, mobile-friendly
\end{frame}

\begin{frame}[t]{Elliptic Curve: Point Addition}
\begin{center}
\includegraphics[width=0.65\textwidth]{charts/02_elliptic_curve/chart.pdf}
\end{center}

\vspace{0.3em}
\small
\textbf{Key Generation:} Choose random $k$ (private key), compute $Q = k \cdot G$ (public key).
\end{frame}

\begin{frame}[t]{Elliptic Curve Mathematics}
\textbf{Elliptic Curve Equation:}
\[
y^2 = x^3 + ax + b
\]

\textbf{Bitcoin's Curve (secp256k1):}
\[
y^2 = x^3 + 7 \quad \text{(parameters: } a=0, b=7 \text{)}
\]

\vspace{0.3cm}
\textbf{Key Mathematical Properties:}
\begin{itemize}
\item Defined over finite field (modulo large prime)
\item Point addition operation creates a group
\item Scalar multiplication: $Q = k \cdot G$ (k = private key, G = generator point, Q = public key)
\item Easy to compute $Q$ from $k$, infeasible to compute $k$ from $Q$
\end{itemize}

\vspace{0.3cm}
\textbf{Intuition:}
\begin{itemize}
\item Private key = random 256-bit number
\item Public key = point on elliptic curve (x, y coordinates)
\item Trapdoor function ensures one-way relationship
\end{itemize}
\end{frame}

\begin{frame}[t]{The Discrete Logarithm Problem}
\textbf{Security Foundation:}

\vspace{0.3cm}
Given:
\begin{itemize}
\item Generator point $G$ (known)
\item Public key $Q = k \cdot G$ (known)
\end{itemize}

Find:
\begin{itemize}
\item Private key $k$ (unknown)
\end{itemize}

\vspace{0.3cm}
\textbf{Computational Complexity:}
\begin{itemize}
\item No known efficient algorithm to solve this
\item Best attack: Pollard's rho algorithm
\item For 256-bit keys: requires $2^{128}$ operations (infeasible)
\item Quantum computers: Shor's algorithm reduces to polynomial time (future threat)
\end{itemize}

\vspace{0.3cm}
\textbf{Practical Implication:}
\begin{itemize}
\item Cannot derive private key from public key
\item Cannot forge signatures without private key
\item Blockchain security relies on this hardness assumption
\end{itemize}
\end{frame}

\section{Digital Signatures}

\begin{frame}[t]{ECDSA: Digital Signature Workflow}
\begin{center}
\includegraphics[width=0.75\textwidth]{charts/04_digital_signature/chart.pdf}
\end{center}

\vspace{0.3em}
\small
\textbf{Properties:} Authentication, Integrity, Non-repudiation, Unforgeability.
\end{frame}

\begin{frame}[t]{Digital Signatures: Purpose and Properties}
\textbf{What is a Digital Signature?}
\begin{itemize}
\item Cryptographic proof that a message was created by the holder of a private key
\item Anyone with the public key can verify authenticity
\item Provides authentication, integrity, and non-repudiation
\end{itemize}

\vspace{0.3cm}
\textbf{Required Properties:}
\begin{enumerate}
\item \textbf{Authentication:} Proves who created the signature
\item \textbf{Integrity:} Any modification to message invalidates signature
\item \textbf{Non-repudiation:} Signer cannot deny creating signature
\item \textbf{Unforgeable:} Impossible to create valid signature without private key
\end{enumerate}

\vspace{0.3cm}
\textbf{Blockchain Applications:}
\begin{itemize}
\item Authorize cryptocurrency transactions
\item Prove ownership of funds
\item Create immutable audit trails
\item Smart contract execution authorization
\end{itemize}
\end{frame}

\begin{frame}[t]{ECDSA: Mathematical Details}
\textbf{Signature Generation:}

Given message $m$ and private key $k$:
\begin{enumerate}
\item Compute message hash: $h = \text{SHA-256}(m)$
\item Generate random nonce: $n$ (must be unique per signature)
\item Compute point: $R = n \cdot G$, extract x-coordinate as $r$
\item Compute: $s = n^{-1}(h + r \cdot k) \mod p$
\item Signature is pair: $(r, s)$
\end{enumerate}

\vspace{0.3cm}
\textbf{Signature Verification:}

Given message $m$, signature $(r, s)$, and public key $Q$:
\begin{enumerate}
\item Compute message hash: $h = \text{SHA-256}(m)$
\item Compute: $u_1 = h \cdot s^{-1} \mod p$, $u_2 = r \cdot s^{-1} \mod p$
\item Compute point: $R' = u_1 \cdot G + u_2 \cdot Q$
\item Extract x-coordinate $r'$
\item Verify: $r' = r$ (signature valid if true)
\end{enumerate}
\end{frame}

\begin{frame}[t]{Critical Security: Nonce Reuse Vulnerability}
\textbf{The Danger of Reusing Nonces:}

\vspace{0.3cm}
If the same nonce $n$ is used to sign two different messages $m_1$ and $m_2$:
\begin{itemize}
\item Attacker obtains signatures: $(r, s_1)$ and $(r, s_2)$
\item Notice that $r$ is the same (same nonce used)
\item Can compute: $s_1 - s_2 = n^{-1}(h_1 - h_2) \mod p$
\item Solve for $n$, then solve for private key $k$
\end{itemize}

\vspace{0.3cm}
\textbf{Real-World Incident: Sony PlayStation 3 (2010)}
\begin{itemize}
\item Sony used ECDSA to sign PS3 firmware
\item Used the same nonce for all signatures
\item Hackers recovered Sony's private key
\item Enabled homebrew software and piracy
\end{itemize}

\vspace{0.3cm}
\textbf{Best Practice:} Use deterministic nonce generation (RFC 6979) - nonce derived from message hash and private key
\end{frame}

\section{Address Derivation}

\begin{frame}[t]{From Private Key to Blockchain Address}
\begin{center}
\includegraphics[width=0.80\textwidth]{charts/03_key_derivation/chart.pdf}
\end{center}

\vspace{0.3em}
\small
\textbf{Security:} One-way functions ensure addresses cannot be reversed to private keys.
\end{frame}

\begin{frame}[t]{Bitcoin Address Derivation Details}
\textbf{Step-by-Step Process:}

\begin{enumerate}
\item Generate random 256-bit private key $k$
\item Compute public key: $Q = k \cdot G$ (33 bytes compressed or 65 bytes uncompressed)
\item Hash public key: $h_1 = \text{SHA-256}(Q)$
\item Hash again: $h_2 = \text{RIPEMD-160}(h_1)$ (20 bytes)
\item Add version byte: $0x00$ for mainnet (21 bytes)
\item Compute checksum: first 4 bytes of $\text{SHA-256}(\text{SHA-256}(\text{version} + h_2))$
\item Concatenate: version + $h_2$ + checksum (25 bytes)
\item Encode in Base58: produces address like ``1A1zP1eP5QGefi2DMPTfTL5SLmv7DivfNa''
\end{enumerate}

\vspace{0.3cm}
\textbf{Why Multiple Hash Functions?}
\begin{itemize}
\item SHA-256: widely trusted, fast
\item RIPEMD-160: reduces address size (160 bits instead of 256 bits)
\item Double hashing: additional security layer
\end{itemize}
\end{frame}

\begin{frame}[t]{Ethereum Address Derivation}
\textbf{Simplified Process:}

\begin{enumerate}
\item Generate random 256-bit private key $k$
\item Compute public key: $Q = k \cdot G$ (uncompressed, 64 bytes excluding prefix)
\item Hash public key: $h = \text{Keccak-256}(Q)$ (32 bytes)
\item Take last 20 bytes of hash
\item Add ``0x'' prefix
\item Result: address like ``0x742d35Cc6634C0532925a3b844Bc9e7595f0bEb''
\end{enumerate}

\vspace{0.3cm}
\textbf{Key Differences from Bitcoin:}
\begin{itemize}
\item Uses Keccak-256 instead of SHA-256 + RIPEMD-160
\item No checksum in base address (EIP-55 adds mixed-case checksum)
\item Hexadecimal encoding instead of Base58
\item Always 20 bytes (40 hex characters)
\end{itemize}
\end{frame}

\begin{frame}[t]{Public Key Compression}
\textbf{Uncompressed Public Key (65 bytes):}
\begin{itemize}
\item Format: $04 \| x \| y$
\item Prefix: 0x04 (1 byte)
\item x-coordinate: 32 bytes
\item y-coordinate: 32 bytes
\end{itemize}

\vspace{0.3cm}
\textbf{Compressed Public Key (33 bytes):}
\begin{itemize}
\item Format: $02/03 \| x$
\item Prefix: 0x02 (if y is even) or 0x03 (if y is odd)
\item x-coordinate: 32 bytes
\item y-coordinate can be reconstructed from x (elliptic curve equation)
\end{itemize}

\vspace{0.3cm}
\textbf{Why Compression Matters:}
\begin{itemize}
\item Reduces blockchain storage by $\approx$50\%
\item Faster transaction propagation
\item Lower transaction fees (Bitcoin SegWit incentivizes compressed keys)
\item Bitcoin: both formats valid, Ethereum: only uncompressed
\end{itemize}
\end{frame}

\section{Wallet Security}

\begin{frame}[t]{Hierarchical Deterministic Wallets (BIP-32)}
\begin{center}
\includegraphics[width=0.68\textwidth]{charts/05_hd_wallet_tree/chart.pdf}
\end{center}

\vspace{0.3em}
\small
\textbf{Advantage:} Backup once (seed phrase), recover all addresses forever.
\end{frame}

\begin{frame}[t]{HD Wallet Details}
\textbf{Problem with Random Key Generation:}
\begin{itemize}
\item Need to back up every new private key separately
\item Managing hundreds of keys becomes impractical
\item Risk of losing individual keys
\end{itemize}

\vspace{0.3cm}
\textbf{HD Wallet Solution:}
\begin{itemize}
\item Generate all keys from single master seed (usually 12-24 words)
\item Deterministic derivation: same seed always produces same keys
\item Hierarchical structure: master key -> account keys -> address keys
\item Only need to back up seed phrase once
\end{itemize}

\vspace{0.3cm}
\textbf{Derivation Path Example:}
\begin{itemize}
\item Bitcoin: \texttt{m/44'/0'/0'/0/0}
\item Ethereum: \texttt{m/44'/60'/0'/0/0}
\item Each level represents: purpose / coin type / account / change / index
\end{itemize}

\vspace{0.3cm}
\textbf{BIP-39 Mnemonic:} Seed phrase = 12-24 English words encoding 128-256 bits of entropy
\end{frame}

\begin{frame}[t]{Multi-Signature Wallets}
\begin{center}
\includegraphics[width=0.78\textwidth]{charts/06_multisig_schemes/chart.pdf}
\end{center}

\vspace{0.3em}
\small
\textbf{Security:} No single point of failure; attacker must compromise multiple keys.
\end{frame}

\begin{frame}[t]{Multi-Signature Use Cases}
\textbf{Concept:}
\begin{itemize}
\item Require multiple signatures to authorize a transaction
\item Common schemes: 2-of-3, 3-of-5, etc.
\item M-of-N threshold: need M signatures out of N total keys
\end{itemize}

\vspace{0.3cm}
\textbf{Use Cases:}
\begin{itemize}
\item Corporate treasuries (require CEO + CFO approval)
\item Estate planning (family members hold keys)
\item Escrow services (buyer + seller + mediator)
\item Enhanced security (attacker must compromise multiple keys)
\end{itemize}

\vspace{0.3cm}
\textbf{Bitcoin Implementation:}
\begin{itemize}
\item P2SH (Pay-to-Script-Hash) addresses start with ``3''
\item Redeem script specifies signature requirements
\end{itemize}

\vspace{0.3cm}
\textbf{Ethereum Implementation:}
\begin{itemize}
\item Smart contract-based (e.g., Gnosis Safe)
\item More flexible logic (time delays, spending limits)
\end{itemize}
\end{frame}

\begin{frame}[t]{Key Security Best Practices}
\textbf{Private Key Protection:}
\begin{itemize}
\item Never share private keys or seed phrases
\item Use hardware wallets (Ledger, Trezor) for large amounts
\item Store backups in multiple secure physical locations
\item Use strong passwords for wallet encryption
\item Avoid storing keys on internet-connected devices
\end{itemize}

\vspace{0.3cm}
\textbf{Common Threats:}
\begin{itemize}
\item Phishing attacks (fake wallet websites)
\item Clipboard malware (replaces copied addresses)
\item Keyloggers (record password entry)
\item SIM swapping (hijack 2FA)
\item Physical theft (unencrypted paper wallets)
\end{itemize}

\vspace{0.3cm}
\textbf{Defense Strategies:}
\begin{itemize}
\item Multi-signature wallets (require M-of-N keys)
\item Air-gapped signing (offline devices)
\item Test small amounts before large transfers
\end{itemize}
\end{frame}

\section{Summary}

\begin{frame}[t]{Key Takeaways}
\begin{itemize}
\item Public key cryptography solves the key distribution problem
\item Private keys must remain absolutely secret
\item ECDSA provides efficient digital signatures for blockchains
\item Nonce reuse is a critical vulnerability
\item Bitcoin and Ethereum use different address derivation schemes
\item HD wallets enable convenient key management from a single seed
\item Multi-signature wallets add an extra security layer
\end{itemize}

\vspace{0.5cm}
\begin{block}{Core Principle}
\textit{``Not your keys, not your coins.''}

If you do not control the private keys to your cryptocurrency addresses, you do not truly own the funds. Exchanges and custodians can be hacked, frozen, or shut down.
\end{block}
\end{frame}

\begin{frame}[t]{Discussion Questions}
\begin{enumerate}
\item Why do blockchains use elliptic curve cryptography instead of RSA?

\item What would happen if two users randomly generated the same private key?

\item How does address derivation protect privacy compared to reusing the same address?

\item Why is the discrete logarithm problem crucial for blockchain security?

\item What are the trade-offs between convenience and security when storing private keys?

\item How would quantum computing impact ECDSA-based blockchains?
\end{enumerate}
\end{frame}

\begin{frame}[t]{Next Lesson Preview: L06 Bitcoin Protocol Deep Dive}
\textbf{Topics to be covered:}
\begin{itemize}
\item Unspent Transaction Output (UTXO) model
\item Bitcoin transaction structure in detail
\item Transaction inputs and outputs
\item Bitcoin Script language
\item Transaction verification process
\item Common transaction types (P2PKH, P2SH, SegWit)
\item Transaction lifecycle from creation to confirmation
\end{itemize}

\vspace{0.5cm}
\textbf{Preparation:}
\begin{itemize}
\item Review how digital signatures work
\item Explore a Bitcoin block explorer (e.g., blockchain.com)
\item Observe transaction structure in real blocks
\end{itemize}
\end{frame}

\begin{frame}[plain]
\vspace{3cm}
\begin{center}
{\Large Thank you}\\[2cm]
{\normalsize Questions?}\\[1cm]
{\small See you in Lesson 6: Bitcoin Protocol Deep Dive}
\end{center}
\end{frame}

\end{document}
