\documentclass[8pt,aspectratio=169]{beamer}
\usetheme{Madrid}
\usepackage[utf8]{inputenc}
\usepackage{graphicx}
\usepackage{booktabs}
\usepackage{hyperref}
\usepackage{amsmath}

\title{Proof of Stake Consensus}
\subtitle{BSc Blockchain, Crypto Economy \& NFTs}
\author{Course Instructor}
\date{Module A: Blockchain Foundations}

\begin{document}

\begin{frame}
\titlepage
\end{frame}

\begin{frame}{Learning Objectives}
By the end of this lesson, you will be able to:
\begin{itemize}
    \item Explain the proof-of-stake consensus mechanism
    \item Describe validator responsibilities: staking, attestation, block proposal
    \item Understand slashing conditions and economic penalties
    \item Analyze Ethereum's Beacon Chain architecture
    \item Distinguish between justification and finality
    \item Calculate staking rewards and economics
    \item Evaluate centralization risks in proof-of-stake systems
\end{itemize}
\end{frame}

\begin{frame}{Proof of Work vs. Proof of Stake}
\begin{columns}[T]
\begin{column}{0.48\textwidth}
\textbf{Proof of Work}
\begin{itemize}
    \item Security via computational cost
    \item Miners compete by hash power
    \item Energy-intensive
    \item Hardware requirements (ASICs)
    \item Block rewards + transaction fees
    \item 51\% attack requires hash rate majority
    \item Used by: Bitcoin, Litecoin, Dogecoin
\end{itemize}
\end{column}

\begin{column}{0.48\textwidth}
\textbf{Proof of Stake}
\begin{itemize}
    \item Security via economic stake
    \item Validators selected by stake weight
    \item Energy-efficient (99\% reduction)
    \item Capital requirements (staking)
    \item Transaction fees only (or low issuance)
    \item 51\% attack requires token majority
    \item Used by: Ethereum, Cardano, Polkadot
\end{itemize}
\end{column}
\end{columns}

\vspace{0.4cm}
\textbf{Core Difference:}
\begin{itemize}
    \item PoW: external resource (electricity) -> security
    \item PoS: internal resource (staked tokens) -> security
\end{itemize}

\vspace{0.3cm}
\textbf{Economic Argument for PoS:}
\begin{itemize}
    \item Attacking network destroys attacker's stake (self-punishment)
    \item PoW attacker retains hardware after attack
    \item PoS aligns economic incentives more directly
\end{itemize}
\end{frame}

\begin{frame}{The Nothing-at-Stake Problem}
\textbf{Theoretical Vulnerability:}

\vspace{0.3cm}
In case of blockchain fork:
\begin{itemize}
    \item PoW miners must choose one fork (cannot mine both simultaneously)
    \item PoS validators can vote on multiple forks simultaneously (no cost)
    \item Rational strategy: vote on all forks to maximize rewards
    \item Prevents consensus from converging on single chain
\end{itemize}

\vspace{0.3cm}
\textbf{Solutions:}

\vspace{0.3cm}
\begin{enumerate}
    \item \textbf{Slashing:}
    \begin{itemize}
        \item Detect validators voting on conflicting blocks
        \item Destroy portion of their stake as penalty
        \item Makes voting on multiple forks costly
    \end{itemize}

    \item \textbf{Finality Gadgets:}
    \begin{itemize}
        \item Checkpoint mechanisms (Casper FFG)
        \item Validators attest to finalized blocks
        \item Cannot reverse finalized blocks without losing stake
    \end{itemize}
\end{enumerate}

\vspace{0.3cm}
\textbf{Practical Observation:}
\begin{itemize}
    \item Well-designed PoS systems have never experienced nothing-at-stake attacks
    \item Slashing and economic incentives effectively mitigate the threat
\end{itemize}
\end{frame}

\begin{frame}{Ethereum's Transition to Proof of Stake}
\textbf{The Merge (September 15, 2022):}
\begin{itemize}
    \item Ethereum transitioned from PoW to PoS
    \item Beacon Chain (PoS) launched December 2020
    \item Ran in parallel with PoW chain for 21 months
    \item Merge combined execution layer (original chain) with consensus layer (Beacon Chain)
    \item No downtime, seamless transition
\end{itemize}

\vspace{0.3cm}
\textbf{Impact:}
\begin{itemize}
    \item Energy consumption reduced by ~99.95\%
    \item ETH issuance reduced by ~90\% (from 13,000 ETH/day to ~1,600 ETH/day)
    \item Block time reduced from ~13s to ~12s
    \item Transaction finality improved (faster finalization)
\end{itemize}

\vspace{0.3cm}
\textbf{Misconceptions Cleared:}
\begin{itemize}
    \item Did NOT reduce gas fees (separate issue, addressed by Layer 2s)
    \item Did NOT increase transaction throughput (scalability via sharding planned later)
    \item Did change consensus mechanism only
\end{itemize}
\end{frame}

\begin{frame}{Ethereum Beacon Chain Architecture}
\textbf{Two-Layer Design:}

\vspace{0.3cm}
\begin{enumerate}
    \item \textbf{Execution Layer (EL):}
    \begin{itemize}
        \item Processes transactions and smart contracts
        \item Maintains Ethereum Virtual Machine (EVM)
        \item Manages account state
        \item Formerly the entire Ethereum blockchain (pre-Merge)
    \end{itemize}

    \item \textbf{Consensus Layer (CL / Beacon Chain):}
    \begin{itemize}
        \item Selects block proposers
        \item Coordinates validators
        \item Manages staking and slashing
        \item Finalizes blocks
    \end{itemize}
\end{enumerate}

\vspace{0.3cm}
\textbf{Communication:}
\begin{itemize}
    \item Execution layer clients: Geth, Nethermind, Erigon, Besu
    \item Consensus layer clients: Prysm, Lighthouse, Teku, Nimbus, Lodestar
    \item Engine API connects EL and CL
    \item Validator runs both clients simultaneously
\end{itemize}
\end{frame}

\begin{frame}{Validator Responsibilities}
\textbf{Core Duties:}

\vspace{0.3cm}
\begin{enumerate}
    \item \textbf{Attestation (every epoch):}
    \begin{itemize}
        \item Vote on head of chain (latest block)
        \item Vote on justified and finalized checkpoints
        \item Submit attestation to network
        \item Attestations aggregated into blocks
    \end{itemize}

    \item \textbf{Block Proposal (occasionally):}
    \begin{itemize}
        \item Selected pseudo-randomly based on stake
        \item Propose new block with transactions
        \item Include attestations from other validators
        \item Earn transaction fees + block reward
    \end{itemize}

    \item \textbf{Sync Committee Participation (rarely):}
    \begin{itemize}
        \item Rotating committee (~27 hours of service)
        \item Help light clients sync to chain
        \item Sign block headers
    \end{itemize}
\end{enumerate}

\vspace{0.3cm}
\textbf{Timing:}
\begin{itemize}
    \item Slot: 12 seconds
    \item Epoch: 32 slots = 6.4 minutes
    \item Validator attests once per epoch
    \item Expected block proposal: every ~2 months (for single validator)
\end{itemize}
\end{frame}

\begin{frame}{Staking Requirements and Process}
\textbf{Minimum Stake:}
\begin{itemize}
    \item 32 ETH per validator
    \item Cannot partially activate (all-or-nothing)
    \item Can run multiple validators (64 ETH = 2 validators, etc.)
\end{itemize}

\vspace{0.3cm}
\textbf{Staking Process:}

\vspace{0.3cm}
\begin{enumerate}
    \item Generate validator keys (BLS12-381 signature scheme)
    \item Deposit 32 ETH to deposit contract on Ethereum mainnet
    \item Run validator client (Prysm, Lighthouse, etc.)
    \item Wait for activation (entry queue, ~24 hours to weeks depending on queue)
    \item Start attesting and proposing blocks
\end{enumerate}

\vspace{0.3cm}
\textbf{Hardware Requirements:}
\begin{itemize}
    \item CPU: 4+ cores
    \item RAM: 16+ GB
    \item Storage: 2+ TB SSD (grows over time)
    \item Internet: reliable, high uptime (99\%+)
    \item Monthly cost: ~\$50-200 (VPS or home setup)
\end{itemize}

\vspace{0.3cm}
\textbf{Staking Alternatives:}
\begin{itemize}
    \item Staking pools (Lido, Rocket Pool): stake any amount
    \item Centralized exchanges (Coinbase, Kraken): custodial staking
    \item Staking-as-a-Service (Allnodes, Staked.us): run validator for you
\end{itemize}
\end{frame}

\begin{frame}{Validator Rewards}
\textbf{Reward Components:}

\vspace{0.3cm}
\begin{enumerate}
    \item \textbf{Attestation Rewards:}
    \begin{itemize}
        \item Earned for timely, correct attestations
        \item Proportional to effective balance (max 32 ETH)
        \item Vote on: head of chain, source checkpoint, target checkpoint
    \end{itemize}

    \item \textbf{Block Proposal Rewards:}
    \begin{itemize}
        \item Transaction fees (priority fees)
        \item Attestation inclusion rewards
        \item Sync committee rewards
    \end{itemize}

    \item \textbf{Sync Committee Rewards:}
    \begin{itemize}
        \item Small bonus for participating in sync committee
    \end{itemize}
\end{enumerate}

\vspace{0.3cm}
\textbf{Annual Percentage Rate (APR):}
\[
\text{APR} \approx \frac{64 \sqrt{N}}{N}
\]
where $N$ = total ETH staked (in millions)

\vspace{0.3cm}
\textbf{Examples (2024-2025):}
\begin{itemize}
    \item 34M+ ETH staked (late 2024): APR $\approx$ 3.2-3.5\%
    \item Over 1 million active validators
    \item More stakers -> lower APR (reward dilution)
\end{itemize}
\end{frame}

\begin{frame}{Penalties and Inactivity Leaks}
\textbf{Missed Attestation Penalty:}
\begin{itemize}
    \item Equal to reward you would have earned
    \item Example: would earn 10,000 gwei -> lose 10,000 gwei
    \item Validator balance decreases
    \item No compounding penalties (linear)
\end{itemize}

\vspace{0.3cm}
\textbf{Inactivity Leak:}
\begin{itemize}
    \item Activated when chain fails to finalize (> 4 epochs without finality)
    \item Offline validators lose stake at increasing rate
    \item Purpose: eject offline validators to restore finality
    \item Quadratic leak rate (accelerates over time)
    \item Leak stops when chain finalizes again
\end{itemize}

\vspace{0.3cm}
\textbf{Example Scenario:}
\begin{itemize}
    \item Network partition: 40\% validators offline
    \item Chain cannot finalize (requires 67\% participation)
    \item Inactivity leak begins
    \item Offline validators lose stake rapidly
    \item Eventually, their stake falls below 33\% threshold
    \item Online validators (60\%) now constitute 67\%+ of remaining stake
    \item Chain resumes finalization
\end{itemize}
\end{frame}

\begin{frame}{Slashing: Severe Penalties}
\textbf{Slashable Offenses:}

\vspace{0.3cm}
\begin{enumerate}
    \item \textbf{Double Proposal:}
    \begin{itemize}
        \item Proposing two different blocks for same slot
        \item Indicates malicious intent or serious software bug
    \end{itemize}

    \item \textbf{Surround Vote:}
    \begin{itemize}
        \item Attesting to two conflicting checkpoint votes
        \item Vote A surrounds vote B (violates Casper FFG rules)
    \end{itemize}

    \item \textbf{Double Vote:}
    \begin{itemize}
        \item Attesting to two different blocks in same epoch
        \item Attempting to create fork
    \end{itemize}
\end{enumerate}

\vspace{0.3cm}
\textbf{Slashing Penalties:}
\begin{itemize}
    \item \textbf{Initial penalty:} 1 ETH (immediate)
    \item \textbf{Correlation penalty:} up to entire stake if many validators slashed simultaneously
    \item \textbf{Forced exit:} validator ejected from active set
    \item \textbf{Withdrawal delay:} 36 days before funds withdrawable
\end{itemize}

\vspace{0.3cm}
\textbf{Correlation Penalty Formula:}
\[
\text{Penalty} = \text{Stake} \times \frac{3 \times \text{Slashed Validators}}{\text{Total Validators}}
\]
\begin{itemize}
    \item If 1\% slashed together: lose ~3\% of stake
    \item If 33\% slashed together: lose ~99\% of stake (catastrophic)
\end{itemize}
\end{frame}

\begin{frame}{Casper FFG: Finality Gadget}
\textbf{Purpose:}
\begin{itemize}
    \item Provide economic finality
    \item Prevent long-range reorganizations
    \item Make finalized blocks irreversible
\end{itemize}

\vspace{0.3cm}
\textbf{Checkpoint Voting:}
\begin{itemize}
    \item Checkpoint = first block of each epoch
    \item Validators vote on checkpoint pairs: (source, target)
    \item Source: last justified checkpoint
    \item Target: current epoch checkpoint
\end{itemize}

\vspace{0.3cm}
\textbf{Justification:}
\begin{itemize}
    \item Checkpoint becomes justified if 67\%+ of validators attest to it
    \item Justification is temporary, can be reverted
\end{itemize}

\vspace{0.3cm}
\textbf{Finalization:}
\begin{itemize}
    \item Checkpoint becomes finalized if:
    \begin{enumerate}
        \item It is justified
        \item Next epoch checkpoint is also justified
    \end{enumerate}
    \item Finalization is permanent (cannot revert without massive slashing)
    \item Typically occurs after 2 epochs (~12.8 minutes)
\end{itemize}

\vspace{0.3cm}
\textbf{Accountable Safety:}
\begin{itemize}
    \item Two conflicting checkpoints cannot both finalize
    \item If they did, 33\%+ of validators would be slashed
    \item Economic guarantee: attacker loses large stake
\end{itemize}
\end{frame}

\begin{frame}{LMD GHOST: Fork Choice Rule}
\textbf{Purpose:}
\begin{itemize}
    \item Determine canonical chain head
    \item Resolve forks before finalization
    \item Ensure validators agree on latest block
\end{itemize}

\vspace{0.3cm}
\textbf{LMD GHOST (Latest Message Driven Greedy Heaviest Observed SubTree):}

\vspace{0.3cm}
\begin{enumerate}
    \item Start at last finalized checkpoint
    \item At each fork, choose subtree with most validator attestations
    \item Recursively descend until leaf (chain head)
    \item Weights updated with each attestation
\end{enumerate}

\vspace{0.3cm}
\textbf{Example:}
\begin{itemize}
    \item Block A has 100 attestations
    \item Block B (competing fork) has 150 attestations
    \item LMD GHOST selects Block B as canonical
    \item Validators build on Block B
\end{itemize}

\vspace{0.3cm}
\textbf{Combination with Casper FFG:}
\begin{itemize}
    \item LMD GHOST: determines head (short-term)
    \item Casper FFG: determines finality (long-term)
    \item Together: provide fast confirmation + eventual certainty
\end{itemize}
\end{frame}

\begin{frame}{Validator Lifecycle}
\textbf{States:}

\vspace{0.3cm}
\begin{enumerate}
    \item \textbf{Deposited:}
    \begin{itemize}
        \item 32 ETH deposited to contract
        \item Waiting in activation queue
        \item Not yet earning rewards
    \end{itemize}

    \item \textbf{Active:}
    \begin{itemize}
        \item Attesting and proposing blocks
        \item Earning rewards or incurring penalties
        \item Can be slashed
    \end{itemize}

    \item \textbf{Exiting:}
    \begin{itemize}
        \item Voluntary exit initiated
        \item Still active for ~1 day (exit queue)
        \item Cannot be canceled
    \end{itemize}

    \item \textbf{Exited:}
    \begin{itemize}
        \item No longer active
        \item Funds locked for 27 hours (sweep delay)
    \end{itemize}

    \item \textbf{Withdrawable:}
    \begin{itemize}
        \item Funds available for withdrawal
        \item Automatic withdrawal to specified address (after Shanghai upgrade, April 2023)
    \end{itemize}
\end{enumerate}

\vspace{0.3cm}
\textbf{Forced Exit:}
\begin{itemize}
    \item Triggered by slashing or balance < 16 ETH
    \item Validator ejected automatically
\end{itemize}
\end{frame}

\begin{frame}{Staking Economics: Risks and Rewards}
\textbf{Rewards:}
\begin{itemize}
    \item Base APR: 3-5\% (varies with total staked)
    \item Transaction fees: variable (0.5-2\% additional during high activity)
    \item MEV (Maximal Extractable Value): 0.5-1\% via MEV-Boost
    \item Total yield: ~4-8\% annually
\end{itemize}

\vspace{0.3cm}
\textbf{Risks:}
\begin{itemize}
    \item \textbf{Slashing:} lose stake due to malicious behavior or bugs (rare)
    \item \textbf{Inactivity penalties:} offline validators lose rewards + leak penalties
    \item \textbf{Opportunity cost:} locked capital (cannot sell during 36-day exit)
    \item \textbf{Technical risk:} node downtime, hardware failure, software bugs
    \item \textbf{Price risk:} ETH price volatility affects total returns
\end{itemize}

\vspace{0.3cm}
\textbf{Break-Even Analysis:}
\begin{itemize}
    \item Setup cost: \$500-2000 (hardware or VPS)
    \item Monthly operating cost: \$50-200
    \item Annual reward (32 ETH at 4\% APR): 1.28 ETH
    \item Reward value (ETH = \$2000): \$2560/year
    \item Annual operating cost: \$600-2400
    \item Net profit: \$160-1960/year (excluding setup cost)
    \item Payback period: 3-15 months
\end{itemize}
\end{frame}

\begin{frame}{Centralization Risks in Proof of Stake}
\textbf{Concerns:}

\vspace{0.3cm}
\begin{enumerate}
    \item \textbf{Wealth Concentration:}
    \begin{itemize}
        \item Rich get richer (compound rewards)
        \item High entry barrier (32 ETH = \$64,000 at \$2000/ETH)
        \item Discourages small holders
    \end{itemize}

    \item \textbf{Exchange Dominance (2024):}
    \begin{itemize}
        \item Lido: ~28\% of staked ETH (down from 32\%)
        \item Coinbase: ~13\%
        \item Top 5 entities: ~55\% of stake
        \item Centralized control risk (improving slowly)
    \end{itemize}

    \item \textbf{Staking Pool Centralization:}
    \begin{itemize}
        \item Users delegate stake to pools
        \item Pool operators control validators
        \item Censorship risk (e.g., OFAC compliance)
    \end{itemize}

    \item \textbf{Client Diversity:}
    \begin{itemize}
        \item Prysm: ~40\% of validators (2023)
        \item Single client bug could finalize invalid chain
        \item Supermajority client creates systemic risk
    \end{itemize}
\end{enumerate}

\vspace{0.3cm}
\textbf{Mitigation Efforts:}
\begin{itemize}
    \item Promote client diversity (incentives for minority clients)
    \item Decentralized staking pools (Rocket Pool, distributed validator technology)
    \item Community norms against >33\% concentration
\end{itemize}
\end{frame}

\begin{frame}{MEV (Maximal Extractable Value)}
\textbf{Definition:}
\begin{itemize}
    \item Additional profit beyond block rewards
    \item Extractable by reordering, inserting, or censoring transactions
    \item Examples: arbitrage, liquidations, sandwich attacks
\end{itemize}

\vspace{0.3cm}
\textbf{How it Works:}
\begin{enumerate}
    \item Searchers find profitable opportunities (e.g., arbitrage between DEXs)
    \item Submit bundles to block builders
    \item Builders construct blocks with MEV transactions
    \item Validators select highest-paying block (via MEV-Boost)
    \item Profit split between searcher, builder, validator
\end{enumerate}

\vspace{0.3cm}
\textbf{MEV-Boost:}
\begin{itemize}
    \item Middleware connecting validators to block builders
    \item Validators outsource block construction
    \item Increases validator rewards by 50-100\%
    \item Used by ~90\% of validators
\end{itemize}

\vspace{0.3cm}
\textbf{Concerns:}
\begin{itemize}
    \item Centralizes block production (few builders dominate)
    \item Censorship risk (builders can exclude transactions)
    \item Users pay hidden costs (sandwich attacks)
    \item Ongoing research: proposer-builder separation, encrypted mempools
\end{itemize}
\end{frame}

\begin{frame}{2024 Milestones: Ethereum PoS Maturity}
\textbf{Network Growth:}
\begin{itemize}
    \item 34M+ ETH staked (28\% of total supply)
    \item 1+ million active validators
    \item Network securing \$400B+ in value
\end{itemize}

\vspace{0.3cm}
\textbf{Protocol Upgrades:}
\begin{itemize}
    \item \textbf{Dencun (March 2024)}: Proto-danksharding (EIP-4844)
    \begin{itemize}
        \item Blob transactions for L2 data availability
        \item L2 fees reduced by 90\%+
    \end{itemize}
    \item \textbf{July 2024}: Spot Ethereum ETFs approved by SEC
    \item Institutional staking products expanding
\end{itemize}

\vspace{0.3cm}
\textbf{Restaking Emergence (EigenLayer):}
\begin{itemize}
    \item \$15B+ in restaked ETH
    \item Validators secure additional networks
    \item New primitive: shared security
    \item Liquid restaking tokens (LRTs) proliferate
\end{itemize}
\end{frame}

\begin{frame}{Key Takeaways}
\begin{itemize}
    \item Proof-of-stake replaces computational work with economic stake
    \item Validators attest to blocks and occasionally propose new blocks
    \item Slashing penalizes malicious behavior, aligning incentives
    \item Casper FFG provides economic finality through checkpoint voting
    \item Ethereum's Beacon Chain reduced energy consumption by 99.95\%
    \item Staking requires 32 ETH and generates 4-8\% annual returns
    \item Centralization risks remain (exchanges, pools, client diversity)
    \item MEV introduces additional revenue but raises censorship concerns
\end{itemize}

\vspace{0.4cm}
\textbf{Core Insight:}

Proof-of-stake shifts blockchain security from external resources (energy) to internal resources (staked capital). Attackers must own and risk a significant portion of the network's value, creating strong economic disincentives.
\end{frame}

\begin{frame}{Discussion Questions}
\begin{enumerate}
    \item How does slashing solve the nothing-at-stake problem?

    \item Why is client diversity critical for proof-of-stake security?

    \item What are the trade-offs between solo staking and using a staking pool?

    \item How does the inactivity leak mechanism help restore finality?

    \item Is proof-of-stake more or less centralized than proof-of-work?

    \item What role does MEV play in validator economics?

    \item How would a 33\% attack differ from a 51\% attack in PoS?
\end{enumerate}
\end{frame}

\begin{frame}{Next Lesson Preview: L10 Consensus Mechanism Comparison}
\textbf{Topics to be covered:}
\begin{itemize}
    \item Comprehensive comparison: PoW vs PoS vs DPoS vs PBFT
    \item Security models and assumptions
    \item Scalability and throughput trade-offs
    \item Energy consumption and sustainability
    \item Decentralization metrics
    \item Finality and confirmation times
    \item Use cases and blockchain selection criteria
\end{itemize}

\vspace{0.5cm}
\textbf{Preparation:}
\begin{itemize}
    \item Review proof-of-work (Lesson 7) and proof-of-stake (Lesson 9)
    \item Read about Delegated Proof of Stake (DPoS) used in EOS, Tron
    \item Explore Byzantine Fault Tolerant (BFT) consensus in Hyperledger, Cosmos
\end{itemize}
\end{frame}

\end{document}
