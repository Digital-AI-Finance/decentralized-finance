\documentclass[8pt,aspectratio=169]{beamer}
\usetheme{Madrid}
\usepackage[utf8]{inputenc}
\usepackage{graphicx}
\usepackage{booktabs}
\usepackage{hyperref}
\usepackage{amsmath}

\newcommand{\bottomnote}[1]{\vfill\footnotesize\textit{#1}}

\title{Proof of Stake Consensus}
\subtitle{BSc Blockchain, Crypto Economy \& NFTs}
\author{Course Instructor}
\date{Module A: Blockchain Foundations}

\begin{document}

\begin{frame}
\titlepage
\end{frame}

\begin{frame}{Learning Objectives}
By the end of this lesson, you will be able to:
\begin{itemize}
    \item Explain the proof-of-stake consensus mechanism
    \item Describe validator responsibilities: staking, attestation, block proposal
    \item Understand slashing conditions and economic penalties
    \item Analyze Ethereum's Beacon Chain architecture
    \item Distinguish between justification and finality
    \item Calculate staking rewards and economics
    \item Evaluate centralization risks in proof-of-stake systems
\end{itemize}
\end{frame}

\begin{frame}{Proof of Work vs. Proof of Stake}
\begin{columns}[T]
\begin{column}{0.48\textwidth}
\textbf{Proof of Work}
\begin{itemize}
    \item Security via computational cost
    \item Miners compete by hash power
    \item Energy-intensive
    \item Hardware requirements (ASICs)
    \item Block rewards + transaction fees
    \item 51\% attack requires hash rate majority
\end{itemize}
\end{column}

\begin{column}{0.48\textwidth}
\textbf{Proof of Stake}
\begin{itemize}
    \item Security via economic stake
    \item Validators selected by stake weight
    \item Energy-efficient (99\% reduction)
    \item Capital requirements (staking)
    \item Transaction fees only (or low issuance)
    \item 51\% attack requires token majority
\end{itemize}
\end{column}
\end{columns}

\vspace{0.3cm}
\textbf{Core Difference:}
\begin{itemize}
    \item PoW: external resource (electricity) $\rightarrow$ security
    \item PoS: internal resource (staked tokens) $\rightarrow$ security
\end{itemize}
\end{frame}

\begin{frame}[t]{PoW vs PoS: Quantitative Comparison}
\begin{center}
\includegraphics[width=0.60\textwidth]{charts/01_pow_vs_pos/chart.pdf}
\end{center}
\bottomnote{PoS dramatically reduces energy while improving finality time and throughput.}
\end{frame}

\begin{frame}{The Nothing-at-Stake Problem}
\textbf{Theoretical Vulnerability:}

\vspace{0.2cm}
In case of blockchain fork:
\begin{itemize}
    \item PoW miners must choose one fork (cannot mine both simultaneously)
    \item PoS validators can vote on multiple forks simultaneously (no cost)
    \item Rational strategy: vote on all forks to maximize rewards
\end{itemize}

\vspace{0.2cm}
\textbf{Solutions:}
\begin{enumerate}
    \item \textbf{Slashing:} Detect and penalize validators voting on conflicting blocks
    \item \textbf{Finality Gadgets:} Checkpoint mechanisms (Casper FFG) prevent reversals
\end{enumerate}

\vspace{0.2cm}
\textbf{Practical Observation:}
\begin{itemize}
    \item Well-designed PoS systems have never experienced nothing-at-stake attacks
    \item Slashing and economic incentives effectively mitigate the threat
\end{itemize}
\end{frame}

\begin{frame}{Ethereum's Transition to Proof of Stake}
\textbf{The Merge (September 15, 2022):}
\begin{itemize}
    \item Ethereum transitioned from PoW to PoS
    \item Beacon Chain (PoS) launched December 2020, ran parallel 21 months
    \item No downtime, seamless transition
\end{itemize}

\vspace{0.3cm}
\textbf{Impact:}
\begin{itemize}
    \item Energy consumption reduced by $\sim$99.95\%
    \item ETH issuance reduced by $\sim$90\% (13,000 $\rightarrow$ 1,600 ETH/day)
    \item Block time: $\sim$13s $\rightarrow$ $\sim$12s
\end{itemize}

\vspace{0.3cm}
\textbf{Misconceptions Cleared:}
\begin{itemize}
    \item Did NOT reduce gas fees (addressed by Layer 2s)
    \item Did NOT increase throughput (scalability via sharding planned later)
    \item Changed consensus mechanism only
\end{itemize}
\end{frame}

\begin{frame}[t]{Ethereum Beacon Chain Architecture}
\begin{center}
\includegraphics[width=0.55\textwidth]{charts/02_beacon_chain/chart.pdf}
\end{center}
\bottomnote{The Merge unified execution and consensus layers under PoS.}
\end{frame}

\begin{frame}{Beacon Chain: Two-Layer Design}
\begin{enumerate}
    \item \textbf{Execution Layer (EL):}
    \begin{itemize}
        \item Processes transactions and smart contracts
        \item Maintains Ethereum Virtual Machine (EVM)
        \item Manages account state
    \end{itemize}

    \item \textbf{Consensus Layer (CL / Beacon Chain):}
    \begin{itemize}
        \item Selects block proposers
        \item Coordinates validators
        \item Manages staking and slashing
        \item Finalizes blocks
    \end{itemize}
\end{enumerate}

\vspace{0.3cm}
\textbf{Client Diversity:}
\begin{itemize}
    \item Execution: Geth, Nethermind, Erigon, Besu
    \item Consensus: Prysm, Lighthouse, Teku, Nimbus, Lodestar
    \item Engine API connects EL and CL
\end{itemize}
\end{frame}

\begin{frame}[t]{Validator Responsibilities}
\begin{center}
\includegraphics[width=0.55\textwidth]{charts/03_validator_duties/chart.pdf}
\end{center}
\bottomnote{Validators perform multiple duties with varying frequencies and reward structures.}
\end{frame}

\begin{frame}{Validator Duties Explained}
\begin{enumerate}
    \item \textbf{Attestation (every epoch):}
    \begin{itemize}
        \item Vote on head of chain (latest block)
        \item Vote on justified and finalized checkpoints
        \item Attestations aggregated into blocks
    \end{itemize}

    \item \textbf{Block Proposal (occasionally):}
    \begin{itemize}
        \item Selected pseudo-randomly based on stake
        \item Propose new block with transactions
        \item Earn transaction fees + block reward
    \end{itemize}

    \item \textbf{Sync Committee (rarely):}
    \begin{itemize}
        \item Rotating committee ($\sim$27 hours of service)
        \item Help light clients sync to chain
    \end{itemize}
\end{enumerate}

\vspace{0.2cm}
\textbf{Timing:} Slot = 12 seconds | Epoch = 32 slots = 6.4 minutes
\end{frame}

\begin{frame}{Staking Requirements and Process}
\textbf{Minimum Stake:}
\begin{itemize}
    \item 32 ETH per validator (all-or-nothing activation)
    \item Can run multiple validators (64 ETH = 2 validators)
\end{itemize}

\vspace{0.2cm}
\textbf{Staking Process:}
\begin{enumerate}
    \item Generate validator keys (BLS12-381 signature scheme)
    \item Deposit 32 ETH to deposit contract
    \item Run validator client (Prysm, Lighthouse, etc.)
    \item Wait for activation (24 hours to weeks)
    \item Start attesting and proposing blocks
\end{enumerate}

\vspace{0.2cm}
\textbf{Hardware Requirements:}
\begin{itemize}
    \item CPU: 4+ cores | RAM: 16+ GB | Storage: 2+ TB SSD
    \item Internet: 99\%+ uptime | Monthly cost: \$50-200
\end{itemize}

\vspace{0.2cm}
\textbf{Alternatives:} Staking pools (Lido, Rocket Pool), exchanges (Coinbase), SaaS
\end{frame}

\begin{frame}[t]{Staking Rewards: APR Curve}
\begin{center}
\includegraphics[width=0.55\textwidth]{charts/05_staking_rewards/chart.pdf}
\end{center}
\bottomnote{More stakers = lower APR due to reward dilution. MEV adds 30\%+ boost.}
\end{frame}

\begin{frame}{Validator Rewards Breakdown}
\textbf{Reward Components:}
\begin{enumerate}
    \item \textbf{Attestation Rewards:} Earned for timely, correct attestations
    \item \textbf{Block Proposal Rewards:} Transaction fees + attestation inclusion
    \item \textbf{Sync Committee Rewards:} Small bonus for participation
\end{enumerate}

\vspace{0.3cm}
\textbf{Annual Percentage Rate (APR):}
\[
\text{APR} \approx \frac{64}{\sqrt{N}}
\]
where $N$ = total ETH staked (in millions)

\vspace{0.3cm}
\textbf{Current State (2024-2025):}
\begin{itemize}
    \item 34M+ ETH staked: APR $\approx$ 3.2-3.5\%
    \item Over 1 million active validators
    \item MEV-Boost adds 30-50\% to base rewards
\end{itemize}
\end{frame}

\begin{frame}{Penalties and Inactivity Leaks}
\textbf{Missed Attestation Penalty:}
\begin{itemize}
    \item Equal to reward you would have earned
    \item Validator balance decreases (linear, no compounding)
\end{itemize}

\vspace{0.3cm}
\textbf{Inactivity Leak:}
\begin{itemize}
    \item Activated when chain fails to finalize ($>$ 4 epochs)
    \item Offline validators lose stake at increasing rate (quadratic)
    \item Purpose: eject offline validators to restore finality
\end{itemize}

\vspace{0.3cm}
\textbf{Example Scenario:}
\begin{itemize}
    \item Network partition: 40\% validators offline
    \item Chain cannot finalize (requires 67\% participation)
    \item Inactivity leak begins, offline validators lose stake rapidly
    \item Eventually chain resumes finalization with remaining validators
\end{itemize}
\end{frame}

\begin{frame}[t]{Slashing: Correlation Penalty}
\begin{center}
\includegraphics[width=0.55\textwidth]{charts/06_slashing_penalties/chart.pdf}
\end{center}
\bottomnote{Coordinated attacks result in catastrophic losses -- strong deterrent against collusion.}
\end{frame}

\begin{frame}{Slashable Offenses}
\textbf{Three Types:}
\begin{enumerate}
    \item \textbf{Double Proposal:} Proposing two blocks for same slot
    \item \textbf{Surround Vote:} Attesting to conflicting checkpoint votes
    \item \textbf{Double Vote:} Attesting to two different blocks in same epoch
\end{enumerate}

\vspace{0.3cm}
\textbf{Slashing Penalties:}
\begin{itemize}
    \item \textbf{Initial penalty:} 1 ETH (immediate)
    \item \textbf{Correlation penalty:} Stake $\times$ 3 $\times$ (slashed/total validators)
    \item \textbf{Forced exit:} validator ejected from active set
    \item \textbf{Withdrawal delay:} 36 days before funds withdrawable
\end{itemize}

\vspace{0.3cm}
\textbf{Penalty Examples:}
\begin{itemize}
    \item 1\% slashed together: lose $\sim$3\% of stake
    \item 33\% slashed together: lose $\sim$99\% of stake (catastrophic)
\end{itemize}
\end{frame}

\begin{frame}[t]{Casper FFG: Finality Mechanism}
\begin{center}
\includegraphics[width=0.60\textwidth]{charts/04_casper_ffg/chart.pdf}
\end{center}
\bottomnote{Finalized blocks cannot be reverted without 33\%+ stake being slashed.}
\end{frame}

\begin{frame}{Casper FFG Explained}
\textbf{Purpose:}
\begin{itemize}
    \item Provide economic finality
    \item Prevent long-range reorganizations
    \item Make finalized blocks irreversible
\end{itemize}

\vspace{0.3cm}
\textbf{Checkpoint Voting:}
\begin{itemize}
    \item Checkpoint = first block of each epoch
    \item Validators vote on checkpoint pairs: (source, target)
\end{itemize}

\vspace{0.3cm}
\textbf{Justification vs Finalization:}
\begin{itemize}
    \item \textbf{Justified:} 67\%+ attestations (temporary, can revert)
    \item \textbf{Finalized:} Justified + next epoch also justified (permanent)
    \item Finalization typically occurs after 2 epochs ($\sim$12.8 minutes)
\end{itemize}

\vspace{0.3cm}
\textbf{Accountable Safety:}
\begin{itemize}
    \item Two conflicting checkpoints cannot both finalize
    \item Economic guarantee: attacker loses massive stake
\end{itemize}
\end{frame}

\begin{frame}{LMD GHOST: Fork Choice Rule}
\textbf{Purpose:} Determine canonical chain head before finalization

\vspace{0.3cm}
\textbf{LMD GHOST (Latest Message Driven Greedy Heaviest Observed SubTree):}
\begin{enumerate}
    \item Start at last finalized checkpoint
    \item At each fork, choose subtree with most validator attestations
    \item Recursively descend until leaf (chain head)
\end{enumerate}

\vspace{0.3cm}
\textbf{Example:}
\begin{itemize}
    \item Block A: 100 attestations | Block B: 150 attestations
    \item LMD GHOST selects Block B as canonical
\end{itemize}

\vspace{0.3cm}
\textbf{Combination with Casper FFG:}
\begin{itemize}
    \item LMD GHOST: determines head (short-term)
    \item Casper FFG: determines finality (long-term)
    \item Together: fast confirmation + eventual certainty
\end{itemize}
\end{frame}

\begin{frame}{Validator Lifecycle}
\textbf{States:}
\begin{enumerate}
    \item \textbf{Deposited:} 32 ETH deposited, waiting in activation queue
    \item \textbf{Active:} Attesting and proposing, earning rewards/penalties
    \item \textbf{Exiting:} Voluntary exit initiated, still active for $\sim$1 day
    \item \textbf{Exited:} No longer active, funds locked for 27 hours
    \item \textbf{Withdrawable:} Funds available (auto-withdrawal after Shanghai)
\end{enumerate}

\vspace{0.3cm}
\textbf{Forced Exit:}
\begin{itemize}
    \item Triggered by slashing or balance $<$ 16 ETH
    \item Validator ejected automatically
\end{itemize}
\end{frame}

\begin{frame}{Staking Economics: Risks and Rewards}
\textbf{Rewards:}
\begin{itemize}
    \item Base APR: 3-5\% (varies with total staked)
    \item Transaction fees: 0.5-2\% during high activity
    \item MEV (via MEV-Boost): 0.5-1\% additional
    \item Total yield: $\sim$4-8\% annually
\end{itemize}

\vspace{0.3cm}
\textbf{Risks:}
\begin{itemize}
    \item \textbf{Slashing:} lose stake due to malicious behavior or bugs
    \item \textbf{Inactivity penalties:} offline validators lose rewards + leak
    \item \textbf{Opportunity cost:} locked capital (36-day exit)
    \item \textbf{Technical risk:} node downtime, hardware failure
    \item \textbf{Price risk:} ETH volatility affects total returns
\end{itemize}
\end{frame}

\begin{frame}[t]{Centralization Risks: Stake Distribution}
\begin{center}
\includegraphics[width=0.50\textwidth]{charts/07_centralization_risks/chart.pdf}
\end{center}
\bottomnote{Top 5 entities control $\sim$53\% of staked ETH -- 33\% threshold gives finality blocking power.}
\end{frame}

\begin{frame}{Centralization Concerns}
\textbf{Key Risks:}
\begin{enumerate}
    \item \textbf{Wealth Concentration:} Rich get richer (compound rewards), high 32 ETH barrier
    \item \textbf{Exchange Dominance:} Lido $\sim$28\%, Coinbase $\sim$13\%, top 5 $\sim$55\%
    \item \textbf{Staking Pool Centralization:} Pool operators control validators
    \item \textbf{Client Diversity:} Prysm $\sim$40\% creates systemic risk
\end{enumerate}

\vspace{0.3cm}
\textbf{Mitigation Efforts:}
\begin{itemize}
    \item Promote client diversity (incentives for minority clients)
    \item Decentralized pools (Rocket Pool, DVT)
    \item Community norms against $>$33\% concentration
    \item Lido share declining, solo stakers growing
\end{itemize}
\end{frame}

\begin{frame}{MEV (Maximal Extractable Value)}
\textbf{Definition:} Additional profit from reordering, inserting, or censoring transactions

\vspace{0.3cm}
\textbf{How it Works:}
\begin{enumerate}
    \item Searchers find opportunities (arbitrage, liquidations)
    \item Submit bundles to block builders
    \item Builders construct MEV-optimized blocks
    \item Validators select highest-paying block (MEV-Boost)
\end{enumerate}

\vspace{0.3cm}
\textbf{MEV-Boost:} Used by $\sim$90\% of validators, increases rewards 50-100\%

\vspace{0.3cm}
\textbf{Concerns:}
\begin{itemize}
    \item Centralizes block production (few builders dominate)
    \item Censorship risk (builders can exclude transactions)
    \item Users pay hidden costs (sandwich attacks)
\end{itemize}
\end{frame}

\begin{frame}{2024-2025 Milestones}
\textbf{Network Growth:}
\begin{itemize}
    \item 34M+ ETH staked (28\% of total supply)
    \item 1+ million active validators
    \item Network securing \$400B+ in value
\end{itemize}

\vspace{0.3cm}
\textbf{Protocol Upgrades:}
\begin{itemize}
    \item \textbf{Dencun (March 2024):} Proto-danksharding (EIP-4844), L2 fees down 90\%+
    \item \textbf{July 2024:} Spot Ethereum ETFs approved by SEC
    \item Institutional staking products expanding
\end{itemize}

\vspace{0.3cm}
\textbf{Restaking (EigenLayer):}
\begin{itemize}
    \item \$15B+ in restaked ETH
    \item Validators secure additional networks
    \item Liquid restaking tokens (LRTs) proliferate
\end{itemize}
\end{frame}

\begin{frame}{Key Takeaways}
\begin{itemize}
    \item Proof-of-stake replaces computational work with economic stake
    \item Validators attest to blocks and occasionally propose new blocks
    \item Slashing penalizes malicious behavior, aligning incentives
    \item Casper FFG provides economic finality through checkpoint voting
    \item Ethereum's Beacon Chain reduced energy consumption by 99.95\%
    \item Staking requires 32 ETH and generates 4-8\% annual returns
    \item Centralization risks remain (exchanges, pools, client diversity)
\end{itemize}

\vspace{0.4cm}
\textbf{Core Insight:}

Proof-of-stake shifts blockchain security from external resources (energy) to internal resources (staked capital). Attackers must own and risk a significant portion of the network's value.
\end{frame}

\begin{frame}{Discussion Questions}
\begin{enumerate}
    \item How does slashing solve the nothing-at-stake problem?
    \item Why is client diversity critical for proof-of-stake security?
    \item What are the trade-offs between solo staking and using a staking pool?
    \item How does the inactivity leak mechanism help restore finality?
    \item Is proof-of-stake more or less centralized than proof-of-work?
    \item What role does MEV play in validator economics?
    \item How would a 33\% attack differ from a 51\% attack in PoS?
\end{enumerate}
\end{frame}

\begin{frame}{Next Lesson Preview: L10 Consensus Mechanism Comparison}
\textbf{Topics to be covered:}
\begin{itemize}
    \item Comprehensive comparison: PoW vs PoS vs DPoS vs PBFT
    \item Security models and assumptions
    \item Scalability and throughput trade-offs
    \item Energy consumption and sustainability
    \item Decentralization metrics
    \item Finality and confirmation times
\end{itemize}

\vspace{0.5cm}
\textbf{Preparation:}
\begin{itemize}
    \item Review proof-of-work (Lesson 7) and proof-of-stake (Lesson 9)
    \item Read about Delegated Proof of Stake (DPoS) used in EOS, Tron
    \item Explore BFT consensus in Hyperledger, Cosmos
\end{itemize}
\end{frame}

\end{document}
