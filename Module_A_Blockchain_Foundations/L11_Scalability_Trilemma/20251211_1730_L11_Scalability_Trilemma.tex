\documentclass[8pt,aspectratio=169]{beamer}
\usetheme{Madrid}
\usepackage[utf8]{inputenc}
\usepackage{graphicx}
\usepackage{booktabs}
\usepackage{hyperref}
\usepackage{amsmath}

\newcommand{\bottomnote}[1]{\vfill\footnotesize\textit{#1}}

\title{Blockchain Scalability Trilemma}
\subtitle{BSc Blockchain, Crypto Economy \& NFTs}
\author{Course Instructor}
\date{Module A: Blockchain Foundations}

\begin{document}

\begin{frame}
\titlepage
\end{frame}

\begin{frame}{Learning Objectives}
By the end of this lesson, you will be able to:
\begin{itemize}
    \item Explain the blockchain scalability trilemma
    \item Analyze trade-offs between security, decentralization, and scalability
    \item Understand Layer 1 scalability bottlenecks
    \item Compare throughput limitations across blockchains
    \item Evaluate vertical vs. horizontal scaling approaches
    \item Recognize emerging scalability solutions
\end{itemize}
\end{frame}

\begin{frame}[t]{The Scalability Trilemma}
\begin{center}
\includegraphics[width=0.55\textwidth]{charts/01_trilemma_triangle/chart.pdf}
\end{center}
\bottomnote{Coined by Vitalik Buterin: a blockchain can achieve at most two of three properties.}
\end{frame}

\begin{frame}{Understanding the Trilemma}
\textbf{The Three Properties:}
\begin{enumerate}
    \item \textbf{Decentralization:} No single entity or small group controls the network
    \item \textbf{Security:} Resistant to attacks, ensures data integrity
    \item \textbf{Scalability:} High transaction throughput and low latency
\end{enumerate}

\vspace{0.3cm}
\textbf{Trade-Off Examples:}
\begin{itemize}
    \item Bitcoin: Decentralized + Secure $\rightarrow$ Low Scalability (7 TPS)
    \item EOS: Scalable + Secure $\rightarrow$ Low Decentralization (21 block producers)
    \item Centralized Database: Scalable + Secure $\rightarrow$ No Decentralization
\end{itemize}

\vspace{0.3cm}
\textbf{Core Challenge:}
\begin{itemize}
    \item Larger blocks $\rightarrow$ fewer nodes can validate (centralization)
    \item Fewer validators $\rightarrow$ faster consensus (centralization)
    \item No free lunch: optimizing one property often degrades another
\end{itemize}
\end{frame}

\begin{frame}{Why Each Property Matters}
\begin{columns}[T]
\begin{column}{0.32\textwidth}
\textbf{Decentralization}
\begin{itemize}
    \item Censorship resistance
    \item Fault tolerance
    \item Trustlessness
    \item No single point of failure
\end{itemize}
\end{column}

\begin{column}{0.32\textwidth}
\textbf{Security}
\begin{itemize}
    \item Immutability
    \item Double-spend prevention
    \item Attack resistance
    \item Data integrity
\end{itemize}
\end{column}

\begin{column}{0.32\textwidth}
\textbf{Scalability}
\begin{itemize}
    \item High throughput (TPS)
    \item Low latency (finality)
    \item Low transaction costs
    \item Mass adoption support
\end{itemize}
\end{column}
\end{columns}

\vspace{0.4cm}
\textbf{Why Blockchain Needs All Three:}
\begin{itemize}
    \item DeFi applications need high throughput AND security
    \item Mass adoption requires scalability without sacrificing decentralization
    \item Competition with traditional payment systems (Visa: 24,000 TPS)
\end{itemize}
\end{frame}

\begin{frame}{Layer 1 Bottlenecks}
\textbf{Fundamental Constraints:}
\begin{enumerate}
    \item \textbf{Block Size:}
    \begin{itemize}
        \item Larger blocks = more TXs, but slower propagation, higher storage
        \item Bitcoin: 1-4 MB | Ethereum: variable (gas limit)
    \end{itemize}

    \item \textbf{Block Time:}
    \begin{itemize}
        \item Faster blocks = higher throughput, but more orphans/forks
        \item Bitcoin: 10 min | Ethereum: 12 sec | Solana: 0.4 sec
    \end{itemize}

    \item \textbf{State Growth:}
    \begin{itemize}
        \item Bitcoin: $\sim$500 GB | Ethereum: $\sim$1 TB (archive)
        \item Full nodes require significant storage
    \end{itemize}

    \item \textbf{Computational Overhead:}
    \begin{itemize}
        \item Every node verifies every transaction
        \item Smart contracts computationally expensive
    \end{itemize}
\end{enumerate}
\end{frame}

\begin{frame}[t]{Layer 1 Throughput Comparison}
\begin{center}
\includegraphics[width=0.55\textwidth]{charts/02_tps_comparison/chart.pdf}
\end{center}
\bottomnote{Inverse correlation: higher TPS generally means fewer nodes and lower decentralization.}
\end{frame}

\begin{frame}{The Block Size Debate: Bitcoin Case Study}
\textbf{Background:}
\begin{itemize}
    \item Bitcoin originally: 1 MB block size limit
    \item As adoption grew: blocks filled up $\rightarrow$ higher fees, slower confirmations
\end{itemize}

\vspace{0.3cm}
\textbf{Big Block Proponents:}
\begin{itemize}
    \item Increase to 8 MB, 32 MB, or unlimited
    \item Immediate throughput increase, low fees
    \item Risk: centralization (fewer can run full nodes)
\end{itemize}

\vspace{0.3cm}
\textbf{Small Block Proponents:}
\begin{itemize}
    \item Keep blocks small to preserve decentralization
    \item Scale via Layer 2 (Lightning Network)
    \item Accept higher on-chain fees
\end{itemize}

\vspace{0.3cm}
\textbf{Outcome (2017):}
\begin{itemize}
    \item Bitcoin Cash hard fork (8 MB blocks)
    \item Bitcoin: SegWit soft fork (effective 1-4 MB)
    \item Market verdict: BTC prioritized decentralization, price > BCH
\end{itemize}
\end{frame}

\begin{frame}[t]{Vertical vs. Horizontal Scaling}
\begin{center}
\includegraphics[width=0.55\textwidth]{charts/05_scaling_approaches/chart.pdf}
\end{center}
\bottomnote{Most major blockchains pursue horizontal scaling to preserve decentralization.}
\end{frame}

\begin{frame}{Scaling Approaches Explained}
\begin{columns}[T]
\begin{column}{0.48\textwidth}
\textbf{Vertical Scaling (Scale Up)}
\begin{itemize}
    \item Increase capacity of individual nodes
    \item Require more powerful hardware
    \item Examples: Solana, EOS
\end{itemize}

\textbf{Advantages:}
\begin{itemize}
    \item Simpler implementation
    \item Immediate throughput gains
\end{itemize}

\textbf{Disadvantages:}
\begin{itemize}
    \item Raises barrier to run nodes
    \item Centralization risk
    \item Hits physical limits
\end{itemize}
\end{column}

\begin{column}{0.48\textwidth}
\textbf{Horizontal Scaling (Scale Out)}
\begin{itemize}
    \item Distribute load across many nodes
    \item Parallel processing
    \item Examples: Sharding, Layer 2, Rollups
\end{itemize}

\textbf{Advantages:}
\begin{itemize}
    \item Preserves decentralization
    \item Theoretically unbounded scaling
\end{itemize}

\textbf{Disadvantages:}
\begin{itemize}
    \item Complex implementation
    \item Cross-shard communication overhead
    \item Longer development timelines
\end{itemize}
\end{column}
\end{columns}
\end{frame}

\begin{frame}[t]{Layer 2 Scaling Solutions}
\begin{center}
\includegraphics[width=0.55\textwidth]{charts/03_layer2_tps/chart.pdf}
\end{center}
\bottomnote{Layer 2 achieves 100-10,000x improvement while inheriting L1 security.}
\end{frame}

\begin{frame}{Layer 2 Types}
\textbf{1. Payment Channels (Lightning Network):}
\begin{itemize}
    \item Open channel with on-chain TX, unlimited off-chain TXs
    \item Use case: micropayments, instant transfers
    \item TPS: theoretically unlimited
\end{itemize}

\vspace{0.3cm}
\textbf{2. Rollups (Optimistic, ZK):}
\begin{itemize}
    \item Bundle hundreds of transactions into one
    \item Optimistic: assume valid, fraud proofs challenge
    \item ZK: cryptographic validity proofs
    \item Use case: DeFi, NFTs, general computation
\end{itemize}

\vspace{0.3cm}
\textbf{3. Sidechains (Polygon):}
\begin{itemize}
    \item Independent blockchain with bridge to main chain
    \item Own consensus mechanism
    \item Weaker security than rollups (separate validator set)
\end{itemize}

\vspace{0.3cm}
\textbf{Current Adoption:} Ethereum L2 TVL: \$40B+ (2024)
\end{frame}

\begin{frame}[t]{State Growth Problem}
\begin{center}
\includegraphics[width=0.55\textwidth]{charts/04_state_growth/chart.pdf}
\end{center}
\bottomnote{Running full nodes becomes prohibitively expensive as state grows.}
\end{frame}

\begin{frame}{State Growth Solutions}
\textbf{Challenge:}
\begin{itemize}
    \item Blockchain size grows unbounded
    \item Ethereum state: $\sim$100 GB (accounts, storage)
    \item Syncing new nodes takes days/weeks
    \item Centralization pressure (only dedicated users run nodes)
\end{itemize}

\vspace{0.3cm}
\textbf{Solutions:}
\begin{enumerate}
    \item \textbf{State Expiry (Ethereum proposal):}
    \begin{itemize}
        \item Inactive state evicted from active storage
        \item Must provide proof to re-activate
    \end{itemize}

    \item \textbf{Pruning:}
    \begin{itemize}
        \item Discard old blockchain history
        \item Keep only recent blocks + current state
    \end{itemize}

    \item \textbf{Rent (Cosmos, EOS):}
    \begin{itemize}
        \item Users pay ongoing fees to store state
        \item Incentivizes cleanup of unused data
    \end{itemize}
\end{enumerate}
\end{frame}

\begin{frame}{Sharding: Horizontal Scaling at Layer 1}
\textbf{Concept:}
\begin{itemize}
    \item Split blockchain into parallel ``shards''
    \item Each shard processes subset of transactions
    \item Aggregate throughput = shards $\times$ per-shard TPS
\end{itemize}

\vspace{0.3cm}
\textbf{Ethereum Sharding Roadmap:}
\begin{itemize}
    \item Current plan: data availability sharding (danksharding)
    \item EIP-4844 proto-danksharding deployed March 2024
    \item Target: 100,000+ TPS via rollups + sharding
\end{itemize}

\vspace{0.3cm}
\textbf{Challenges:}
\begin{itemize}
    \item Cross-shard communication complexity
    \item Shard takeover attacks (low-stake shards vulnerable)
    \item State management (which shard holds which data?)
\end{itemize}

\vspace{0.3cm}
\textbf{Other Sharded Chains:} Zilliqa, NEAR Protocol, Elrond
\end{frame}

\begin{frame}[t]{Trilemma Positioning}
\begin{center}
\includegraphics[width=0.55\textwidth]{charts/06_trilemma_positions/chart.pdf}
\end{center}
\bottomnote{Ethereum + L2 approaches the ``ideal'' corner without sacrificing security.}
\end{frame}

\begin{frame}{Breaking the Trilemma: Possible?}
\textbf{Optimistic View:}
\begin{itemize}
    \item Layer 2 separates execution from settlement
    \item Rollups achieve scalability without sacrificing L1 security
    \item Data availability sampling enables sharding
    \item Zero-knowledge proofs compress computation
\end{itemize}

\vspace{0.3cm}
\textbf{Pessimistic View:}
\begin{itemize}
    \item Layer 2 introduces new trust assumptions (sequencers, bridges)
    \item Increased complexity $\rightarrow$ more attack vectors
    \item Still bound by data availability bandwidth
\end{itemize}

\vspace{0.3cm}
\textbf{Consensus View:}
\begin{itemize}
    \item Trilemma remains but can be mitigated
    \item Modular architecture (separate execution, consensus, DA)
    \item Accept specialization: L1 for security, L2 for scale
    \item Not ``solving'' trilemma but ``navigating'' it intelligently
\end{itemize}
\end{frame}

\begin{frame}{Key Takeaways}
\begin{itemize}
    \item Scalability trilemma: cannot maximize decentralization, security, and scalability simultaneously
    \item Layer 1 bottlenecks: block size, block time, state growth
    \item Vertical scaling (bigger nodes) increases centralization
    \item Horizontal scaling (sharding, Layer 2) preserves decentralization
    \item Layer 2 solutions achieve 100-1000x throughput improvement
    \item State growth threatens long-term node decentralization
    \item Real-world TPS often far below theoretical limits
\end{itemize}

\vspace{0.4cm}
\textbf{Design Philosophy:}

Accept that trade-offs exist. Bitcoin prioritizes decentralization/security. Solana prioritizes scalability. Ethereum aims for balance via Layer 2.
\end{frame}

\begin{frame}{Discussion Questions}
\begin{enumerate}
    \item Why cannot a blockchain simply increase block size indefinitely?
    \item How do Layer 2 solutions differ from sidechains in terms of security?
    \item What are the implications of state growth for decentralization?
    \item Can the trilemma be ``solved'' or only mitigated?
    \item How does sharding introduce new security challenges?
    \item Why do most blockchains have lower actual TPS than theoretical TPS?
\end{enumerate}
\end{frame}

\begin{frame}{Next Lesson Preview: L12 Lab - Block Explorer Analysis}
\textbf{Lab activities:}
\begin{itemize}
    \item Navigate Etherscan and Blockstream block explorers
    \item Analyze transaction details (inputs, outputs, fees, confirmations)
    \item Trace transaction lifecycle from mempool to confirmation
    \item Examine block structure (header, transactions, miner rewards)
    \item Practice forensic blockchain analysis
\end{itemize}

\vspace{0.5cm}
\textbf{Preparation:}
\begin{itemize}
    \item Review Bitcoin transaction structure (Lesson 6)
    \item Ensure access to Etherscan.io and Blockstream.info
\end{itemize}
\end{frame}

\end{document}
