\documentclass[8pt,aspectratio=169]{beamer}
\usetheme{Madrid}
\usepackage[utf8]{inputenc}
\usepackage{graphicx}
\usepackage{booktabs}
\usepackage{hyperref}
\usepackage{amsmath}

\title{Blockchain Scalability Trilemma}
\subtitle{BSc Blockchain, Crypto Economy \& NFTs}
\author{Course Instructor}
\date{Module A: Blockchain Foundations}

\begin{document}

\begin{frame}
\titlepage
\end{frame}

\begin{frame}{Learning Objectives}
By the end of this lesson, you will be able to:
\begin{itemize}
    \item Explain the blockchain scalability trilemma
    \item Analyze trade-offs between security, decentralization, and scalability
    \item Understand Layer 1 scalability bottlenecks
    \item Compare throughput limitations across blockchains
    \item Evaluate vertical vs. horizontal scaling approaches
    \item Recognize emerging scalability solutions
    \item Assess real-world blockchain performance
\end{itemize}
\end{frame}

\begin{frame}{The Scalability Trilemma}
\textbf{Coined by Vitalik Buterin (Ethereum co-founder):}

\vspace{0.3cm}
A blockchain can achieve at most two of these three properties:

\vspace{0.3cm}
\begin{enumerate}
    \item \textbf{Decentralization:} No single entity or small group controls the network
    \item \textbf{Security:} Resistant to attacks, ensures data integrity
    \item \textbf{Scalability:} High transaction throughput and low latency
\end{enumerate}

\vspace{0.3cm}
\textbf{Trade-Off Examples:}
\begin{itemize}
    \item Bitcoin: Decentralized + Secure -> Low Scalability (7 TPS)
    \item EOS: Scalable + Secure -> Low Decentralization (21 block producers)
    \item Centralized Database: Scalable + Secure -> No Decentralization (single operator)
\end{itemize}

\vspace{0.3cm}
\textbf{Core Challenge:}
\begin{itemize}
    \item Increasing throughput typically requires:
    \begin{itemize}
        \item Larger blocks -> fewer nodes can validate (centralization)
        \item Fewer validators -> faster consensus (centralization)
        \item Weaker security assumptions -> attack vulnerability (insecurity)
    \end{itemize}
    \item No free lunch: optimizing one property often degrades another
\end{itemize}
\end{frame}

\begin{frame}{Why Decentralization Matters}
\textbf{Benefits of Decentralization:}

\vspace{0.3cm}
\begin{enumerate}
    \item \textbf{Censorship Resistance:}
    \begin{itemize}
        \item No single entity can block transactions
        \item Critical for financial freedom
        \item Enables permissionless participation
    \end{itemize}

    \item \textbf{Fault Tolerance:}
    \begin{itemize}
        \item Network survives node failures
        \item No single point of failure
        \item Geographic redundancy
    \end{itemize}

    \item \textbf{Trustlessness:}
    \begin{itemize}
        \item Users do not need to trust any central authority
        \item Anyone can verify blockchain state
        \item Reduces corruption and rent-seeking
    \end{itemize}
\end{enumerate}

\vspace{0.3cm}
\textbf{Centralization Risks:}
\begin{itemize}
    \item Government coercion (shut down operators)
    \item Regulatory capture (compliance requirements exclude users)
    \item Monopolistic behavior (rent extraction, censorship)
    \item Trust assumptions (defeats blockchain purpose)
\end{itemize}
\end{frame}

\begin{frame}{Why Security Matters}
\textbf{Security Requirements:}

\vspace{0.3cm}
\begin{enumerate}
    \item \textbf{Immutability:}
    \begin{itemize}
        \item Confirmed transactions cannot be reversed
        \item Historical data cannot be altered
        \item Audit trail permanence
    \end{itemize}

    \item \textbf{Double-Spend Prevention:}
    \begin{itemize}
        \item Same funds cannot be spent twice
        \item Consensus ensures single valid transaction history
        \item Critical for monetary applications
    \end{itemize}

    \item \textbf{Attack Resistance:}
    \begin{itemize}
        \item 51\% attack economically infeasible
        \item Sybil attack prevented by consensus mechanism
        \item Network remains available despite attacks
    \end{itemize}
\end{enumerate}

\vspace{0.3cm}
\textbf{Security Failures:}
\begin{itemize}
    \item 51\% attacks: Bitcoin Gold (2018), Ethereum Classic (2020)
    \item Smart contract exploits: The DAO (2016), Poly Network (2021)
    \item Weak consensus: various small chains (low hash rate)
\end{itemize}
\end{frame}

\begin{frame}{Why Scalability Matters}
\textbf{Scalability Metrics:}

\vspace{0.3cm}
\begin{enumerate}
    \item \textbf{Throughput (TPS):}
    \begin{itemize}
        \item Transactions processed per second
        \item Bitcoin: 7 TPS
        \item Visa: 24,000 TPS (peak: 65,000 TPS)
    \end{itemize}

    \item \textbf{Latency (Confirmation Time):}
    \begin{itemize}
        \item Time until transaction finality
        \item Bitcoin: 1 hour (6 confirmations)
        \item Credit card: instant authorization (settlement in days)
    \end{itemize}

    \item \textbf{Cost per Transaction:}
    \begin{itemize}
        \item Gas fees in Ethereum
        \item Bitcoin transaction fees
        \item Affects microtransaction viability
    \end{itemize}
\end{enumerate}

\vspace{0.3cm}
\textbf{Why Blockchain Needs Scalability:}
\begin{itemize}
    \item Mass adoption requires handling millions of users
    \item DeFi applications need high throughput
    \item Micropayments require low fees
    \item Gaming and social media need instant confirmation
    \item Competition with traditional payment systems
\end{itemize}
\end{frame}

\begin{frame}{Layer 1 Bottlenecks}
\textbf{Fundamental Constraints:}

\vspace{0.3cm}
\begin{enumerate}
    \item \textbf{Block Size:}
    \begin{itemize}
        \item Larger blocks = more transactions per block
        \item But: slower propagation, higher storage requirements
        \item Bitcoin: 1-4 MB (SegWit), Ethereum: variable (gas limit)
        \item Increasing block size reduces decentralization (fewer can run full nodes)
    \end{itemize}

    \item \textbf{Block Time:}
    \begin{itemize}
        \item Faster blocks = higher throughput
        \item But: more orphaned blocks (forks), lower security
        \item Bitcoin: 10 min, Ethereum: 12 sec, Solana: 0.4 sec
        \item Trade-off: speed vs. consensus stability
    \end{itemize}

    \item \textbf{State Growth:}
    \begin{itemize}
        \item Blockchain size grows indefinitely
        \item Bitcoin: ~500 GB, Ethereum: ~1 TB (archive node)
        \item Full nodes require significant storage
        \item Pruning reduces storage but limits historical queries
    \end{itemize}

    \item \textbf{Computational Overhead:}
    \begin{itemize}
        \item Every node verifies every transaction
        \item Smart contract execution computationally expensive
        \item Parallel processing limited by sequential dependencies
    \end{itemize}
\end{enumerate}
\end{frame}

\begin{frame}{The Block Size Debate: Bitcoin Case Study}
\textbf{Background:}
\begin{itemize}
    \item Bitcoin originally: 1 MB block size limit
    \item As adoption grew, blocks filled up -> higher fees, slower confirmations
    \item Community divided on solution
\end{itemize}

\vspace{0.3cm}
\textbf{Big Block Proponents:}
\begin{itemize}
    \item Increase block size to 8 MB, 32 MB, or unlimited
    \item Immediate throughput increase
    \item Maintain low fees
    \item Risk: centralization (fewer can run full nodes)
\end{itemize}

\vspace{0.3cm}
\textbf{Small Block Proponents:}
\begin{itemize}
    \item Keep blocks small to preserve decentralization
    \item Scale via Layer 2 (Lightning Network)
    \item Maintain full node accessibility
    \item Accept higher on-chain fees
\end{itemize}

\vspace{0.3cm}
\textbf{Outcome:}
\begin{itemize}
    \item 2017: Bitcoin Cash hard fork (8 MB blocks)
    \item Bitcoin: SegWit soft fork (effective 1-4 MB)
    \item Competing visions split community
    \item Bitcoin prioritized decentralization, BTC price > BCH price (market verdict)
\end{itemize}
\end{frame}

\begin{frame}{Vertical vs. Horizontal Scaling}
\begin{columns}[T]
\begin{column}{0.48\textwidth}
\textbf{Vertical Scaling (Scale Up)}
\begin{itemize}
    \item Increase capacity of individual nodes
    \item Require more powerful hardware
    \item Larger blocks, faster processing
    \item Examples: Solana, EOS
\end{itemize}

\vspace{0.3cm}
\textbf{Advantages:}
\begin{itemize}
    \item Simpler implementation
    \item Immediate throughput gains
    \item No protocol changes needed
\end{itemize}

\vspace{0.3cm}
\textbf{Disadvantages:}
\begin{itemize}
    \item Raises barrier to run nodes
    \item Centralization risk
    \item Hardware costs grow linearly
    \item Eventually hits physical limits
\end{itemize}
\end{column}

\begin{column}{0.48\textwidth}
\textbf{Horizontal Scaling (Scale Out)}
\begin{itemize}
    \item Distribute load across many nodes
    \item Parallel processing
    \item Sharding, Layer 2 solutions
    \item Examples: Ethereum sharding, Polkadot parachains
\end{itemize}

\vspace{0.3cm}
\textbf{Advantages:}
\begin{itemize}
    \item Preserves decentralization
    \item Theoretically unbounded scaling
    \item Nodes remain affordable
\end{itemize}

\vspace{0.3cm}
\textbf{Disadvantages:}
\begin{itemize}
    \item Complex implementation
    \item Cross-shard communication overhead
    \item Security challenges (shard attacks)
    \item Longer development timelines
\end{itemize}
\end{column}
\end{columns}

\vspace{0.4cm}
\textbf{Trend:}
Most major blockchains pursuing horizontal scaling (sharding, Layer 2) to preserve decentralization while increasing throughput.
\end{frame}

\begin{frame}{Throughput Comparison}
\textbf{Layer 1 Throughput (TPS):}

\vspace{0.3cm}
\begin{tabular}{lrrr}
\toprule
\textbf{Blockchain} & \textbf{TPS} & \textbf{Block Time} & \textbf{Nodes} \\
\midrule
Bitcoin & 7 & 10 min & 15,000 \\
Ethereum & 30 & 12 sec & 7,000 \\
Litecoin & 56 & 2.5 min & 2,000 \\
Cardano & 250 & 20 sec & 3,000 \\
Polkadot & 1,000 & 6 sec & 1,000 \\
Solana & 65,000 & 0.4 sec & 2,000 \\
EOS & 4,000 & 0.5 sec & 21 \\
Avalanche & 4,500 & 1 sec & 1,300 \\
\midrule
Visa (comparison) & 24,000 & instant & centralized \\
\bottomrule
\end{tabular}

\vspace{0.4cm}
\textbf{Observations:}
\begin{itemize}
    \item Inverse correlation: higher TPS -> fewer nodes (generally)
    \item Solana achieves high TPS via vertical scaling (expensive hardware)
    \item Bitcoin/Ethereum prioritize decentralization over throughput
    \item No blockchain matches Visa TPS at Layer 1 without centralization
\end{itemize}
\end{frame}

\begin{frame}{Real-World Performance vs. Theoretical Limits}
\textbf{Advertised vs. Actual TPS:}

\vspace{0.3cm}
\begin{tabular}{lrr}
\toprule
\textbf{Blockchain} & \textbf{Theoretical TPS} & \textbf{Actual Avg. TPS} \\
\midrule
Bitcoin & 7 & 3-5 \\
Ethereum & 30 & 12-15 \\
EOS & 4,000 & 50-200 \\
Solana & 65,000 & 2,000-3,000 \\
Cardano & 250 & 10-20 \\
\bottomrule
\end{tabular}

\vspace{0.4cm}
\textbf{Why the Discrepancy?}
\begin{itemize}
    \item \textbf{Network congestion:} actual demand varies
    \item \textbf{Transaction complexity:} simple transfers vs. smart contracts
    \item \textbf{Spam filtering:} anti-DoS measures limit throughput
    \item \textbf{Economic constraints:} high gas fees discourage frivolous transactions
    \item \textbf{Benchmarking conditions:} theoretical max assumes ideal conditions
\end{itemize}

\vspace{0.3cm}
\textbf{Lesson:}
Advertised TPS often misleading. Real-world performance depends on transaction mix, network health, and economic factors.
\end{frame}

\begin{frame}{Layer 2 Scaling Solutions}
\textbf{Concept:}
\begin{itemize}
    \item Move transactions off main chain (Layer 1)
    \item Settle periodically on Layer 1
    \item Inherit Layer 1 security guarantees
\end{itemize}

\vspace{0.3cm}
\textbf{Types of Layer 2:}

\vspace{0.3cm}
\begin{enumerate}
    \item \textbf{Payment Channels (Lightning Network):}
    \begin{itemize}
        \item Open channel with on-chain transaction
        \item Unlimited off-chain transactions
        \item Close channel to settle on-chain
        \item Use case: micropayments, instant transfers
    \end{itemize}

    \item \textbf{Rollups (Optimistic, ZK):}
    \begin{itemize}
        \item Bundle hundreds of transactions into one
        \item Post compressed data to Layer 1
        \item Optimistic: assume valid, fraud proofs challenge
        \item ZK: cryptographic validity proofs
        \item Use case: DeFi, NFTs, general computation
    \end{itemize}

    \item \textbf{Sidechains (Polygon):}
    \begin{itemize}
        \item Independent blockchain with bridge to main chain
        \item Own consensus mechanism
        \item Weaker security than rollups (separate validator set)
        \item Use case: high-throughput applications
    \end{itemize}
\end{enumerate}
\end{frame}

\begin{frame}{Layer 2 Throughput Comparison}
\textbf{Layer 2 Performance:}

\vspace{0.3cm}
\begin{tabular}{lrr}
\toprule
\textbf{Layer 2} & \textbf{TPS} & \textbf{Security Model} \\
\midrule
Lightning Network (Bitcoin) & 1,000,000+ & Payment channels \\
Arbitrum (Ethereum) & 4,000 & Optimistic rollup \\
Optimism (Ethereum) & 2,000 & Optimistic rollup \\
zkSync (Ethereum) & 2,000 & ZK rollup \\
StarkNet (Ethereum) & 10,000+ & ZK rollup \\
Polygon PoS (sidechain) & 7,000 & Separate consensus \\
\midrule
Ethereum Layer 1 & 30 & Full security \\
\bottomrule
\end{tabular}

\vspace{0.4cm}
\textbf{Trade-offs:}
\begin{itemize}
    \item Rollups: inherit L1 security, moderate throughput (100-1000x)
    \item Sidechains: weaker security, higher throughput
    \item Payment channels: unlimited throughput, limited use case (payments only)
\end{itemize}

\vspace{0.3cm}
\textbf{Current Adoption:}
\begin{itemize}
    \item Ethereum Layer 2 TVL (Total Value Locked): \$40+ billion (2024)
    \item Lightning Network capacity: 5,000+ BTC
    \item Layer 2 becoming dominant for everyday transactions
\end{itemize}
\end{frame}

\begin{frame}{Sharding: Horizontal Scaling at Layer 1}
\textbf{Concept:}
\begin{itemize}
    \item Split blockchain into parallel ``shards''
    \item Each shard processes subset of transactions
    \item Aggregate throughput = shards $\times$ per-shard TPS
\end{itemize}

\vspace{0.3cm}
\textbf{Ethereum Sharding Roadmap:}
\begin{itemize}
    \item Originally: 64 shards planned
    \item Current plan (post-Merge): data availability sharding (danksharding)
    \item Rollups use shards for data, execute off-chain
    \item Target: 100,000+ TPS via rollups + sharding
    \item Timeline: 2024-2026 (EIP-4844 proto-danksharding deployed March 2024)
\end{itemize}

\vspace{0.3cm}
\textbf{Challenges:}
\begin{itemize}
    \item Cross-shard communication complexity
    \item Shard takeover attacks (low-stake shards vulnerable)
    \item State management (which shard holds which data?)
    \item Validator assignment (random, secure rotation)
\end{itemize}

\vspace{0.3cm}
\textbf{Other Sharded Chains:}
\begin{itemize}
    \item Zilliqa: 4 shards, ~2,500 TPS
    \item NEAR Protocol: dynamic sharding, ~100,000 TPS target
    \item Elrond: 4 shards, ~15,000 TPS
\end{itemize}
\end{frame}

\begin{frame}{State Growth Problem}
\textbf{Challenge:}
\begin{itemize}
    \item Blockchain size grows unbounded
    \item Ethereum state: ~100 GB (account balances, smart contract storage)
    \item Bitcoin UTXO set: ~5 GB
    \item Storage costs increase over time
\end{itemize}

\vspace{0.3cm}
\textbf{Consequences:}
\begin{itemize}
    \item Harder to run full nodes (requires SSDs, terabytes of storage)
    \item Syncing new nodes takes days/weeks
    \item Centralization pressure (only dedicated users run full nodes)
\end{itemize}

\vspace{0.3cm}
\textbf{Solutions:}

\vspace{0.3cm}
\begin{enumerate}
    \item \textbf{State Expiry (Ethereum proposal):}
    \begin{itemize}
        \item Inactive state evicted from active storage
        \item Must provide proof to re-activate
        \item Reduces live state size
    \end{itemize}

    \item \textbf{Pruning:}
    \begin{itemize}
        \item Discard old blockchain history
        \item Keep only recent blocks + current state
        \item Trade-off: cannot verify full history
    \end{itemize}

    \item \textbf{Rent (Cosmos, EOS):}
    \begin{itemize}
        \item Users pay ongoing fees to store state
        \item Incentivizes cleanup of unused data
    \end{itemize}
\end{enumerate}
\end{frame}

\begin{frame}{Nakamoto's Dilemma Formalized}
\textbf{Mathematical Model:}

\vspace{0.3cm}
Let:
\begin{itemize}
    \item $f$ = block creation rate (blocks per second)
    \item $\beta$ = block size (bytes)
    \item $\Delta$ = network propagation delay (seconds)
\end{itemize}

\vspace{0.3cm}
Security requires:
\[
f \cdot \beta \cdot \Delta < \text{constant}
\]

\vspace{0.3cm}
\textbf{Interpretation:}
\begin{itemize}
    \item Increase throughput ($f \cdot \beta$) -> increase orphan rate
    \item High orphan rate -> security degrades (wasted hash power)
    \item To maintain security: throughput $\times$ latency $<$ threshold
\end{itemize}

\vspace{0.3cm}
\textbf{Implications:}
\begin{itemize}
    \item Cannot arbitrarily increase block size or frequency
    \item Network latency sets fundamental limit
    \item Bitcoin conservatively below limit (prioritizes security)
    \item Solana pushes limit (requires fast network, powerful nodes)
\end{itemize}
\end{frame}

\begin{frame}{Breaking the Trilemma: Possible?}
\textbf{Optimistic View:}
\begin{itemize}
    \item Layer 2 solutions separate execution from settlement
    \item Rollups achieve scalability without sacrificing L1 security/decentralization
    \item Data availability sampling enables sharding without full nodes downloading all data
    \item Zero-knowledge proofs compress computation (verify in constant time)
\end{itemize}

\vspace{0.3cm}
\textbf{Pessimistic View:}
\begin{itemize}
    \item Layer 2 introduces new trust assumptions (sequencers, bridges)
    \item Increased system complexity -> more attack vectors
    \item Still bound by data availability bandwidth
    \item No escaping fundamental trade-offs, only shifting them
\end{itemize}

\vspace{0.3cm}
\textbf{Consensus View:}
\begin{itemize}
    \item Trilemma remains but can be mitigated
    \item Modular architecture (separate execution, consensus, data availability)
    \item Accept specialization: Layer 1 for security, Layer 2 for scale
    \item Future: 100,000+ TPS with acceptable decentralization/security trade-offs
    \item Not ``solving'' trilemma but ``navigating'' it intelligently
\end{itemize}
\end{frame}

\begin{frame}{Key Takeaways}
\begin{itemize}
    \item Scalability trilemma: cannot maximize decentralization, security, and scalability simultaneously
    \item Layer 1 bottlenecks: block size, block time, state growth, computational overhead
    \item Vertical scaling (bigger nodes) increases centralization
    \item Horizontal scaling (sharding, Layer 2) preserves decentralization
    \item Layer 2 solutions achieve 100-1000x throughput improvement
    \item Sharding enables parallel transaction processing
    \item Real-world TPS often far below theoretical limits
    \item State growth threatens long-term node decentralization
\end{itemize}

\vspace{0.4cm}
\textbf{Design Philosophy:}

Accept that trade-offs exist. Choose priorities explicitly: Bitcoin prioritizes decentralization and security over scalability. Solana prioritizes scalability over decentralization. Ethereum aims for balance via Layer 2.
\end{frame}

\begin{frame}{Discussion Questions}
\begin{enumerate}
    \item Why cannot a blockchain simply increase block size indefinitely to achieve scalability?

    \item How do Layer 2 solutions differ from sidechains in terms of security?

    \item What are the implications of state growth for blockchain decentralization?

    \item Can the trilemma be ``solved'' or only mitigated?

    \item How does sharding introduce new security challenges?

    \item Why do most blockchains have lower actual TPS than theoretical TPS?

    \item What role does network latency play in blockchain throughput limits?
\end{enumerate}
\end{frame}

\begin{frame}{Next Lesson Preview: L12 Lab - Block Explorer Analysis}
\textbf{Lab activities:}
\begin{itemize}
    \item Navigate Etherscan and Blockstream block explorers
    \item Analyze transaction details (inputs, outputs, fees, confirmations)
    \item Trace transaction lifecycle from mempool to confirmation
    \item Examine block structure (header, transactions, miner rewards)
    \item Investigate address activity and balances
    \item Identify transaction patterns (exchanges, mixers, smart contracts)
    \item Practice forensic blockchain analysis
\end{itemize}

\vspace{0.5cm}
\textbf{Preparation:}
\begin{itemize}
    \item Review Bitcoin transaction structure (Lesson 6)
    \item Ensure access to Etherscan.io and Blockstream.info
    \item Prepare questions about specific transactions or addresses to investigate
\end{itemize}
\end{frame}

\end{document}
