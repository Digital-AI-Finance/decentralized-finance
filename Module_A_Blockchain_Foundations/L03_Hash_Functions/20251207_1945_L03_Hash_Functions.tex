\documentclass[8pt,aspectratio=169]{beamer}
\usetheme{Madrid}
\usepackage{graphicx}
\usepackage{booktabs}
\usepackage{adjustbox}
\usepackage{multicol}
\usepackage{amsmath}
\usepackage{amssymb}

% Color definitions
\definecolor{mlblue}{RGB}{0,102,204}
\definecolor{mlpurple}{RGB}{51,51,178}
\definecolor{mllavender}{RGB}{173,173,224}
\definecolor{mllavender2}{RGB}{193,193,232}
\definecolor{mllavender3}{RGB}{204,204,235}
\definecolor{mllavender4}{RGB}{214,214,239}
\definecolor{mlorange}{RGB}{255, 127, 14}
\definecolor{mlgreen}{RGB}{44, 160, 44}
\definecolor{mlred}{RGB}{214, 39, 40}
\definecolor{mlgray}{RGB}{127, 127, 127}
\definecolor{lightgray}{RGB}{240, 240, 240}
\definecolor{midgray}{RGB}{180, 180, 180}

% Apply custom colors to Madrid theme
\setbeamercolor{palette primary}{bg=mllavender3,fg=mlpurple}
\setbeamercolor{palette secondary}{bg=mllavender2,fg=mlpurple}
\setbeamercolor{palette tertiary}{bg=mllavender,fg=white}
\setbeamercolor{palette quaternary}{bg=mlpurple,fg=white}
\setbeamercolor{structure}{fg=mlpurple}
\setbeamercolor{section in toc}{fg=mlpurple}
\setbeamercolor{subsection in toc}{fg=mlblue}
\setbeamercolor{title}{fg=mlpurple}
\setbeamercolor{frametitle}{fg=mlpurple,bg=mllavender3}
\setbeamercolor{block title}{bg=mllavender2,fg=mlpurple}
\setbeamercolor{block body}{bg=mllavender4,fg=black}

\setbeamertemplate{navigation symbols}{}
\setbeamertemplate{itemize items}[circle]
\setbeamertemplate{enumerate items}[default]
\setbeamersize{text margin left=5mm,text margin right=5mm}

\title{Lesson 3: Cryptographic Hash Functions}
\subtitle{Module A: Blockchain Foundations}
\author{BSc Blockchain \& Cryptocurrency}
\institute{University Course}
\date{2025}

\begin{document}

\begin{frame}[plain]
\titlepage
\end{frame}

\begin{frame}[t]{Learning Objectives}
By the end of this lesson, you will be able to:

\begin{enumerate}
\item Define cryptographic hash functions and their key properties
\item Explain the avalanche effect with concrete examples
\item Understand SHA-256 algorithm and its role in Bitcoin
\item Analyze collision resistance and the birthday paradox
\item Construct and verify Merkle trees step-by-step
\item Identify real-world applications of hash functions beyond blockchain
\end{enumerate}

\vspace{1em}
\textbf{Prerequisites:} L02 - DLT Concepts (Merkle trees introduction)
\end{frame}

\section{Hash Function Basics}

\begin{frame}[t]{What is a Hash Function?}
\begin{block}{Hash Function Definition}
A \textbf{hash function} is a mathematical function that takes an input (message) of arbitrary length and produces a fixed-size output (hash/digest).
\end{block}

\vspace{0.5em}
\textbf{Mathematical Notation:}
$$H: \{0,1\}^* \rightarrow \{0,1\}^n$$

Where $\{0,1\}^*$ is any binary string, and $\{0,1\}^n$ is a fixed $n$-bit output.

\vspace{0.5em}
\textbf{Example (SHA-256):}
\begin{itemize}
\item Input: ``Hello World'' (any length)
\item Output: \texttt{a591a6d40bf420404...} (always 256 bits = 64 hex characters)
\end{itemize}

\vspace{0.5em}
\textit{Hash functions are ``digital fingerprints'' - unique identifiers for data}
\end{frame}

\begin{frame}[t]{Non-Cryptographic vs. Cryptographic Hashes}
\begin{columns}[T]
\column{0.48\textwidth}
\textbf{Non-Cryptographic Hashes}
\begin{itemize}
\item Fast computation
\item Designed for hash tables, checksums
\item NOT resistant to adversarial attacks
\item Collision attacks are easy
\end{itemize}

\vspace{0.5em}
\textbf{Examples:}
\begin{itemize}
\item CRC32 (cyclic redundancy check)
\item MD5 (broken, not cryptographic anymore)
\item MurmurHash (fast, non-crypto)
\end{itemize}

\vspace{0.5em}
\textit{Use Case:} Hash tables in programming languages

\column{0.48\textwidth}
\textbf{Cryptographic Hashes}
\begin{itemize}
\item Slower, but secure
\item Collision resistant
\item Preimage resistant
\item Unpredictable (pseudorandom)
\end{itemize}

\vspace{0.5em}
\textbf{Examples:}
\begin{itemize}
\item SHA-256 (Bitcoin, SSL/TLS)
\item SHA-3 (Keccak, used in Ethereum)
\item BLAKE2 (fast, modern)
\end{itemize}

\vspace{0.5em}
\textit{Use Case:} Digital signatures, blockchain, passwords
\end{columns}

\vspace{0.5em}
\textbf{Key Difference:} Cryptographic hashes must withstand deliberate attacks
\end{frame}

\section{Core Properties}

\begin{frame}[t]{Property 1: Deterministic}
\begin{block}{Deterministic Property}
The same input always produces the same output. No randomness involved.
\end{block}

\vspace{0.5em}
\textbf{Example:}
\begin{itemize}
\item $H(\text{``blockchain''})$ = \texttt{ef7797e13d3a75526946a3bcf00daec9fc9c9c4d51ddc7cc5df888f74dd434d1}
\item Computing this hash 1,000 times yields the same result every time
\end{itemize}

\vspace{0.5em}
\textbf{Why This Matters:}
\begin{itemize}
\item Enables verification: Anyone can recompute the hash
\item Makes auditing possible: Deterministic proofs
\item Foundation for digital signatures
\end{itemize}

\vspace{0.5em}
\textbf{Contrast with Random Functions:}
\begin{itemize}
\item Random: $f(x)$ might return different values each time
\item Hash: $H(x)$ is a pure function (functional programming)
\end{itemize}
\end{frame}

\begin{frame}[t]{Property 2: Fixed-Length Output}
\begin{block}{Fixed-Length Property}
Regardless of input size, the hash output is always the same fixed length.
\end{block}

\vspace{0.5em}
\textbf{Examples (SHA-256):}
\begin{itemize}
\item $H(\text{``a''})$ = 256 bits (64 hex characters)
\item $H(\text{entire Bitcoin whitepaper})$ = 256 bits (same length)
\item $H(\text{1 GB video file})$ = 256 bits (still 64 hex chars)
\end{itemize}

\vspace{0.5em}
\textbf{Common Hash Sizes:}
\begin{table}[h]
\scriptsize
\begin{tabular}{lll}
\toprule
\textbf{Algorithm} & \textbf{Output Size (bits)} & \textbf{Hex Characters} \\
\midrule
MD5 (broken) & 128 & 32 \\
SHA-1 (deprecated) & 160 & 40 \\
SHA-256 (Bitcoin) & 256 & 64 \\
SHA-512 & 512 & 128 \\
\bottomrule
\end{tabular}
\end{table}

\vspace{0.5em}
\textbf{Implication:} Infinite inputs map to finite outputs $\Rightarrow$ collisions must exist (pigeonhole principle)
\end{frame}

\begin{frame}[t]{Property 3: Avalanche Effect (Sensitivity)}
\begin{block}{Avalanche Effect}
A tiny change in input (even 1 bit) causes a massive, unpredictable change in the output. Approximately 50\% of output bits flip.
\end{block}

\vspace{0.5em}
\textbf{Example (SHA-256):}
\begin{itemize}
\item Input: ``blockchain''
  \begin{itemize}
  \item Hash: \texttt{ef7797e13d3a75526946a3bcf00daec9fc9c9c4d51ddc7cc5df888f74dd434d1}
  \end{itemize}
\item Input: ``Blockchain'' (capital B)
  \begin{itemize}
  \item Hash: \texttt{625da44e4eaf58d61cf048d168aa6f5e492dea166d8bb54ec06c30de07db57e1}
  \end{itemize}
\end{itemize}

\vspace{0.5em}
\textbf{Analysis:}
\begin{itemize}
\item Only 1 bit changed in input (ASCII: b = \texttt{01100010}, B = \texttt{01000010})
\item Output is completely different (no pattern recognizable)
\item Approximately 128 out of 256 bits flipped (50\%)
\end{itemize}

\vspace{0.5em}
\textit{Visualizing this effect will be the focus of Lab 4}
\end{frame}

\begin{frame}[t]{Property 4: Preimage Resistance (One-Way)}
\begin{block}{Preimage Resistance}
Given a hash output $h$, it is computationally infeasible to find any input $m$ such that $H(m) = h$.
\end{block}

\vspace{0.5em}
\textbf{Analogy:} Easy to scramble an egg, impossible to unscramble it

\vspace{0.5em}
\textbf{Mathematical Formulation:}
\begin{itemize}
\item Computing $h = H(m)$ is fast ($\approx$ microseconds)
\item Finding $m$ given $h$ requires trying all $2^{256}$ possibilities (for SHA-256)
\item At 1 trillion hashes/second, this would take $10^{58}$ years
\end{itemize}

\vspace{0.5em}
\textbf{Practical Implications:}
\begin{itemize}
\item Password storage: Store $H(\text{password})$, not password itself
\item Blockchain integrity: Cannot reverse-engineer block data from hash
\item Commitment schemes: Hash your choice before revealing
\end{itemize}

\vspace{0.5em}
\textit{Exception: Rainbow tables for weak passwords (mitigated by salting)}
\end{frame}

\begin{frame}[t]{Property 5: Second Preimage Resistance}
\begin{block}{Second Preimage Resistance}
Given input $m_1$ and its hash $h = H(m_1)$, it is computationally infeasible to find a different input $m_2 \neq m_1$ such that $H(m_2) = h$.
\end{block}

\vspace{0.5em}
\textbf{Scenario:}
\begin{itemize}
\item You sign a contract: ``Pay Alice \$1,000''
\item Hash: $H(\text{contract}) = h$
\item Attacker tries to find alternate message: ``Pay Bob \$1,000,000'' with same hash $h$
\item Second preimage resistance prevents this attack
\end{itemize}

\vspace{0.5em}
\textbf{Difference from Preimage Resistance:}
\begin{itemize}
\item \textbf{Preimage}: Given $h$, find any $m$ where $H(m) = h$
\item \textbf{Second Preimage}: Given $m_1$ and $h = H(m_1)$, find different $m_2$ where $H(m_2) = h$
\end{itemize}

\vspace{0.5em}
\textit{Second preimage attacks are easier than preimage attacks, but still infeasible for strong hashes}
\end{frame}

\begin{frame}[t]{Property 6: Collision Resistance}
\begin{block}{Collision Resistance}
It is computationally infeasible to find any two different inputs $m_1 \neq m_2$ such that $H(m_1) = H(m_2)$.
\end{block}

\vspace{0.5em}
\textbf{Theoretical Guarantee:}
\begin{itemize}
\item Pigeonhole principle: Collisions MUST exist (infinite inputs, finite outputs)
\item But finding them should be practically impossible
\end{itemize}

\vspace{0.5em}
\textbf{Birthday Paradox Attack:}
\begin{itemize}
\item For $n$-bit hash, finding collision requires $\approx 2^{n/2}$ attempts (not $2^n$)
\item SHA-256: $2^{128}$ operations needed ($\approx 10^{38}$ hashes)
\item Still infeasible with current technology
\end{itemize}

\vspace{0.5em}
\textbf{Broken Examples:}
\begin{itemize}
\item MD5 collisions found in 2004 (now insecure)
\item SHA-1 collisions demonstrated in 2017 (deprecated for security)
\end{itemize}
\end{frame}

\begin{frame}[t]{The Birthday Paradox}
\textbf{Classic Probability Problem:}

\vspace{0.5em}
\textit{How many people needed in a room for 50\% probability that two share a birthday?}

\vspace{0.5em}
Answer: Only 23 people (counterintuitive!)

\vspace{0.5em}
\textbf{Why It Matters for Hashing:}
\begin{itemize}
\item With 365 possible birthdays (like hash outputs)
\item Finding a collision takes $\sqrt{365} \approx 23$ samples
\item Generalized: For $N$ possible outputs, collision in $\approx \sqrt{N}$ attempts
\end{itemize}

\vspace{0.5em}
\textbf{Application to SHA-256:}
\begin{itemize}
\item Total possible hashes: $N = 2^{256}$
\item Collision search: $\sqrt{2^{256}} = 2^{128}$ attempts
\item At 1 trillion hashes/second: $10^{19}$ years
\end{itemize}

\vspace{0.5em}
\textit{This is why we need large hash outputs: Collision resistance scales as $2^{n/2}$}
\end{frame}

\section{SHA-256 Algorithm}

\begin{frame}[t]{SHA-256 Overview}
\begin{block}{SHA-256}
\textbf{Secure Hash Algorithm 256-bit} is a cryptographic hash function designed by the NSA, published by NIST in 2001 as part of the SHA-2 family.
\end{block}

\vspace{0.5em}
\textbf{Specifications:}
\begin{itemize}
\item Output: 256 bits (32 bytes, 64 hex characters)
\item Block size: 512 bits (processes data in 512-bit chunks)
\item Internal state: Eight 32-bit words (256 bits total)
\item Rounds: 64 compression iterations per block
\end{itemize}

\vspace{0.5em}
\textbf{Uses in Bitcoin:}
\begin{itemize}
\item Block hashing: Double SHA-256 ($\text{SHA256}(\text{SHA256}(\text{header}))$)
\item Transaction IDs: SHA-256 hash of transaction data
\item Address generation: SHA-256 + RIPEMD-160
\item Merkle tree construction
\end{itemize}
\end{frame}

\begin{frame}[t]{SHA-256 High-Level Process}
\textbf{Step-by-Step:}

\begin{enumerate}
\item \textbf{Padding}: Append bits to make length $\equiv 448 \pmod{512}$
  \begin{itemize}
  \item Add single 1 bit, then zeros, then 64-bit length field
  \end{itemize}

\item \textbf{Parsing}: Break padded message into 512-bit blocks

\item \textbf{Initialize}: Set eight 32-bit hash values (constants from square roots of first 8 primes)

\item \textbf{Compression}: For each block:
  \begin{itemize}
  \item Expand 512 bits to 2,048 bits (message schedule)
  \item 64 rounds of bitwise operations (AND, XOR, rotate, add)
  \item Update internal state
  \end{itemize}

\item \textbf{Output}: Concatenate final 8 words into 256-bit hash
\end{enumerate}

\vspace{0.5em}
\textit{Details involve modular arithmetic and bitwise logic (beyond scope, but implementations widely available)}
\end{frame}

\begin{frame}[t]{SHA-256 Example Calculation}
\textbf{Input:} ``abc'' (3 bytes)

\vspace{0.5em}
\textbf{Step 1 - Padding:}
\begin{itemize}
\item Binary: \texttt{01100001 01100010 01100011} (24 bits)
\item Add \texttt{1} bit: \texttt{01100001 01100010 01100011 1...}
\item Pad with zeros until length $\equiv 448 \pmod{512}$
\item Append 64-bit length: \texttt{...0000011000} (24 in binary)
\end{itemize}

\vspace{0.5em}
\textbf{Step 2 - Initialize Hash Values (first 8 primes):}
\begin{itemize}
\item $H_0 = \texttt{6a09e667}$, $H_1 = \texttt{bb67ae85}$, ..., $H_7 = \texttt{5be0cd19}$
\end{itemize}

\vspace{0.5em}
\textbf{Step 3 - Compression (64 rounds):}
\begin{itemize}
\item Complex bitwise operations (Ch, Maj, $\Sigma_0$, $\Sigma_1$, etc.)
\end{itemize}

\vspace{0.5em}
\textbf{Final Output:}
$$H(\text{``abc''}) = \texttt{ba7816bf8f01cfea414140de5dae2223b00361a396177a9cb410ff61f20015ad}$$
\end{frame}

\begin{frame}[t]{Double SHA-256 in Bitcoin}
\begin{block}{Why Double Hashing?}
Bitcoin uses $\text{SHA256}(\text{SHA256}(x))$ instead of single SHA-256 to guard against length-extension attacks (theoretical vulnerability in Merkle-Damgård construction).
\end{block}

\vspace{0.5em}
\textbf{Process:}
\begin{enumerate}
\item Compute $h_1 = \text{SHA256}(\text{data})$
\item Compute $h_2 = \text{SHA256}(h_1)$
\item Use $h_2$ as final hash
\end{enumerate}

\vspace{0.5em}
\textbf{Example - Bitcoin Block Hash:}
\begin{itemize}
\item Input: 80-byte block header
\item First hash: $h_1 = \text{SHA256}(\text{header})$
\item Second hash: $h_2 = \text{SHA256}(h_1)$
\item Result: $h_2$ must be below difficulty target to be valid
\end{itemize}

\vspace{0.5em}
\textit{Performance: Modern hardware computes millions of double-SHA256 per second}
\end{frame}

\section{Merkle Trees}

\begin{frame}[t]{Merkle Tree Construction}
\textbf{Goal:} Efficiently verify transaction inclusion without downloading all data

\vspace{0.5em}
\textbf{Construction Algorithm:}
\begin{enumerate}
\item Start with $n$ transactions: $Tx_1, Tx_2, ..., Tx_n$
\item Hash each transaction: $H_1 = H(Tx_1), H_2 = H(Tx_2), ..., H_n = H(Tx_n)$
\item Pair and hash: $H_{12} = H(H_1 || H_2)$, $H_{34} = H(H_3 || H_4)$, ...
\item If odd number, duplicate last hash: $H_{nn} = H(H_n || H_n)$
\item Repeat until single root hash (Merkle Root)
\end{enumerate}

\vspace{0.5em}
\textbf{Example (4 transactions):}
\begin{itemize}
\item Level 0: $Tx_1, Tx_2, Tx_3, Tx_4$
\item Level 1: $H_1, H_2, H_3, H_4$
\item Level 2: $H_{12} = H(H_1 || H_2)$, $H_{34} = H(H_3 || H_4)$
\item Level 3 (Root): $R = H(H_{12} || H_{34})$
\end{itemize}
\end{frame}

\begin{frame}[t]{Merkle Proof Verification}
\textbf{Scenario:} Light client wants to verify $Tx_3$ is in block (4 transactions total)

\vspace{0.5em}
\textbf{Verifier Has:}
\begin{itemize}
\item Block header with Merkle Root $R$
\item Transaction $Tx_3$
\end{itemize}

\vspace{0.5em}
\textbf{Prover Sends (Merkle Proof):}
\begin{itemize}
\item $H_4$ (sibling of $H_3$)
\item $H_{12}$ (sibling of $H_{34}$)
\end{itemize}

\vspace{0.5em}
\textbf{Verification Steps:}
\begin{enumerate}
\item Compute $H_3 = H(Tx_3)$
\item Compute $H_{34} = H(H_3 || H_4)$ (using provided $H_4$)
\item Compute $R' = H(H_{12} || H_{34})$ (using provided $H_{12}$)
\item If $R' = R$, then $Tx_3$ is proven to be in the block
\end{enumerate}

\vspace{0.5em}
\textbf{Efficiency:} Only 2 hashes sent instead of 3 other transactions
\end{frame}

\begin{frame}[t]{Merkle Tree Efficiency Analysis}
\textbf{Space Complexity:}
\begin{itemize}
\item For $n$ transactions, tree has $\approx 2n$ nodes
\item Merkle proof requires $\log_2(n)$ hashes
\end{itemize}

\vspace{0.5em}
\textbf{Proof Size Comparison:}
\begin{table}[h]
\scriptsize
\begin{tabular}{lll}
\toprule
\textbf{Transactions in Block} & \textbf{Full Data} & \textbf{Merkle Proof} \\
\midrule
10 & $\approx$ 2.5 KB & 4 hashes (128 bytes) \\
100 & $\approx$ 25 KB & 7 hashes (224 bytes) \\
1,000 & $\approx$ 250 KB & 10 hashes (320 bytes) \\
10,000 & $\approx$ 2.5 MB & 14 hashes (448 bytes) \\
\bottomrule
\end{tabular}
\end{table}

\vspace{0.5em}
\textbf{Real-World Impact:}
\begin{itemize}
\item Bitcoin block: $\approx$ 2,000 transactions $\Rightarrow$ 11-hash proof ($\approx$ 352 bytes)
\item Full block: $\approx$ 1 MB
\item Savings: $\frac{352}{1,000,000} = 0.035\%$ of data needed
\end{itemize}
\end{frame}

\begin{frame}[t]{Merkle Tree Variants}
\begin{columns}[T]
\column{0.48\textwidth}
\textbf{Binary Merkle Tree}
\begin{itemize}
\item Each node has 2 children
\item Used in Bitcoin
\item Proof size: $O(\log_2 n)$
\item Simple to implement
\end{itemize}

\vspace{0.5em}
\textbf{Merkle Patricia Trie (Ethereum)}
\begin{itemize}
\item Combines Merkle tree + Patricia trie
\item Supports key-value storage
\item Enables state root (all accounts)
\item More complex, but more powerful
\end{itemize}

\column{0.48\textwidth}
\textbf{Verkle Trees (Future Ethereum)}
\begin{itemize}
\item Uses polynomial commitments
\item Constant-size proofs (regardless of tree size)
\item Enables stateless clients
\item Based on vector commitments
\end{itemize}

\vspace{0.5em}
\textbf{Sparse Merkle Trees}
\begin{itemize}
\item Fixed depth (e.g., 256 levels)
\item Most branches empty (pruned)
\item Supports non-membership proofs
\item Used in some zero-knowledge systems
\end{itemize}
\end{columns}
\end{frame}

\section{Applications Beyond Blockchain}

\begin{frame}[t]{Password Storage}
\textbf{Problem:} Storing passwords in plaintext is insecure (database breach exposes all passwords)

\vspace{0.5em}
\textbf{Solution:} Store $H(\text{password})$ instead

\begin{enumerate}
\item User creates account with password ``mypassword123''
\item System computes $h = H(\text{``mypassword123''})$
\item Database stores only $h$, not the password
\item Login: User enters password, system hashes it, compares with stored $h$
\item Attacker with database cannot reverse $h$ to get password (preimage resistance)
\end{enumerate}

\vspace{0.5em}
\textbf{Enhancements:}
\begin{itemize}
\item \textbf{Salting}: Add random data before hashing to prevent rainbow tables
  \begin{itemize}
  \item Store $h = H(\text{password} || \text{salt})$ and salt
  \end{itemize}
\item \textbf{Key Stretching}: Use slow hash functions (bcrypt, Argon2) to resist brute-force
\end{itemize}
\end{frame}

\begin{frame}[t]{File Integrity Verification}
\textbf{Use Case:} Ensure downloaded file hasn't been tampered with

\vspace{0.5em}
\textbf{Process:}
\begin{enumerate}
\item Software developer publishes file hash on official website
  \begin{itemize}
  \item Example: Ubuntu ISO hash on ubuntu.com
  \end{itemize}
\item User downloads file from mirror site
\item User computes hash of downloaded file locally
\item User compares computed hash with official hash
\item If hashes match, file is authentic and unmodified
\end{enumerate}

\vspace{0.5em}
\textbf{Real-World Example:}
\begin{itemize}
\item Download Ubuntu 24.04 LTS ISO (4 GB)
\item Official SHA-256: \texttt{c2e6f4dc37ac944e2f8b21de00e9610c79c61d11...}
\item Compute: \texttt{sha256sum ubuntu-24.04-desktop-amd64.iso}
\item Match confirms integrity
\end{itemize}

\vspace{0.5em}
\textit{Critical for security: Prevents man-in-the-middle attacks during downloads}
\end{frame}

\begin{frame}[t]{Git Version Control}
\textbf{How Git Uses Hash Functions:}

\vspace{0.5em}
\begin{itemize}
\item Every commit has a SHA-1 hash (Git transitioning to SHA-256)
\item Hash is computed from:
  \begin{itemize}
  \item Commit message
  \item Author/committer metadata
  \item Tree object (directory structure)
  \item Parent commit hash(es)
  \end{itemize}
\item Forms a Merkle tree of commits (immutable history)
\end{itemize}

\vspace{0.5em}
\textbf{Example:}
\begin{itemize}
\item Commit: \texttt{a3f5c79...} points to parent \texttt{b2e4d31...}
\item Changing history requires recomputing all subsequent hashes
\item Makes history tampering detectable
\end{itemize}

\vspace{0.5em}
\textbf{Similarity to Blockchain:}
\begin{itemize}
\item Git is a content-addressed storage system
\item Hashes link commits, just like blocks in blockchain
\item Both rely on collision resistance for integrity
\end{itemize}
\end{frame}

\begin{frame}[t]{Digital Signatures \& SSL/TLS}
\textbf{Digital Signatures (Overview - more in L05):}
\begin{itemize}
\item Hash the message: $h = H(m)$
\item Sign the hash with private key: $\sigma = \text{Sign}(h, sk)$
\item Verify signature: $\text{Verify}(\sigma, h, pk)$
\item Signing hash (32 bytes) is faster than signing entire message (MB+)
\end{itemize}

\vspace{0.5em}
\textbf{SSL/TLS (HTTPS):}
\begin{itemize}
\item Certificate authorities (CAs) sign website certificates
\item CA computes $h = H(\text{certificate})$
\item CA signs $h$ with private key
\item Your browser verifies signature using CA's public key
\item Ensures you're connected to legitimate website, not impostor
\end{itemize}

\vspace{0.5em}
\textbf{Hash Functions Used:}
\begin{itemize}
\item Modern TLS 1.3: SHA-256, SHA-384
\item Legacy systems: SHA-1 (deprecated due to collision attacks)
\end{itemize}
\end{frame}

\begin{frame}[t]{Proof-of-Work (Bitcoin Mining)}
\textbf{Mining Process:}
\begin{enumerate}
\item Collect transactions into candidate block
\item Construct block header (80 bytes)
\item Compute: $h = \text{SHA256}(\text{SHA256}(\text{header}))$
\item Check if $h < \text{target}$ (difficulty requirement)
\item If no: increment nonce (4-byte counter), repeat step 3
\item If yes: broadcast valid block, receive reward
\end{enumerate}

\vspace{0.5em}
\textbf{Example Difficulty (Block 800,000):}
\begin{itemize}
\item Target: \texttt{0000000000000000000...} (19 leading zeros)
\item Probability of success per hash: $\frac{1}{2^{76}} \approx 10^{-23}$
\item Network hash rate: $\approx$ 400 EH/s (exahashes/second)
\item Average time to find block: 10 minutes (by design)
\end{itemize}

\vspace{0.5em}
\textbf{Why Hashes Enable PoW:}
\begin{itemize}
\item Unpredictable output (no shortcut, must try all nonces)
\item Fast verification (anyone can check $h < \text{target}$ instantly)
\end{itemize}
\end{frame}

\section{Summary}

\begin{frame}[t]{Key Takeaways}
\textbf{What You Should Remember:}

\begin{enumerate}
\item \textbf{Hash Functions}: Transform arbitrary input to fixed-size output (digital fingerprints)
\item \textbf{Core Properties}: Deterministic, fixed-length, avalanche effect, preimage resistance, collision resistance
\item \textbf{SHA-256}: 256-bit output, used in Bitcoin (double hashing), computationally infeasible to break
\item \textbf{Collision Resistance}: Birthday paradox reduces attack from $2^n$ to $2^{n/2}$ operations
\item \textbf{Merkle Trees}: Binary hash trees enable $O(\log n)$ proofs for transaction inclusion
\item \textbf{Applications}: Passwords, file integrity, Git, digital signatures, Proof-of-Work
\end{enumerate}

\vspace{0.5em}
\begin{block}{Critical Insight}
Cryptographic hash functions are the foundation of blockchain security. Without collision resistance and preimage resistance, the entire system collapses.
\end{block}
\end{frame}

\begin{frame}[t]{Discussion Questions}
\textbf{Consider and discuss:}

\begin{enumerate}
\item \textbf{Quantum Computing Threat}: Will quantum computers break SHA-256?
  \begin{itemize}
  \item Research: Grover's algorithm reduces preimage attack to $2^{128}$ operations
  \end{itemize}

\item \textbf{Hash Function Lifespan}: When should we migrate to SHA-3 or BLAKE3?
  \begin{itemize}
  \item Consider: Coordination challenge in decentralized networks
  \end{itemize}

\item \textbf{Environmental Impact}: Can we reduce PoW energy consumption without sacrificing security?
  \begin{itemize}
  \item Explore: Alternative consensus mechanisms (PoS, PoSpace)
  \end{itemize}
\end{enumerate}
\end{frame}

\begin{frame}[t]{References \& Resources}
\begin{columns}[T]
\column{0.48\textwidth}
\textbf{Standards \& Specs}
\begin{itemize}
\item NIST FIPS 180-4 (SHA-2 specification)
\item RFC 6234 (SHA and HMAC-SHA)
\item Keccak Team (SHA-3 documentation)
\end{itemize}

\vspace{0.5em}
\textbf{Academic Papers}
\begin{itemize}
\item Merkle (1988): \textit{Digital Signature Based on Hash Functions}
\item Wang et al. (2005): \textit{Finding Collisions in SHA-1}
\end{itemize}

\column{0.48\textwidth}
\textbf{Tools}
\begin{itemize}
\item \texttt{sha256sum} (Linux command-line)
\item Online SHA-256 calculator
\item Python \texttt{hashlib} library
\end{itemize}

\vspace{0.5em}
\textbf{Learning Resources}
\begin{itemize}
\item Computerphile: ``Hashing Algorithms''
\item Khan Academy: Cryptography course
\item 3Blue1Brown: ``But what is a hash function?''
\end{itemize}
\end{columns}
\end{frame}

\begin{frame}[t]{Next Lesson Preview}
\textbf{L04: Lab - Hash Experiments}

\vspace{0.5em}
Hands-on exercises:
\begin{itemize}
\item Generate SHA-256 hashes using Python \texttt{hashlib}
\item Visualize the avalanche effect (1-bit input change)
\item Build a Merkle tree from scratch
\item Verify Merkle proofs manually
\item Experiment with collision search (limited scale)
\item Analyze hash distribution properties
\end{itemize}

\vspace{0.5em}
\textbf{Required Setup:} Python 3.8+, Jupyter Notebook or Python IDE

\vspace{0.5em}
\textbf{Deliverables:} Lab report with code snippets and visualizations
\end{frame}

\begin{frame}[plain]
\vspace{3cm}
\begin{center}
{\Large Thank you}\\[2cm]
{\normalsize Questions?}\\[1cm]
{\small See you in Lab 4: Hash Experiments}
\end{center}
\end{frame}

\end{document}
