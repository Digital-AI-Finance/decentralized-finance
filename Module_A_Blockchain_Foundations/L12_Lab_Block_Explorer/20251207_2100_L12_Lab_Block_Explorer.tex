\documentclass[8pt,aspectratio=169]{beamer}
\usetheme{Madrid}
\usepackage[utf8]{inputenc}
\usepackage{graphicx}
\usepackage{booktabs}
\usepackage{hyperref}

\title{Lab Session: Block Explorer Analysis}
\subtitle{BSc Blockchain, Crypto Economy \& NFTs}
\author{Course Instructor}
\date{Module A: Blockchain Foundations}

\begin{document}

\begin{frame}
\titlepage
\end{frame}

\begin{frame}{Learning Objectives}
By the end of this lab session, you will be able to:
\begin{itemize}
    \item Navigate Etherscan and Blockstream block explorers effectively
    \item Analyze transaction details: inputs, outputs, fees, confirmations
    \item Trace transaction lifecycle from broadcast to confirmation
    \item Examine block structure and miner information
    \item Investigate address activity, balances, and transaction history
    \item Identify transaction patterns and entity types
    \item Conduct basic blockchain forensic analysis
\end{itemize}
\end{frame}

\begin{frame}{Lab Overview}
\textbf{Structure:}
\begin{enumerate}
    \item Introduction to block explorers (10 minutes)
    \item Exercise 1: Analyzing Bitcoin transactions (20 minutes)
    \item Exercise 2: Examining Ethereum transactions (20 minutes)
    \item Exercise 3: Block analysis and mining (15 minutes)
    \item Exercise 4: Address investigation (15 minutes)
    \item Exercise 5: Forensic case study (20 minutes)
    \item Wrap-up and deliverables (10 minutes)
\end{enumerate}

\vspace{0.3cm}
\textbf{Total Duration:} 110 minutes

\vspace{0.3cm}
\textbf{Prerequisites:}
\begin{itemize}
    \item Understanding of transaction structure (Lesson 6)
    \item Basic knowledge of Bitcoin and Ethereum
    \item Web browser with internet access
\end{itemize}
\end{frame}

\begin{frame}{What is a Block Explorer?}
\textbf{Definition:}
\begin{itemize}
    \item Web-based interface to query blockchain data
    \item Provides human-readable view of blocks, transactions, addresses
    \item Operates by indexing blockchain data (runs full node + database)
    \item Free and publicly accessible
\end{itemize}

\vspace{0.3cm}
\textbf{Major Block Explorers:}

\vspace{0.3cm}
\textbf{Bitcoin:}
\begin{itemize}
    \item Blockstream.info (used by Blockstream, privacy-focused)
    \item Blockchain.com (oldest, most popular)
    \item Mempool.space (real-time mempool visualization)
\end{itemize}

\vspace{0.3cm}
\textbf{Ethereum:}
\begin{itemize}
    \item Etherscan.io (most comprehensive, contract verification)
    \item Blockscout.com (open-source alternative)
    \item Ethplorer.io (token-focused)
\end{itemize}

\vspace{0.3cm}
\textbf{Key Features:}
\begin{itemize}
    \item Search by transaction hash, block number, address
    \item View transaction status and confirmations
    \item Examine smart contract code and interactions
    \item Track token transfers (ERC-20, ERC-721)
\end{itemize}
\end{frame}

\begin{frame}{Exercise 1: Analyzing a Bitcoin Transaction}
\textbf{Objective:} Understand Bitcoin UTXO model through real transaction

\vspace{0.3cm}
\textbf{Instructions:}

\vspace{0.3cm}
\begin{enumerate}
    \item Visit: \url{https://blockstream.info}
    \item Search for transaction hash:\\
    \texttt{f4184fc596403b9d638783cf57adfe4c75c605f6356fbc91338530e9831e9e16}\\
    (This is the first-ever Bitcoin transaction between Satoshi and Hal Finney, 2009)

    \item Examine transaction details and answer:
    \begin{itemize}
        \item How many inputs and outputs?
        \item What is the total amount transferred?
        \item What is the transaction fee?
        \item Which block included this transaction?
        \item How many confirmations does it have now?
    \end{itemize}

    \item Click on input address:
    \begin{itemize}
        \item Observe sender's previous transactions
        \item How was the UTXO created?
    \end{itemize}

    \item Click on output address:
    \begin{itemize}
        \item Was this output spent or unspent?
        \item If spent, in which transaction?
    \end{itemize}
\end{enumerate}
\end{frame}

\begin{frame}{Exercise 1: Key Observations}
\textbf{Transaction Structure:}
\begin{itemize}
    \item \textbf{Inputs:} Reference to previous transaction output (txid + index)
    \item \textbf{Outputs:} New UTXOs with amounts and recipient addresses
    \item \textbf{Fee:} Difference between input sum and output sum
\end{itemize}

\vspace{0.3cm}
\textbf{Understanding Confirmations:}
\begin{itemize}
    \item Confirmations = number of blocks built on top of this transaction's block
    \item More confirmations = higher confidence transaction is final
    \item Historic transactions: 800,000+ confirmations
    \item Recent transactions: 1-10 confirmations
\end{itemize}

\vspace{0.3cm}
\textbf{UTXO Tracing:}
\begin{itemize}
    \item Follow the chain of transactions backward (where did funds come from?)
    \item Follow forward (where did funds go?)
    \item Useful for: auditing, forensics, privacy analysis
\end{itemize}

\vspace{0.3cm}
\textbf{Questions to Consider:}
\begin{itemize}
    \item Why do some transactions have many inputs or outputs?
    \item How can you identify change addresses?
    \item What does a very high fee indicate?
\end{itemize}
\end{frame}

\begin{frame}{Exercise 2: Analyzing an Ethereum Transaction}
\textbf{Objective:} Understand Ethereum account model and gas fees

\vspace{0.3cm}
\textbf{Instructions:}

\vspace{0.3cm}
\begin{enumerate}
    \item Visit: \url{https://etherscan.io}
    \item Search for a recent transaction (click ``Latest Transactions'' on homepage)
    \item Select a transaction and examine:
    \begin{itemize}
        \item Transaction hash
        \item Status (Success / Failed)
        \item Block number
        \item Timestamp
        \item From address (sender)
        \item To address (recipient or contract)
        \item Value transferred (ETH amount)
        \item Gas used vs. gas limit
        \item Gas price and total transaction fee
    \end{itemize}

    \item If transaction involves smart contract:
    \begin{itemize}
        \item Click ``Logs'' tab to see emitted events
        \item Identify function called (e.g., ``transfer'' for ERC-20)
        \item Observe internal transactions (contract calls)
    \end{itemize}

    \item Click on sender address:
    \begin{itemize}
        \item View ETH balance
        \item Examine transaction history
        \item Check token holdings (ERC-20, NFTs)
    \end{itemize}
\end{enumerate}
\end{frame}

\begin{frame}{Exercise 2: Gas Mechanics}
\textbf{Understanding Gas:}

\vspace{0.3cm}
\begin{itemize}
    \item \textbf{Gas Limit:} Maximum gas user willing to pay
    \item \textbf{Gas Used:} Actual gas consumed by transaction
    \item \textbf{Gas Price:} Price per unit of gas (in gwei, 1 gwei = $10^{-9}$ ETH)
    \item \textbf{Transaction Fee:} Gas Used $\times$ Gas Price
\end{itemize}

\vspace{0.3cm}
\textbf{Post-EIP-1559 (August 2021):}
\begin{itemize}
    \item \textbf{Base Fee:} Algorithmically determined, burned
    \item \textbf{Priority Fee (Tip):} Paid to miner/validator
    \item \textbf{Max Fee:} Maximum total fee user willing to pay
\end{itemize}

\vspace{0.3cm}
\textbf{Example Calculation:}
\begin{itemize}
    \item Gas Used: 21,000 (simple ETH transfer)
    \item Base Fee: 30 gwei
    \item Priority Fee: 2 gwei
    \item Total Fee: $21,000 \times (30 + 2) = 672,000$ gwei = 0.000672 ETH
    \item At ETH = \$2,000: fee = \$1.34
\end{itemize}

\vspace{0.3cm}
\textbf{Failed Transactions:}
\begin{itemize}
    \item Gas still consumed (computation executed)
    \item State changes reverted
    \item Common causes: out of gas, failed assertion, contract error
\end{itemize}
\end{frame}

\begin{frame}{Exercise 3: Block Analysis}
\textbf{Objective:} Understand block structure and mining/validation

\vspace{0.3cm}
\textbf{Bitcoin Block Analysis:}

\vspace{0.3cm}
\begin{enumerate}
    \item On Blockstream.info, search for block 800,000 (milestone block)
    \item Examine block header:
    \begin{itemize}
        \item Block hash
        \item Previous block hash (forms chain)
        \item Merkle root (commitment to all transactions)
        \item Timestamp
        \item Difficulty and nonce (proof-of-work)
    \end{itemize}
    \item Identify coinbase transaction (first transaction):
    \begin{itemize}
        \item Block reward amount (6.25 BTC as of 2024)
        \item Transaction fees collected
        \item Miner address (who mined this block?)
    \end{itemize}
    \item Count transactions in block
    \item Calculate average transaction fee
\end{enumerate}

\vspace{0.3cm}
\textbf{Ethereum Block Analysis:}

\vspace{0.3cm}
\begin{enumerate}
    \item On Etherscan.io, view latest block
    \item Examine block details:
    \begin{itemize}
        \item Validator (who proposed block?)
        \item Gas used / gas limit
        \item Base fee per gas
        \item Burnt fees (EIP-1559)
    \end{itemize}
\end{enumerate}
\end{frame}

\begin{frame}{Exercise 4: Address Investigation}
\textbf{Objective:} Analyze address activity and identify entity types

\vspace{0.3cm}
\textbf{Instructions:}

\vspace{0.3cm}
\begin{enumerate}
    \item \textbf{Exchange Address (Example: Binance Hot Wallet)}
    \begin{itemize}
        \item Search Etherscan for: \texttt{0x28C6c06298d514Db089934071355E5743bf21d60}
        \item Observe transaction volume (thousands per day)
        \item Note large balances (millions of USD)
        \item Identify patterns: frequent deposits and withdrawals
        \item Tag: Etherscan labels it ``Binance 14''
    \end{itemize}

    \item \textbf{Smart Contract Address}
    \begin{itemize}
        \item Search for Uniswap V3 Router: \texttt{0xE592427A0AEce92De3Edee1F18E0157C05861564}
        \item Click ``Contract'' tab to view verified source code
        \item Examine recent transactions (all contract interactions)
        \item Notice: no ETH balance needed (users pay gas)
    \end{itemize}

    \item \textbf{Individual User Address}
    \begin{itemize}
        \item Use your own MetaMask address (from Lab 8)
        \item View transaction history
        \item Check token balances
        \item Compare activity level with exchange address
    \end{itemize}
\end{enumerate}
\end{frame}

\begin{frame}{Exercise 4: Entity Identification Patterns}
\textbf{How to Identify Address Types:}

\vspace{0.3cm}
\textbf{Exchange Addresses:}
\begin{itemize}
    \item Very high transaction volume (1000s per day)
    \item Large balances (millions of USD)
    \item Many unique counterparties
    \item Often labeled by block explorers
    \item Pattern: users deposit -> internal accounting -> users withdraw
\end{itemize}

\vspace{0.3cm}
\textbf{Smart Contracts:}
\begin{itemize}
    \item ``Contract'' label in block explorer
    \item Transactions show ``Contract Interaction''
    \item Code visible (if verified)
    \item Receives transactions, never initiates (unless self-destruct)
\end{itemize}

\vspace{0.3cm}
\textbf{Individual Users:}
\begin{itemize}
    \item Low-to-moderate transaction volume
    \item Smaller balances
    \item Irregular transaction timing
    \item Mix of incoming and outgoing transactions
\end{itemize}

\vspace{0.3cm}
\textbf{Miners/Validators:}
\begin{itemize}
    \item Receive coinbase rewards (Bitcoin)
    \item Receive block rewards + fees (Ethereum)
    \item Regular income stream every N blocks
\end{itemize}
\end{frame}

\begin{frame}{Exercise 5: Forensic Case Study}
\textbf{Scenario: Tracking Stolen Funds}

\vspace{0.3cm}
Imagine you are a blockchain analyst investigating a theft. Your task is to trace the flow of stolen funds.

\vspace{0.3cm}
\textbf{Case Details:}
\begin{itemize}
    \item Victim address (hypothetical): \texttt{0xVICTIM}
    \item Attacker address (hypothetical): \texttt{0xATTACKER}
    \item Transaction hash of theft: \texttt{0xSTEAL}
    \item Amount stolen: 100 ETH
\end{itemize}

\vspace{0.3cm}
\textbf{Investigation Steps:}

\vspace{0.3cm}
\begin{enumerate}
    \item \textbf{Confirm Theft:}
    \begin{itemize}
        \item Search for transaction \texttt{0xSTEAL}
        \item Verify 100 ETH moved from victim to attacker
    \end{itemize}

    \item \textbf{Trace Funds:}
    \begin{itemize}
        \item Click on attacker address
        \item View subsequent transactions
        \item Identify where 100 ETH went (direct transfer? split? mixer?)
    \end{itemize}

    \item \textbf{Follow the Chain:}
    \begin{itemize}
        \item If funds moved to another address, continue tracing
        \item If funds sent to exchange, investigation pauses (off-chain)
        \item If funds sent to mixer/tumbler, tracing becomes difficult
    \end{itemize}

    \item \textbf{Document Path:}
    \begin{itemize}
        \item Create flowchart: Victim -> Attacker -> Address A -> Address B -> Exchange
        \item Note transaction hashes at each step
        \item Identify opportunities for recovery (e.g., frozen exchange account)
    \end{itemize}
\end{enumerate}
\end{frame}

\begin{frame}{Exercise 5: Privacy and Mixing}
\textbf{Blockchain Privacy Challenges:}

\vspace{0.3cm}
\begin{itemize}
    \item All transactions publicly visible
    \item Addresses pseudonymous but traceable
    \item Address reuse links multiple transactions to same entity
    \item Heuristics identify common ownership (e.g., inputs from same transaction)
\end{itemize}

\vspace{0.3cm}
\textbf{Privacy Techniques:}

\vspace{0.3cm}
\begin{enumerate}
    \item \textbf{Mixers/Tumblers (e.g., Tornado Cash):}
    \begin{itemize}
        \item Pool funds from many users
        \item Withdraw to new address
        \item Breaks on-chain link
        \item Controversial: used by criminals but also privacy advocates
    \end{itemize}

    \item \textbf{CoinJoin (Bitcoin):}
    \begin{itemize}
        \item Multiple users combine transactions
        \item Obfuscates sender-receiver mapping
        \item Used by Wasabi Wallet, Samourai Wallet
    \end{itemize}

    \item \textbf{Privacy Coins (Monero, Zcash):}
    \begin{itemize}
        \item Built-in privacy (ring signatures, zero-knowledge proofs)
        \item Transactions not traceable
        \item Trade-off: regulatory scrutiny, exchange delistings
    \end{itemize}
\end{enumerate}

\vspace{0.3cm}
\textbf{Ethical Considerations:}
\begin{itemize}
    \item Privacy as a right vs. transparency for accountability
    \item Law enforcement vs. individual freedom
    \item Blockchain analytics industry (Chainalysis, Elliptic)
\end{itemize}
\end{frame}

\begin{frame}{Advanced Block Explorer Features}
\textbf{Etherscan Tools:}

\vspace{0.3cm}
\begin{enumerate}
    \item \textbf{Contract Verification:}
    \begin{itemize}
        \item Developers upload source code
        \item Etherscan compiles and matches bytecode
        \item Users can read contract logic before interacting
    \end{itemize}

    \item \textbf{Token Tracker:}
    \begin{itemize}
        \item View all ERC-20 tokens
        \item See total supply, holders, transfers
        \item Identify top holders
    \end{itemize}

    \item \textbf{Gas Tracker:}
    \begin{itemize}
        \item Real-time gas price recommendations
        \item Historical gas price charts
        \item Gas usage by contract (which dApps are expensive?)
    \end{itemize}

    \item \textbf{Charts and Analytics:}
    \begin{itemize}
        \item Daily transaction count
        \item Network utilization
        \item ETH supply and burn rate
        \item Validator statistics
    \end{itemize}
\end{enumerate}

\vspace{0.3cm}
\textbf{Mempool Explorers:}
\begin{itemize}
    \item Mempool.space (Bitcoin): visualize pending transactions
    \item See fee distribution, block template predictions
    \item Useful for fee estimation
\end{itemize}
\end{frame}

\begin{frame}{Lab Deliverables}
\textbf{Submit the following:}

\vspace{0.3cm}
\begin{enumerate}
    \item \textbf{Transaction Analysis Report (2-3 pages PDF):}
    \begin{itemize}
        \item Exercise 1: Bitcoin transaction hash, input/output summary, fee analysis
        \item Exercise 2: Ethereum transaction hash, gas breakdown, sender/recipient details
        \item Exercise 3: Block number analyzed, coinbase/validator info, statistics
        \item Exercise 4: Three addresses investigated with entity type identification
        \item Exercise 5: Forensic case flowchart (real or hypothetical scenario)
    \end{itemize}

    \item \textbf{Screenshots:}
    \begin{itemize}
        \item Bitcoin transaction details page
        \item Ethereum transaction details page
        \item Block details page
        \item Address activity page
        \item (Annotate screenshots with key observations)
    \end{itemize}

    \item \textbf{Reflection Questions (1 page):}
    \begin{itemize}
        \item How does blockchain transparency affect privacy?
        \item What are legitimate uses of transaction mixers?
        \item How can forensic analysis help recover stolen funds?
        \item What limitations exist in blockchain forensics?
    \end{itemize}

    \item \textbf{Bonus (Optional):}
    \begin{itemize}
        \item Analyze the same address on Bitcoin and Ethereum (if applicable)
        \item Investigate a recent high-profile hack using block explorer
    \end{itemize}
\end{enumerate}
\end{frame}

\begin{frame}{Key Takeaways}
\begin{itemize}
    \item Block explorers provide transparency into all blockchain activity
    \item Bitcoin uses UTXO model: trace inputs and outputs
    \item Ethereum uses account model: track balances and nonces
    \item Gas fees vary based on network congestion and transaction complexity
    \item Addresses can be identified by transaction patterns (exchanges, contracts, users)
    \item Blockchain forensics is powerful but faces challenges with mixers and privacy coins
    \item Verified smart contracts allow users to audit code before interacting
    \item Mempool analysis helps estimate fees and transaction timing
\end{itemize}

\vspace{0.5cm}
\textbf{Real-World Applications:}
\begin{itemize}
    \item Auditing and compliance
    \item Fraud investigation and fund recovery
    \item Market analysis (whale watching, exchange flows)
    \item Security research (identifying attack patterns)
\end{itemize}
\end{frame}

\begin{frame}{Discussion Questions}
\begin{enumerate}
    \item How does Bitcoin's UTXO model differ from Ethereum's account model in terms of privacy?

    \item Why are transaction fees so variable on Ethereum compared to Bitcoin?

    \item What are the ethical implications of blockchain transparency?

    \item How can someone enhance their privacy when using public blockchains?

    \item What role do block explorers play in blockchain adoption?

    \item How might regulation affect block explorer operations in the future?
\end{enumerate}
\end{frame}

\begin{frame}{Module A Summary and Next Steps}
\textbf{What You Learned in Module A:}

\vspace{0.3cm}
\begin{itemize}
    \item Blockchain fundamentals: decentralization, immutability, transparency
    \item Distributed ledger technology and consensus mechanisms
    \item Hash functions and cryptographic security
    \item Public key cryptography and digital signatures
    \item Bitcoin protocol: UTXO model, transactions, scripting
    \item Proof-of-work and proof-of-stake consensus
    \item Consensus mechanism trade-offs and comparisons
    \item Scalability trilemma and Layer 2 solutions
    \item Practical skills: wallet setup, transaction analysis, block exploration
\end{itemize}

\vspace{0.5cm}
\textbf{Preparation for Module B (Smart Contracts and DeFi):}
\begin{itemize}
    \item Review Ethereum fundamentals
    \item Explore DeFi applications (Uniswap, Aave, Compound)
    \item Familiarize yourself with Solidity programming concepts
    \item Keep MetaMask wallet active with testnet ETH
\end{itemize}
\end{frame}

\end{document}
