\documentclass[8pt,aspectratio=169]{beamer}
\usetheme{Madrid}
\usepackage[utf8]{inputenc}
\usepackage{graphicx}
\usepackage{booktabs}
\usepackage{hyperref}
\usepackage{listings}

\newcommand{\bottomnote}[1]{\vfill\par\noindent\footnotesize\textit{#1}}

\title{Lab Session: Hash Function Experiments}
\subtitle{BSc Blockchain, Crypto Economy \& NFTs}
\author{Course Instructor}
\date{Module A: Blockchain Foundations}

\begin{document}

\begin{frame}
\titlepage
\end{frame}

\begin{frame}{Learning Objectives}
By the end of this lab session, you will be able to:
\begin{itemize}
    \item Use Python's \texttt{hashlib} library to compute SHA-256 hashes
    \item Demonstrate the avalanche effect experimentally
    \item Understand collision resistance through brute-force attempts
    \item Build a simple proof-of-work mining simulation
    \item Verify hash-based integrity in practical scenarios
\end{itemize}
\end{frame}

\begin{frame}[t]{Lab Session Structure}
\begin{center}
\includegraphics[width=0.70\textwidth]{charts/01_lab_workflow/chart.pdf}
\end{center}
\bottomnote{Total duration: 90 minutes with 4 hands-on exercises}
\end{frame}

\begin{frame}{Environment Setup}
\textbf{Required Libraries:}
\begin{itemize}
    \item \texttt{hashlib} (standard library)
    \item \texttt{time} (standard library)
    \item \texttt{json} (standard library)
\end{itemize}

\vspace{0.3cm}
\textbf{Setup Instructions:}
\begin{enumerate}
    \item Create a new directory: \texttt{hash\_lab}
    \item Create a Python file: \texttt{hash\_experiments.py}
    \item Import required libraries
    \item Test installation by computing a simple hash
\end{enumerate}

\vspace{0.3cm}
\textbf{Verification:} Hash ``Hello, Blockchain!'' and verify output
\end{frame}

\begin{frame}{Exercise 1: Basic Hashing}
\textbf{Objectives:}
\begin{itemize}
    \item Compute SHA-256 hashes of strings
    \item Compute SHA-256 hashes of files
    \item Compare different hash algorithms (MD5, SHA-1, SHA-256)
\end{itemize}

\vspace{0.3cm}
\textbf{Tasks:}
\begin{enumerate}
    \item Create a function that takes a string and returns its SHA-256 hash
    \item Hash: ``Blockchain'', ``blockchain'', ``Blockchain '' (with space)
    \item Compare the outputs and observe differences
    \item Create a text file and compute its hash
    \item Modify one character and recompute
\end{enumerate}

\vspace{0.3cm}
\textbf{Expected Outcome:} Tiny input changes produce completely different hashes
\end{frame}

\begin{frame}{Exercise 2: Avalanche Effect Demonstration}
\textbf{Objective:} Experimentally verify the avalanche effect

\vspace{0.3cm}
\textbf{The Avalanche Effect:}
\begin{itemize}
    \item Changing a single bit should change approximately 50\% of output bits
    \item Critical property for cryptographic security
\end{itemize}

\vspace{0.3cm}
\textbf{Tasks:}
\begin{enumerate}
    \item Create a function that compares two hashes bit-by-bit
    \item Hash ``The quick brown fox jumps over the lazy dog''
    \item Hash ``The quick brown fox jumps over the lazy dof'' (last letter changed)
    \item Count how many bits differ between the two hashes
    \item Calculate the percentage of bits that changed
\end{enumerate}

\vspace{0.3cm}
\textbf{Expected Result:} Approximately 50\% of bits should differ
\end{frame}

\begin{frame}{Exercise 3: Simple Proof-of-Work Mining}
\textbf{Objective:} Build a basic mining simulation

\vspace{0.3cm}
\textbf{Concept Review:}
\begin{itemize}
    \item Mining = finding a nonce such that hash meets difficulty target
    \item Difficulty target = hash must start with N leading zeros
    \item No shortcut: must try different nonces sequentially
\end{itemize}

\vspace{0.3cm}
\textbf{Tasks:}
\begin{enumerate}
    \item Create a block structure (number, data, previous hash, nonce)
    \item Implement mining function that increments nonce until valid
    \item Mine blocks with difficulty 1, 2, 3, 4
    \item Record time taken and nonces tried for each difficulty
\end{enumerate}
\end{frame}

\begin{frame}[t]{Mining Difficulty: Exponential Growth}
\begin{center}
\includegraphics[width=0.55\textwidth]{charts/02_mining_difficulty/chart.pdf}
\end{center}
\bottomnote{Each additional zero increases difficulty by approximately 16x}
\end{frame}

\begin{frame}{Exercise 4: Hash Chain Verification}
\textbf{Objective:} Build and verify a simple blockchain

\vspace{0.3cm}
\textbf{Tasks:}
\begin{enumerate}
    \item Create a genesis block (first block with previous\_hash = ``0'')
    \item Create a function to add new blocks
    \item Build a chain of 5 blocks
    \item Implement a verification function that checks:
    \begin{itemize}
        \item Each block's hash is valid
        \item Each block correctly references previous hash
    \end{itemize}
    \item Tamper with block 3's data and observe verification failure
\end{enumerate}

\vspace{0.3cm}
\textbf{Expected Outcome:} Understand immutability through hash chains
\end{frame}

\begin{frame}[t]{Chain Integrity Verification}
\begin{center}
\includegraphics[width=0.65\textwidth]{charts/03_chain_verification/chart.pdf}
\end{center}
\bottomnote{Tampering breaks hash links, making modifications immediately detectable}
\end{frame}

\begin{frame}{Lab Deliverables}
\textbf{Submit the following:}

\begin{enumerate}
    \item \textbf{Python script} (\texttt{hash\_experiments.py}) containing:
    \begin{itemize}
        \item All four exercises implemented
        \item Clear function names and comments
        \item Test cases demonstrating functionality
    \end{itemize}

    \item \textbf{Lab report} (PDF, 2-3 pages) including:
    \begin{itemize}
        \item Avalanche effect results (bit difference percentages)
        \item Mining performance table (difficulty vs. time)
        \item Screenshot of chain verification before and after tampering
    \end{itemize}
\end{enumerate}

\vspace{0.3cm}
\textbf{Submission Deadline:} One week from lab session date
\end{frame}

\begin{frame}{Key Takeaways}
\begin{itemize}
    \item Hash functions are easy to compute but infeasible to reverse
    \item The avalanche effect ensures unpredictable output changes
    \item Proof-of-work mining is computationally expensive by design
    \item Hash chains create tamper-evident data structures
    \item Blockchain immutability comes from re-mining cost
\end{itemize}

\vspace{0.5cm}
\textbf{Real-World Applications:}
\begin{itemize}
    \item Bitcoin/Ethereum mining
    \item Git version control (commit hashes)
    \item File integrity verification (checksums)
\end{itemize}
\end{frame}

\begin{frame}{Discussion Questions}
\begin{enumerate}
    \item Why is it important that hash functions are deterministic?
    \item Did you notice any patterns in which nonces produced valid hashes?
    \item If you wanted to modify block 3 of a 100-block chain, how many hashes would you need to recompute?
    \item How does increasing difficulty affect blockchain security?
    \item What would happen if a hash function did not exhibit the avalanche effect?
\end{enumerate}
\end{frame}

\end{document}
