\documentclass[8pt,aspectratio=169]{beamer}
\usetheme{Madrid}
\usepackage[utf8]{inputenc}
\usepackage{graphicx}
\usepackage{booktabs}
\usepackage{hyperref}
\usepackage{listings}

\title{Lab Session: Hash Function Experiments}
\subtitle{BSc Blockchain, Crypto Economy \& NFTs}
\author{Course Instructor}
\date{Module A: Blockchain Foundations}

\begin{document}

\begin{frame}
\titlepage
\end{frame}

\begin{frame}{Learning Objectives}
By the end of this lab session, you will be able to:
\begin{itemize}
    \item Use Python's \texttt{hashlib} library to compute SHA-256 hashes
    \item Demonstrate the avalanche effect experimentally
    \item Understand collision resistance through brute-force attempts
    \item Build a simple proof-of-work mining simulation
    \item Verify hash-based integrity in practical scenarios
\end{itemize}
\end{frame}

\begin{frame}{Lab Overview}
\textbf{Structure:}
\begin{enumerate}
    \item Environment setup (5 minutes)
    \item Exercise 1: Basic hashing (15 minutes)
    \item Exercise 2: Avalanche effect demonstration (20 minutes)
    \item Exercise 3: Simple proof-of-work mining (30 minutes)
    \item Exercise 4: Hash chain verification (20 minutes)
    \item Wrap-up and deliverables (10 minutes)
\end{enumerate}

\vspace{0.3cm}
\textbf{Total Duration:} 90 minutes

\vspace{0.3cm}
\textbf{Prerequisites:}
\begin{itemize}
    \item Python 3.8+ installed
    \item Basic Python programming knowledge
    \item Understanding of hash functions from Lesson 3
\end{itemize}
\end{frame}

\begin{frame}{Environment Setup}
\textbf{Required Libraries:}
\begin{itemize}
    \item \texttt{hashlib} (standard library)
    \item \texttt{time} (standard library)
    \item \texttt{json} (standard library)
\end{itemize}

\vspace{0.3cm}
\textbf{Setup Instructions:}
\begin{enumerate}
    \item Create a new directory: \texttt{hash\_lab}
    \item Create a Python file: \texttt{hash\_experiments.py}
    \item Import required libraries
    \item Test installation by computing a simple hash
\end{enumerate}

\vspace{0.3cm}
\textbf{Verification Command:}

Check that running a basic hash function produces the expected output for the string ``Hello, Blockchain!''
\end{frame}

\begin{frame}{Exercise 1: Basic Hashing}
\textbf{Objectives:}
\begin{itemize}
    \item Compute SHA-256 hashes of strings
    \item Compute SHA-256 hashes of files
    \item Compare different hash algorithms (MD5, SHA-1, SHA-256)
\end{itemize}

\vspace{0.3cm}
\textbf{Tasks:}
\begin{enumerate}
    \item Create a function that takes a string and returns its SHA-256 hash
    \item Hash the following inputs:
    \begin{itemize}
        \item ``Blockchain''
        \item ``blockchain'' (lowercase)
        \item ``Blockchain '' (with trailing space)
    \end{itemize}
    \item Compare the outputs and observe differences
    \item Create a text file and compute its hash
    \item Modify one character in the file and recompute
\end{enumerate}

\vspace{0.3cm}
\textbf{Expected Outcome:} Understand that tiny input changes produce completely different hashes
\end{frame}

\begin{frame}{Exercise 2: Avalanche Effect Demonstration}
\textbf{Objective:} Experimentally verify the avalanche effect

\vspace{0.3cm}
\textbf{The Avalanche Effect:}
\begin{itemize}
    \item Changing a single bit in input should change approximately 50\% of output bits
    \item Critical property for cryptographic security
    \item Makes pattern detection impossible
\end{itemize}

\vspace{0.3cm}
\textbf{Tasks:}
\begin{enumerate}
    \item Create a function that compares two hashes bit-by-bit
    \item Hash the string ``The quick brown fox jumps over the lazy dog''
    \item Hash the string ``The quick brown fox jumps over the lazy dof'' (last letter changed)
    \item Count how many bits differ between the two hashes
    \item Calculate the percentage of bits that changed
    \item Repeat with other single-character modifications
\end{enumerate}

\vspace{0.3cm}
\textbf{Expected Result:} Approximately 50\% of bits should differ
\end{frame}

\begin{frame}{Exercise 2: Implementation Hints}
\textbf{Converting Hash to Binary:}
\begin{itemize}
    \item Hash output is hexadecimal (64 characters for SHA-256)
    \item Convert each hex digit to 4 binary bits
    \item Total: 256 bits
\end{itemize}

\vspace{0.3cm}
\textbf{Bit Comparison:}
\begin{itemize}
    \item Use XOR operation to find differing bits
    \item Count the number of 1s in the XOR result
    \item Divide by 256 to get percentage
\end{itemize}

\vspace{0.3cm}
\textbf{Test Cases:}
\begin{itemize}
    \item Same string should have 0\% difference
    \item One character change should have ~50\% difference
    \item Completely different strings should have ~50\% difference
\end{itemize}
\end{frame}

\begin{frame}{Exercise 3: Simple Proof-of-Work Mining}
\textbf{Objective:} Build a basic mining simulation

\vspace{0.3cm}
\textbf{Concept Review:}
\begin{itemize}
    \item Mining = finding a nonce such that hash(block\_data + nonce) meets difficulty target
    \item Difficulty target = hash must start with N leading zeros
    \item No shortcut: must try different nonces sequentially
\end{itemize}

\vspace{0.3cm}
\textbf{Tasks:}
\begin{enumerate}
    \item Create a block structure with:
    \begin{itemize}
        \item Block number
        \item Data (e.g., ``Transaction: Alice sends 10 BTC to Bob'')
        \item Previous block hash
        \item Nonce (initially 0)
    \end{itemize}
    \item Implement a mining function that increments nonce until hash starts with N zeros
    \item Mine blocks with difficulty 1, 2, 3, 4
    \item Record time taken and nonces tried for each difficulty
\end{enumerate}
\end{frame}

\begin{frame}{Exercise 3: Implementation Structure}
\textbf{Block Data Structure:}
\begin{itemize}
    \item Combine all fields into a single string
    \item Format: ``block\_number:data:previous\_hash:nonce''
    \item Hash this concatenated string
\end{itemize}

\vspace{0.3cm}
\textbf{Mining Algorithm:}
\begin{enumerate}
    \item Start with nonce = 0
    \item Compute hash of block data + nonce
    \item Check if hash meets difficulty (starts with N zeros)
    \item If yes: return nonce
    \item If no: increment nonce and repeat
\end{enumerate}

\vspace{0.3cm}
\textbf{Difficulty Verification:}
\begin{itemize}
    \item Difficulty 1: hash starts with ``0''
    \item Difficulty 2: hash starts with ``00''
    \item Difficulty 3: hash starts with ``000''
    \item Each additional zero increases difficulty by ~16x
\end{itemize}
\end{frame}

\begin{frame}{Exercise 3: Expected Results}
\textbf{Performance Observations:}

\vspace{0.3cm}
\begin{tabular}{lrr}
\toprule
\textbf{Difficulty} & \textbf{Avg. Attempts} & \textbf{Approx. Time} \\
\midrule
1 zero  & ~16        & < 1 second \\
2 zeros & ~256       & < 1 second \\
3 zeros & ~4,096     & 1-5 seconds \\
4 zeros & ~65,536    & 10-60 seconds \\
5 zeros & ~1,048,576 & 5-20 minutes \\
\bottomrule
\end{tabular}

\vspace{0.3cm}
\textbf{Key Insights:}
\begin{itemize}
    \item Exponential growth in computation time
    \item No way to predict nonce value
    \item Bitcoin uses difficulty ~19 leading zeros (current)
    \item Real mining uses specialized hardware (ASICs)
\end{itemize}
\end{frame}

\begin{frame}{Exercise 4: Hash Chain Verification}
\textbf{Objective:} Build and verify a simple blockchain

\vspace{0.3cm}
\textbf{Tasks:}
\begin{enumerate}
    \item Create a genesis block (first block with previous\_hash = ``0'')
    \item Create a function to add new blocks that:
    \begin{itemize}
        \item Takes previous block's hash
        \item Includes new transaction data
        \item Mines with difficulty 2
    \end{itemize}
    \item Build a chain of 5 blocks
    \item Implement a verification function that checks:
    \begin{itemize}
        \item Each block's hash is valid
        \item Each block correctly references previous hash
        \item Chain integrity from genesis to tip
    \end{itemize}
    \item Tamper with block 3's data and observe verification failure
\end{enumerate}

\vspace{0.3cm}
\textbf{Expected Outcome:} Understand immutability through hash chains
\end{frame}

\begin{frame}{Exercise 4: Tamper Detection}
\textbf{Scenario:} Attacker modifies a transaction in the middle of the chain

\vspace{0.3cm}
\textbf{Experiment:}
\begin{enumerate}
    \item Build a valid 5-block chain
    \item Verify the entire chain (should pass)
    \item Modify the data in block 3
    \item Run verification again (should fail)
\end{enumerate}

\vspace{0.3cm}
\textbf{Why Verification Fails:}
\begin{itemize}
    \item Changing block 3's data changes its hash
    \item Block 4 references the old hash of block 3
    \item Hash chain breaks at this link
    \item Verification detects mismatch
\end{itemize}

\vspace{0.3cm}
\textbf{Discussion Question:} What would an attacker need to do to successfully modify block 3?

\vspace{0.2cm}
\textit{Answer: Re-mine block 3 and all subsequent blocks (computationally expensive)}
\end{frame}

\begin{frame}{Lab Deliverables}
\textbf{Submit the following:}

\begin{enumerate}
    \item \textbf{Python script} (\texttt{hash\_experiments.py}) containing:
    \begin{itemize}
        \item All four exercises implemented
        \item Clear function names and comments
        \item Test cases demonstrating functionality
    \end{itemize}

    \item \textbf{Lab report} (PDF, 2-3 pages) including:
    \begin{itemize}
        \item Exercise 2: Avalanche effect results (bit difference percentages)
        \item Exercise 3: Mining performance table (difficulty vs. time)
        \item Exercise 4: Screenshot of chain verification before and after tampering
        \item Brief reflection on blockchain immutability
    \end{itemize}

    \item \textbf{Bonus (optional):} Implement SHA-1 collision detection using known collision examples
\end{enumerate}

\vspace{0.3cm}
\textbf{Submission Deadline:} One week from lab session date

\vspace{0.2cm}
\textbf{Grading:} Pass/Fail based on completeness and correctness
\end{frame}

\begin{frame}{Key Takeaways}
\begin{itemize}
    \item Hash functions are easy to compute but infeasible to reverse
    \item The avalanche effect ensures that small input changes produce unpredictable output changes
    \item Proof-of-work mining is computationally expensive by design
    \item Hash chains create tamper-evident data structures
    \item Blockchain immutability comes from the cost of re-mining modified blocks
\end{itemize}

\vspace{0.5cm}
\textbf{Real-World Applications:}
\begin{itemize}
    \item Bitcoin/Ethereum mining
    \item Git version control (commit hashes)
    \item File integrity verification (checksums)
    \item Password storage (salted hashes)
\end{itemize}
\end{frame}

\begin{frame}{Discussion Questions}
\begin{enumerate}
    \item Why is it important that hash functions are deterministic (same input always produces same output)?

    \item In your mining experiments, did you notice any patterns in which nonces produced valid hashes?

    \item If you wanted to modify a transaction in block 3 of a 100-block chain, approximately how many hashes would you need to recompute?

    \item How does increasing the mining difficulty affect the security of a blockchain?

    \item What would happen if a hash function did not exhibit the avalanche effect?
\end{enumerate}
\end{frame}

\begin{frame}{Next Lesson Preview: L05 Public Key Cryptography}
\textbf{Topics to be covered:}
\begin{itemize}
    \item Asymmetric encryption fundamentals
    \item Elliptic Curve Digital Signature Algorithm (ECDSA)
    \item Digital signatures and verification
    \item Public/private key pair generation
    \item Bitcoin and Ethereum address derivation
    \item Key security best practices
\end{itemize}

\vspace{0.5cm}
\textbf{Preparation:}
\begin{itemize}
    \item Review symmetric vs. asymmetric encryption concepts
    \item Read about public key infrastructure (PKI)
    \item Explore how digital signatures differ from physical signatures
\end{itemize}
\end{frame}

\end{document}
