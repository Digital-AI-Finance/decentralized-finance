\documentclass[8pt,aspectratio=169]{beamer}
\usetheme{Madrid}
\usepackage{graphicx}
\usepackage{booktabs}
\usepackage{adjustbox}
\usepackage{multicol}
\usepackage{amsmath}
\usepackage{amssymb}

% Color definitions
\definecolor{mlblue}{RGB}{0,102,204}
\definecolor{mlpurple}{RGB}{51,51,178}
\definecolor{mllavender}{RGB}{173,173,224}
\definecolor{mllavender2}{RGB}{193,193,232}
\definecolor{mllavender3}{RGB}{204,204,235}
\definecolor{mllavender4}{RGB}{214,214,239}
\definecolor{mlorange}{RGB}{255, 127, 14}
\definecolor{mlgreen}{RGB}{44, 160, 44}
\definecolor{mlred}{RGB}{214, 39, 40}
\definecolor{mlgray}{RGB}{127, 127, 127}
\definecolor{lightgray}{RGB}{240, 240, 240}
\definecolor{midgray}{RGB}{180, 180, 180}

% Apply custom colors to Madrid theme
\setbeamercolor{palette primary}{bg=mllavender3,fg=mlpurple}
\setbeamercolor{palette secondary}{bg=mllavender2,fg=mlpurple}
\setbeamercolor{palette tertiary}{bg=mllavender,fg=white}
\setbeamercolor{palette quaternary}{bg=mlpurple,fg=white}

\setbeamercolor{structure}{fg=mlpurple}
\setbeamercolor{section in toc}{fg=mlpurple}
\setbeamercolor{subsection in toc}{fg=mlblue}
\setbeamercolor{title}{fg=mlpurple}
\setbeamercolor{frametitle}{fg=mlpurple,bg=mllavender3}
\setbeamercolor{block title}{bg=mllavender2,fg=mlpurple}
\setbeamercolor{block body}{bg=mllavender4,fg=black}

% Remove navigation symbols
\setbeamertemplate{navigation symbols}{}

% Clean itemize/enumerate
\setbeamertemplate{itemize items}[circle]
\setbeamertemplate{enumerate items}[default]

% Reduce margins for more content space
\setbeamersize{text margin left=5mm,text margin right=5mm}

% Command for bottom annotation (Madrid-style)
\newcommand{\bottomnote}[1]{%
\vfill
\vspace{-2mm}
\textcolor{mllavender2}{\rule{\textwidth}{0.4pt}}
\vspace{1mm}
\footnotesize
\textbf{#1}
}

\title{Lesson 1: What is Blockchain?}
\subtitle{Module A: Blockchain Foundations}
\author{MSc Blockchain \& Cryptocurrency}
\institute{Digital Finance Program}
\date{2025}

\begin{document}

% Title slide
\begin{frame}[plain]
\titlepage
\end{frame}

% Learning Objectives
\begin{frame}[t]{Learning Objectives}
\textbf{By the end of this lesson, you will be able to:}
\begin{enumerate}
\item Define blockchain as a cryptographically-secured distributed ledger
\item Trace the historical evolution from 1991 to 2025
\item Explain the double-spending problem and its consensus-based solution
\item Formalize the hash chain structure: $H(B_n) = H(\text{header}_n || \text{prev\_hash}_{n-1})$
\item Compare centralized, decentralized, and distributed architectures
\end{enumerate}

\vspace{0.5em}
\textbf{Prerequisites:} Cryptographic hash functions, basic probability theory

\bottomnote{MSc level: Full mathematical rigor expected in subsequent slides}
\end{frame}

% Definition with Math
\begin{frame}[t]{Formal Definition of Blockchain}
\begin{columns}[T]
\column{0.48\textwidth}
\textbf{Mathematical Definition}

A blockchain $\mathcal{B}$ is an ordered sequence of blocks:
$$\mathcal{B} = (B_0, B_1, \ldots, B_n)$$

Where each block $B_i$ contains:
\begin{itemize}
\item Header $h_i$ with metadata
\item Transaction set $T_i = \{tx_1, \ldots, tx_k\}$
\item Hash pointer: $H(B_{i-1})$
\end{itemize}

\textbf{Integrity Constraint:}
$$\forall i > 0: B_i.\text{prev} = H(B_{i-1})$$

\column{0.48\textwidth}
\begin{center}
\includegraphics[width=0.95\columnwidth]{charts/04_block_structure/chart.pdf}
\end{center}
\end{columns}

\bottomnote{The hash pointer creates a tamper-evident data structure}
\end{frame}

% Timeline Chart
\begin{frame}[t]{Historical Evolution: 1991--2025}
\begin{center}
\includegraphics[width=0.7\textwidth]{charts/01_blockchain_timeline/chart.pdf}
\end{center}

\bottomnote{Key inflection points: 2008 (Nakamoto), 2015 (Ethereum), 2024 (Bitcoin ETFs)}
\end{frame}

% Network Topology - Centralized
\begin{frame}[t]{Centralized Architecture}
\begin{center}
\includegraphics[width=0.45\textwidth]{charts/02a_centralized_network/chart.pdf}
\end{center}
\vspace{-2mm}
\footnotesize\textbf{Characteristics:} Single authority, $10^6$ TPS, $<$10ms latency

\bottomnote{Traditional systems: banks, exchanges, cloud services}
\end{frame}

% Network Topology - Decentralized
\begin{frame}[t]{Decentralized Architecture}
\begin{center}
\includegraphics[width=0.45\textwidth]{charts/02b_decentralized_network/chart.pdf}
\end{center}
\vspace{-2mm}
\footnotesize\textbf{Characteristics:} Multiple hubs, federated trust, $10^3$ TPS

\bottomnote{Examples: Federated exchanges, consortium blockchains}
\end{frame}

% Network Topology - Distributed
\begin{frame}[t]{Distributed Architecture (Blockchain)}
\begin{center}
\includegraphics[width=0.45\textwidth]{charts/02c_distributed_network/chart.pdf}
\end{center}
\vspace{-2mm}
\footnotesize\textbf{Characteristics:} No hierarchy, cryptographic trust, $10^1$ TPS

\bottomnote{Blockchain trades performance for trustlessness}
\end{frame}

% Double Spending Problem - The Attack
\begin{frame}[t]{The Double-Spending Problem}
\begin{center}
\includegraphics[width=0.6\textwidth]{charts/03a_double_spending_problem/chart.pdf}
\end{center}

\textbf{Problem:} Prevent $\text{Transfer}(a, A \to B) \land \text{Transfer}(a, A \to C)$

\bottomnote{Digital files can be copied infinitely --- no inherent scarcity in bits}
\end{frame}

% Double Spending Solution
\begin{frame}[t]{The Blockchain Solution}
\begin{center}
\includegraphics[width=0.6\textwidth]{charts/03b_double_spending_solution/chart.pdf}
\end{center}

\textbf{Solution:} Distributed consensus determines transaction ordering

\bottomnote{Nakamoto's key insight: Use computational work to achieve probabilistic finality}
\end{frame}

% Hash Chain Integrity - Valid
\begin{frame}[t]{Valid Blockchain: Hash Chain Integrity}
\begin{center}
\includegraphics[width=0.65\textwidth]{charts/05a_valid_chain/chart.pdf}
\end{center}

\bottomnote{Hash pointers create a tamper-evident linked data structure}
\end{frame}

% Hash Chain Integrity - Tampered
\begin{frame}[t]{Tampered Blockchain: Detection}
\begin{center}
\includegraphics[width=0.65\textwidth]{charts/05b_tampered_chain/chart.pdf}
\end{center}

\bottomnote{Modifying $B_k$ requires recomputing all subsequent hashes: $O(n-k) \times 2^{76}$ ops}
\end{frame}

% Merkle Tree Integration
\begin{frame}[t]{Merkle Tree: Efficient Transaction Verification}
\begin{columns}[T]
\column{0.55\textwidth}
\textbf{Structure}

Binary hash tree over transactions:
$$\text{Root} = H(H(H(tx_1)||H(tx_2)) || H(H(tx_3)||H(tx_4)))$$

\textbf{Verification Complexity:}
\begin{itemize}
\item Full verification: $O(n)$ hashes
\item Merkle proof: $O(\log n)$ hashes
\item SPV clients use proofs, not full chain
\end{itemize}

\textbf{Bitcoin Block Header:}\\
32-byte Merkle root commits to all transactions

\column{0.42\textwidth}
\begin{center}
\includegraphics[width=\columnwidth]{../../charts/merkle_tree/chart.pdf}
\end{center}
\end{columns}

\bottomnote{Merkle trees enable lightweight clients: verify transactions without downloading full blocks}
\end{frame}

% Key Properties
\begin{frame}[t]{Core Properties and Guarantees}
\begin{columns}[T]
\column{0.48\textwidth}
\textbf{Cryptographic Properties}
\begin{itemize}
\item \textbf{Collision Resistance:}\\
$\Pr[H(x) = H(y) \land x \neq y] \approx 2^{-128}$
\item \textbf{Preimage Resistance:}\\
Given $h$, infeasible to find $x: H(x) = h$
\item \textbf{Avalanche Effect:}\\
1-bit change $\Rightarrow$ 50\% output bits flip
\end{itemize}

\column{0.48\textwidth}
\textbf{System Properties}
\begin{itemize}
\item \textbf{Liveness:}\\
Valid transactions eventually confirmed
\item \textbf{Safety:}\\
No double-spends with $>k$ confirmations
\item \textbf{Consistency:}\\
All honest nodes agree on prefix
\end{itemize}
\end{columns}

\vspace{0.5em}
\textbf{Probability of Reversal} (after $k$ confirmations, attacker with $q < 0.5$ hashrate):
$$P(\text{reversal}) < \left(\frac{q}{1-q}\right)^k$$

\bottomnote{6 confirmations $\Rightarrow$ reversal probability $< 0.1\%$ for $q = 0.3$}
\end{frame}

% Use Cases
\begin{frame}[t]{Real-World Applications (2025)}
\begin{columns}[T]
\column{0.48\textwidth}
\textbf{Finance \& Payments}
\begin{itemize}
\item Bitcoin: \$1.2T market cap, \$50B daily volume
\item Stablecoins: USDT/USDC \$150B+ circulation
\item DeFi TVL: \$80B across protocols
\item Bitcoin ETFs: \$50B+ AUM (Jan 2024 launch)
\end{itemize}

\vspace{0.3em}
\textbf{Enterprise}
\begin{itemize}
\item IBM Food Trust: 500+ organizations
\item JPMorgan Onyx: \$1B+ daily settlements
\item Maersk TradeLens: 1.5B shipping events
\end{itemize}

\column{0.48\textwidth}
\textbf{Government \& CBDC}
\begin{itemize}
\item China e-CNY: 260M+ wallets
\item EU Digital Euro: Pilot phase 2024
\item 130+ countries exploring CBDCs
\end{itemize}

\vspace{0.3em}
\textbf{Emerging Applications}
\begin{itemize}
\item Real-World Assets (RWA): \$5B+ tokenized
\item Decentralized Identity (DID)
\item Supply chain provenance
\item Carbon credit verification
\end{itemize}
\end{columns}

\bottomnote{Source: DeFi Llama, CoinGecko, Atlantic Council CBDC Tracker (Dec 2024)}
\end{frame}

% Bitcoin Transactions Growth
\begin{frame}[t]{Network Adoption: Bitcoin Transactions}
\begin{center}
\includegraphics[width=0.6\textwidth]{charts/06_bitcoin_transactions/chart.pdf}
\end{center}
\vspace{-2mm}
\footnotesize From $<$1K daily transactions (2010) to $>$600K daily (2024)

\bottomnote{Transaction volume indicates real economic activity on the network}
\end{frame}

% Hash Rate Growth
\begin{frame}[t]{Network Security: Hash Rate Growth}
\begin{center}
\includegraphics[width=0.6\textwidth]{charts/07_hash_rate_growth/chart.pdf}
\end{center}
\vspace{-2mm}
\footnotesize 750 EH/s = $7.5 \times 10^{20}$ SHA-256 hashes per second

\bottomnote{Higher hash rate = more computational cost to attack the network}
\end{frame}

% Crypto Market Cap
\begin{frame}[t]{Market Growth: Cryptocurrency Capitalization}
\begin{center}
\includegraphics[width=0.6\textwidth]{charts/08_crypto_market_cap/chart.pdf}
\end{center}
\vspace{-2mm}
\footnotesize Total market cap: \$3.5T (Dec 2024); Bitcoin dominance: $\sim$57\%

\bottomnote{Market cap growth reflects institutional adoption and mainstream acceptance}
\end{frame}

% Blockchain Trilemma
\begin{frame}[t]{The Blockchain Trilemma}
\begin{center}
\includegraphics[width=0.45\textwidth]{charts/09_blockchain_trilemma/chart.pdf}
\end{center}
\vspace{-3mm}
\bottomnote{Layer 2 solutions attempt to optimize all three dimensions}
\end{frame}

% When to Use Blockchain
\begin{frame}[t]{Decision Framework: When to Use Blockchain}
\begin{columns}[T]
\column{0.48\textwidth}
\textbf{Use Blockchain When:}
\begin{itemize}
\item[\textcolor{mlgreen}{$\checkmark$}] Multiple writers, no trusted party
\item[\textcolor{mlgreen}{$\checkmark$}] Immutable audit trail required
\item[\textcolor{mlgreen}{$\checkmark$}] Disintermediation creates value
\item[\textcolor{mlgreen}{$\checkmark$}] Censorship resistance needed
\item[\textcolor{mlgreen}{$\checkmark$}] Cross-organizational data sharing
\end{itemize}

\column{0.48\textwidth}
\textbf{Use Traditional DB When:}
\begin{itemize}
\item[\textcolor{mlred}{$\times$}] Single organization controls data
\item[\textcolor{mlred}{$\times$}] High throughput required ($>$10K TPS)
\item[\textcolor{mlred}{$\times$}] Data deletion/modification needed
\item[\textcolor{mlred}{$\times$}] Strong privacy requirements
\item[\textcolor{mlred}{$\times$}] Existing solutions work well
\end{itemize}
\end{columns}

\vspace{0.5em}
\textbf{Decision Heuristic:}
$$\text{Blockchain Value} \propto \frac{\text{Trust Deficit} \times \text{Coordination Benefit}}{\text{Performance Requirements}}$$

\bottomnote{Most enterprise ``blockchain'' projects could use a replicated database}
\end{frame}

% Summary
\begin{frame}[t]{Key Takeaways}
\textbf{Core Concepts:}
\begin{enumerate}
\item Blockchain = hash-chained blocks + distributed consensus + cryptographic signatures
\item Double-spending solved via total ordering through consensus mechanism
\item Immutability achieved through computational intractability of hash chain modification
\item Trade-off: Performance $\leftrightarrow$ Trustlessness $\leftrightarrow$ Decentralization
\end{enumerate}

\vspace{0.5em}
\textbf{Mathematical Foundations:}
\begin{itemize}
\item Hash functions: $H: \{0,1\}^* \to \{0,1\}^{256}$ (SHA-256)
\item Merkle trees: $O(\log n)$ verification complexity
\item Reversal probability: Exponential decay with confirmations
\end{itemize}

\vspace{0.5em}
\textbf{Next Lesson:} L02 -- Distributed Ledger Technology (DLT) deep dive

\bottomnote{Blockchain is a tool, not a solution --- evaluate against specific requirements}
\end{frame}

\end{document}
