\documentclass[8pt,aspectratio=169]{beamer}
\usetheme{Madrid}
\usepackage{graphicx}
\usepackage{booktabs}
\usepackage{adjustbox}
\usepackage{multicol}
\usepackage{amsmath}
\usepackage{amssymb}

% Color definitions
\definecolor{mlblue}{RGB}{0,102,204}
\definecolor{mlpurple}{RGB}{51,51,178}
\definecolor{mllavender}{RGB}{173,173,224}
\definecolor{mllavender2}{RGB}{193,193,232}
\definecolor{mllavender3}{RGB}{204,204,235}
\definecolor{mllavender4}{RGB}{214,214,239}
\definecolor{mlorange}{RGB}{255, 127, 14}
\definecolor{mlgreen}{RGB}{44, 160, 44}
\definecolor{mlred}{RGB}{214, 39, 40}
\definecolor{mlgray}{RGB}{127, 127, 127}

% Additional colors for template compatibility
\definecolor{lightgray}{RGB}{240, 240, 240}
\definecolor{midgray}{RGB}{180, 180, 180}

% Apply custom colors to Madrid theme
\setbeamercolor{palette primary}{bg=mllavender3,fg=mlpurple}
\setbeamercolor{palette secondary}{bg=mllavender2,fg=mlpurple}
\setbeamercolor{palette tertiary}{bg=mllavender,fg=white}
\setbeamercolor{palette quaternary}{bg=mlpurple,fg=white}

\setbeamercolor{structure}{fg=mlpurple}
\setbeamercolor{section in toc}{fg=mlpurple}
\setbeamercolor{subsection in toc}{fg=mlblue}
\setbeamercolor{title}{fg=mlpurple}
\setbeamercolor{frametitle}{fg=mlpurple,bg=mllavender3}
\setbeamercolor{block title}{bg=mllavender2,fg=mlpurple}
\setbeamercolor{block body}{bg=mllavender4,fg=black}

% Remove navigation symbols
\setbeamertemplate{navigation symbols}{}

% Clean itemize/enumerate
\setbeamertemplate{itemize items}[circle]
\setbeamertemplate{enumerate items}[default]

% Reduce margins for more content space
\setbeamersize{text margin left=5mm,text margin right=5mm}

% Command for bottom annotation (Madrid-style)
\newcommand{\bottomnote}[1]{%
\vfill
\vspace{-2mm}
\textcolor{mllavender2}{\rule{\textwidth}{0.4pt}}
\vspace{1mm}
\footnotesize
\textbf{#1}
}

% Command for compact list spacing
\newcommand{\compactlist}{%
\setlength{\itemsep}{0pt}%
\setlength{\parskip}{0pt}%
\setlength{\parsep}{0pt}%
}

% Command for chart placeholder with safe sizing
\newcommand{\chartplaceholder}[2][5cm]{%
\begin{center}
\begin{adjustbox}{max width=0.95\textwidth, max height=#1}
\framebox[\textwidth][c]{%
\rule{0pt}{#1}%
\textcolor{midgray}{[#2]}%
}
\end{adjustbox}
\end{center}
}

\title{Lesson 1: What is Blockchain?}
\subtitle{History, Motivation \& Core Concepts}
\author{BSc Blockchain, Crypto Economy \& NFTs}
\institute{FHGR School of Management}
\date{Winter Semester 2025}

\begin{document}

% ==================== TITLE SLIDE ====================
\begin{frame}[plain]
\titlepage
\end{frame}

% ==================== LEARNING OBJECTIVES ====================
\begin{frame}[t]{Learning Objectives}
\textbf{By the end of this lesson, you will be able to:}

\begin{enumerate}
\item Define blockchain and identify its core components
\item Trace the historical evolution from 1991 to present day
\item Understand the double spending problem and how blockchain solves it
\item Identify real-world applications across industries
\end{enumerate}

\vspace{1em}
\textbf{Duration:} 45 minutes

\bottomnote{This lesson establishes the foundation for understanding distributed ledger technology}
\end{frame}

% ==================== AGENDA ====================
\begin{frame}[t]{Lesson Agenda}
\begin{columns}[T]
\column{0.48\textwidth}
\textbf{Part 1: Foundation (20 min)}
\begin{itemize}
\item Historical timeline 1991-2025
\item The double spending problem
\item Why previous digital cash failed
\end{itemize}

\column{0.48\textwidth}
\textbf{Part 2: Core Concepts (25 min)}
\begin{itemize}
\item Block structure and chain linking
\item Distributed ledger architecture
\item Immutability and decentralization
\item Real-world applications
\end{itemize}
\end{columns}

\vspace{1em}
\textbf{Format:} Interactive lecture with visual demonstrations

\bottomnote{Questions welcome throughout the session}
\end{frame}

% ==================== HISTORICAL TIMELINE 1 ====================
\begin{frame}[t]{Historical Timeline: Pre-Bitcoin Era}
\chartplaceholder{5.5cm}{Chart: Timeline visualization 1991-2008}

\vspace{0.5em}
\begin{columns}[T]
\column{0.48\textwidth}
\textbf{1991: Cryptographic Foundations}
\begin{itemize}
\item Haber \& Stornetta: timestamping
\item First blockchain-like structure
\item Published in Journal of Cryptology
\end{itemize}

\column{0.48\textwidth}
\textbf{2004: Reusable Proof-of-Work}
\begin{itemize}
\item Hal Finney introduces RPOW
\item Attempt at digital scarcity
\item Required trusted servers
\end{itemize}
\end{columns}

\bottomnote{Foundation concepts existed decades before Bitcoin}
\end{frame}

% ==================== HISTORICAL TIMELINE 2 ====================
\begin{frame}[t]{Historical Timeline: The Bitcoin Revolution}
\chartplaceholder{5.5cm}{Chart: Timeline visualization 2008-2015}

\vspace{0.5em}
\begin{columns}[T]
\column{0.31\textwidth}
\textbf{October 31, 2008}\\
Satoshi whitepaper:\\
``Bitcoin: A Peer-to-Peer Electronic Cash System''

\column{0.31\textwidth}
\textbf{January 3, 2009}\\
Genesis block mined\\
Network goes live\\
Block reward: 50 BTC

\column{0.31\textwidth}
\textbf{July 30, 2015}\\
Ethereum launches\\
Smart contracts enabled\\
Blockchain 2.0 era begins
\end{columns}

\bottomnote{Bitcoin solved the double spending problem without trusted third parties}
\end{frame}

% ==================== HISTORICAL TIMELINE 3 ====================
\begin{frame}[t]{Historical Timeline: Institutional Adoption}
\chartplaceholder{5.5cm}{Chart: Timeline visualization 2020-2025 with major milestones}

\vspace{0.5em}
\begin{columns}[T]
\column{0.48\textwidth}
\textbf{2020-2024: Corporate Entry}
\begin{itemize}
\item PayPal adds crypto support (2020)
\item Tesla invests \$1.5B in Bitcoin (2021)
\item El Salvador adopts Bitcoin as legal tender (2021)
\item BlackRock files for Bitcoin ETF (2023)
\end{itemize}

\column{0.48\textwidth}
\textbf{2024-2025: Mainstream Integration}
\begin{itemize}
\item Bitcoin ETF approvals (Jan 2024)
\item Institutional custody solutions
\item Central bank digital currencies (CBDCs)
\item 559-617 million global crypto users
\end{itemize}
\end{columns}

\bottomnote{From fringe technology to institutional asset class in 15 years}
\end{frame}

% ==================== THE PROBLEM: DOUBLE SPENDING 1 ====================
\begin{frame}[t]{The Problem: Double Spending}
\begin{columns}[T]
\column{0.48\textwidth}
\textbf{Physical Cash: No Problem}

\chartplaceholder{4cm}{Chart: Physical money exchange diagram}

\begin{itemize}
\item Alice gives \$10 bill to Bob
\item Alice no longer has the bill
\item Physical scarcity prevents duplication
\end{itemize}

\column{0.48\textwidth}
\textbf{Digital Cash: Major Problem}

\chartplaceholder{4cm}{Chart: Digital file duplication diagram}

\begin{itemize}
\item Digital files are easily copied
\item Alice could send same \$10 to Bob and Carol
\item No inherent scarcity in digital format
\end{itemize}
\end{columns}

\bottomnote{Double spending was the fundamental barrier to digital currency}
\end{frame}

% ==================== THE PROBLEM: DOUBLE SPENDING 2 ====================
\begin{frame}[t]{Traditional Solution vs. Blockchain Solution}
\begin{columns}[T]
\column{0.48\textwidth}
\textbf{Centralized Solution (Banks)}

\chartplaceholder{4.5cm}{Chart: Centralized ledger with bank as middleman}

\begin{itemize}
\item[\textcolor{mlgreen}{+}] Prevents double spending
\item[\textcolor{mlgreen}{+}] Familiar and trusted
\item[\textcolor{mlorange}{-}] Single point of failure
\item[\textcolor{mlorange}{-}] Requires intermediary fees
\item[\textcolor{mlorange}{-}] Limited access and hours
\end{itemize}

\column{0.48\textwidth}
\textbf{Blockchain Solution}

\chartplaceholder{4.5cm}{Chart: Distributed ledger with multiple nodes}

\begin{itemize}
\item[\textcolor{mlgreen}{+}] Prevents double spending
\item[\textcolor{mlgreen}{+}] No intermediary needed
\item[\textcolor{mlgreen}{+}] 24/7 global access
\item[\textcolor{mlgreen}{+}] Censorship resistant
\item[\textcolor{mlorange}{-}] New technology to learn
\end{itemize}
\end{columns}

\bottomnote{Blockchain achieves trust through mathematics and cryptography, not institutions}
\end{frame}

% ==================== CORE CONCEPT: BLOCK STRUCTURE ====================
\begin{frame}[t]{Core Concept 1: Block Structure}
\chartplaceholder{5cm}{Chart: Detailed block anatomy showing header and body}

\vspace{0.5em}
\begin{columns}[T]
\column{0.48\textwidth}
\textbf{Block Header}
\begin{itemize}
\item Previous block hash (link to chain)
\item Timestamp
\item Nonce (for proof-of-work)
\item Merkle root (transaction summary)
\end{itemize}

\column{0.48\textwidth}
\textbf{Block Body}
\begin{itemize}
\item List of verified transactions
\item Bitcoin: avg 2,000-3,000 transactions
\item Block size: 1-4 MB (Bitcoin)
\item Created every 10 minutes (Bitcoin)
\end{itemize}
\end{columns}

\bottomnote{Each block is a container of transactions cryptographically linked to the previous block}
\end{frame}

% ==================== CORE CONCEPT: CHAIN LINKING ====================
\begin{frame}[t]{Core Concept 2: Cryptographic Chain}
\chartplaceholder{5.5cm}{Chart: Chain of blocks showing hash connections}

\vspace{0.5em}
\textbf{How Blocks Link Together:}
\begin{itemize}
\item Each block contains the hash of the previous block
\item Hash functions create unique digital fingerprints
\item Changing any data changes the hash
\item Broken hash link = invalid chain
\end{itemize}

\vspace{0.5em}
\textbf{Example:} Block 100 contains hash of Block 99, which contains hash of Block 98, etc.

\bottomnote{The chain structure creates an immutable record stretching back to the genesis block}
\end{frame}

% ==================== CORE CONCEPT: DISTRIBUTED LEDGER ====================
\begin{frame}[t]{Core Concept 3: Distributed Ledger}
\chartplaceholder{5cm}{Chart: Network diagram showing multiple nodes with identical ledgers}

\vspace{0.5em}
\begin{columns}[T]
\column{0.48\textwidth}
\textbf{Key Characteristics}
\begin{itemize}
\item Thousands of identical copies
\item Bitcoin: 15,000+ full nodes
\item Each node validates independently
\item Consensus determines truth
\end{itemize}

\column{0.48\textwidth}
\textbf{Benefits}
\begin{itemize}
\item No single point of failure
\item High availability (99.99\%+)
\item Transparent and auditable
\item Resistant to censorship
\end{itemize}
\end{columns}

\bottomnote{Distribution creates resilience and eliminates dependence on any single entity}
\end{frame}

% ==================== CORE CONCEPT: IMMUTABILITY ====================
\begin{frame}[t]{Core Concept 4: Immutability}
\begin{columns}[T]
\column{0.48\textwidth}
\textbf{Why Changes are Impossible}

\chartplaceholder{4.5cm}{Chart: Cascade effect of changing one block}

\begin{enumerate}
\item Attacker changes Block 100
\item Hash of Block 100 changes
\item Block 101 now points to wrong hash
\item Must recalculate Block 101 hash
\item Must recalculate all subsequent blocks
\item Must do this faster than network
\end{enumerate}

\column{0.48\textwidth}
\textbf{Mathematical Security}

\begin{itemize}
\item Bitcoin network: 400+ EH/s
\item Would need 51\% of computing power
\item Cost: billions of dollars in hardware
\item Economically irrational for attacker
\item Deeper blocks = more secure
\end{itemize}

\vspace{0.5em}
\textbf{Best Practice:} Wait 6 confirmations (60 min) for large transactions

\end{columns}

\bottomnote{Immutability comes from computational difficulty, not legal enforcement}
\end{frame}

% ==================== CORE CONCEPT: DECENTRALIZATION ====================
\begin{frame}[t]{Core Concept 5: Decentralization}
\chartplaceholder{5cm}{Chart: Centralized vs Decentralized vs Distributed network topology}

\vspace{0.5em}
\begin{columns}[T]
\column{0.31\textwidth}
\textbf{Centralized}
\begin{itemize}
\item Single authority
\item Fast decisions
\item Single point of failure
\end{itemize}

\column{0.31\textwidth}
\textbf{Decentralized}
\begin{itemize}
\item Multiple authorities
\item Federated control
\item Resilient to attacks
\end{itemize}

\column{0.31\textwidth}
\textbf{Distributed (Blockchain)}
\begin{itemize}
\item No authorities
\item Peer-to-peer consensus
\item Maximum resilience
\end{itemize}
\end{columns}

\bottomnote{True blockchain systems are distributed, not just decentralized}
\end{frame}

% ==================== KEY STATISTICS ====================
\begin{frame}[t]{Blockchain by the Numbers}
\begin{columns}[T]
\column{0.48\textwidth}
\chartplaceholder{5cm}{Chart: Market statistics visualization}

\column{0.48\textwidth}
\textbf{Market Size (2024-2025)}
\begin{itemize}
\item Bitcoin market cap: \$1.79 trillion
\item Total crypto market cap: \$3+ trillion
\item Daily trading volume: \$100+ billion
\end{itemize}

\vspace{0.5em}
\textbf{User Adoption}
\begin{itemize}
\item Global crypto users: 559-617 million
\item 28\% of US adults own cryptocurrency
\item Growth rate: 34\% year-over-year
\end{itemize}

\vspace{0.5em}
\textbf{Network Activity}
\begin{itemize}
\item Bitcoin transactions: 300,000-400,000/day
\item Ethereum transactions: 1.2 million/day
\item Combined: 2,000+ blockchain networks
\end{itemize}
\end{columns}

\bottomnote{From zero to mainstream adoption in 15 years}
\end{frame}

% ==================== APPLICATIONS 1 ====================
\begin{frame}[t]{Real-World Applications: Finance \& Beyond}
\begin{columns}[T]
\column{0.48\textwidth}
\textbf{Financial Services}
\begin{itemize}
\item Cross-border payments (Ripple, Stellar)
\item Decentralized finance (Uniswap, Aave)
\item Tokenized assets (real estate, stocks)
\item Stablecoins (USDC, USDT)
\end{itemize}

\vspace{0.5em}
\textbf{Supply Chain}
\begin{itemize}
\item Walmart: food traceability
\item Maersk: shipping logistics
\item De Beers: diamond provenance
\item Pharmaceutical authentication
\end{itemize}

\column{0.48\textwidth}
\textbf{Digital Identity}
\begin{itemize}
\item Self-sovereign identity (SSI)
\item Credential verification
\item Healthcare records
\item Academic diplomas
\end{itemize}

\vspace{0.5em}
\textbf{Government \& Public Sector}
\begin{itemize}
\item Land registries (Georgia, Sweden)
\item Voting systems (pilot projects)
\item Public benefit distribution
\item Tax collection and auditing
\end{itemize}
\end{columns}

\bottomnote{Blockchain is transitioning from cryptocurrency to enterprise infrastructure}
\end{frame}

% ==================== APPLICATIONS 2 ====================
\begin{frame}[t]{Career Opportunities in Blockchain}
\chartplaceholder{5cm}{Chart: Job market and salary data visualization}

\vspace{0.5em}
\begin{columns}[T]
\column{0.48\textwidth}
\textbf{Technical Roles}
\begin{itemize}
\item Blockchain developer: \$100k-\$180k
\item Smart contract auditor: \$120k-\$200k
\item Protocol engineer: \$150k-\$250k
\item Cryptography specialist
\end{itemize}

\column{0.48\textwidth}
\textbf{Business Roles}
\begin{itemize}
\item Blockchain consultant: \$90k-\$150k
\item Product manager: \$110k-\$180k
\item Legal and compliance specialist
\item Tokenomics designer
\end{itemize}
\end{columns}

\vspace{0.5em}
\textbf{Industry Demand:} LinkedIn shows 395\% increase in blockchain job postings since 2020

\bottomnote{This course prepares you for the fastest-growing sector in technology}
\end{frame}

% ==================== SUMMARY ====================
\begin{frame}[t]{Summary: Key Takeaways}
\begin{columns}[T]
\column{0.48\textwidth}
\textbf{What We Learned}
\begin{enumerate}
\item Blockchain evolved from 1991 research to \$3T market
\item Solves double spending without intermediaries
\item Core components: blocks, chain, distribution, immutability
\item Applications span finance, supply chain, identity
\end{enumerate}

\column{0.48\textwidth}
\textbf{Next Lesson Preview}
\begin{itemize}
\item L02: Distributed Ledger Technology Deep Dive
\item Consensus mechanisms
\item Byzantine Fault Tolerance
\item Network architecture
\end{itemize}
\end{columns}

\vspace{1em}
\textbf{Preparation:} Review Satoshi Nakamoto whitepaper (provided in course materials)

\bottomnote{Questions? Office hours: Thursdays 14:00-16:00}
\end{frame}

% ==================== CLOSING SLIDE ====================
\begin{frame}[plain]
\vspace{3cm}
\begin{center}
{\Large Thank you}\\[2cm]
{\normalsize Questions?}\\[1cm]
{\small Next: Lesson 2 -- Distributed Ledger Technology}
\end{center}
\end{frame}

\end{document}
