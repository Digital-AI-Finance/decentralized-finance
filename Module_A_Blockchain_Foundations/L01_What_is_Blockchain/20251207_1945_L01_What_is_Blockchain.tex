\documentclass[8pt,aspectratio=169]{beamer}
\usetheme{Madrid}
\usepackage{graphicx}
\usepackage{booktabs}
\usepackage{adjustbox}
\usepackage{multicol}
\usepackage{amsmath}
\usepackage{amssymb}

% Color definitions
\definecolor{mlblue}{RGB}{0,102,204}
\definecolor{mlpurple}{RGB}{51,51,178}
\definecolor{mllavender}{RGB}{173,173,224}
\definecolor{mllavender2}{RGB}{193,193,232}
\definecolor{mllavender3}{RGB}{204,204,235}
\definecolor{mllavender4}{RGB}{214,214,239}
\definecolor{mlorange}{RGB}{255, 127, 14}
\definecolor{mlgreen}{RGB}{44, 160, 44}
\definecolor{mlred}{RGB}{214, 39, 40}
\definecolor{mlgray}{RGB}{127, 127, 127}
\definecolor{lightgray}{RGB}{240, 240, 240}
\definecolor{midgray}{RGB}{180, 180, 180}

% Apply custom colors to Madrid theme
\setbeamercolor{palette primary}{bg=mllavender3,fg=mlpurple}
\setbeamercolor{palette secondary}{bg=mllavender2,fg=mlpurple}
\setbeamercolor{palette tertiary}{bg=mllavender,fg=white}
\setbeamercolor{palette quaternary}{bg=mlpurple,fg=white}

\setbeamercolor{structure}{fg=mlpurple}
\setbeamercolor{section in toc}{fg=mlpurple}
\setbeamercolor{subsection in toc}{fg=mlblue}
\setbeamercolor{title}{fg=mlpurple}
\setbeamercolor{frametitle}{fg=mlpurple,bg=mllavender3}
\setbeamercolor{block title}{bg=mllavender2,fg=mlpurple}
\setbeamercolor{block body}{bg=mllavender4,fg=black}

% Remove navigation symbols
\setbeamertemplate{navigation symbols}{}

% Clean itemize/enumerate
\setbeamertemplate{itemize items}[circle]
\setbeamertemplate{enumerate items}[default]

% Reduce margins for more content space
\setbeamersize{text margin left=5mm,text margin right=5mm}

\title{Lesson 1: What is Blockchain?}
\subtitle{Module A: Blockchain Foundations}
\author{BSc Blockchain \& Cryptocurrency}
\institute{University Course}
\date{2025}

\begin{document}

% Title slide
\begin{frame}[plain]
\titlepage
\end{frame}

% Learning objectives
\begin{frame}[t]{Learning Objectives}
By the end of this lesson, you will be able to:

\begin{enumerate}
\item Explain what blockchain technology is and how it works
\item Describe the historical evolution from 1991 to 2025
\item Understand the double-spending problem and how blockchain solves it
\item Identify key properties that make blockchain unique
\item Compare centralized vs. decentralized systems
\item Recognize real-world blockchain use cases across industries
\end{enumerate}

\vspace{1em}
\textbf{Prerequisites:} Basic understanding of databases and networks
\end{frame}

% Section: Introduction
\section{Introduction}

\begin{frame}[t]{What is Blockchain? A First Definition}
\begin{block}{Blockchain Definition}
A \textbf{blockchain} is a distributed, immutable ledger that records transactions in a chain of blocks, secured by cryptography and maintained by a decentralized network of nodes.
\end{block}

\vspace{0.5em}
\textbf{Key Components:}
\begin{itemize}
\item \textbf{Distributed}: No single point of control or failure
\item \textbf{Immutable}: Once recorded, data cannot be altered retroactively
\item \textbf{Chain of Blocks}: Data organized in sequential, linked blocks
\item \textbf{Cryptography}: Mathematical techniques ensure security and integrity
\item \textbf{Consensus}: Network agrees on the state of the ledger
\end{itemize}

\vspace{0.5em}
\textit{Think of it as a shared spreadsheet that everyone can read, but no one can cheat.}
\end{frame}

\begin{frame}[t]{How Does a Blockchain Work?}
\begin{columns}[T]
\column{0.48\textwidth}
\textbf{Traditional Database}
\begin{itemize}
\item Central server stores all data
\item Single administrator controls access
\item Users trust the central authority
\item Fast but vulnerable to attacks
\item Single point of failure
\end{itemize}

\vspace{0.5em}
\textit{Example:} Your bank's database

\column{0.48\textwidth}
\textbf{Blockchain System}
\begin{itemize}
\item Data replicated across many nodes
\item No single controller (decentralized)
\item Users verify through consensus
\item More resilient, slower transactions
\item No single point of failure
\end{itemize}

\vspace{0.5em}
\textit{Example:} Bitcoin network
\end{columns}

\vspace{1em}
\textbf{Key Insight:} Blockchain trades speed for trust and security
\end{frame}

% Section: Historical Evolution
\section{Historical Evolution}

\begin{frame}[t]{Timeline: 1991-2008 (Pre-Bitcoin)}
\textbf{The Building Blocks}

\begin{itemize}
\item \textbf{1991}: Stuart Haber \& Scott Stornetta propose cryptographically secured chain of blocks
\item \textbf{1992}: Merkle trees incorporated to improve efficiency
\item \textbf{1998}: Nick Szabo designs ``Bit Gold'' (precursor to Bitcoin)
\item \textbf{2004}: Hal Finney creates ``Reusable Proof of Work'' (RPOW)
\item \textbf{2008}: \textcolor{mlpurple}{\textbf{Satoshi Nakamoto}} publishes Bitcoin whitepaper (Oct 31)
\end{itemize}

\vspace{0.5em}
\begin{block}{Key Innovation}
Bitcoin solved the \textbf{double-spending problem} without a trusted third party by combining cryptographic techniques with economic incentives (proof-of-work mining).
\end{block}
\end{frame}

\begin{frame}[t]{Timeline: 2009-2015 (Bitcoin Era)}
\textbf{The First Cryptocurrency}

\begin{itemize}
\item \textbf{2009 Jan}: Bitcoin network launches (Genesis Block mined)
\item \textbf{2010 May}: First real-world transaction (10,000 BTC for 2 pizzas)
\item \textbf{2011}: Alternative cryptocurrencies emerge (Litecoin, Namecoin)
\item \textbf{2013}: Bitcoin price exceeds \$1,000 for first time
\item \textbf{2014}: Ethereum whitepaper published by Vitalik Buterin
\item \textbf{2015}: Ethereum mainnet launches (introducing smart contracts)
\end{itemize}

\vspace{0.5em}
\textbf{Key Development:} Shift from ``blockchain as currency'' to ``blockchain as platform''
\end{frame}

\begin{frame}[t]{Timeline: 2016-2020 (Enterprise Adoption)}
\textbf{Blockchain Goes Mainstream}

\begin{itemize}
\item \textbf{2016}: Enterprise blockchain platforms (Hyperledger Fabric)
\item \textbf{2017}: ICO boom (Initial Coin Offerings raise \$5.6 billion)
\item \textbf{2018}: Security Token Offerings (STOs) emerge
\item \textbf{2019}: Facebook announces Libra (later Diem, now defunct)
\item \textbf{2020}: DeFi (Decentralized Finance) explosion (\$15B to \$100B+ TVL)
\item \textbf{2020}: Central banks explore CBDCs (Digital currencies)
\end{itemize}

\vspace{0.5em}
\textbf{Trend:} From public cryptocurrencies to private enterprise blockchains
\end{frame}

\begin{frame}[t]{Timeline: 2021-2025 (Maturity \& Regulation)}
\textbf{Current State of Blockchain}

\begin{itemize}
\item \textbf{2021}: NFT boom (\$25 billion in sales)
\item \textbf{2022}: Ethereum transitions to Proof-of-Stake (``The Merge'')
\item \textbf{2022}: FTX collapse highlights need for regulation
\item \textbf{2023}: EU's MiCA regulation passes (Markets in Crypto-Assets)
\item \textbf{2024}: Bitcoin ETFs approved in major markets
\item \textbf{2025}: Institutional adoption accelerates (BlackRock, Fidelity)
\end{itemize}

\vspace{0.5em}
\textbf{Current Focus:} Scalability, sustainability, regulatory compliance
\end{frame}

% Section: The Double-Spending Problem
\section{The Double-Spending Problem}

\begin{frame}[t]{What is Double-Spending?}
\begin{columns}[T]
\column{0.48\textwidth}
\textbf{Physical Cash}
\begin{itemize}
\item You have a \$10 bill
\item You give it to Alice
\item Now Alice has the bill
\item You \textbf{cannot} give the same bill to Bob
\item Physical scarcity prevents double-spending
\end{itemize}

\column{0.48\textwidth}
\textbf{Digital Money (Without Blockchain)}
\begin{itemize}
\item You have a digital file: ``10 coins''
\item You send it to Alice
\item You \textbf{still have} a copy of the file
\item You \textbf{can} send the same file to Bob
\item \textcolor{mlred}{Problem:} Digital files are easily copied
\end{itemize}
\end{columns}

\vspace{1em}
\begin{block}{The Challenge}
How do we create digital scarcity without a trusted central authority (like a bank)?
\end{block}
\end{frame}

\begin{frame}[t]{Traditional Solution: Trusted Third Party}
\textbf{Banks Prevent Double-Spending}

\begin{enumerate}
\item Alice wants to send \$10 to Bob
\item Alice's bank checks: Does Alice have \$10?
\item If YES: Bank deducts \$10 from Alice's account
\item Bank adds \$10 to Bob's account
\item Bank updates its central ledger
\end{enumerate}

\vspace{0.5em}
\textbf{Advantages:}
\begin{itemize}
\item Fast transactions
\item Easy to reverse errors
\item Regulatory oversight
\end{itemize}

\vspace{0.5em}
\textbf{Disadvantages:}
\begin{itemize}
\item Must trust the bank
\item Single point of failure
\item Censorship possible
\item High fees for international transfers
\end{itemize}
\end{frame}

\begin{frame}[t]{Blockchain Solution: Distributed Consensus}
\textbf{How Bitcoin Prevents Double-Spending}

\begin{enumerate}
\item Alice broadcasts: ``Send 1 BTC to Bob'' to the entire network
\item Thousands of nodes receive and verify the transaction
\item Miners collect transactions into a new block
\item Miners compete to solve a cryptographic puzzle (Proof-of-Work)
\item First miner to solve puzzle broadcasts the block
\item Other nodes verify and add block to their chain
\item Bob's wallet shows 1 BTC after 6 confirmations ($\approx$ 60 minutes)
\end{enumerate}

\vspace{0.5em}
\textbf{Key Insight:} The longest chain (most computational work) represents the true history. Rewriting history requires more computing power than the entire network combined.
\end{frame}

% Section: Key Properties
\section{Key Properties of Blockchain}

\begin{frame}[t]{Property 1: Decentralization}
\begin{columns}[T]
\column{0.48\textwidth}
\textbf{What It Means:}
\begin{itemize}
\item No central authority controls the network
\item Thousands of nodes maintain copies
\item Decisions made by consensus
\item Anyone can join the network
\end{itemize}

\vspace{0.5em}
\textbf{Benefits:}
\begin{itemize}
\item Censorship resistance
\item No single point of failure
\item Transparent governance
\end{itemize}

\column{0.48\textwidth}
\textbf{Trade-offs:}
\begin{itemize}
\item Slower decision-making
\item Harder to upgrade
\item More energy-intensive
\item Lower transaction throughput
\end{itemize}

\vspace{0.5em}
\textbf{Example:} Bitcoin has $\approx$ 15,000 full nodes worldwide
\end{columns}

\vspace{0.5em}
\textit{Decentralization is a spectrum, not a binary choice}
\end{frame}

\begin{frame}[t]{Property 2: Immutability}
\begin{block}{Immutability Definition}
Once data is written to a blockchain and confirmed, it becomes \textbf{extremely difficult} (practically impossible) to alter or delete.
\end{block}

\vspace{0.5em}
\textbf{How It Works:}
\begin{enumerate}
\item Each block contains a cryptographic hash of the previous block
\item Changing data in Block N would change its hash
\item This breaks the link to Block N+1
\item Attacker must recalculate hashes for ALL subsequent blocks
\item Attacker must do this faster than the honest network
\end{enumerate}

\vspace{0.5em}
\textbf{Practical Implications:}
\begin{itemize}
\item Permanent audit trail
\item Cannot ``cook the books''
\item Mistakes are difficult to fix
\item Requires careful design of smart contracts
\end{itemize}
\end{frame}

\begin{frame}[t]{Property 3: Transparency}
\begin{columns}[T]
\column{0.48\textwidth}
\textbf{Public Blockchains:}
\begin{itemize}
\item All transactions visible to everyone
\item Anyone can verify the entire history
\item Pseudonymous (addresses, not names)
\item Full auditability
\end{itemize}

\vspace{0.5em}
\textit{Example:} You can view every Bitcoin transaction ever made on blockchain explorers

\column{0.48\textwidth}
\textbf{Private/Permissioned Blockchains:}
\begin{itemize}
\item Restricted read/write access
\item Known participants only
\item Selective transparency
\item Enterprise use cases
\end{itemize}

\vspace{0.5em}
\textit{Example:} Hyperledger Fabric for supply chain tracking
\end{columns}

\vspace{1em}
\textbf{Privacy Paradox:} Transparent ledger + pseudonymous addresses = partial privacy
\end{frame}

\begin{frame}[t]{Property 4: Security}
\textbf{Multi-Layered Security}

\begin{enumerate}
\item \textbf{Cryptographic Hashing}: SHA-256 ensures data integrity
\item \textbf{Digital Signatures}: ECDSA proves ownership of assets
\item \textbf{Consensus Mechanisms}: Proof-of-Work/Proof-of-Stake prevent attacks
\item \textbf{Network Distribution}: No single point to attack
\item \textbf{Economic Incentives}: Attacking costs more than potential gain
\end{enumerate}

\vspace{0.5em}
\textbf{Common Attack Vectors:}
\begin{itemize}
\item \textbf{51\% Attack}: Attacker controls majority of mining power
\item \textbf{Smart Contract Bugs}: Coding errors (e.g., DAO hack 2016)
\item \textbf{Private Key Theft}: Wallet compromise
\item \textbf{Exchange Hacks}: Centralized weak points
\end{itemize}

\vspace{0.5em}
\textit{The blockchain itself is secure; the ecosystem around it may not be}
\end{frame}

% Section: Centralized vs Decentralized
\section{Centralized vs. Decentralized Systems}

\begin{frame}[t]{System Architecture Comparison}
\begin{columns}[T]
\column{0.48\textwidth}
\textbf{Centralized Systems}
\begin{itemize}
\item Single entity controls data
\item Fast decision-making
\item Efficient resource use
\item Easy to upgrade
\item Clear accountability
\item User-friendly interfaces
\end{itemize}

\vspace{0.5em}
\textbf{Examples:}
\begin{itemize}
\item Traditional banks
\item Facebook, Google
\item Amazon AWS
\end{itemize}

\column{0.48\textwidth}
\textbf{Decentralized Systems}
\begin{itemize}
\item Distributed control
\item Slower consensus required
\item Resource-intensive
\item Difficult upgrades
\item Shared responsibility
\item Often technical UX
\end{itemize}

\vspace{0.5em}
\textbf{Examples:}
\begin{itemize}
\item Bitcoin, Ethereum
\item IPFS (file storage)
\item Tor network
\end{itemize}
\end{columns}

\vspace{0.5em}
\textit{Neither is inherently better; it depends on the use case}
\end{frame}

\begin{frame}[t]{When to Use Blockchain vs. Traditional Database}
\begin{columns}[T]
\column{0.48\textwidth}
\textbf{Use Blockchain When:}
\begin{itemize}
\item Multiple parties need to write data
\item Parties don't fully trust each other
\item Immutable audit trail required
\item Removing intermediaries adds value
\item Transparency is critical
\item Censorship resistance needed
\end{itemize}

\vspace{0.5em}
\textcolor{mlgreen}{Good fit:} Supply chain tracking, land registries, cross-border payments

\column{0.48\textwidth}
\textbf{Use Traditional Database When:}
\begin{itemize}
\item Single organization controls data
\item High transaction throughput needed
\item Data updates/deletions required
\item Strong privacy is essential
\item Existing solutions work well
\item Energy efficiency matters
\end{itemize}

\vspace{0.5em}
\textcolor{mlred}{Poor fit:} Social media posts, personal files, high-frequency trading
\end{columns}

\vspace{0.5em}
\textit{Blockchain is not a solution looking for a problem; it solves specific trust challenges}
\end{frame}

% Section: Use Cases
\section{Real-World Use Cases}

\begin{frame}[t]{Finance \& Banking}
\textbf{Cryptocurrency \& Payments}
\begin{itemize}
\item \textbf{Bitcoin}: Peer-to-peer electronic cash system (market cap \$800B+)
\item \textbf{Stablecoins}: USDC, USDT (pegged to fiat currencies)
\item \textbf{Cross-border Payments}: Ripple/XRP for banks
\item \textbf{Remittances}: Stellar for low-cost international transfers
\end{itemize}

\vspace{0.5em}
\textbf{Decentralized Finance (DeFi)}
\begin{itemize}
\item \textbf{Lending/Borrowing}: Aave, Compound (no credit checks)
\item \textbf{Decentralized Exchanges}: Uniswap, PancakeSwap
\item \textbf{Yield Farming}: Earn interest on crypto holdings
\item \textbf{Derivatives}: Perpetual swaps, options trading
\end{itemize}

\vspace{0.5em}
\textit{DeFi Total Value Locked (TVL): \$50B+ as of 2025}
\end{frame}

\begin{frame}[t]{Supply Chain \& Logistics}
\textbf{Transparency \& Traceability}

\begin{itemize}
\item \textbf{Walmart + IBM Food Trust}: Track produce from farm to store
  \begin{itemize}
  \item Reduced food recall time from 7 days to 2.2 seconds
  \end{itemize}
\item \textbf{Maersk + TradeLens}: Shipping container tracking
  \begin{itemize}
  \item 150+ organizations, 1.5B shipping events logged
  \end{itemize}
\item \textbf{De Beers}: Diamond provenance on blockchain (Tracr platform)
\item \textbf{VeChain}: Luxury goods authentication (Louis Vuitton)
\end{itemize}

\vspace{0.5em}
\textbf{Benefits:}
\begin{itemize}
\item Reduce counterfeiting
\item Improve food safety
\item Streamline customs processes
\item Verify ethical sourcing
\end{itemize}
\end{frame}

\begin{frame}[t]{Healthcare}
\textbf{Patient Data Management}

\begin{itemize}
\item \textbf{Electronic Health Records}: Patients control access to medical data
\item \textbf{Drug Traceability}: Combat counterfeit pharmaceuticals
\item \textbf{Clinical Trials}: Transparent, tamper-proof trial data
\item \textbf{Insurance Claims}: Automated, fraud-resistant processing
\end{itemize}

\vspace{0.5em}
\textbf{Example Projects:}
\begin{itemize}
\item \textbf{MedRec} (MIT): Patient-centered EHR system
\item \textbf{Chronicled}: Pharmaceutical supply chain verification
\item \textbf{Guardtime}: Estonian national health records on blockchain
\end{itemize}

\vspace{0.5em}
\textbf{Challenges:} GDPR compliance (right to be forgotten vs. immutability)
\end{frame}

\begin{frame}[t]{Government \& Public Services}
\textbf{Digital Identity \& Voting}

\begin{itemize}
\item \textbf{Estonia e-Residency}: Digital identity for 100,000+ global citizens
\item \textbf{Dubai Land Registry}: All real estate on blockchain by 2025
\item \textbf{Georgia}: Land titles recorded on blockchain since 2016
\item \textbf{Voatz}: Mobile blockchain voting (limited pilots in West Virginia)
\end{itemize}

\vspace{0.5em}
\textbf{Central Bank Digital Currencies (CBDCs):}
\begin{itemize}
\item \textbf{China}: Digital Yuan (e-CNY) in widespread use
\item \textbf{European Union}: Digital Euro pilot programs
\item \textbf{Bahamas}: Sand Dollar (fully launched 2020)
\end{itemize}

\vspace{0.5em}
\textit{90+ countries exploring CBDCs as of 2025}
\end{frame}

\begin{frame}[t]{Other Emerging Use Cases}
\begin{columns}[T]
\column{0.48\textwidth}
\textbf{Energy \& Sustainability}
\begin{itemize}
\item Peer-to-peer energy trading
\item Carbon credit tracking
\item Renewable energy certificates
\item Electric vehicle charging networks
\end{itemize}

\vspace{0.5em}
\textbf{Media \& Entertainment}
\begin{itemize}
\item NFTs (art, music, gaming)
\item Royalty distribution
\item Content licensing
\item Anti-piracy measures
\end{itemize}

\column{0.48\textwidth}
\textbf{Education}
\begin{itemize}
\item Academic credential verification
\item Lifelong learning records
\item Decentralized universities
\item Micro-credentialing
\end{itemize}

\vspace{0.5em}
\textbf{Internet of Things (IoT)}
\begin{itemize}
\item Device identity management
\item Secure firmware updates
\item Machine-to-machine payments
\item Data marketplace
\end{itemize}
\end{columns}
\end{frame}

% Key Takeaways
\section{Summary}

\begin{frame}[t]{Key Takeaways}
\textbf{What You Should Remember:}

\begin{enumerate}
\item \textbf{Blockchain} is a distributed, immutable ledger secured by cryptography
\item \textbf{Historical Evolution}: From 1991 research to 2025 institutional adoption
\item \textbf{Double-Spending Solution}: Consensus eliminates need for trusted intermediaries
\item \textbf{Core Properties}: Decentralization, immutability, transparency, security
\item \textbf{Architecture Choice}: Centralized for speed/efficiency, decentralized for trust
\item \textbf{Use Cases}: Finance, supply chain, healthcare, government, and beyond
\end{enumerate}

\vspace{0.5em}
\begin{block}{Critical Insight}
Blockchain is not a universal solution. It trades computational efficiency for trust and resilience. The key question is: \textit{Does your problem require decentralized trust?}
\end{block}
\end{frame}

% Discussion Questions
\begin{frame}[t]{Discussion Questions}
\textbf{Consider and discuss:}

\begin{enumerate}
\item \textbf{Trust vs. Efficiency}: In what scenarios is the trade-off worth it?
  \begin{itemize}
  \item Example: International remittances vs. buying coffee
  \end{itemize}

\item \textbf{Privacy Paradox}: How can we balance transparency with user privacy?
  \begin{itemize}
  \item Explore: Zero-knowledge proofs, private transactions
  \end{itemize}

\item \textbf{Environmental Impact}: Is Proof-of-Work's energy consumption justified?
  \begin{itemize}
  \item Compare: PoW vs. PoS vs. traditional banking infrastructure
  \end{itemize}
\end{enumerate}

\vspace{0.5em}
\textit{Prepare your thoughts for next session's discussion}
\end{frame}

% References
\begin{frame}[t]{References \& Resources}
\begin{columns}[T]
\column{0.48\textwidth}
\textbf{Foundational Papers}
\begin{itemize}
\item Nakamoto (2008): \textit{Bitcoin Whitepaper}
\item Buterin (2014): \textit{Ethereum Whitepaper}
\item Haber \& Stornetta (1991): \textit{Timestamping Digital Documents}
\end{itemize}

\vspace{0.5em}
\textbf{Books}
\begin{itemize}
\item Antonopoulos (2023): \textit{Mastering Bitcoin}
\item Narayanan et al. (2016): \textit{Bitcoin and Cryptocurrency Technologies}
\end{itemize}

\column{0.48\textwidth}
\textbf{Online Resources}
\begin{itemize}
\item Blockchain.com Explorer
\item Ethereum.org Documentation
\item CoinDesk Research
\item MIT OpenCourseWare: Blockchain
\end{itemize}

\vspace{0.5em}
\textbf{Industry Reports}
\begin{itemize}
\item Gartner Blockchain Trends
\item PwC Blockchain Survey
\item CB Insights: Crypto Trends
\end{itemize}
\end{columns}
\end{frame}

% Next Lesson Preview
\begin{frame}[t]{Next Lesson Preview}
\textbf{L02: Distributed Ledger Technology (DLT)}

\vspace{0.5em}
We will explore:
\begin{itemize}
\item Deep dive into DLT concepts and architectures
\item The Byzantine Generals Problem and consensus challenges
\item Network topologies (centralized, decentralized, distributed)
\item Anatomy of a block (headers, transactions, Merkle trees)
\item Types of nodes (full nodes, light nodes, miners)
\item Permissioned vs. permissionless blockchains
\end{itemize}

\vspace{0.5em}
\textbf{Preparation:} Review basic networking concepts and data structures (hash tables, trees)
\end{frame}

\begin{frame}[plain]
\vspace{3cm}
\begin{center}
{\Large Thank you}\\[2cm]
{\normalsize Questions?}\\[1cm]
{\small See you in Lesson 2: DLT Concepts}
\end{center}
\end{frame}

\end{document}
